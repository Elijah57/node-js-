\section{Console}\label{console}

\begin{quote}
Stability: 2 - Stable
\end{quote}

The \texttt{node:console} module provides a simple debugging console
that is similar to the JavaScript console mechanism provided by web
browsers.

The module exports two specific components:

\begin{itemize}
\tightlist
\item
  A \texttt{Console} class with methods such as \texttt{console.log()},
  \texttt{console.error()}, and \texttt{console.warn()} that can be used
  to write to any Node.js stream.
\item
  A global \texttt{console} instance configured to write to
  \href{process.md\#processstdout}{\texttt{process.stdout}} and
  \href{process.md\#processstderr}{\texttt{process.stderr}}. The global
  \texttt{console} can be used without calling
  \texttt{require(\textquotesingle{}node:console\textquotesingle{})}.
\end{itemize}

\emph{\textbf{Warning}}: The global console object's methods are neither
consistently synchronous like the browser APIs they resemble, nor are
they consistently asynchronous like all other Node.js streams. See the
\href{process.md\#a-note-on-process-io}{note on process I/O} for more
information.

Example using the global \texttt{console}:

\begin{Shaded}
\begin{Highlighting}[]
\BuiltInTok{console}\OperatorTok{.}\FunctionTok{log}\NormalTok{(}\StringTok{\textquotesingle{}hello world\textquotesingle{}}\NormalTok{)}\OperatorTok{;}
\CommentTok{// Prints: hello world, to stdout}
\BuiltInTok{console}\OperatorTok{.}\FunctionTok{log}\NormalTok{(}\StringTok{\textquotesingle{}hello \%s\textquotesingle{}}\OperatorTok{,} \StringTok{\textquotesingle{}world\textquotesingle{}}\NormalTok{)}\OperatorTok{;}
\CommentTok{// Prints: hello world, to stdout}
\BuiltInTok{console}\OperatorTok{.}\FunctionTok{error}\NormalTok{(}\KeywordTok{new} \BuiltInTok{Error}\NormalTok{(}\StringTok{\textquotesingle{}Whoops, something bad happened\textquotesingle{}}\NormalTok{))}\OperatorTok{;}
\CommentTok{// Prints error message and stack trace to stderr:}
\CommentTok{//   Error: Whoops, something bad happened}
\CommentTok{//     at [eval]:5:15}
\CommentTok{//     at Script.runInThisContext (node:vm:132:18)}
\CommentTok{//     at Object.runInThisContext (node:vm:309:38)}
\CommentTok{//     at node:internal/process/execution:77:19}
\CommentTok{//     at [eval]{-}wrapper:6:22}
\CommentTok{//     at evalScript (node:internal/process/execution:76:60)}
\CommentTok{//     at node:internal/main/eval\_string:23:3}

\KeywordTok{const}\NormalTok{ name }\OperatorTok{=} \StringTok{\textquotesingle{}Will Robinson\textquotesingle{}}\OperatorTok{;}
\BuiltInTok{console}\OperatorTok{.}\FunctionTok{warn}\NormalTok{(}\VerbatimStringTok{\textasciigrave{}Danger }\SpecialCharTok{$\{}\NormalTok{name}\SpecialCharTok{\}}\VerbatimStringTok{! Danger!\textasciigrave{}}\NormalTok{)}\OperatorTok{;}
\CommentTok{// Prints: Danger Will Robinson! Danger!, to stderr}
\end{Highlighting}
\end{Shaded}

Example using the \texttt{Console} class:

\begin{Shaded}
\begin{Highlighting}[]
\KeywordTok{const}\NormalTok{ out }\OperatorTok{=} \FunctionTok{getStreamSomehow}\NormalTok{()}\OperatorTok{;}
\KeywordTok{const}\NormalTok{ err }\OperatorTok{=} \FunctionTok{getStreamSomehow}\NormalTok{()}\OperatorTok{;}
\KeywordTok{const}\NormalTok{ myConsole }\OperatorTok{=} \KeywordTok{new} \BuiltInTok{console}\OperatorTok{.}\FunctionTok{Console}\NormalTok{(out}\OperatorTok{,}\NormalTok{ err)}\OperatorTok{;}

\NormalTok{myConsole}\OperatorTok{.}\FunctionTok{log}\NormalTok{(}\StringTok{\textquotesingle{}hello world\textquotesingle{}}\NormalTok{)}\OperatorTok{;}
\CommentTok{// Prints: hello world, to out}
\NormalTok{myConsole}\OperatorTok{.}\FunctionTok{log}\NormalTok{(}\StringTok{\textquotesingle{}hello \%s\textquotesingle{}}\OperatorTok{,} \StringTok{\textquotesingle{}world\textquotesingle{}}\NormalTok{)}\OperatorTok{;}
\CommentTok{// Prints: hello world, to out}
\NormalTok{myConsole}\OperatorTok{.}\FunctionTok{error}\NormalTok{(}\KeywordTok{new} \BuiltInTok{Error}\NormalTok{(}\StringTok{\textquotesingle{}Whoops, something bad happened\textquotesingle{}}\NormalTok{))}\OperatorTok{;}
\CommentTok{// Prints: [Error: Whoops, something bad happened], to err}

\KeywordTok{const}\NormalTok{ name }\OperatorTok{=} \StringTok{\textquotesingle{}Will Robinson\textquotesingle{}}\OperatorTok{;}
\NormalTok{myConsole}\OperatorTok{.}\FunctionTok{warn}\NormalTok{(}\VerbatimStringTok{\textasciigrave{}Danger }\SpecialCharTok{$\{}\NormalTok{name}\SpecialCharTok{\}}\VerbatimStringTok{! Danger!\textasciigrave{}}\NormalTok{)}\OperatorTok{;}
\CommentTok{// Prints: Danger Will Robinson! Danger!, to err}
\end{Highlighting}
\end{Shaded}

\subsection{\texorpdfstring{Class:
\texttt{Console}}{Class: Console}}\label{class-console}

The \texttt{Console} class can be used to create a simple logger with
configurable output streams and can be accessed using either
\texttt{require(\textquotesingle{}node:console\textquotesingle{}).Console}
or \texttt{console.Console} (or their destructured counterparts):

\begin{Shaded}
\begin{Highlighting}[]
\KeywordTok{const}\NormalTok{ \{ }\BuiltInTok{Console}\NormalTok{ \} }\OperatorTok{=} \PreprocessorTok{require}\NormalTok{(}\StringTok{\textquotesingle{}node:console\textquotesingle{}}\NormalTok{)}\OperatorTok{;}
\end{Highlighting}
\end{Shaded}

\begin{Shaded}
\begin{Highlighting}[]
\KeywordTok{const}\NormalTok{ \{ }\BuiltInTok{Console}\NormalTok{ \} }\OperatorTok{=} \BuiltInTok{console}\OperatorTok{;}
\end{Highlighting}
\end{Shaded}

\subsubsection{\texorpdfstring{\texttt{new\ Console(stdout{[},\ stderr{]}{[},\ ignoreErrors{]})}}{new Console(stdout{[}, stderr{]}{[}, ignoreErrors{]})}}\label{new-consolestdout-stderr-ignoreerrors}

\subsubsection{\texorpdfstring{\texttt{new\ Console(options)}}{new Console(options)}}\label{new-consoleoptions}

\begin{itemize}
\tightlist
\item
  \texttt{options} \{Object\}

  \begin{itemize}
  \tightlist
  \item
    \texttt{stdout} \{stream.Writable\}
  \item
    \texttt{stderr} \{stream.Writable\}
  \item
    \texttt{ignoreErrors} \{boolean\} Ignore errors when writing to the
    underlying streams. \textbf{Default:} \texttt{true}.
  \item
    \texttt{colorMode} \{boolean\textbar string\} Set color support for
    this \texttt{Console} instance. Setting to \texttt{true} enables
    coloring while inspecting values. Setting to \texttt{false} disables
    coloring while inspecting values. Setting to
    \texttt{\textquotesingle{}auto\textquotesingle{}} makes color
    support depend on the value of the \texttt{isTTY} property and the
    value returned by \texttt{getColorDepth()} on the respective stream.
    This option can not be used, if \texttt{inspectOptions.colors} is
    set as well. \textbf{Default:}
    \texttt{\textquotesingle{}auto\textquotesingle{}}.
  \item
    \texttt{inspectOptions} \{Object\} Specifies options that are passed
    along to
    \href{util.md\#utilinspectobject-options}{\texttt{util.inspect()}}.
  \item
    \texttt{groupIndentation} \{number\} Set group indentation.
    \textbf{Default:} \texttt{2}.
  \end{itemize}
\end{itemize}

Creates a new \texttt{Console} with one or two writable stream
instances. \texttt{stdout} is a writable stream to print log or info
output. \texttt{stderr} is used for warning or error output. If
\texttt{stderr} is not provided, \texttt{stdout} is used for
\texttt{stderr}.

\begin{Shaded}
\begin{Highlighting}[]
\KeywordTok{const}\NormalTok{ output }\OperatorTok{=}\NormalTok{ fs}\OperatorTok{.}\FunctionTok{createWriteStream}\NormalTok{(}\StringTok{\textquotesingle{}./stdout.log\textquotesingle{}}\NormalTok{)}\OperatorTok{;}
\KeywordTok{const}\NormalTok{ errorOutput }\OperatorTok{=}\NormalTok{ fs}\OperatorTok{.}\FunctionTok{createWriteStream}\NormalTok{(}\StringTok{\textquotesingle{}./stderr.log\textquotesingle{}}\NormalTok{)}\OperatorTok{;}
\CommentTok{// Custom simple logger}
\KeywordTok{const}\NormalTok{ logger }\OperatorTok{=} \KeywordTok{new} \BuiltInTok{Console}\NormalTok{(\{ }\DataTypeTok{stdout}\OperatorTok{:}\NormalTok{ output}\OperatorTok{,} \DataTypeTok{stderr}\OperatorTok{:}\NormalTok{ errorOutput \})}\OperatorTok{;}
\CommentTok{// use it like console}
\KeywordTok{const}\NormalTok{ count }\OperatorTok{=} \DecValTok{5}\OperatorTok{;}
\NormalTok{logger}\OperatorTok{.}\FunctionTok{log}\NormalTok{(}\StringTok{\textquotesingle{}count: \%d\textquotesingle{}}\OperatorTok{,}\NormalTok{ count)}\OperatorTok{;}
\CommentTok{// In stdout.log: count 5}
\end{Highlighting}
\end{Shaded}

The global \texttt{console} is a special \texttt{Console} whose output
is sent to \href{process.md\#processstdout}{\texttt{process.stdout}} and
\href{process.md\#processstderr}{\texttt{process.stderr}}. It is
equivalent to calling:

\begin{Shaded}
\begin{Highlighting}[]
\KeywordTok{new} \BuiltInTok{Console}\NormalTok{(\{ }\DataTypeTok{stdout}\OperatorTok{:} \BuiltInTok{process}\OperatorTok{.}\AttributeTok{stdout}\OperatorTok{,} \DataTypeTok{stderr}\OperatorTok{:} \BuiltInTok{process}\OperatorTok{.}\AttributeTok{stderr}\NormalTok{ \})}\OperatorTok{;}
\end{Highlighting}
\end{Shaded}

\subsubsection{\texorpdfstring{\texttt{console.assert(value{[},\ ...message{]})}}{console.assert(value{[}, ...message{]})}}\label{console.assertvalue-...message}

\begin{itemize}
\tightlist
\item
  \texttt{value} \{any\} The value tested for being truthy.
\item
  \texttt{...message} \{any\} All arguments besides \texttt{value} are
  used as error message.
\end{itemize}

\texttt{console.assert()} writes a message if \texttt{value} is
\href{https://developer.mozilla.org/en-US/docs/Glossary/Falsy}{falsy} or
omitted. It only writes a message and does not otherwise affect
execution. The output always starts with \texttt{"Assertion\ failed"}.
If provided, \texttt{message} is formatted using
\href{util.md\#utilformatformat-args}{\texttt{util.format()}}.

If \texttt{value} is
\href{https://developer.mozilla.org/en-US/docs/Glossary/Truthy}{truthy},
nothing happens.

\begin{Shaded}
\begin{Highlighting}[]
\BuiltInTok{console}\OperatorTok{.}\FunctionTok{assert}\NormalTok{(}\KeywordTok{true}\OperatorTok{,} \StringTok{\textquotesingle{}does nothing\textquotesingle{}}\NormalTok{)}\OperatorTok{;}

\BuiltInTok{console}\OperatorTok{.}\FunctionTok{assert}\NormalTok{(}\KeywordTok{false}\OperatorTok{,} \StringTok{\textquotesingle{}Whoops \%s work\textquotesingle{}}\OperatorTok{,} \StringTok{\textquotesingle{}didn}\SpecialCharTok{\textbackslash{}\textquotesingle{}}\StringTok{t\textquotesingle{}}\NormalTok{)}\OperatorTok{;}
\CommentTok{// Assertion failed: Whoops didn\textquotesingle{}t work}

\BuiltInTok{console}\OperatorTok{.}\FunctionTok{assert}\NormalTok{()}\OperatorTok{;}
\CommentTok{// Assertion failed}
\end{Highlighting}
\end{Shaded}

\subsubsection{\texorpdfstring{\texttt{console.clear()}}{console.clear()}}\label{console.clear}

When \texttt{stdout} is a TTY, calling \texttt{console.clear()} will
attempt to clear the TTY. When \texttt{stdout} is not a TTY, this method
does nothing.

The specific operation of \texttt{console.clear()} can vary across
operating systems and terminal types. For most Linux operating systems,
\texttt{console.clear()} operates similarly to the \texttt{clear} shell
command. On Windows, \texttt{console.clear()} will clear only the output
in the current terminal viewport for the Node.js binary.

\subsubsection{\texorpdfstring{\texttt{console.count({[}label{]})}}{console.count({[}label{]})}}\label{console.countlabel}

\begin{itemize}
\tightlist
\item
  \texttt{label} \{string\} The display label for the counter.
  \textbf{Default:}
  \texttt{\textquotesingle{}default\textquotesingle{}}.
\end{itemize}

Maintains an internal counter specific to \texttt{label} and outputs to
\texttt{stdout} the number of times \texttt{console.count()} has been
called with the given \texttt{label}.

\begin{Shaded}
\begin{Highlighting}[]
\OperatorTok{\textgreater{}} \BuiltInTok{console}\OperatorTok{.}\FunctionTok{count}\NormalTok{()}
\ImportTok{default}\OperatorTok{:} \DecValTok{1}
\KeywordTok{undefined}
\OperatorTok{\textgreater{}} \BuiltInTok{console}\OperatorTok{.}\FunctionTok{count}\NormalTok{(}\StringTok{\textquotesingle{}default\textquotesingle{}}\NormalTok{)}
\ImportTok{default}\OperatorTok{:} \DecValTok{2}
\KeywordTok{undefined}
\OperatorTok{\textgreater{}} \BuiltInTok{console}\OperatorTok{.}\FunctionTok{count}\NormalTok{(}\StringTok{\textquotesingle{}abc\textquotesingle{}}\NormalTok{)}
\NormalTok{abc}\OperatorTok{:} \DecValTok{1}
\KeywordTok{undefined}
\OperatorTok{\textgreater{}} \BuiltInTok{console}\OperatorTok{.}\FunctionTok{count}\NormalTok{(}\StringTok{\textquotesingle{}xyz\textquotesingle{}}\NormalTok{)}
\NormalTok{xyz}\OperatorTok{:} \DecValTok{1}
\KeywordTok{undefined}
\OperatorTok{\textgreater{}} \BuiltInTok{console}\OperatorTok{.}\FunctionTok{count}\NormalTok{(}\StringTok{\textquotesingle{}abc\textquotesingle{}}\NormalTok{)}
\NormalTok{abc}\OperatorTok{:} \DecValTok{2}
\KeywordTok{undefined}
\OperatorTok{\textgreater{}} \BuiltInTok{console}\OperatorTok{.}\FunctionTok{count}\NormalTok{()}
\ImportTok{default}\OperatorTok{:} \DecValTok{3}
\KeywordTok{undefined}
\OperatorTok{\textgreater{}}
\end{Highlighting}
\end{Shaded}

\subsubsection{\texorpdfstring{\texttt{console.countReset({[}label{]})}}{console.countReset({[}label{]})}}\label{console.countresetlabel}

\begin{itemize}
\tightlist
\item
  \texttt{label} \{string\} The display label for the counter.
  \textbf{Default:}
  \texttt{\textquotesingle{}default\textquotesingle{}}.
\end{itemize}

Resets the internal counter specific to \texttt{label}.

\begin{Shaded}
\begin{Highlighting}[]
\OperatorTok{\textgreater{}} \BuiltInTok{console}\OperatorTok{.}\FunctionTok{count}\NormalTok{(}\StringTok{\textquotesingle{}abc\textquotesingle{}}\NormalTok{)}\OperatorTok{;}
\NormalTok{abc}\OperatorTok{:} \DecValTok{1}
\KeywordTok{undefined}
\OperatorTok{\textgreater{}} \BuiltInTok{console}\OperatorTok{.}\FunctionTok{countReset}\NormalTok{(}\StringTok{\textquotesingle{}abc\textquotesingle{}}\NormalTok{)}\OperatorTok{;}
\KeywordTok{undefined}
\OperatorTok{\textgreater{}} \BuiltInTok{console}\OperatorTok{.}\FunctionTok{count}\NormalTok{(}\StringTok{\textquotesingle{}abc\textquotesingle{}}\NormalTok{)}\OperatorTok{;}
\NormalTok{abc}\OperatorTok{:} \DecValTok{1}
\KeywordTok{undefined}
\OperatorTok{\textgreater{}}
\end{Highlighting}
\end{Shaded}

\subsubsection{\texorpdfstring{\texttt{console.debug(data{[},\ ...args{]})}}{console.debug(data{[}, ...args{]})}}\label{console.debugdata-...args}

\begin{itemize}
\tightlist
\item
  \texttt{data} \{any\}
\item
  \texttt{...args} \{any\}
\end{itemize}

The \texttt{console.debug()} function is an alias for
\hyperref[consolelogdata-args]{\texttt{console.log()}}.

\subsubsection{\texorpdfstring{\texttt{console.dir(obj{[},\ options{]})}}{console.dir(obj{[}, options{]})}}\label{console.dirobj-options}

\begin{itemize}
\tightlist
\item
  \texttt{obj} \{any\}
\item
  \texttt{options} \{Object\}

  \begin{itemize}
  \tightlist
  \item
    \texttt{showHidden} \{boolean\} If \texttt{true} then the object's
    non-enumerable and symbol properties will be shown too.
    \textbf{Default:} \texttt{false}.
  \item
    \texttt{depth} \{number\} Tells
    \href{util.md\#utilinspectobject-options}{\texttt{util.inspect()}}
    how many times to recurse while formatting the object. This is
    useful for inspecting large complicated objects. To make it recurse
    indefinitely, pass \texttt{null}. \textbf{Default:} \texttt{2}.
  \item
    \texttt{colors} \{boolean\} If \texttt{true}, then the output will
    be styled with ANSI color codes. Colors are customizable; see
    \href{util.md\#customizing-utilinspect-colors}{customizing
    \texttt{util.inspect()} colors}. \textbf{Default:} \texttt{false}.
  \end{itemize}
\end{itemize}

Uses \href{util.md\#utilinspectobject-options}{\texttt{util.inspect()}}
on \texttt{obj} and prints the resulting string to \texttt{stdout}. This
function bypasses any custom \texttt{inspect()} function defined on
\texttt{obj}.

\subsubsection{\texorpdfstring{\texttt{console.dirxml(...data)}}{console.dirxml(...data)}}\label{console.dirxml...data}

\begin{itemize}
\tightlist
\item
  \texttt{...data} \{any\}
\end{itemize}

This method calls \texttt{console.log()} passing it the arguments
received. This method does not produce any XML formatting.

\subsubsection{\texorpdfstring{\texttt{console.error({[}data{]}{[},\ ...args{]})}}{console.error({[}data{]}{[}, ...args{]})}}\label{console.errordata-...args}

\begin{itemize}
\tightlist
\item
  \texttt{data} \{any\}
\item
  \texttt{...args} \{any\}
\end{itemize}

Prints to \texttt{stderr} with newline. Multiple arguments can be
passed, with the first used as the primary message and all additional
used as substitution values similar to printf(3) (the arguments are all
passed to
\href{util.md\#utilformatformat-args}{\texttt{util.format()}}).

\begin{Shaded}
\begin{Highlighting}[]
\KeywordTok{const}\NormalTok{ code }\OperatorTok{=} \DecValTok{5}\OperatorTok{;}
\BuiltInTok{console}\OperatorTok{.}\FunctionTok{error}\NormalTok{(}\StringTok{\textquotesingle{}error \#\%d\textquotesingle{}}\OperatorTok{,}\NormalTok{ code)}\OperatorTok{;}
\CommentTok{// Prints: error \#5, to stderr}
\BuiltInTok{console}\OperatorTok{.}\FunctionTok{error}\NormalTok{(}\StringTok{\textquotesingle{}error\textquotesingle{}}\OperatorTok{,}\NormalTok{ code)}\OperatorTok{;}
\CommentTok{// Prints: error 5, to stderr}
\end{Highlighting}
\end{Shaded}

If formatting elements (e.g.~\texttt{\%d}) are not found in the first
string then
\href{util.md\#utilinspectobject-options}{\texttt{util.inspect()}} is
called on each argument and the resulting string values are
concatenated. See
\href{util.md\#utilformatformat-args}{\texttt{util.format()}} for more
information.

\subsubsection{\texorpdfstring{\texttt{console.group({[}...label{]})}}{console.group({[}...label{]})}}\label{console.group...label}

\begin{itemize}
\tightlist
\item
  \texttt{...label} \{any\}
\end{itemize}

Increases indentation of subsequent lines by spaces for
\texttt{groupIndentation} length.

If one or more \texttt{label}s are provided, those are printed first
without the additional indentation.

\subsubsection{\texorpdfstring{\texttt{console.groupCollapsed()}}{console.groupCollapsed()}}\label{console.groupcollapsed}

An alias for \hyperref[consolegrouplabel]{\texttt{console.group()}}.

\subsubsection{\texorpdfstring{\texttt{console.groupEnd()}}{console.groupEnd()}}\label{console.groupend}

Decreases indentation of subsequent lines by spaces for
\texttt{groupIndentation} length.

\subsubsection{\texorpdfstring{\texttt{console.info({[}data{]}{[},\ ...args{]})}}{console.info({[}data{]}{[}, ...args{]})}}\label{console.infodata-...args}

\begin{itemize}
\tightlist
\item
  \texttt{data} \{any\}
\item
  \texttt{...args} \{any\}
\end{itemize}

The \texttt{console.info()} function is an alias for
\hyperref[consolelogdata-args]{\texttt{console.log()}}.

\subsubsection{\texorpdfstring{\texttt{console.log({[}data{]}{[},\ ...args{]})}}{console.log({[}data{]}{[}, ...args{]})}}\label{console.logdata-...args}

\begin{itemize}
\tightlist
\item
  \texttt{data} \{any\}
\item
  \texttt{...args} \{any\}
\end{itemize}

Prints to \texttt{stdout} with newline. Multiple arguments can be
passed, with the first used as the primary message and all additional
used as substitution values similar to printf(3) (the arguments are all
passed to
\href{util.md\#utilformatformat-args}{\texttt{util.format()}}).

\begin{Shaded}
\begin{Highlighting}[]
\KeywordTok{const}\NormalTok{ count }\OperatorTok{=} \DecValTok{5}\OperatorTok{;}
\BuiltInTok{console}\OperatorTok{.}\FunctionTok{log}\NormalTok{(}\StringTok{\textquotesingle{}count: \%d\textquotesingle{}}\OperatorTok{,}\NormalTok{ count)}\OperatorTok{;}
\CommentTok{// Prints: count: 5, to stdout}
\BuiltInTok{console}\OperatorTok{.}\FunctionTok{log}\NormalTok{(}\StringTok{\textquotesingle{}count:\textquotesingle{}}\OperatorTok{,}\NormalTok{ count)}\OperatorTok{;}
\CommentTok{// Prints: count: 5, to stdout}
\end{Highlighting}
\end{Shaded}

See \href{util.md\#utilformatformat-args}{\texttt{util.format()}} for
more information.

\subsubsection{\texorpdfstring{\texttt{console.table(tabularData{[},\ properties{]})}}{console.table(tabularData{[}, properties{]})}}\label{console.tabletabulardata-properties}

\begin{itemize}
\tightlist
\item
  \texttt{tabularData} \{any\}
\item
  \texttt{properties} \{string{[}{]}\} Alternate properties for
  constructing the table.
\end{itemize}

Try to construct a table with the columns of the properties of
\texttt{tabularData} (or use \texttt{properties}) and rows of
\texttt{tabularData} and log it. Falls back to just logging the argument
if it can't be parsed as tabular.

\begin{Shaded}
\begin{Highlighting}[]
\CommentTok{// These can\textquotesingle{}t be parsed as tabular data}
\BuiltInTok{console}\OperatorTok{.}\FunctionTok{table}\NormalTok{(}\BuiltInTok{Symbol}\NormalTok{())}\OperatorTok{;}
\CommentTok{// Symbol()}

\BuiltInTok{console}\OperatorTok{.}\FunctionTok{table}\NormalTok{(}\KeywordTok{undefined}\NormalTok{)}\OperatorTok{;}
\CommentTok{// undefined}

\BuiltInTok{console}\OperatorTok{.}\FunctionTok{table}\NormalTok{([\{ }\DataTypeTok{a}\OperatorTok{:} \DecValTok{1}\OperatorTok{,} \DataTypeTok{b}\OperatorTok{:} \StringTok{\textquotesingle{}Y\textquotesingle{}}\NormalTok{ \}}\OperatorTok{,}\NormalTok{ \{ }\DataTypeTok{a}\OperatorTok{:} \StringTok{\textquotesingle{}Z\textquotesingle{}}\OperatorTok{,} \DataTypeTok{b}\OperatorTok{:} \DecValTok{2}\NormalTok{ \}])}\OperatorTok{;}
\CommentTok{// ┌─────────┬─────┬─────┐}
\CommentTok{// │ (index) │ a   │ b   │}
\CommentTok{// ├─────────┼─────┼─────┤}
\CommentTok{// │ 0       │ 1   │ \textquotesingle{}Y\textquotesingle{} │}
\CommentTok{// │ 1       │ \textquotesingle{}Z\textquotesingle{} │ 2   │}
\CommentTok{// └─────────┴─────┴─────┘}

\BuiltInTok{console}\OperatorTok{.}\FunctionTok{table}\NormalTok{([\{ }\DataTypeTok{a}\OperatorTok{:} \DecValTok{1}\OperatorTok{,} \DataTypeTok{b}\OperatorTok{:} \StringTok{\textquotesingle{}Y\textquotesingle{}}\NormalTok{ \}}\OperatorTok{,}\NormalTok{ \{ }\DataTypeTok{a}\OperatorTok{:} \StringTok{\textquotesingle{}Z\textquotesingle{}}\OperatorTok{,} \DataTypeTok{b}\OperatorTok{:} \DecValTok{2}\NormalTok{ \}]}\OperatorTok{,}\NormalTok{ [}\StringTok{\textquotesingle{}a\textquotesingle{}}\NormalTok{])}\OperatorTok{;}
\CommentTok{// ┌─────────┬─────┐}
\CommentTok{// │ (index) │ a   │}
\CommentTok{// ├─────────┼─────┤}
\CommentTok{// │ 0       │ 1   │}
\CommentTok{// │ 1       │ \textquotesingle{}Z\textquotesingle{} │}
\CommentTok{// └─────────┴─────┘}
\end{Highlighting}
\end{Shaded}

\subsubsection{\texorpdfstring{\texttt{console.time({[}label{]})}}{console.time({[}label{]})}}\label{console.timelabel}

\begin{itemize}
\tightlist
\item
  \texttt{label} \{string\} \textbf{Default:}
  \texttt{\textquotesingle{}default\textquotesingle{}}
\end{itemize}

Starts a timer that can be used to compute the duration of an operation.
Timers are identified by a unique \texttt{label}. Use the same
\texttt{label} when calling
\hyperref[consoletimeendlabel]{\texttt{console.timeEnd()}} to stop the
timer and output the elapsed time in suitable time units to
\texttt{stdout}. For example, if the elapsed time is 3869ms,
\texttt{console.timeEnd()} displays ``3.869s''.

\subsubsection{\texorpdfstring{\texttt{console.timeEnd({[}label{]})}}{console.timeEnd({[}label{]})}}\label{console.timeendlabel}

\begin{itemize}
\tightlist
\item
  \texttt{label} \{string\} \textbf{Default:}
  \texttt{\textquotesingle{}default\textquotesingle{}}
\end{itemize}

Stops a timer that was previously started by calling
\hyperref[consoletimelabel]{\texttt{console.time()}} and prints the
result to \texttt{stdout}:

\begin{Shaded}
\begin{Highlighting}[]
\BuiltInTok{console}\OperatorTok{.}\FunctionTok{time}\NormalTok{(}\StringTok{\textquotesingle{}bunch{-}of{-}stuff\textquotesingle{}}\NormalTok{)}\OperatorTok{;}
\CommentTok{// Do a bunch of stuff.}
\BuiltInTok{console}\OperatorTok{.}\FunctionTok{timeEnd}\NormalTok{(}\StringTok{\textquotesingle{}bunch{-}of{-}stuff\textquotesingle{}}\NormalTok{)}\OperatorTok{;}
\CommentTok{// Prints: bunch{-}of{-}stuff: 225.438ms}
\end{Highlighting}
\end{Shaded}

\subsubsection{\texorpdfstring{\texttt{console.timeLog({[}label{]}{[},\ ...data{]})}}{console.timeLog({[}label{]}{[}, ...data{]})}}\label{console.timeloglabel-...data}

\begin{itemize}
\tightlist
\item
  \texttt{label} \{string\} \textbf{Default:}
  \texttt{\textquotesingle{}default\textquotesingle{}}
\item
  \texttt{...data} \{any\}
\end{itemize}

For a timer that was previously started by calling
\hyperref[consoletimelabel]{\texttt{console.time()}}, prints the elapsed
time and other \texttt{data} arguments to \texttt{stdout}:

\begin{Shaded}
\begin{Highlighting}[]
\BuiltInTok{console}\OperatorTok{.}\FunctionTok{time}\NormalTok{(}\StringTok{\textquotesingle{}process\textquotesingle{}}\NormalTok{)}\OperatorTok{;}
\KeywordTok{const}\NormalTok{ value }\OperatorTok{=} \FunctionTok{expensiveProcess1}\NormalTok{()}\OperatorTok{;} \CommentTok{// Returns 42}
\BuiltInTok{console}\OperatorTok{.}\FunctionTok{timeLog}\NormalTok{(}\StringTok{\textquotesingle{}process\textquotesingle{}}\OperatorTok{,}\NormalTok{ value)}\OperatorTok{;}
\CommentTok{// Prints "process: 365.227ms 42".}
\FunctionTok{doExpensiveProcess2}\NormalTok{(value)}\OperatorTok{;}
\BuiltInTok{console}\OperatorTok{.}\FunctionTok{timeEnd}\NormalTok{(}\StringTok{\textquotesingle{}process\textquotesingle{}}\NormalTok{)}\OperatorTok{;}
\end{Highlighting}
\end{Shaded}

\subsubsection{\texorpdfstring{\texttt{console.trace({[}message{]}{[},\ ...args{]})}}{console.trace({[}message{]}{[}, ...args{]})}}\label{console.tracemessage-...args}

\begin{itemize}
\tightlist
\item
  \texttt{message} \{any\}
\item
  \texttt{...args} \{any\}
\end{itemize}

Prints to \texttt{stderr} the string
\texttt{\textquotesingle{}Trace:\ \textquotesingle{}}, followed by the
\href{util.md\#utilformatformat-args}{\texttt{util.format()}} formatted
message and stack trace to the current position in the code.

\begin{Shaded}
\begin{Highlighting}[]
\BuiltInTok{console}\OperatorTok{.}\FunctionTok{trace}\NormalTok{(}\StringTok{\textquotesingle{}Show me\textquotesingle{}}\NormalTok{)}\OperatorTok{;}
\CommentTok{// Prints: (stack trace will vary based on where trace is called)}
\CommentTok{//  Trace: Show me}
\CommentTok{//    at repl:2:9}
\CommentTok{//    at REPLServer.defaultEval (repl.js:248:27)}
\CommentTok{//    at bound (domain.js:287:14)}
\CommentTok{//    at REPLServer.runBound [as eval] (domain.js:300:12)}
\CommentTok{//    at REPLServer.\textless{}anonymous\textgreater{} (repl.js:412:12)}
\CommentTok{//    at emitOne (events.js:82:20)}
\CommentTok{//    at REPLServer.emit (events.js:169:7)}
\CommentTok{//    at REPLServer.Interface.\_onLine (readline.js:210:10)}
\CommentTok{//    at REPLServer.Interface.\_line (readline.js:549:8)}
\CommentTok{//    at REPLServer.Interface.\_ttyWrite (readline.js:826:14)}
\end{Highlighting}
\end{Shaded}

\subsubsection{\texorpdfstring{\texttt{console.warn({[}data{]}{[},\ ...args{]})}}{console.warn({[}data{]}{[}, ...args{]})}}\label{console.warndata-...args}

\begin{itemize}
\tightlist
\item
  \texttt{data} \{any\}
\item
  \texttt{...args} \{any\}
\end{itemize}

The \texttt{console.warn()} function is an alias for
\hyperref[consoleerrordata-args]{\texttt{console.error()}}.

\subsection{Inspector only methods}\label{inspector-only-methods}

The following methods are exposed by the V8 engine in the general API
but do not display anything unless used in conjunction with the
\href{debugger.md}{inspector} (\texttt{-\/-inspect} flag).

\subsubsection{\texorpdfstring{\texttt{console.profile({[}label{]})}}{console.profile({[}label{]})}}\label{console.profilelabel}

\begin{itemize}
\tightlist
\item
  \texttt{label} \{string\}
\end{itemize}

This method does not display anything unless used in the inspector. The
\texttt{console.profile()} method starts a JavaScript CPU profile with
an optional label until
\hyperref[consoleprofileendlabel]{\texttt{console.profileEnd()}} is
called. The profile is then added to the \textbf{Profile} panel of the
inspector.

\begin{Shaded}
\begin{Highlighting}[]
\BuiltInTok{console}\OperatorTok{.}\FunctionTok{profile}\NormalTok{(}\StringTok{\textquotesingle{}MyLabel\textquotesingle{}}\NormalTok{)}\OperatorTok{;}
\CommentTok{// Some code}
\BuiltInTok{console}\OperatorTok{.}\FunctionTok{profileEnd}\NormalTok{(}\StringTok{\textquotesingle{}MyLabel\textquotesingle{}}\NormalTok{)}\OperatorTok{;}
\CommentTok{// Adds the profile \textquotesingle{}MyLabel\textquotesingle{} to the Profiles panel of the inspector.}
\end{Highlighting}
\end{Shaded}

\subsubsection{\texorpdfstring{\texttt{console.profileEnd({[}label{]})}}{console.profileEnd({[}label{]})}}\label{console.profileendlabel}

\begin{itemize}
\tightlist
\item
  \texttt{label} \{string\}
\end{itemize}

This method does not display anything unless used in the inspector.
Stops the current JavaScript CPU profiling session if one has been
started and prints the report to the \textbf{Profiles} panel of the
inspector. See
\hyperref[consoleprofilelabel]{\texttt{console.profile()}} for an
example.

If this method is called without a label, the most recently started
profile is stopped.

\subsubsection{\texorpdfstring{\texttt{console.timeStamp({[}label{]})}}{console.timeStamp({[}label{]})}}\label{console.timestamplabel}

\begin{itemize}
\tightlist
\item
  \texttt{label} \{string\}
\end{itemize}

This method does not display anything unless used in the inspector. The
\texttt{console.timeStamp()} method adds an event with the label
\texttt{\textquotesingle{}label\textquotesingle{}} to the
\textbf{Timeline} panel of the inspector.
