\section{About this documentation}\label{about-this-documentation}

Welcome to the official API reference documentation for Node.js!

Node.js is a JavaScript runtime built on the \href{https://v8.dev/}{V8
JavaScript engine}.

\subsection{Contributing}\label{contributing}

Report errors in this documentation in
\href{https://github.com/nodejs/node/issues/new}{the issue tracker}. See
\href{https://github.com/nodejs/node/blob/HEAD/CONTRIBUTING.md}{the
contributing guide} for directions on how to submit pull requests.

\subsection{Stability index}\label{stability-index}

Throughout the documentation are indications of a section's stability.
Some APIs are so proven and so relied upon that they are unlikely to
ever change at all. Others are brand new and experimental, or known to
be hazardous.

The stability indices are as follows:

\begin{quote}
Stability: 0 - Deprecated. The feature may emit warnings. Backward
compatibility is not guaranteed.
\end{quote}

\begin{quote}
Stability: 1 - Experimental. The feature is not subject to
\href{https://semver.org/}{semantic versioning} rules. Non-backward
compatible changes or removal may occur in any future release. Use of
the feature is not recommended in production environments.

Experimental features are subdivided into stages:

\begin{itemize}
\tightlist
\item
  1.0 - Early development. Experimental features at this stage are
  unfinished and subject to substantial change.
\item
  1.1 - Active development. Experimental features at this stage are
  nearing minimum viability.
\item
  1.2 - Release candidate. Experimental features at this stage are
  hopefully ready to become stable. No further breaking changes are
  anticipated but may still occur in response to user feedback. We
  encourage user testing and feedback so that we can know that this
  feature is ready to be marked as stable.
\end{itemize}
\end{quote}

\begin{quote}
Stability: 2 - Stable. Compatibility with the npm ecosystem is a high
priority.
\end{quote}

\begin{quote}
Stability: 3 - Legacy. Although this feature is unlikely to be removed
and is still covered by semantic versioning guarantees, it is no longer
actively maintained, and other alternatives are available.
\end{quote}

Features are marked as legacy rather than being deprecated if their use
does no harm, and they are widely relied upon within the npm ecosystem.
Bugs found in legacy features are unlikely to be fixed.

Use caution when making use of Experimental features, particularly when
authoring libraries. Users may not be aware that experimental features
are being used. Bugs or behavior changes may surprise users when
Experimental API modifications occur. To avoid surprises, use of an
Experimental feature may need a command-line flag. Experimental features
may also emit a \href{process.md\#event-warning}{warning}.

\subsection{Stability overview}\label{stability-overview}

\subsection{JSON output}\label{json-output}

Every \texttt{.html} document has a corresponding \texttt{.json}
document. This is for IDEs and other utilities that consume the
documentation.

\subsection{System calls and man
pages}\label{system-calls-and-man-pages}

Node.js functions which wrap a system call will document that. The docs
link to the corresponding man pages which describe how the system call
works.

Most Unix system calls have Windows analogues. Still, behavior
differences may be unavoidable.
