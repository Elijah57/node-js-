\section{UDP/datagram sockets}\label{udpdatagram-sockets}

\begin{quote}
Stability: 2 - Stable
\end{quote}

The \texttt{node:dgram} module provides an implementation of UDP
datagram sockets.

\begin{Shaded}
\begin{Highlighting}[]
\ImportTok{import}\NormalTok{ dgram }\ImportTok{from} \StringTok{\textquotesingle{}node:dgram\textquotesingle{}}\OperatorTok{;}

\KeywordTok{const}\NormalTok{ server }\OperatorTok{=}\NormalTok{ dgram}\OperatorTok{.}\FunctionTok{createSocket}\NormalTok{(}\StringTok{\textquotesingle{}udp4\textquotesingle{}}\NormalTok{)}\OperatorTok{;}

\NormalTok{server}\OperatorTok{.}\FunctionTok{on}\NormalTok{(}\StringTok{\textquotesingle{}error\textquotesingle{}}\OperatorTok{,}\NormalTok{ (err) }\KeywordTok{=\textgreater{}}\NormalTok{ \{}
  \BuiltInTok{console}\OperatorTok{.}\FunctionTok{error}\NormalTok{(}\VerbatimStringTok{\textasciigrave{}server error:}\SpecialCharTok{\textbackslash{}n$\{}\NormalTok{err}\OperatorTok{.}\AttributeTok{stack}\SpecialCharTok{\}}\VerbatimStringTok{\textasciigrave{}}\NormalTok{)}\OperatorTok{;}
\NormalTok{  server}\OperatorTok{.}\FunctionTok{close}\NormalTok{()}\OperatorTok{;}
\NormalTok{\})}\OperatorTok{;}

\NormalTok{server}\OperatorTok{.}\FunctionTok{on}\NormalTok{(}\StringTok{\textquotesingle{}message\textquotesingle{}}\OperatorTok{,}\NormalTok{ (msg}\OperatorTok{,}\NormalTok{ rinfo) }\KeywordTok{=\textgreater{}}\NormalTok{ \{}
  \BuiltInTok{console}\OperatorTok{.}\FunctionTok{log}\NormalTok{(}\VerbatimStringTok{\textasciigrave{}server got: }\SpecialCharTok{$\{}\NormalTok{msg}\SpecialCharTok{\}}\VerbatimStringTok{ from }\SpecialCharTok{$\{}\NormalTok{rinfo}\OperatorTok{.}\AttributeTok{address}\SpecialCharTok{\}}\VerbatimStringTok{:}\SpecialCharTok{$\{}\NormalTok{rinfo}\OperatorTok{.}\AttributeTok{port}\SpecialCharTok{\}}\VerbatimStringTok{\textasciigrave{}}\NormalTok{)}\OperatorTok{;}
\NormalTok{\})}\OperatorTok{;}

\NormalTok{server}\OperatorTok{.}\FunctionTok{on}\NormalTok{(}\StringTok{\textquotesingle{}listening\textquotesingle{}}\OperatorTok{,}\NormalTok{ () }\KeywordTok{=\textgreater{}}\NormalTok{ \{}
  \KeywordTok{const}\NormalTok{ address }\OperatorTok{=}\NormalTok{ server}\OperatorTok{.}\FunctionTok{address}\NormalTok{()}\OperatorTok{;}
  \BuiltInTok{console}\OperatorTok{.}\FunctionTok{log}\NormalTok{(}\VerbatimStringTok{\textasciigrave{}server listening }\SpecialCharTok{$\{}\NormalTok{address}\OperatorTok{.}\AttributeTok{address}\SpecialCharTok{\}}\VerbatimStringTok{:}\SpecialCharTok{$\{}\NormalTok{address}\OperatorTok{.}\AttributeTok{port}\SpecialCharTok{\}}\VerbatimStringTok{\textasciigrave{}}\NormalTok{)}\OperatorTok{;}
\NormalTok{\})}\OperatorTok{;}

\NormalTok{server}\OperatorTok{.}\FunctionTok{bind}\NormalTok{(}\DecValTok{41234}\NormalTok{)}\OperatorTok{;}
\CommentTok{// Prints: server listening 0.0.0.0:41234}
\end{Highlighting}
\end{Shaded}

\begin{Shaded}
\begin{Highlighting}[]
\KeywordTok{const}\NormalTok{ dgram }\OperatorTok{=} \PreprocessorTok{require}\NormalTok{(}\StringTok{\textquotesingle{}node:dgram\textquotesingle{}}\NormalTok{)}\OperatorTok{;}
\KeywordTok{const}\NormalTok{ server }\OperatorTok{=}\NormalTok{ dgram}\OperatorTok{.}\FunctionTok{createSocket}\NormalTok{(}\StringTok{\textquotesingle{}udp4\textquotesingle{}}\NormalTok{)}\OperatorTok{;}

\NormalTok{server}\OperatorTok{.}\FunctionTok{on}\NormalTok{(}\StringTok{\textquotesingle{}error\textquotesingle{}}\OperatorTok{,}\NormalTok{ (err) }\KeywordTok{=\textgreater{}}\NormalTok{ \{}
  \BuiltInTok{console}\OperatorTok{.}\FunctionTok{error}\NormalTok{(}\VerbatimStringTok{\textasciigrave{}server error:}\SpecialCharTok{\textbackslash{}n$\{}\NormalTok{err}\OperatorTok{.}\AttributeTok{stack}\SpecialCharTok{\}}\VerbatimStringTok{\textasciigrave{}}\NormalTok{)}\OperatorTok{;}
\NormalTok{  server}\OperatorTok{.}\FunctionTok{close}\NormalTok{()}\OperatorTok{;}
\NormalTok{\})}\OperatorTok{;}

\NormalTok{server}\OperatorTok{.}\FunctionTok{on}\NormalTok{(}\StringTok{\textquotesingle{}message\textquotesingle{}}\OperatorTok{,}\NormalTok{ (msg}\OperatorTok{,}\NormalTok{ rinfo) }\KeywordTok{=\textgreater{}}\NormalTok{ \{}
  \BuiltInTok{console}\OperatorTok{.}\FunctionTok{log}\NormalTok{(}\VerbatimStringTok{\textasciigrave{}server got: }\SpecialCharTok{$\{}\NormalTok{msg}\SpecialCharTok{\}}\VerbatimStringTok{ from }\SpecialCharTok{$\{}\NormalTok{rinfo}\OperatorTok{.}\AttributeTok{address}\SpecialCharTok{\}}\VerbatimStringTok{:}\SpecialCharTok{$\{}\NormalTok{rinfo}\OperatorTok{.}\AttributeTok{port}\SpecialCharTok{\}}\VerbatimStringTok{\textasciigrave{}}\NormalTok{)}\OperatorTok{;}
\NormalTok{\})}\OperatorTok{;}

\NormalTok{server}\OperatorTok{.}\FunctionTok{on}\NormalTok{(}\StringTok{\textquotesingle{}listening\textquotesingle{}}\OperatorTok{,}\NormalTok{ () }\KeywordTok{=\textgreater{}}\NormalTok{ \{}
  \KeywordTok{const}\NormalTok{ address }\OperatorTok{=}\NormalTok{ server}\OperatorTok{.}\FunctionTok{address}\NormalTok{()}\OperatorTok{;}
  \BuiltInTok{console}\OperatorTok{.}\FunctionTok{log}\NormalTok{(}\VerbatimStringTok{\textasciigrave{}server listening }\SpecialCharTok{$\{}\NormalTok{address}\OperatorTok{.}\AttributeTok{address}\SpecialCharTok{\}}\VerbatimStringTok{:}\SpecialCharTok{$\{}\NormalTok{address}\OperatorTok{.}\AttributeTok{port}\SpecialCharTok{\}}\VerbatimStringTok{\textasciigrave{}}\NormalTok{)}\OperatorTok{;}
\NormalTok{\})}\OperatorTok{;}

\NormalTok{server}\OperatorTok{.}\FunctionTok{bind}\NormalTok{(}\DecValTok{41234}\NormalTok{)}\OperatorTok{;}
\CommentTok{// Prints: server listening 0.0.0.0:41234}
\end{Highlighting}
\end{Shaded}

\subsection{\texorpdfstring{Class:
\texttt{dgram.Socket}}{Class: dgram.Socket}}\label{class-dgram.socket}

\begin{itemize}
\tightlist
\item
  Extends: \{EventEmitter\}
\end{itemize}

Encapsulates the datagram functionality.

New instances of \texttt{dgram.Socket} are created using
\hyperref[dgramcreatesocketoptions-callback]{\texttt{dgram.createSocket()}}.
The \texttt{new} keyword is not to be used to create
\texttt{dgram.Socket} instances.

\subsubsection{\texorpdfstring{Event:
\texttt{\textquotesingle{}close\textquotesingle{}}}{Event: \textquotesingle close\textquotesingle{}}}\label{event-close}

The \texttt{\textquotesingle{}close\textquotesingle{}} event is emitted
after a socket is closed with
\hyperref[socketclosecallback]{\texttt{close()}}. Once triggered, no new
\texttt{\textquotesingle{}message\textquotesingle{}} events will be
emitted on this socket.

\subsubsection{\texorpdfstring{Event:
\texttt{\textquotesingle{}connect\textquotesingle{}}}{Event: \textquotesingle connect\textquotesingle{}}}\label{event-connect}

The \texttt{\textquotesingle{}connect\textquotesingle{}} event is
emitted after a socket is associated to a remote address as a result of
a successful
\hyperref[socketconnectport-address-callback]{\texttt{connect()}} call.

\subsubsection{\texorpdfstring{Event:
\texttt{\textquotesingle{}error\textquotesingle{}}}{Event: \textquotesingle error\textquotesingle{}}}\label{event-error}

\begin{itemize}
\tightlist
\item
  \texttt{exception} \{Error\}
\end{itemize}

The \texttt{\textquotesingle{}error\textquotesingle{}} event is emitted
whenever any error occurs. The event handler function is passed a single
\texttt{Error} object.

\subsubsection{\texorpdfstring{Event:
\texttt{\textquotesingle{}listening\textquotesingle{}}}{Event: \textquotesingle listening\textquotesingle{}}}\label{event-listening}

The \texttt{\textquotesingle{}listening\textquotesingle{}} event is
emitted once the \texttt{dgram.Socket} is addressable and can receive
data. This happens either explicitly with \texttt{socket.bind()} or
implicitly the first time data is sent using \texttt{socket.send()}.
Until the \texttt{dgram.Socket} is listening, the underlying system
resources do not exist and calls such as \texttt{socket.address()} and
\texttt{socket.setTTL()} will fail.

\subsubsection{\texorpdfstring{Event:
\texttt{\textquotesingle{}message\textquotesingle{}}}{Event: \textquotesingle message\textquotesingle{}}}\label{event-message}

The \texttt{\textquotesingle{}message\textquotesingle{}} event is
emitted when a new datagram is available on a socket. The event handler
function is passed two arguments: \texttt{msg} and \texttt{rinfo}.

\begin{itemize}
\tightlist
\item
  \texttt{msg} \{Buffer\} The message.
\item
  \texttt{rinfo} \{Object\} Remote address information.

  \begin{itemize}
  \tightlist
  \item
    \texttt{address} \{string\} The sender address.
  \item
    \texttt{family} \{string\} The address family
    (\texttt{\textquotesingle{}IPv4\textquotesingle{}} or
    \texttt{\textquotesingle{}IPv6\textquotesingle{}}).
  \item
    \texttt{port} \{number\} The sender port.
  \item
    \texttt{size} \{number\} The message size.
  \end{itemize}
\end{itemize}

If the source address of the incoming packet is an IPv6 link-local
address, the interface name is added to the \texttt{address}. For
example, a packet received on the \texttt{en0} interface might have the
address field set to
\texttt{\textquotesingle{}fe80::2618:1234:ab11:3b9c\%en0\textquotesingle{}},
where \texttt{\textquotesingle{}\%en0\textquotesingle{}} is the
interface name as a zone ID suffix.

\subsubsection{\texorpdfstring{\texttt{socket.addMembership(multicastAddress{[},\ multicastInterface{]})}}{socket.addMembership(multicastAddress{[}, multicastInterface{]})}}\label{socket.addmembershipmulticastaddress-multicastinterface}

\begin{itemize}
\tightlist
\item
  \texttt{multicastAddress} \{string\}
\item
  \texttt{multicastInterface} \{string\}
\end{itemize}

Tells the kernel to join a multicast group at the given
\texttt{multicastAddress} and \texttt{multicastInterface} using the
\texttt{IP\_ADD\_MEMBERSHIP} socket option. If the
\texttt{multicastInterface} argument is not specified, the operating
system will choose one interface and will add membership to it. To add
membership to every available interface, call \texttt{addMembership}
multiple times, once per interface.

When called on an unbound socket, this method will implicitly bind to a
random port, listening on all interfaces.

When sharing a UDP socket across multiple \texttt{cluster} workers, the
\texttt{socket.addMembership()} function must be called only once or an
\texttt{EADDRINUSE} error will occur:

\begin{Shaded}
\begin{Highlighting}[]
\ImportTok{import}\NormalTok{ cluster }\ImportTok{from} \StringTok{\textquotesingle{}node:cluster\textquotesingle{}}\OperatorTok{;}
\ImportTok{import}\NormalTok{ dgram }\ImportTok{from} \StringTok{\textquotesingle{}node:dgram\textquotesingle{}}\OperatorTok{;}

\ControlFlowTok{if}\NormalTok{ (cluster}\OperatorTok{.}\AttributeTok{isPrimary}\NormalTok{) \{}
\NormalTok{  cluster}\OperatorTok{.}\FunctionTok{fork}\NormalTok{()}\OperatorTok{;} \CommentTok{// Works ok.}
\NormalTok{  cluster}\OperatorTok{.}\FunctionTok{fork}\NormalTok{()}\OperatorTok{;} \CommentTok{// Fails with EADDRINUSE.}
\NormalTok{\} }\ControlFlowTok{else}\NormalTok{ \{}
  \KeywordTok{const}\NormalTok{ s }\OperatorTok{=}\NormalTok{ dgram}\OperatorTok{.}\FunctionTok{createSocket}\NormalTok{(}\StringTok{\textquotesingle{}udp4\textquotesingle{}}\NormalTok{)}\OperatorTok{;}
\NormalTok{  s}\OperatorTok{.}\FunctionTok{bind}\NormalTok{(}\DecValTok{1234}\OperatorTok{,}\NormalTok{ () }\KeywordTok{=\textgreater{}}\NormalTok{ \{}
\NormalTok{    s}\OperatorTok{.}\FunctionTok{addMembership}\NormalTok{(}\StringTok{\textquotesingle{}224.0.0.114\textquotesingle{}}\NormalTok{)}\OperatorTok{;}
\NormalTok{  \})}\OperatorTok{;}
\NormalTok{\}}
\end{Highlighting}
\end{Shaded}

\begin{Shaded}
\begin{Highlighting}[]
\KeywordTok{const}\NormalTok{ cluster }\OperatorTok{=} \PreprocessorTok{require}\NormalTok{(}\StringTok{\textquotesingle{}node:cluster\textquotesingle{}}\NormalTok{)}\OperatorTok{;}
\KeywordTok{const}\NormalTok{ dgram }\OperatorTok{=} \PreprocessorTok{require}\NormalTok{(}\StringTok{\textquotesingle{}node:dgram\textquotesingle{}}\NormalTok{)}\OperatorTok{;}

\ControlFlowTok{if}\NormalTok{ (cluster}\OperatorTok{.}\AttributeTok{isPrimary}\NormalTok{) \{}
\NormalTok{  cluster}\OperatorTok{.}\FunctionTok{fork}\NormalTok{()}\OperatorTok{;} \CommentTok{// Works ok.}
\NormalTok{  cluster}\OperatorTok{.}\FunctionTok{fork}\NormalTok{()}\OperatorTok{;} \CommentTok{// Fails with EADDRINUSE.}
\NormalTok{\} }\ControlFlowTok{else}\NormalTok{ \{}
  \KeywordTok{const}\NormalTok{ s }\OperatorTok{=}\NormalTok{ dgram}\OperatorTok{.}\FunctionTok{createSocket}\NormalTok{(}\StringTok{\textquotesingle{}udp4\textquotesingle{}}\NormalTok{)}\OperatorTok{;}
\NormalTok{  s}\OperatorTok{.}\FunctionTok{bind}\NormalTok{(}\DecValTok{1234}\OperatorTok{,}\NormalTok{ () }\KeywordTok{=\textgreater{}}\NormalTok{ \{}
\NormalTok{    s}\OperatorTok{.}\FunctionTok{addMembership}\NormalTok{(}\StringTok{\textquotesingle{}224.0.0.114\textquotesingle{}}\NormalTok{)}\OperatorTok{;}
\NormalTok{  \})}\OperatorTok{;}
\NormalTok{\}}
\end{Highlighting}
\end{Shaded}

\subsubsection{\texorpdfstring{\texttt{socket.addSourceSpecificMembership(sourceAddress,\ groupAddress{[},\ multicastInterface{]})}}{socket.addSourceSpecificMembership(sourceAddress, groupAddress{[}, multicastInterface{]})}}\label{socket.addsourcespecificmembershipsourceaddress-groupaddress-multicastinterface}

\begin{itemize}
\tightlist
\item
  \texttt{sourceAddress} \{string\}
\item
  \texttt{groupAddress} \{string\}
\item
  \texttt{multicastInterface} \{string\}
\end{itemize}

Tells the kernel to join a source-specific multicast channel at the
given \texttt{sourceAddress} and \texttt{groupAddress}, using the
\texttt{multicastInterface} with the
\texttt{IP\_ADD\_SOURCE\_MEMBERSHIP} socket option. If the
\texttt{multicastInterface} argument is not specified, the operating
system will choose one interface and will add membership to it. To add
membership to every available interface, call
\texttt{socket.addSourceSpecificMembership()} multiple times, once per
interface.

When called on an unbound socket, this method will implicitly bind to a
random port, listening on all interfaces.

\subsubsection{\texorpdfstring{\texttt{socket.address()}}{socket.address()}}\label{socket.address}

\begin{itemize}
\tightlist
\item
  Returns: \{Object\}
\end{itemize}

Returns an object containing the address information for a socket. For
UDP sockets, this object will contain \texttt{address}, \texttt{family},
and \texttt{port} properties.

This method throws \texttt{EBADF} if called on an unbound socket.

\subsubsection{\texorpdfstring{\texttt{socket.bind({[}port{]}{[},\ address{]}{[},\ callback{]})}}{socket.bind({[}port{]}{[}, address{]}{[}, callback{]})}}\label{socket.bindport-address-callback}

\begin{itemize}
\tightlist
\item
  \texttt{port} \{integer\}
\item
  \texttt{address} \{string\}
\item
  \texttt{callback} \{Function\} with no parameters. Called when binding
  is complete.
\end{itemize}

For UDP sockets, causes the \texttt{dgram.Socket} to listen for datagram
messages on a named \texttt{port} and optional \texttt{address}. If
\texttt{port} is not specified or is \texttt{0}, the operating system
will attempt to bind to a random port. If \texttt{address} is not
specified, the operating system will attempt to listen on all addresses.
Once binding is complete, a
\texttt{\textquotesingle{}listening\textquotesingle{}} event is emitted
and the optional \texttt{callback} function is called.

Specifying both a \texttt{\textquotesingle{}listening\textquotesingle{}}
event listener and passing a \texttt{callback} to the
\texttt{socket.bind()} method is not harmful but not very useful.

A bound datagram socket keeps the Node.js process running to receive
datagram messages.

If binding fails, an \texttt{\textquotesingle{}error\textquotesingle{}}
event is generated. In rare case (e.g. attempting to bind with a closed
socket), an \href{errors.md\#class-error}{\texttt{Error}} may be thrown.

Example of a UDP server listening on port 41234:

\begin{Shaded}
\begin{Highlighting}[]
\ImportTok{import}\NormalTok{ dgram }\ImportTok{from} \StringTok{\textquotesingle{}node:dgram\textquotesingle{}}\OperatorTok{;}

\KeywordTok{const}\NormalTok{ server }\OperatorTok{=}\NormalTok{ dgram}\OperatorTok{.}\FunctionTok{createSocket}\NormalTok{(}\StringTok{\textquotesingle{}udp4\textquotesingle{}}\NormalTok{)}\OperatorTok{;}

\NormalTok{server}\OperatorTok{.}\FunctionTok{on}\NormalTok{(}\StringTok{\textquotesingle{}error\textquotesingle{}}\OperatorTok{,}\NormalTok{ (err) }\KeywordTok{=\textgreater{}}\NormalTok{ \{}
  \BuiltInTok{console}\OperatorTok{.}\FunctionTok{error}\NormalTok{(}\VerbatimStringTok{\textasciigrave{}server error:}\SpecialCharTok{\textbackslash{}n$\{}\NormalTok{err}\OperatorTok{.}\AttributeTok{stack}\SpecialCharTok{\}}\VerbatimStringTok{\textasciigrave{}}\NormalTok{)}\OperatorTok{;}
\NormalTok{  server}\OperatorTok{.}\FunctionTok{close}\NormalTok{()}\OperatorTok{;}
\NormalTok{\})}\OperatorTok{;}

\NormalTok{server}\OperatorTok{.}\FunctionTok{on}\NormalTok{(}\StringTok{\textquotesingle{}message\textquotesingle{}}\OperatorTok{,}\NormalTok{ (msg}\OperatorTok{,}\NormalTok{ rinfo) }\KeywordTok{=\textgreater{}}\NormalTok{ \{}
  \BuiltInTok{console}\OperatorTok{.}\FunctionTok{log}\NormalTok{(}\VerbatimStringTok{\textasciigrave{}server got: }\SpecialCharTok{$\{}\NormalTok{msg}\SpecialCharTok{\}}\VerbatimStringTok{ from }\SpecialCharTok{$\{}\NormalTok{rinfo}\OperatorTok{.}\AttributeTok{address}\SpecialCharTok{\}}\VerbatimStringTok{:}\SpecialCharTok{$\{}\NormalTok{rinfo}\OperatorTok{.}\AttributeTok{port}\SpecialCharTok{\}}\VerbatimStringTok{\textasciigrave{}}\NormalTok{)}\OperatorTok{;}
\NormalTok{\})}\OperatorTok{;}

\NormalTok{server}\OperatorTok{.}\FunctionTok{on}\NormalTok{(}\StringTok{\textquotesingle{}listening\textquotesingle{}}\OperatorTok{,}\NormalTok{ () }\KeywordTok{=\textgreater{}}\NormalTok{ \{}
  \KeywordTok{const}\NormalTok{ address }\OperatorTok{=}\NormalTok{ server}\OperatorTok{.}\FunctionTok{address}\NormalTok{()}\OperatorTok{;}
  \BuiltInTok{console}\OperatorTok{.}\FunctionTok{log}\NormalTok{(}\VerbatimStringTok{\textasciigrave{}server listening }\SpecialCharTok{$\{}\NormalTok{address}\OperatorTok{.}\AttributeTok{address}\SpecialCharTok{\}}\VerbatimStringTok{:}\SpecialCharTok{$\{}\NormalTok{address}\OperatorTok{.}\AttributeTok{port}\SpecialCharTok{\}}\VerbatimStringTok{\textasciigrave{}}\NormalTok{)}\OperatorTok{;}
\NormalTok{\})}\OperatorTok{;}

\NormalTok{server}\OperatorTok{.}\FunctionTok{bind}\NormalTok{(}\DecValTok{41234}\NormalTok{)}\OperatorTok{;}
\CommentTok{// Prints: server listening 0.0.0.0:41234}
\end{Highlighting}
\end{Shaded}

\begin{Shaded}
\begin{Highlighting}[]
\KeywordTok{const}\NormalTok{ dgram }\OperatorTok{=} \PreprocessorTok{require}\NormalTok{(}\StringTok{\textquotesingle{}node:dgram\textquotesingle{}}\NormalTok{)}\OperatorTok{;}
\KeywordTok{const}\NormalTok{ server }\OperatorTok{=}\NormalTok{ dgram}\OperatorTok{.}\FunctionTok{createSocket}\NormalTok{(}\StringTok{\textquotesingle{}udp4\textquotesingle{}}\NormalTok{)}\OperatorTok{;}

\NormalTok{server}\OperatorTok{.}\FunctionTok{on}\NormalTok{(}\StringTok{\textquotesingle{}error\textquotesingle{}}\OperatorTok{,}\NormalTok{ (err) }\KeywordTok{=\textgreater{}}\NormalTok{ \{}
  \BuiltInTok{console}\OperatorTok{.}\FunctionTok{error}\NormalTok{(}\VerbatimStringTok{\textasciigrave{}server error:}\SpecialCharTok{\textbackslash{}n$\{}\NormalTok{err}\OperatorTok{.}\AttributeTok{stack}\SpecialCharTok{\}}\VerbatimStringTok{\textasciigrave{}}\NormalTok{)}\OperatorTok{;}
\NormalTok{  server}\OperatorTok{.}\FunctionTok{close}\NormalTok{()}\OperatorTok{;}
\NormalTok{\})}\OperatorTok{;}

\NormalTok{server}\OperatorTok{.}\FunctionTok{on}\NormalTok{(}\StringTok{\textquotesingle{}message\textquotesingle{}}\OperatorTok{,}\NormalTok{ (msg}\OperatorTok{,}\NormalTok{ rinfo) }\KeywordTok{=\textgreater{}}\NormalTok{ \{}
  \BuiltInTok{console}\OperatorTok{.}\FunctionTok{log}\NormalTok{(}\VerbatimStringTok{\textasciigrave{}server got: }\SpecialCharTok{$\{}\NormalTok{msg}\SpecialCharTok{\}}\VerbatimStringTok{ from }\SpecialCharTok{$\{}\NormalTok{rinfo}\OperatorTok{.}\AttributeTok{address}\SpecialCharTok{\}}\VerbatimStringTok{:}\SpecialCharTok{$\{}\NormalTok{rinfo}\OperatorTok{.}\AttributeTok{port}\SpecialCharTok{\}}\VerbatimStringTok{\textasciigrave{}}\NormalTok{)}\OperatorTok{;}
\NormalTok{\})}\OperatorTok{;}

\NormalTok{server}\OperatorTok{.}\FunctionTok{on}\NormalTok{(}\StringTok{\textquotesingle{}listening\textquotesingle{}}\OperatorTok{,}\NormalTok{ () }\KeywordTok{=\textgreater{}}\NormalTok{ \{}
  \KeywordTok{const}\NormalTok{ address }\OperatorTok{=}\NormalTok{ server}\OperatorTok{.}\FunctionTok{address}\NormalTok{()}\OperatorTok{;}
  \BuiltInTok{console}\OperatorTok{.}\FunctionTok{log}\NormalTok{(}\VerbatimStringTok{\textasciigrave{}server listening }\SpecialCharTok{$\{}\NormalTok{address}\OperatorTok{.}\AttributeTok{address}\SpecialCharTok{\}}\VerbatimStringTok{:}\SpecialCharTok{$\{}\NormalTok{address}\OperatorTok{.}\AttributeTok{port}\SpecialCharTok{\}}\VerbatimStringTok{\textasciigrave{}}\NormalTok{)}\OperatorTok{;}
\NormalTok{\})}\OperatorTok{;}

\NormalTok{server}\OperatorTok{.}\FunctionTok{bind}\NormalTok{(}\DecValTok{41234}\NormalTok{)}\OperatorTok{;}
\CommentTok{// Prints: server listening 0.0.0.0:41234}
\end{Highlighting}
\end{Shaded}

\subsubsection{\texorpdfstring{\texttt{socket.bind(options{[},\ callback{]})}}{socket.bind(options{[}, callback{]})}}\label{socket.bindoptions-callback}

\begin{itemize}
\tightlist
\item
  \texttt{options} \{Object\} Required. Supports the following
  properties:

  \begin{itemize}
  \tightlist
  \item
    \texttt{port} \{integer\}
  \item
    \texttt{address} \{string\}
  \item
    \texttt{exclusive} \{boolean\}
  \item
    \texttt{fd} \{integer\}
  \end{itemize}
\item
  \texttt{callback} \{Function\}
\end{itemize}

For UDP sockets, causes the \texttt{dgram.Socket} to listen for datagram
messages on a named \texttt{port} and optional \texttt{address} that are
passed as properties of an \texttt{options} object passed as the first
argument. If \texttt{port} is not specified or is \texttt{0}, the
operating system will attempt to bind to a random port. If
\texttt{address} is not specified, the operating system will attempt to
listen on all addresses. Once binding is complete, a
\texttt{\textquotesingle{}listening\textquotesingle{}} event is emitted
and the optional \texttt{callback} function is called.

The \texttt{options} object may contain a \texttt{fd} property. When a
\texttt{fd} greater than \texttt{0} is set, it will wrap around an
existing socket with the given file descriptor. In this case, the
properties of \texttt{port} and \texttt{address} will be ignored.

Specifying both a \texttt{\textquotesingle{}listening\textquotesingle{}}
event listener and passing a \texttt{callback} to the
\texttt{socket.bind()} method is not harmful but not very useful.

The \texttt{options} object may contain an additional \texttt{exclusive}
property that is used when using \texttt{dgram.Socket} objects with the
\href{cluster.md}{\texttt{cluster}} module. When \texttt{exclusive} is
set to \texttt{false} (the default), cluster workers will use the same
underlying socket handle allowing connection handling duties to be
shared. When \texttt{exclusive} is \texttt{true}, however, the handle is
not shared and attempted port sharing results in an error.

A bound datagram socket keeps the Node.js process running to receive
datagram messages.

If binding fails, an \texttt{\textquotesingle{}error\textquotesingle{}}
event is generated. In rare case (e.g. attempting to bind with a closed
socket), an \href{errors.md\#class-error}{\texttt{Error}} may be thrown.

An example socket listening on an exclusive port is shown below.

\begin{Shaded}
\begin{Highlighting}[]
\NormalTok{socket}\OperatorTok{.}\FunctionTok{bind}\NormalTok{(\{}
  \DataTypeTok{address}\OperatorTok{:} \StringTok{\textquotesingle{}localhost\textquotesingle{}}\OperatorTok{,}
  \DataTypeTok{port}\OperatorTok{:} \DecValTok{8000}\OperatorTok{,}
  \DataTypeTok{exclusive}\OperatorTok{:} \KeywordTok{true}\OperatorTok{,}
\NormalTok{\})}\OperatorTok{;}
\end{Highlighting}
\end{Shaded}

\subsubsection{\texorpdfstring{\texttt{socket.close({[}callback{]})}}{socket.close({[}callback{]})}}\label{socket.closecallback}

\begin{itemize}
\tightlist
\item
  \texttt{callback} \{Function\} Called when the socket has been closed.
\end{itemize}

Close the underlying socket and stop listening for data on it. If a
callback is provided, it is added as a listener for the
\hyperref[event-close]{\texttt{\textquotesingle{}close\textquotesingle{}}}
event.

\subsubsection{\texorpdfstring{\texttt{socket{[}Symbol.asyncDispose{]}()}}{socket{[}Symbol.asyncDispose{]}()}}\label{socketsymbol.asyncdispose}

\begin{quote}
Stability: 1 - Experimental
\end{quote}

Calls \hyperref[socketclosecallback]{\texttt{socket.close()}} and
returns a promise that fulfills when the socket has closed.

\subsubsection{\texorpdfstring{\texttt{socket.connect(port{[},\ address{]}{[},\ callback{]})}}{socket.connect(port{[}, address{]}{[}, callback{]})}}\label{socket.connectport-address-callback}

\begin{itemize}
\tightlist
\item
  \texttt{port} \{integer\}
\item
  \texttt{address} \{string\}
\item
  \texttt{callback} \{Function\} Called when the connection is completed
  or on error.
\end{itemize}

Associates the \texttt{dgram.Socket} to a remote address and port. Every
message sent by this handle is automatically sent to that destination.
Also, the socket will only receive messages from that remote peer.
Trying to call \texttt{connect()} on an already connected socket will
result in an
\href{errors.md\#err_socket_dgram_is_connected}{\texttt{ERR\_SOCKET\_DGRAM\_IS\_CONNECTED}}
exception. If \texttt{address} is not provided,
\texttt{\textquotesingle{}127.0.0.1\textquotesingle{}} (for
\texttt{udp4} sockets) or
\texttt{\textquotesingle{}::1\textquotesingle{}} (for \texttt{udp6}
sockets) will be used by default. Once the connection is complete, a
\texttt{\textquotesingle{}connect\textquotesingle{}} event is emitted
and the optional \texttt{callback} function is called. In case of
failure, the \texttt{callback} is called or, failing this, an
\texttt{\textquotesingle{}error\textquotesingle{}} event is emitted.

\subsubsection{\texorpdfstring{\texttt{socket.disconnect()}}{socket.disconnect()}}\label{socket.disconnect}

A synchronous function that disassociates a connected
\texttt{dgram.Socket} from its remote address. Trying to call
\texttt{disconnect()} on an unbound or already disconnected socket will
result in an
\href{errors.md\#err_socket_dgram_not_connected}{\texttt{ERR\_SOCKET\_DGRAM\_NOT\_CONNECTED}}
exception.

\subsubsection{\texorpdfstring{\texttt{socket.dropMembership(multicastAddress{[},\ multicastInterface{]})}}{socket.dropMembership(multicastAddress{[}, multicastInterface{]})}}\label{socket.dropmembershipmulticastaddress-multicastinterface}

\begin{itemize}
\tightlist
\item
  \texttt{multicastAddress} \{string\}
\item
  \texttt{multicastInterface} \{string\}
\end{itemize}

Instructs the kernel to leave a multicast group at
\texttt{multicastAddress} using the \texttt{IP\_DROP\_MEMBERSHIP} socket
option. This method is automatically called by the kernel when the
socket is closed or the process terminates, so most apps will never have
reason to call this.

If \texttt{multicastInterface} is not specified, the operating system
will attempt to drop membership on all valid interfaces.

\subsubsection{\texorpdfstring{\texttt{socket.dropSourceSpecificMembership(sourceAddress,\ groupAddress{[},\ multicastInterface{]})}}{socket.dropSourceSpecificMembership(sourceAddress, groupAddress{[}, multicastInterface{]})}}\label{socket.dropsourcespecificmembershipsourceaddress-groupaddress-multicastinterface}

\begin{itemize}
\tightlist
\item
  \texttt{sourceAddress} \{string\}
\item
  \texttt{groupAddress} \{string\}
\item
  \texttt{multicastInterface} \{string\}
\end{itemize}

Instructs the kernel to leave a source-specific multicast channel at the
given \texttt{sourceAddress} and \texttt{groupAddress} using the
\texttt{IP\_DROP\_SOURCE\_MEMBERSHIP} socket option. This method is
automatically called by the kernel when the socket is closed or the
process terminates, so most apps will never have reason to call this.

If \texttt{multicastInterface} is not specified, the operating system
will attempt to drop membership on all valid interfaces.

\subsubsection{\texorpdfstring{\texttt{socket.getRecvBufferSize()}}{socket.getRecvBufferSize()}}\label{socket.getrecvbuffersize}

\begin{itemize}
\tightlist
\item
  Returns: \{number\} the \texttt{SO\_RCVBUF} socket receive buffer size
  in bytes.
\end{itemize}

This method throws
\href{errors.md\#err_socket_buffer_size}{\texttt{ERR\_SOCKET\_BUFFER\_SIZE}}
if called on an unbound socket.

\subsubsection{\texorpdfstring{\texttt{socket.getSendBufferSize()}}{socket.getSendBufferSize()}}\label{socket.getsendbuffersize}

\begin{itemize}
\tightlist
\item
  Returns: \{number\} the \texttt{SO\_SNDBUF} socket send buffer size in
  bytes.
\end{itemize}

This method throws
\href{errors.md\#err_socket_buffer_size}{\texttt{ERR\_SOCKET\_BUFFER\_SIZE}}
if called on an unbound socket.

\subsubsection{\texorpdfstring{\texttt{socket.getSendQueueSize()}}{socket.getSendQueueSize()}}\label{socket.getsendqueuesize}

\begin{itemize}
\tightlist
\item
  Returns: \{number\} Number of bytes queued for sending.
\end{itemize}

\subsubsection{\texorpdfstring{\texttt{socket.getSendQueueCount()}}{socket.getSendQueueCount()}}\label{socket.getsendqueuecount}

\begin{itemize}
\tightlist
\item
  Returns: \{number\} Number of send requests currently in the queue
  awaiting to be processed.
\end{itemize}

\subsubsection{\texorpdfstring{\texttt{socket.ref()}}{socket.ref()}}\label{socket.ref}

\begin{itemize}
\tightlist
\item
  Returns: \{dgram.Socket\}
\end{itemize}

By default, binding a socket will cause it to block the Node.js process
from exiting as long as the socket is open. The \texttt{socket.unref()}
method can be used to exclude the socket from the reference counting
that keeps the Node.js process active. The \texttt{socket.ref()} method
adds the socket back to the reference counting and restores the default
behavior.

Calling \texttt{socket.ref()} multiples times will have no additional
effect.

The \texttt{socket.ref()} method returns a reference to the socket so
calls can be chained.

\subsubsection{\texorpdfstring{\texttt{socket.remoteAddress()}}{socket.remoteAddress()}}\label{socket.remoteaddress}

\begin{itemize}
\tightlist
\item
  Returns: \{Object\}
\end{itemize}

Returns an object containing the \texttt{address}, \texttt{family}, and
\texttt{port} of the remote endpoint. This method throws an
\href{errors.md\#err_socket_dgram_not_connected}{\texttt{ERR\_SOCKET\_DGRAM\_NOT\_CONNECTED}}
exception if the socket is not connected.

\subsubsection{\texorpdfstring{\texttt{socket.send(msg{[},\ offset,\ length{]}{[},\ port{]}{[},\ address{]}{[},\ callback{]})}}{socket.send(msg{[}, offset, length{]}{[}, port{]}{[}, address{]}{[}, callback{]})}}\label{socket.sendmsg-offset-length-port-address-callback}

\begin{itemize}
\tightlist
\item
  \texttt{msg}
  \{Buffer\textbar TypedArray\textbar DataView\textbar string\textbar Array\}
  Message to be sent.
\item
  \texttt{offset} \{integer\} Offset in the buffer where the message
  starts.
\item
  \texttt{length} \{integer\} Number of bytes in the message.
\item
  \texttt{port} \{integer\} Destination port.
\item
  \texttt{address} \{string\} Destination host name or IP address.
\item
  \texttt{callback} \{Function\} Called when the message has been sent.
\end{itemize}

Broadcasts a datagram on the socket. For connectionless sockets, the
destination \texttt{port} and \texttt{address} must be specified.
Connected sockets, on the other hand, will use their associated remote
endpoint, so the \texttt{port} and \texttt{address} arguments must not
be set.

The \texttt{msg} argument contains the message to be sent. Depending on
its type, different behavior can apply. If \texttt{msg} is a
\texttt{Buffer}, any \texttt{TypedArray} or a \texttt{DataView}, the
\texttt{offset} and \texttt{length} specify the offset within the
\texttt{Buffer} where the message begins and the number of bytes in the
message, respectively. If \texttt{msg} is a \texttt{String}, then it is
automatically converted to a \texttt{Buffer} with
\texttt{\textquotesingle{}utf8\textquotesingle{}} encoding. With
messages that contain multi-byte characters, \texttt{offset} and
\texttt{length} will be calculated with respect to
\href{buffer.md\#static-method-bufferbytelengthstring-encoding}{byte
length} and not the character position. If \texttt{msg} is an array,
\texttt{offset} and \texttt{length} must not be specified.

The \texttt{address} argument is a string. If the value of
\texttt{address} is a host name, DNS will be used to resolve the address
of the host. If \texttt{address} is not provided or otherwise nullish,
\texttt{\textquotesingle{}127.0.0.1\textquotesingle{}} (for
\texttt{udp4} sockets) or
\texttt{\textquotesingle{}::1\textquotesingle{}} (for \texttt{udp6}
sockets) will be used by default.

If the socket has not been previously bound with a call to
\texttt{bind}, the socket is assigned a random port number and is bound
to the ``all interfaces'' address
(\texttt{\textquotesingle{}0.0.0.0\textquotesingle{}} for \texttt{udp4}
sockets, \texttt{\textquotesingle{}::0\textquotesingle{}} for
\texttt{udp6} sockets.)

An optional \texttt{callback} function may be specified to as a way of
reporting DNS errors or for determining when it is safe to reuse the
\texttt{buf} object. DNS lookups delay the time to send for at least one
tick of the Node.js event loop.

The only way to know for sure that the datagram has been sent is by
using a \texttt{callback}. If an error occurs and a \texttt{callback} is
given, the error will be passed as the first argument to the
\texttt{callback}. If a \texttt{callback} is not given, the error is
emitted as an \texttt{\textquotesingle{}error\textquotesingle{}} event
on the \texttt{socket} object.

Offset and length are optional but both \emph{must} be set if either are
used. They are supported only when the first argument is a
\texttt{Buffer}, a \texttt{TypedArray}, or a \texttt{DataView}.

This method throws
\href{errors.md\#err_socket_bad_port}{\texttt{ERR\_SOCKET\_BAD\_PORT}}
if called on an unbound socket.

Example of sending a UDP packet to a port on \texttt{localhost};

\begin{Shaded}
\begin{Highlighting}[]
\ImportTok{import}\NormalTok{ dgram }\ImportTok{from} \StringTok{\textquotesingle{}node:dgram\textquotesingle{}}\OperatorTok{;}
\ImportTok{import}\NormalTok{ \{ }\BuiltInTok{Buffer}\NormalTok{ \} }\ImportTok{from} \StringTok{\textquotesingle{}node:buffer\textquotesingle{}}\OperatorTok{;}

\KeywordTok{const}\NormalTok{ message }\OperatorTok{=} \BuiltInTok{Buffer}\OperatorTok{.}\FunctionTok{from}\NormalTok{(}\StringTok{\textquotesingle{}Some bytes\textquotesingle{}}\NormalTok{)}\OperatorTok{;}
\KeywordTok{const}\NormalTok{ client }\OperatorTok{=}\NormalTok{ dgram}\OperatorTok{.}\FunctionTok{createSocket}\NormalTok{(}\StringTok{\textquotesingle{}udp4\textquotesingle{}}\NormalTok{)}\OperatorTok{;}
\NormalTok{client}\OperatorTok{.}\FunctionTok{send}\NormalTok{(message}\OperatorTok{,} \DecValTok{41234}\OperatorTok{,} \StringTok{\textquotesingle{}localhost\textquotesingle{}}\OperatorTok{,}\NormalTok{ (err) }\KeywordTok{=\textgreater{}}\NormalTok{ \{}
\NormalTok{  client}\OperatorTok{.}\FunctionTok{close}\NormalTok{()}\OperatorTok{;}
\NormalTok{\})}\OperatorTok{;}
\end{Highlighting}
\end{Shaded}

\begin{Shaded}
\begin{Highlighting}[]
\KeywordTok{const}\NormalTok{ dgram }\OperatorTok{=} \PreprocessorTok{require}\NormalTok{(}\StringTok{\textquotesingle{}node:dgram\textquotesingle{}}\NormalTok{)}\OperatorTok{;}
\KeywordTok{const}\NormalTok{ \{ }\BuiltInTok{Buffer}\NormalTok{ \} }\OperatorTok{=} \PreprocessorTok{require}\NormalTok{(}\StringTok{\textquotesingle{}node:buffer\textquotesingle{}}\NormalTok{)}\OperatorTok{;}

\KeywordTok{const}\NormalTok{ message }\OperatorTok{=} \BuiltInTok{Buffer}\OperatorTok{.}\FunctionTok{from}\NormalTok{(}\StringTok{\textquotesingle{}Some bytes\textquotesingle{}}\NormalTok{)}\OperatorTok{;}
\KeywordTok{const}\NormalTok{ client }\OperatorTok{=}\NormalTok{ dgram}\OperatorTok{.}\FunctionTok{createSocket}\NormalTok{(}\StringTok{\textquotesingle{}udp4\textquotesingle{}}\NormalTok{)}\OperatorTok{;}
\NormalTok{client}\OperatorTok{.}\FunctionTok{send}\NormalTok{(message}\OperatorTok{,} \DecValTok{41234}\OperatorTok{,} \StringTok{\textquotesingle{}localhost\textquotesingle{}}\OperatorTok{,}\NormalTok{ (err) }\KeywordTok{=\textgreater{}}\NormalTok{ \{}
\NormalTok{  client}\OperatorTok{.}\FunctionTok{close}\NormalTok{()}\OperatorTok{;}
\NormalTok{\})}\OperatorTok{;}
\end{Highlighting}
\end{Shaded}

Example of sending a UDP packet composed of multiple buffers to a port
on \texttt{127.0.0.1};

\begin{Shaded}
\begin{Highlighting}[]
\ImportTok{import}\NormalTok{ dgram }\ImportTok{from} \StringTok{\textquotesingle{}node:dgram\textquotesingle{}}\OperatorTok{;}
\ImportTok{import}\NormalTok{ \{ }\BuiltInTok{Buffer}\NormalTok{ \} }\ImportTok{from} \StringTok{\textquotesingle{}node:buffer\textquotesingle{}}\OperatorTok{;}

\KeywordTok{const}\NormalTok{ buf1 }\OperatorTok{=} \BuiltInTok{Buffer}\OperatorTok{.}\FunctionTok{from}\NormalTok{(}\StringTok{\textquotesingle{}Some \textquotesingle{}}\NormalTok{)}\OperatorTok{;}
\KeywordTok{const}\NormalTok{ buf2 }\OperatorTok{=} \BuiltInTok{Buffer}\OperatorTok{.}\FunctionTok{from}\NormalTok{(}\StringTok{\textquotesingle{}bytes\textquotesingle{}}\NormalTok{)}\OperatorTok{;}
\KeywordTok{const}\NormalTok{ client }\OperatorTok{=}\NormalTok{ dgram}\OperatorTok{.}\FunctionTok{createSocket}\NormalTok{(}\StringTok{\textquotesingle{}udp4\textquotesingle{}}\NormalTok{)}\OperatorTok{;}
\NormalTok{client}\OperatorTok{.}\FunctionTok{send}\NormalTok{([buf1}\OperatorTok{,}\NormalTok{ buf2]}\OperatorTok{,} \DecValTok{41234}\OperatorTok{,}\NormalTok{ (err) }\KeywordTok{=\textgreater{}}\NormalTok{ \{}
\NormalTok{  client}\OperatorTok{.}\FunctionTok{close}\NormalTok{()}\OperatorTok{;}
\NormalTok{\})}\OperatorTok{;}
\end{Highlighting}
\end{Shaded}

\begin{Shaded}
\begin{Highlighting}[]
\KeywordTok{const}\NormalTok{ dgram }\OperatorTok{=} \PreprocessorTok{require}\NormalTok{(}\StringTok{\textquotesingle{}node:dgram\textquotesingle{}}\NormalTok{)}\OperatorTok{;}
\KeywordTok{const}\NormalTok{ \{ }\BuiltInTok{Buffer}\NormalTok{ \} }\OperatorTok{=} \PreprocessorTok{require}\NormalTok{(}\StringTok{\textquotesingle{}node:buffer\textquotesingle{}}\NormalTok{)}\OperatorTok{;}

\KeywordTok{const}\NormalTok{ buf1 }\OperatorTok{=} \BuiltInTok{Buffer}\OperatorTok{.}\FunctionTok{from}\NormalTok{(}\StringTok{\textquotesingle{}Some \textquotesingle{}}\NormalTok{)}\OperatorTok{;}
\KeywordTok{const}\NormalTok{ buf2 }\OperatorTok{=} \BuiltInTok{Buffer}\OperatorTok{.}\FunctionTok{from}\NormalTok{(}\StringTok{\textquotesingle{}bytes\textquotesingle{}}\NormalTok{)}\OperatorTok{;}
\KeywordTok{const}\NormalTok{ client }\OperatorTok{=}\NormalTok{ dgram}\OperatorTok{.}\FunctionTok{createSocket}\NormalTok{(}\StringTok{\textquotesingle{}udp4\textquotesingle{}}\NormalTok{)}\OperatorTok{;}
\NormalTok{client}\OperatorTok{.}\FunctionTok{send}\NormalTok{([buf1}\OperatorTok{,}\NormalTok{ buf2]}\OperatorTok{,} \DecValTok{41234}\OperatorTok{,}\NormalTok{ (err) }\KeywordTok{=\textgreater{}}\NormalTok{ \{}
\NormalTok{  client}\OperatorTok{.}\FunctionTok{close}\NormalTok{()}\OperatorTok{;}
\NormalTok{\})}\OperatorTok{;}
\end{Highlighting}
\end{Shaded}

Sending multiple buffers might be faster or slower depending on the
application and operating system. Run benchmarks to determine the
optimal strategy on a case-by-case basis. Generally speaking, however,
sending multiple buffers is faster.

Example of sending a UDP packet using a socket connected to a port on
\texttt{localhost}:

\begin{Shaded}
\begin{Highlighting}[]
\ImportTok{import}\NormalTok{ dgram }\ImportTok{from} \StringTok{\textquotesingle{}node:dgram\textquotesingle{}}\OperatorTok{;}
\ImportTok{import}\NormalTok{ \{ }\BuiltInTok{Buffer}\NormalTok{ \} }\ImportTok{from} \StringTok{\textquotesingle{}node:buffer\textquotesingle{}}\OperatorTok{;}

\KeywordTok{const}\NormalTok{ message }\OperatorTok{=} \BuiltInTok{Buffer}\OperatorTok{.}\FunctionTok{from}\NormalTok{(}\StringTok{\textquotesingle{}Some bytes\textquotesingle{}}\NormalTok{)}\OperatorTok{;}
\KeywordTok{const}\NormalTok{ client }\OperatorTok{=}\NormalTok{ dgram}\OperatorTok{.}\FunctionTok{createSocket}\NormalTok{(}\StringTok{\textquotesingle{}udp4\textquotesingle{}}\NormalTok{)}\OperatorTok{;}
\NormalTok{client}\OperatorTok{.}\FunctionTok{connect}\NormalTok{(}\DecValTok{41234}\OperatorTok{,} \StringTok{\textquotesingle{}localhost\textquotesingle{}}\OperatorTok{,}\NormalTok{ (err) }\KeywordTok{=\textgreater{}}\NormalTok{ \{}
\NormalTok{  client}\OperatorTok{.}\FunctionTok{send}\NormalTok{(message}\OperatorTok{,}\NormalTok{ (err) }\KeywordTok{=\textgreater{}}\NormalTok{ \{}
\NormalTok{    client}\OperatorTok{.}\FunctionTok{close}\NormalTok{()}\OperatorTok{;}
\NormalTok{  \})}\OperatorTok{;}
\NormalTok{\})}\OperatorTok{;}
\end{Highlighting}
\end{Shaded}

\begin{Shaded}
\begin{Highlighting}[]
\KeywordTok{const}\NormalTok{ dgram }\OperatorTok{=} \PreprocessorTok{require}\NormalTok{(}\StringTok{\textquotesingle{}node:dgram\textquotesingle{}}\NormalTok{)}\OperatorTok{;}
\KeywordTok{const}\NormalTok{ \{ }\BuiltInTok{Buffer}\NormalTok{ \} }\OperatorTok{=} \PreprocessorTok{require}\NormalTok{(}\StringTok{\textquotesingle{}node:buffer\textquotesingle{}}\NormalTok{)}\OperatorTok{;}

\KeywordTok{const}\NormalTok{ message }\OperatorTok{=} \BuiltInTok{Buffer}\OperatorTok{.}\FunctionTok{from}\NormalTok{(}\StringTok{\textquotesingle{}Some bytes\textquotesingle{}}\NormalTok{)}\OperatorTok{;}
\KeywordTok{const}\NormalTok{ client }\OperatorTok{=}\NormalTok{ dgram}\OperatorTok{.}\FunctionTok{createSocket}\NormalTok{(}\StringTok{\textquotesingle{}udp4\textquotesingle{}}\NormalTok{)}\OperatorTok{;}
\NormalTok{client}\OperatorTok{.}\FunctionTok{connect}\NormalTok{(}\DecValTok{41234}\OperatorTok{,} \StringTok{\textquotesingle{}localhost\textquotesingle{}}\OperatorTok{,}\NormalTok{ (err) }\KeywordTok{=\textgreater{}}\NormalTok{ \{}
\NormalTok{  client}\OperatorTok{.}\FunctionTok{send}\NormalTok{(message}\OperatorTok{,}\NormalTok{ (err) }\KeywordTok{=\textgreater{}}\NormalTok{ \{}
\NormalTok{    client}\OperatorTok{.}\FunctionTok{close}\NormalTok{()}\OperatorTok{;}
\NormalTok{  \})}\OperatorTok{;}
\NormalTok{\})}\OperatorTok{;}
\end{Highlighting}
\end{Shaded}

\paragraph{Note about UDP datagram
size}\label{note-about-udp-datagram-size}

The maximum size of an IPv4/v6 datagram depends on the \texttt{MTU}
(Maximum Transmission Unit) and on the \texttt{Payload\ Length} field
size.

\begin{itemize}
\item
  The \texttt{Payload\ Length} field is 16 bits wide, which means that a
  normal payload cannot exceed 64K octets including the internet header
  and data (65,507 bytes = 65,535 − 8 bytes UDP header − 20 bytes IP
  header); this is generally true for loopback interfaces, but such long
  datagram messages are impractical for most hosts and networks.
\item
  The \texttt{MTU} is the largest size a given link layer technology can
  support for datagram messages. For any link, IPv4 mandates a minimum
  \texttt{MTU} of 68 octets, while the recommended \texttt{MTU} for IPv4
  is 576 (typically recommended as the \texttt{MTU} for dial-up type
  applications), whether they arrive whole or in fragments.

  For IPv6, the minimum \texttt{MTU} is 1280 octets. However, the
  mandatory minimum fragment reassembly buffer size is 1500 octets. The
  value of 68 octets is very small, since most current link layer
  technologies, like Ethernet, have a minimum \texttt{MTU} of 1500.
\end{itemize}

It is impossible to know in advance the MTU of each link through which a
packet might travel. Sending a datagram greater than the receiver
\texttt{MTU} will not work because the packet will get silently dropped
without informing the source that the data did not reach its intended
recipient.

\subsubsection{\texorpdfstring{\texttt{socket.setBroadcast(flag)}}{socket.setBroadcast(flag)}}\label{socket.setbroadcastflag}

\begin{itemize}
\tightlist
\item
  \texttt{flag} \{boolean\}
\end{itemize}

Sets or clears the \texttt{SO\_BROADCAST} socket option. When set to
\texttt{true}, UDP packets may be sent to a local interface's broadcast
address.

This method throws \texttt{EBADF} if called on an unbound socket.

\subsubsection{\texorpdfstring{\texttt{socket.setMulticastInterface(multicastInterface)}}{socket.setMulticastInterface(multicastInterface)}}\label{socket.setmulticastinterfacemulticastinterface}

\begin{itemize}
\tightlist
\item
  \texttt{multicastInterface} \{string\}
\end{itemize}

\emph{All references to scope in this section are referring to
\href{https://en.wikipedia.org/wiki/IPv6_address\#Scoped_literal_IPv6_addresses}{IPv6
Zone Indices}, which are defined by
\href{https://tools.ietf.org/html/rfc4007}{RFC 4007}. In string form, an
IP with a scope index is written as
\texttt{\textquotesingle{}IP\%scope\textquotesingle{}} where scope is an
interface name or interface number.}

Sets the default outgoing multicast interface of the socket to a chosen
interface or back to system interface selection. The
\texttt{multicastInterface} must be a valid string representation of an
IP from the socket's family.

For IPv4 sockets, this should be the IP configured for the desired
physical interface. All packets sent to multicast on the socket will be
sent on the interface determined by the most recent successful use of
this call.

For IPv6 sockets, \texttt{multicastInterface} should include a scope to
indicate the interface as in the examples that follow. In IPv6,
individual \texttt{send} calls can also use explicit scope in addresses,
so only packets sent to a multicast address without specifying an
explicit scope are affected by the most recent successful use of this
call.

This method throws \texttt{EBADF} if called on an unbound socket.

\paragraph{Example: IPv6 outgoing multicast
interface}\label{example-ipv6-outgoing-multicast-interface}

On most systems, where scope format uses the interface name:

\begin{Shaded}
\begin{Highlighting}[]
\KeywordTok{const}\NormalTok{ socket }\OperatorTok{=}\NormalTok{ dgram}\OperatorTok{.}\FunctionTok{createSocket}\NormalTok{(}\StringTok{\textquotesingle{}udp6\textquotesingle{}}\NormalTok{)}\OperatorTok{;}

\NormalTok{socket}\OperatorTok{.}\FunctionTok{bind}\NormalTok{(}\DecValTok{1234}\OperatorTok{,}\NormalTok{ () }\KeywordTok{=\textgreater{}}\NormalTok{ \{}
\NormalTok{  socket}\OperatorTok{.}\FunctionTok{setMulticastInterface}\NormalTok{(}\StringTok{\textquotesingle{}::\%eth1\textquotesingle{}}\NormalTok{)}\OperatorTok{;}
\NormalTok{\})}\OperatorTok{;}
\end{Highlighting}
\end{Shaded}

On Windows, where scope format uses an interface number:

\begin{Shaded}
\begin{Highlighting}[]
\KeywordTok{const}\NormalTok{ socket }\OperatorTok{=}\NormalTok{ dgram}\OperatorTok{.}\FunctionTok{createSocket}\NormalTok{(}\StringTok{\textquotesingle{}udp6\textquotesingle{}}\NormalTok{)}\OperatorTok{;}

\NormalTok{socket}\OperatorTok{.}\FunctionTok{bind}\NormalTok{(}\DecValTok{1234}\OperatorTok{,}\NormalTok{ () }\KeywordTok{=\textgreater{}}\NormalTok{ \{}
\NormalTok{  socket}\OperatorTok{.}\FunctionTok{setMulticastInterface}\NormalTok{(}\StringTok{\textquotesingle{}::\%2\textquotesingle{}}\NormalTok{)}\OperatorTok{;}
\NormalTok{\})}\OperatorTok{;}
\end{Highlighting}
\end{Shaded}

\paragraph{Example: IPv4 outgoing multicast
interface}\label{example-ipv4-outgoing-multicast-interface}

All systems use an IP of the host on the desired physical interface:

\begin{Shaded}
\begin{Highlighting}[]
\KeywordTok{const}\NormalTok{ socket }\OperatorTok{=}\NormalTok{ dgram}\OperatorTok{.}\FunctionTok{createSocket}\NormalTok{(}\StringTok{\textquotesingle{}udp4\textquotesingle{}}\NormalTok{)}\OperatorTok{;}

\NormalTok{socket}\OperatorTok{.}\FunctionTok{bind}\NormalTok{(}\DecValTok{1234}\OperatorTok{,}\NormalTok{ () }\KeywordTok{=\textgreater{}}\NormalTok{ \{}
\NormalTok{  socket}\OperatorTok{.}\FunctionTok{setMulticastInterface}\NormalTok{(}\StringTok{\textquotesingle{}10.0.0.2\textquotesingle{}}\NormalTok{)}\OperatorTok{;}
\NormalTok{\})}\OperatorTok{;}
\end{Highlighting}
\end{Shaded}

\paragraph{Call results}\label{call-results}

A call on a socket that is not ready to send or no longer open may throw
a \emph{Not running} \href{errors.md\#class-error}{\texttt{Error}}.

If \texttt{multicastInterface} can not be parsed into an IP then an
\emph{EINVAL}
\href{errors.md\#class-systemerror}{\texttt{System\ Error}} is thrown.

On IPv4, if \texttt{multicastInterface} is a valid address but does not
match any interface, or if the address does not match the family then a
\href{errors.md\#class-systemerror}{\texttt{System\ Error}} such as
\texttt{EADDRNOTAVAIL} or \texttt{EPROTONOSUP} is thrown.

On IPv6, most errors with specifying or omitting scope will result in
the socket continuing to use (or returning to) the system's default
interface selection.

A socket's address family's ANY address (IPv4
\texttt{\textquotesingle{}0.0.0.0\textquotesingle{}} or IPv6
\texttt{\textquotesingle{}::\textquotesingle{}}) can be used to return
control of the sockets default outgoing interface to the system for
future multicast packets.

\subsubsection{\texorpdfstring{\texttt{socket.setMulticastLoopback(flag)}}{socket.setMulticastLoopback(flag)}}\label{socket.setmulticastloopbackflag}

\begin{itemize}
\tightlist
\item
  \texttt{flag} \{boolean\}
\end{itemize}

Sets or clears the \texttt{IP\_MULTICAST\_LOOP} socket option. When set
to \texttt{true}, multicast packets will also be received on the local
interface.

This method throws \texttt{EBADF} if called on an unbound socket.

\subsubsection{\texorpdfstring{\texttt{socket.setMulticastTTL(ttl)}}{socket.setMulticastTTL(ttl)}}\label{socket.setmulticastttlttl}

\begin{itemize}
\tightlist
\item
  \texttt{ttl} \{integer\}
\end{itemize}

Sets the \texttt{IP\_MULTICAST\_TTL} socket option. While TTL generally
stands for ``Time to Live'', in this context it specifies the number of
IP hops that a packet is allowed to travel through, specifically for
multicast traffic. Each router or gateway that forwards a packet
decrements the TTL. If the TTL is decremented to 0 by a router, it will
not be forwarded.

The \texttt{ttl} argument may be between 0 and 255. The default on most
systems is \texttt{1}.

This method throws \texttt{EBADF} if called on an unbound socket.

\subsubsection{\texorpdfstring{\texttt{socket.setRecvBufferSize(size)}}{socket.setRecvBufferSize(size)}}\label{socket.setrecvbuffersizesize}

\begin{itemize}
\tightlist
\item
  \texttt{size} \{integer\}
\end{itemize}

Sets the \texttt{SO\_RCVBUF} socket option. Sets the maximum socket
receive buffer in bytes.

This method throws
\href{errors.md\#err_socket_buffer_size}{\texttt{ERR\_SOCKET\_BUFFER\_SIZE}}
if called on an unbound socket.

\subsubsection{\texorpdfstring{\texttt{socket.setSendBufferSize(size)}}{socket.setSendBufferSize(size)}}\label{socket.setsendbuffersizesize}

\begin{itemize}
\tightlist
\item
  \texttt{size} \{integer\}
\end{itemize}

Sets the \texttt{SO\_SNDBUF} socket option. Sets the maximum socket send
buffer in bytes.

This method throws
\href{errors.md\#err_socket_buffer_size}{\texttt{ERR\_SOCKET\_BUFFER\_SIZE}}
if called on an unbound socket.

\subsubsection{\texorpdfstring{\texttt{socket.setTTL(ttl)}}{socket.setTTL(ttl)}}\label{socket.setttlttl}

\begin{itemize}
\tightlist
\item
  \texttt{ttl} \{integer\}
\end{itemize}

Sets the \texttt{IP\_TTL} socket option. While TTL generally stands for
``Time to Live'', in this context it specifies the number of IP hops
that a packet is allowed to travel through. Each router or gateway that
forwards a packet decrements the TTL. If the TTL is decremented to 0 by
a router, it will not be forwarded. Changing TTL values is typically
done for network probes or when multicasting.

The \texttt{ttl} argument may be between 1 and 255. The default on most
systems is 64.

This method throws \texttt{EBADF} if called on an unbound socket.

\subsubsection{\texorpdfstring{\texttt{socket.unref()}}{socket.unref()}}\label{socket.unref}

\begin{itemize}
\tightlist
\item
  Returns: \{dgram.Socket\}
\end{itemize}

By default, binding a socket will cause it to block the Node.js process
from exiting as long as the socket is open. The \texttt{socket.unref()}
method can be used to exclude the socket from the reference counting
that keeps the Node.js process active, allowing the process to exit even
if the socket is still listening.

Calling \texttt{socket.unref()} multiple times will have no additional
effect.

The \texttt{socket.unref()} method returns a reference to the socket so
calls can be chained.

\subsection{\texorpdfstring{\texttt{node:dgram} module
functions}{node:dgram module functions}}\label{nodedgram-module-functions}

\subsubsection{\texorpdfstring{\texttt{dgram.createSocket(options{[},\ callback{]})}}{dgram.createSocket(options{[}, callback{]})}}\label{dgram.createsocketoptions-callback}

\begin{itemize}
\tightlist
\item
  \texttt{options} \{Object\} Available options are:

  \begin{itemize}
  \tightlist
  \item
    \texttt{type} \{string\} The family of socket. Must be either
    \texttt{\textquotesingle{}udp4\textquotesingle{}} or
    \texttt{\textquotesingle{}udp6\textquotesingle{}}. Required.
  \item
    \texttt{reuseAddr} \{boolean\} When \texttt{true}
    \hyperref[socketbindport-address-callback]{\texttt{socket.bind()}}
    will reuse the address, even if another process has already bound a
    socket on it. \textbf{Default:} \texttt{false}.
  \item
    \texttt{ipv6Only} \{boolean\} Setting \texttt{ipv6Only} to
    \texttt{true} will disable dual-stack support, i.e., binding to
    address \texttt{::} won't make \texttt{0.0.0.0} be bound.
    \textbf{Default:} \texttt{false}.
  \item
    \texttt{recvBufferSize} \{number\} Sets the \texttt{SO\_RCVBUF}
    socket value.
  \item
    \texttt{sendBufferSize} \{number\} Sets the \texttt{SO\_SNDBUF}
    socket value.
  \item
    \texttt{lookup} \{Function\} Custom lookup function.
    \textbf{Default:}
    \href{dns.md\#dnslookuphostname-options-callback}{\texttt{dns.lookup()}}.
  \item
    \texttt{signal} \{AbortSignal\} An AbortSignal that may be used to
    close a socket.
  \end{itemize}
\item
  \texttt{callback} \{Function\} Attached as a listener for
  \texttt{\textquotesingle{}message\textquotesingle{}} events. Optional.
\item
  Returns: \{dgram.Socket\}
\end{itemize}

Creates a \texttt{dgram.Socket} object. Once the socket is created,
calling
\hyperref[socketbindport-address-callback]{\texttt{socket.bind()}} will
instruct the socket to begin listening for datagram messages. When
\texttt{address} and \texttt{port} are not passed to
\hyperref[socketbindport-address-callback]{\texttt{socket.bind()}} the
method will bind the socket to the ``all interfaces'' address on a
random port (it does the right thing for both \texttt{udp4} and
\texttt{udp6} sockets). The bound address and port can be retrieved
using \hyperref[socketaddress]{\texttt{socket.address().address}} and
\hyperref[socketaddress]{\texttt{socket.address().port}}.

If the \texttt{signal} option is enabled, calling \texttt{.abort()} on
the corresponding \texttt{AbortController} is similar to calling
\texttt{.close()} on the socket:

\begin{Shaded}
\begin{Highlighting}[]
\KeywordTok{const}\NormalTok{ controller }\OperatorTok{=} \KeywordTok{new} \FunctionTok{AbortController}\NormalTok{()}\OperatorTok{;}
\KeywordTok{const}\NormalTok{ \{ signal \} }\OperatorTok{=}\NormalTok{ controller}\OperatorTok{;}
\KeywordTok{const}\NormalTok{ server }\OperatorTok{=}\NormalTok{ dgram}\OperatorTok{.}\FunctionTok{createSocket}\NormalTok{(\{ }\DataTypeTok{type}\OperatorTok{:} \StringTok{\textquotesingle{}udp4\textquotesingle{}}\OperatorTok{,}\NormalTok{ signal \})}\OperatorTok{;}
\NormalTok{server}\OperatorTok{.}\FunctionTok{on}\NormalTok{(}\StringTok{\textquotesingle{}message\textquotesingle{}}\OperatorTok{,}\NormalTok{ (msg}\OperatorTok{,}\NormalTok{ rinfo) }\KeywordTok{=\textgreater{}}\NormalTok{ \{}
  \BuiltInTok{console}\OperatorTok{.}\FunctionTok{log}\NormalTok{(}\VerbatimStringTok{\textasciigrave{}server got: }\SpecialCharTok{$\{}\NormalTok{msg}\SpecialCharTok{\}}\VerbatimStringTok{ from }\SpecialCharTok{$\{}\NormalTok{rinfo}\OperatorTok{.}\AttributeTok{address}\SpecialCharTok{\}}\VerbatimStringTok{:}\SpecialCharTok{$\{}\NormalTok{rinfo}\OperatorTok{.}\AttributeTok{port}\SpecialCharTok{\}}\VerbatimStringTok{\textasciigrave{}}\NormalTok{)}\OperatorTok{;}
\NormalTok{\})}\OperatorTok{;}
\CommentTok{// Later, when you want to close the server.}
\NormalTok{controller}\OperatorTok{.}\FunctionTok{abort}\NormalTok{()}\OperatorTok{;}
\end{Highlighting}
\end{Shaded}

\subsubsection{\texorpdfstring{\texttt{dgram.createSocket(type{[},\ callback{]})}}{dgram.createSocket(type{[}, callback{]})}}\label{dgram.createsockettype-callback}

\begin{itemize}
\tightlist
\item
  \texttt{type} \{string\} Either
  \texttt{\textquotesingle{}udp4\textquotesingle{}} or
  \texttt{\textquotesingle{}udp6\textquotesingle{}}.
\item
  \texttt{callback} \{Function\} Attached as a listener to
  \texttt{\textquotesingle{}message\textquotesingle{}} events.
\item
  Returns: \{dgram.Socket\}
\end{itemize}

Creates a \texttt{dgram.Socket} object of the specified \texttt{type}.

Once the socket is created, calling
\hyperref[socketbindport-address-callback]{\texttt{socket.bind()}} will
instruct the socket to begin listening for datagram messages. When
\texttt{address} and \texttt{port} are not passed to
\hyperref[socketbindport-address-callback]{\texttt{socket.bind()}} the
method will bind the socket to the ``all interfaces'' address on a
random port (it does the right thing for both \texttt{udp4} and
\texttt{udp6} sockets). The bound address and port can be retrieved
using \hyperref[socketaddress]{\texttt{socket.address().address}} and
\hyperref[socketaddress]{\texttt{socket.address().port}}.
