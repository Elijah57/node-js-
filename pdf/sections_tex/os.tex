\section{OS}\label{os}

\begin{quote}
Stability: 2 - Stable
\end{quote}

The \texttt{node:os} module provides operating system-related utility
methods and properties. It can be accessed using:

\begin{Shaded}
\begin{Highlighting}[]
\KeywordTok{const}\NormalTok{ os }\OperatorTok{=} \PreprocessorTok{require}\NormalTok{(}\StringTok{\textquotesingle{}node:os\textquotesingle{}}\NormalTok{)}\OperatorTok{;}
\end{Highlighting}
\end{Shaded}

\subsection{\texorpdfstring{\texttt{os.EOL}}{os.EOL}}\label{os.eol}

\begin{itemize}
\tightlist
\item
  \{string\}
\end{itemize}

The operating system-specific end-of-line marker.

\begin{itemize}
\tightlist
\item
  \texttt{\textbackslash{}n} on POSIX
\item
  \texttt{\textbackslash{}r\textbackslash{}n} on Windows
\end{itemize}

\subsection{\texorpdfstring{\texttt{os.availableParallelism()}}{os.availableParallelism()}}\label{os.availableparallelism}

\begin{itemize}
\tightlist
\item
  Returns: \{integer\}
\end{itemize}

Returns an estimate of the default amount of parallelism a program
should use. Always returns a value greater than zero.

This function is a small wrapper about libuv's
\href{https://docs.libuv.org/en/v1.x/misc.html\#c.uv_available_parallelism}{\texttt{uv\_available\_parallelism()}}.

\subsection{\texorpdfstring{\texttt{os.arch()}}{os.arch()}}\label{os.arch}

\begin{itemize}
\tightlist
\item
  Returns: \{string\}
\end{itemize}

Returns the operating system CPU architecture for which the Node.js
binary was compiled. Possible values are
\texttt{\textquotesingle{}arm\textquotesingle{}},
\texttt{\textquotesingle{}arm64\textquotesingle{}},
\texttt{\textquotesingle{}ia32\textquotesingle{}},
\texttt{\textquotesingle{}loong64\textquotesingle{}},
\texttt{\textquotesingle{}mips\textquotesingle{}},
\texttt{\textquotesingle{}mipsel\textquotesingle{}},
\texttt{\textquotesingle{}ppc\textquotesingle{}},
\texttt{\textquotesingle{}ppc64\textquotesingle{}},
\texttt{\textquotesingle{}riscv64\textquotesingle{}},
\texttt{\textquotesingle{}s390\textquotesingle{}},
\texttt{\textquotesingle{}s390x\textquotesingle{}}, and
\texttt{\textquotesingle{}x64\textquotesingle{}}.

The return value is equivalent to
\href{process.md\#processarch}{\texttt{process.arch}}.

\subsection{\texorpdfstring{\texttt{os.constants}}{os.constants}}\label{os.constants}

\begin{itemize}
\tightlist
\item
  \{Object\}
\end{itemize}

Contains commonly used operating system-specific constants for error
codes, process signals, and so on. The specific constants defined are
described in \hyperref[os-constants]{OS constants}.

\subsection{\texorpdfstring{\texttt{os.cpus()}}{os.cpus()}}\label{os.cpus}

\begin{itemize}
\tightlist
\item
  Returns: \{Object{[}{]}\}
\end{itemize}

Returns an array of objects containing information about each logical
CPU core. The array will be empty if no CPU information is available,
such as if the \texttt{/proc} file system is unavailable.

The properties included on each object include:

\begin{itemize}
\tightlist
\item
  \texttt{model} \{string\}
\item
  \texttt{speed} \{number\} (in MHz)
\item
  \texttt{times} \{Object\}

  \begin{itemize}
  \tightlist
  \item
    \texttt{user} \{number\} The number of milliseconds the CPU has
    spent in user mode.
  \item
    \texttt{nice} \{number\} The number of milliseconds the CPU has
    spent in nice mode.
  \item
    \texttt{sys} \{number\} The number of milliseconds the CPU has spent
    in sys mode.
  \item
    \texttt{idle} \{number\} The number of milliseconds the CPU has
    spent in idle mode.
  \item
    \texttt{irq} \{number\} The number of milliseconds the CPU has spent
    in irq mode.
  \end{itemize}
\end{itemize}

\begin{Shaded}
\begin{Highlighting}[]
\NormalTok{[}
\NormalTok{  \{}
    \DataTypeTok{model}\OperatorTok{:} \StringTok{\textquotesingle{}Intel(R) Core(TM) i7 CPU         860  @ 2.80GHz\textquotesingle{}}\OperatorTok{,}
    \DataTypeTok{speed}\OperatorTok{:} \DecValTok{2926}\OperatorTok{,}
    \DataTypeTok{times}\OperatorTok{:}\NormalTok{ \{}
      \DataTypeTok{user}\OperatorTok{:} \DecValTok{252020}\OperatorTok{,}
      \DataTypeTok{nice}\OperatorTok{:} \DecValTok{0}\OperatorTok{,}
      \DataTypeTok{sys}\OperatorTok{:} \DecValTok{30340}\OperatorTok{,}
      \DataTypeTok{idle}\OperatorTok{:} \DecValTok{1070356870}\OperatorTok{,}
      \DataTypeTok{irq}\OperatorTok{:} \DecValTok{0}\OperatorTok{,}
\NormalTok{    \}}\OperatorTok{,}
\NormalTok{  \}}\OperatorTok{,}
\NormalTok{  \{}
    \DataTypeTok{model}\OperatorTok{:} \StringTok{\textquotesingle{}Intel(R) Core(TM) i7 CPU         860  @ 2.80GHz\textquotesingle{}}\OperatorTok{,}
    \DataTypeTok{speed}\OperatorTok{:} \DecValTok{2926}\OperatorTok{,}
    \DataTypeTok{times}\OperatorTok{:}\NormalTok{ \{}
      \DataTypeTok{user}\OperatorTok{:} \DecValTok{306960}\OperatorTok{,}
      \DataTypeTok{nice}\OperatorTok{:} \DecValTok{0}\OperatorTok{,}
      \DataTypeTok{sys}\OperatorTok{:} \DecValTok{26980}\OperatorTok{,}
      \DataTypeTok{idle}\OperatorTok{:} \DecValTok{1071569080}\OperatorTok{,}
      \DataTypeTok{irq}\OperatorTok{:} \DecValTok{0}\OperatorTok{,}
\NormalTok{    \}}\OperatorTok{,}
\NormalTok{  \}}\OperatorTok{,}
\NormalTok{  \{}
    \DataTypeTok{model}\OperatorTok{:} \StringTok{\textquotesingle{}Intel(R) Core(TM) i7 CPU         860  @ 2.80GHz\textquotesingle{}}\OperatorTok{,}
    \DataTypeTok{speed}\OperatorTok{:} \DecValTok{2926}\OperatorTok{,}
    \DataTypeTok{times}\OperatorTok{:}\NormalTok{ \{}
      \DataTypeTok{user}\OperatorTok{:} \DecValTok{248450}\OperatorTok{,}
      \DataTypeTok{nice}\OperatorTok{:} \DecValTok{0}\OperatorTok{,}
      \DataTypeTok{sys}\OperatorTok{:} \DecValTok{21750}\OperatorTok{,}
      \DataTypeTok{idle}\OperatorTok{:} \DecValTok{1070919370}\OperatorTok{,}
      \DataTypeTok{irq}\OperatorTok{:} \DecValTok{0}\OperatorTok{,}
\NormalTok{    \}}\OperatorTok{,}
\NormalTok{  \}}\OperatorTok{,}
\NormalTok{  \{}
    \DataTypeTok{model}\OperatorTok{:} \StringTok{\textquotesingle{}Intel(R) Core(TM) i7 CPU         860  @ 2.80GHz\textquotesingle{}}\OperatorTok{,}
    \DataTypeTok{speed}\OperatorTok{:} \DecValTok{2926}\OperatorTok{,}
    \DataTypeTok{times}\OperatorTok{:}\NormalTok{ \{}
      \DataTypeTok{user}\OperatorTok{:} \DecValTok{256880}\OperatorTok{,}
      \DataTypeTok{nice}\OperatorTok{:} \DecValTok{0}\OperatorTok{,}
      \DataTypeTok{sys}\OperatorTok{:} \DecValTok{19430}\OperatorTok{,}
      \DataTypeTok{idle}\OperatorTok{:} \DecValTok{1070905480}\OperatorTok{,}
      \DataTypeTok{irq}\OperatorTok{:} \DecValTok{20}\OperatorTok{,}
\NormalTok{    \}}\OperatorTok{,}
\NormalTok{  \}}\OperatorTok{,}
\NormalTok{]}
\end{Highlighting}
\end{Shaded}

\texttt{nice} values are POSIX-only. On Windows, the \texttt{nice}
values of all processors are always 0.

\texttt{os.cpus().length} should not be used to calculate the amount of
parallelism available to an application. Use
\hyperref[osavailableparallelism]{\texttt{os.availableParallelism()}}
for this purpose.

\subsection{\texorpdfstring{\texttt{os.devNull}}{os.devNull}}\label{os.devnull}

\begin{itemize}
\tightlist
\item
  \{string\}
\end{itemize}

The platform-specific file path of the null device.

\begin{itemize}
\tightlist
\item
  \texttt{\textbackslash{}\textbackslash{}.\textbackslash{}nul} on
  Windows
\item
  \texttt{/dev/null} on POSIX
\end{itemize}

\subsection{\texorpdfstring{\texttt{os.endianness()}}{os.endianness()}}\label{os.endianness}

\begin{itemize}
\tightlist
\item
  Returns: \{string\}
\end{itemize}

Returns a string identifying the endianness of the CPU for which the
Node.js binary was compiled.

Possible values are \texttt{\textquotesingle{}BE\textquotesingle{}} for
big endian and \texttt{\textquotesingle{}LE\textquotesingle{}} for
little endian.

\subsection{\texorpdfstring{\texttt{os.freemem()}}{os.freemem()}}\label{os.freemem}

\begin{itemize}
\tightlist
\item
  Returns: \{integer\}
\end{itemize}

Returns the amount of free system memory in bytes as an integer.

\subsection{\texorpdfstring{\texttt{os.getPriority({[}pid{]})}}{os.getPriority({[}pid{]})}}\label{os.getprioritypid}

\begin{itemize}
\tightlist
\item
  \texttt{pid} \{integer\} The process ID to retrieve scheduling
  priority for. \textbf{Default:} \texttt{0}.
\item
  Returns: \{integer\}
\end{itemize}

Returns the scheduling priority for the process specified by
\texttt{pid}. If \texttt{pid} is not provided or is \texttt{0}, the
priority of the current process is returned.

\subsection{\texorpdfstring{\texttt{os.homedir()}}{os.homedir()}}\label{os.homedir}

\begin{itemize}
\tightlist
\item
  Returns: \{string\}
\end{itemize}

Returns the string path of the current user's home directory.

On POSIX, it uses the \texttt{\$HOME} environment variable if defined.
Otherwise it uses the
\href{https://en.wikipedia.org/wiki/User_identifier\#Effective_user_ID}{effective
UID} to look up the user's home directory.

On Windows, it uses the \texttt{USERPROFILE} environment variable if
defined. Otherwise it uses the path to the profile directory of the
current user.

\subsection{\texorpdfstring{\texttt{os.hostname()}}{os.hostname()}}\label{os.hostname}

\begin{itemize}
\tightlist
\item
  Returns: \{string\}
\end{itemize}

Returns the host name of the operating system as a string.

\subsection{\texorpdfstring{\texttt{os.loadavg()}}{os.loadavg()}}\label{os.loadavg}

\begin{itemize}
\tightlist
\item
  Returns: \{number{[}{]}\}
\end{itemize}

Returns an array containing the 1, 5, and 15 minute load averages.

The load average is a measure of system activity calculated by the
operating system and expressed as a fractional number.

The load average is a Unix-specific concept. On Windows, the return
value is always \texttt{{[}0,\ 0,\ 0{]}}.

\subsection{\texorpdfstring{\texttt{os.machine()}}{os.machine()}}\label{os.machine}

\begin{itemize}
\tightlist
\item
  Returns \{string\}
\end{itemize}

Returns the machine type as a string, such as \texttt{arm},
\texttt{arm64}, \texttt{aarch64}, \texttt{mips}, \texttt{mips64},
\texttt{ppc64}, \texttt{ppc64le}, \texttt{s390}, \texttt{s390x},
\texttt{i386}, \texttt{i686}, \texttt{x86\_64}.

On POSIX systems, the machine type is determined by calling
\href{https://linux.die.net/man/3/uname}{\texttt{uname(3)}}. On Windows,
\texttt{RtlGetVersion()} is used, and if it is not available,
\texttt{GetVersionExW()} will be used. See
\url{https://en.wikipedia.org/wiki/Uname\#Examples} for more
information.

\subsection{\texorpdfstring{\texttt{os.networkInterfaces()}}{os.networkInterfaces()}}\label{os.networkinterfaces}

\begin{itemize}
\tightlist
\item
  Returns: \{Object\}
\end{itemize}

Returns an object containing network interfaces that have been assigned
a network address.

Each key on the returned object identifies a network interface. The
associated value is an array of objects that each describe an assigned
network address.

The properties available on the assigned network address object include:

\begin{itemize}
\tightlist
\item
  \texttt{address} \{string\} The assigned IPv4 or IPv6 address
\item
  \texttt{netmask} \{string\} The IPv4 or IPv6 network mask
\item
  \texttt{family} \{string\} Either \texttt{IPv4} or \texttt{IPv6}
\item
  \texttt{mac} \{string\} The MAC address of the network interface
\item
  \texttt{internal} \{boolean\} \texttt{true} if the network interface
  is a loopback or similar interface that is not remotely accessible;
  otherwise \texttt{false}
\item
  \texttt{scopeid} \{number\} The numeric IPv6 scope ID (only specified
  when \texttt{family} is \texttt{IPv6})
\item
  \texttt{cidr} \{string\} The assigned IPv4 or IPv6 address with the
  routing prefix in CIDR notation. If the \texttt{netmask} is invalid,
  this property is set to \texttt{null}.
\end{itemize}

\begin{Shaded}
\begin{Highlighting}[]
\NormalTok{\{}
  \DataTypeTok{lo}\OperatorTok{:}\NormalTok{ [}
\NormalTok{    \{}
      \DataTypeTok{address}\OperatorTok{:} \StringTok{\textquotesingle{}127.0.0.1\textquotesingle{}}\OperatorTok{,}
      \DataTypeTok{netmask}\OperatorTok{:} \StringTok{\textquotesingle{}255.0.0.0\textquotesingle{}}\OperatorTok{,}
      \DataTypeTok{family}\OperatorTok{:} \StringTok{\textquotesingle{}IPv4\textquotesingle{}}\OperatorTok{,}
      \DataTypeTok{mac}\OperatorTok{:} \StringTok{\textquotesingle{}00:00:00:00:00:00\textquotesingle{}}\OperatorTok{,}
      \DataTypeTok{internal}\OperatorTok{:} \KeywordTok{true}\OperatorTok{,}
      \DataTypeTok{cidr}\OperatorTok{:} \StringTok{\textquotesingle{}127.0.0.1/8\textquotesingle{}}
\NormalTok{    \}}\OperatorTok{,}
\NormalTok{    \{}
      \DataTypeTok{address}\OperatorTok{:} \StringTok{\textquotesingle{}::1\textquotesingle{}}\OperatorTok{,}
      \DataTypeTok{netmask}\OperatorTok{:} \StringTok{\textquotesingle{}ffff:ffff:ffff:ffff:ffff:ffff:ffff:ffff\textquotesingle{}}\OperatorTok{,}
      \DataTypeTok{family}\OperatorTok{:} \StringTok{\textquotesingle{}IPv6\textquotesingle{}}\OperatorTok{,}
      \DataTypeTok{mac}\OperatorTok{:} \StringTok{\textquotesingle{}00:00:00:00:00:00\textquotesingle{}}\OperatorTok{,}
      \DataTypeTok{scopeid}\OperatorTok{:} \DecValTok{0}\OperatorTok{,}
      \DataTypeTok{internal}\OperatorTok{:} \KeywordTok{true}\OperatorTok{,}
      \DataTypeTok{cidr}\OperatorTok{:} \StringTok{\textquotesingle{}::1/128\textquotesingle{}}
\NormalTok{    \}}
\NormalTok{  ]}\OperatorTok{,}
  \DataTypeTok{eth0}\OperatorTok{:}\NormalTok{ [}
\NormalTok{    \{}
      \DataTypeTok{address}\OperatorTok{:} \StringTok{\textquotesingle{}192.168.1.108\textquotesingle{}}\OperatorTok{,}
      \DataTypeTok{netmask}\OperatorTok{:} \StringTok{\textquotesingle{}255.255.255.0\textquotesingle{}}\OperatorTok{,}
      \DataTypeTok{family}\OperatorTok{:} \StringTok{\textquotesingle{}IPv4\textquotesingle{}}\OperatorTok{,}
      \DataTypeTok{mac}\OperatorTok{:} \StringTok{\textquotesingle{}01:02:03:0a:0b:0c\textquotesingle{}}\OperatorTok{,}
      \DataTypeTok{internal}\OperatorTok{:} \KeywordTok{false}\OperatorTok{,}
      \DataTypeTok{cidr}\OperatorTok{:} \StringTok{\textquotesingle{}192.168.1.108/24\textquotesingle{}}
\NormalTok{    \}}\OperatorTok{,}
\NormalTok{    \{}
      \DataTypeTok{address}\OperatorTok{:} \StringTok{\textquotesingle{}fe80::a00:27ff:fe4e:66a1\textquotesingle{}}\OperatorTok{,}
      \DataTypeTok{netmask}\OperatorTok{:} \StringTok{\textquotesingle{}ffff:ffff:ffff:ffff::\textquotesingle{}}\OperatorTok{,}
      \DataTypeTok{family}\OperatorTok{:} \StringTok{\textquotesingle{}IPv6\textquotesingle{}}\OperatorTok{,}
      \DataTypeTok{mac}\OperatorTok{:} \StringTok{\textquotesingle{}01:02:03:0a:0b:0c\textquotesingle{}}\OperatorTok{,}
      \DataTypeTok{scopeid}\OperatorTok{:} \DecValTok{1}\OperatorTok{,}
      \DataTypeTok{internal}\OperatorTok{:} \KeywordTok{false}\OperatorTok{,}
      \DataTypeTok{cidr}\OperatorTok{:} \StringTok{\textquotesingle{}fe80::a00:27ff:fe4e:66a1/64\textquotesingle{}}
\NormalTok{    \}}
\NormalTok{  ]}
\NormalTok{\}}
\end{Highlighting}
\end{Shaded}

\subsection{\texorpdfstring{\texttt{os.platform()}}{os.platform()}}\label{os.platform}

\begin{itemize}
\tightlist
\item
  Returns: \{string\}
\end{itemize}

Returns a string identifying the operating system platform for which the
Node.js binary was compiled. The value is set at compile time. Possible
values are \texttt{\textquotesingle{}aix\textquotesingle{}},
\texttt{\textquotesingle{}darwin\textquotesingle{}},
\texttt{\textquotesingle{}freebsd\textquotesingle{}},\texttt{\textquotesingle{}linux\textquotesingle{}},
\texttt{\textquotesingle{}openbsd\textquotesingle{}},
\texttt{\textquotesingle{}sunos\textquotesingle{}}, and
\texttt{\textquotesingle{}win32\textquotesingle{}}.

The return value is equivalent to
\href{process.md\#processplatform}{\texttt{process.platform}}.

The value \texttt{\textquotesingle{}android\textquotesingle{}} may also
be returned if Node.js is built on the Android operating system.
\href{https://github.com/nodejs/node/blob/HEAD/BUILDING.md\#androidandroid-based-devices-eg-firefox-os}{Android
support is experimental}.

\subsection{\texorpdfstring{\texttt{os.release()}}{os.release()}}\label{os.release}

\begin{itemize}
\tightlist
\item
  Returns: \{string\}
\end{itemize}

Returns the operating system as a string.

On POSIX systems, the operating system release is determined by calling
\href{https://linux.die.net/man/3/uname}{\texttt{uname(3)}}. On Windows,
\texttt{GetVersionExW()} is used. See
\url{https://en.wikipedia.org/wiki/Uname\#Examples} for more
information.

\subsection{\texorpdfstring{\texttt{os.setPriority({[}pid,\ {]}priority)}}{os.setPriority({[}pid, {]}priority)}}\label{os.setprioritypid-priority}

\begin{itemize}
\tightlist
\item
  \texttt{pid} \{integer\} The process ID to set scheduling priority
  for. \textbf{Default:} \texttt{0}.
\item
  \texttt{priority} \{integer\} The scheduling priority to assign to the
  process.
\end{itemize}

Attempts to set the scheduling priority for the process specified by
\texttt{pid}. If \texttt{pid} is not provided or is \texttt{0}, the
process ID of the current process is used.

The \texttt{priority} input must be an integer between \texttt{-20}
(high priority) and \texttt{19} (low priority). Due to differences
between Unix priority levels and Windows priority classes,
\texttt{priority} is mapped to one of six priority constants in
\texttt{os.constants.priority}. When retrieving a process priority
level, this range mapping may cause the return value to be slightly
different on Windows. To avoid confusion, set \texttt{priority} to one
of the priority constants.

On Windows, setting priority to \texttt{PRIORITY\_HIGHEST} requires
elevated user privileges. Otherwise the set priority will be silently
reduced to \texttt{PRIORITY\_HIGH}.

\subsection{\texorpdfstring{\texttt{os.tmpdir()}}{os.tmpdir()}}\label{os.tmpdir}

\begin{itemize}
\tightlist
\item
  Returns: \{string\}
\end{itemize}

Returns the operating system's default directory for temporary files as
a string.

\subsection{\texorpdfstring{\texttt{os.totalmem()}}{os.totalmem()}}\label{os.totalmem}

\begin{itemize}
\tightlist
\item
  Returns: \{integer\}
\end{itemize}

Returns the total amount of system memory in bytes as an integer.

\subsection{\texorpdfstring{\texttt{os.type()}}{os.type()}}\label{os.type}

\begin{itemize}
\tightlist
\item
  Returns: \{string\}
\end{itemize}

Returns the operating system name as returned by
\href{https://linux.die.net/man/3/uname}{\texttt{uname(3)}}. For
example, it returns \texttt{\textquotesingle{}Linux\textquotesingle{}}
on Linux, \texttt{\textquotesingle{}Darwin\textquotesingle{}} on macOS,
and \texttt{\textquotesingle{}Windows\_NT\textquotesingle{}} on Windows.

See \url{https://en.wikipedia.org/wiki/Uname\#Examples} for additional
information about the output of running
\href{https://linux.die.net/man/3/uname}{\texttt{uname(3)}} on various
operating systems.

\subsection{\texorpdfstring{\texttt{os.uptime()}}{os.uptime()}}\label{os.uptime}

\begin{itemize}
\tightlist
\item
  Returns: \{integer\}
\end{itemize}

Returns the system uptime in number of seconds.

\subsection{\texorpdfstring{\texttt{os.userInfo({[}options{]})}}{os.userInfo({[}options{]})}}\label{os.userinfooptions}

\begin{itemize}
\tightlist
\item
  \texttt{options} \{Object\}

  \begin{itemize}
  \tightlist
  \item
    \texttt{encoding} \{string\} Character encoding used to interpret
    resulting strings. If \texttt{encoding} is set to
    \texttt{\textquotesingle{}buffer\textquotesingle{}}, the
    \texttt{username}, \texttt{shell}, and \texttt{homedir} values will
    be \texttt{Buffer} instances. \textbf{Default:}
    \texttt{\textquotesingle{}utf8\textquotesingle{}}.
  \end{itemize}
\item
  Returns: \{Object\}
\end{itemize}

Returns information about the currently effective user. On POSIX
platforms, this is typically a subset of the password file. The returned
object includes the \texttt{username}, \texttt{uid}, \texttt{gid},
\texttt{shell}, and \texttt{homedir}. On Windows, the \texttt{uid} and
\texttt{gid} fields are \texttt{-1}, and \texttt{shell} is
\texttt{null}.

The value of \texttt{homedir} returned by \texttt{os.userInfo()} is
provided by the operating system. This differs from the result of
\texttt{os.homedir()}, which queries environment variables for the home
directory before falling back to the operating system response.

Throws a \href{errors.md\#class-systemerror}{\texttt{SystemError}} if a
user has no \texttt{username} or \texttt{homedir}.

\subsection{\texorpdfstring{\texttt{os.version()}}{os.version()}}\label{os.version}

\begin{itemize}
\tightlist
\item
  Returns \{string\}
\end{itemize}

Returns a string identifying the kernel version.

On POSIX systems, the operating system release is determined by calling
\href{https://linux.die.net/man/3/uname}{\texttt{uname(3)}}. On Windows,
\texttt{RtlGetVersion()} is used, and if it is not available,
\texttt{GetVersionExW()} will be used. See
\url{https://en.wikipedia.org/wiki/Uname\#Examples} for more
information.

\subsection{OS constants}\label{os-constants}

The following constants are exported by \texttt{os.constants}.

Not all constants will be available on every operating system.

\subsubsection{Signal constants}\label{signal-constants}

The following signal constants are exported by
\texttt{os.constants.signals}.

Constant

Description

SIGHUP

Sent to indicate when a controlling terminal is closed or a parent
process exits.

SIGINT

Sent to indicate when a user wishes to interrupt a process (Ctrl+C).

SIGQUIT

Sent to indicate when a user wishes to terminate a process and perform a
core dump.

SIGILL

Sent to a process to notify that it has attempted to perform an illegal,
malformed, unknown, or privileged instruction.

SIGTRAP

Sent to a process when an exception has occurred.

SIGABRT

Sent to a process to request that it abort.

SIGIOT

Synonym for SIGABRT

SIGBUS

Sent to a process to notify that it has caused a bus error.

SIGFPE

Sent to a process to notify that it has performed an illegal arithmetic
operation.

SIGKILL

Sent to a process to terminate it immediately.

SIGUSR1 SIGUSR2

Sent to a process to identify user-defined conditions.

SIGSEGV

Sent to a process to notify of a segmentation fault.

SIGPIPE

Sent to a process when it has attempted to write to a disconnected pipe.

SIGALRM

Sent to a process when a system timer elapses.

SIGTERM

Sent to a process to request termination.

SIGCHLD

Sent to a process when a child process terminates.

SIGSTKFLT

Sent to a process to indicate a stack fault on a coprocessor.

SIGCONT

Sent to instruct the operating system to continue a paused process.

SIGSTOP

Sent to instruct the operating system to halt a process.

SIGTSTP

Sent to a process to request it to stop.

SIGBREAK

Sent to indicate when a user wishes to interrupt a process.

SIGTTIN

Sent to a process when it reads from the TTY while in the background.

SIGTTOU

Sent to a process when it writes to the TTY while in the background.

SIGURG

Sent to a process when a socket has urgent data to read.

SIGXCPU

Sent to a process when it has exceeded its limit on CPU usage.

SIGXFSZ

Sent to a process when it grows a file larger than the maximum allowed.

SIGVTALRM

Sent to a process when a virtual timer has elapsed.

SIGPROF

Sent to a process when a system timer has elapsed.

SIGWINCH

Sent to a process when the controlling terminal has changed its size.

SIGIO

Sent to a process when I/O is available.

SIGPOLL

Synonym for SIGIO

SIGLOST

Sent to a process when a file lock has been lost.

SIGPWR

Sent to a process to notify of a power failure.

SIGINFO

Synonym for SIGPWR

SIGSYS

Sent to a process to notify of a bad argument.

SIGUNUSED

Synonym for SIGSYS

\subsubsection{Error constants}\label{error-constants}

The following error constants are exported by
\texttt{os.constants.errno}.

\paragraph{POSIX error constants}\label{posix-error-constants}

Constant

Description

E2BIG

Indicates that the list of arguments is longer than expected.

EACCES

Indicates that the operation did not have sufficient permissions.

EADDRINUSE

Indicates that the network address is already in use.

EADDRNOTAVAIL

Indicates that the network address is currently unavailable for use.

EAFNOSUPPORT

Indicates that the network address family is not supported.

EAGAIN

Indicates that there is no data available and to try the operation again
later.

EALREADY

Indicates that the socket already has a pending connection in progress.

EBADF

Indicates that a file descriptor is not valid.

EBADMSG

Indicates an invalid data message.

EBUSY

Indicates that a device or resource is busy.

ECANCELED

Indicates that an operation was canceled.

ECHILD

Indicates that there are no child processes.

ECONNABORTED

Indicates that the network connection has been aborted.

ECONNREFUSED

Indicates that the network connection has been refused.

ECONNRESET

Indicates that the network connection has been reset.

EDEADLK

Indicates that a resource deadlock has been avoided.

EDESTADDRREQ

Indicates that a destination address is required.

EDOM

Indicates that an argument is out of the domain of the function.

EDQUOT

Indicates that the disk quota has been exceeded.

EEXIST

Indicates that the file already exists.

EFAULT

Indicates an invalid pointer address.

EFBIG

Indicates that the file is too large.

EHOSTUNREACH

Indicates that the host is unreachable.

EIDRM

Indicates that the identifier has been removed.

EILSEQ

Indicates an illegal byte sequence.

EINPROGRESS

Indicates that an operation is already in progress.

EINTR

Indicates that a function call was interrupted.

EINVAL

Indicates that an invalid argument was provided.

EIO

Indicates an otherwise unspecified I/O error.

EISCONN

Indicates that the socket is connected.

EISDIR

Indicates that the path is a directory.

ELOOP

Indicates too many levels of symbolic links in a path.

EMFILE

Indicates that there are too many open files.

EMLINK

Indicates that there are too many hard links to a file.

EMSGSIZE

Indicates that the provided message is too long.

EMULTIHOP

Indicates that a multihop was attempted.

ENAMETOOLONG

Indicates that the filename is too long.

ENETDOWN

Indicates that the network is down.

ENETRESET

Indicates that the connection has been aborted by the network.

ENETUNREACH

Indicates that the network is unreachable.

ENFILE

Indicates too many open files in the system.

ENOBUFS

Indicates that no buffer space is available.

ENODATA

Indicates that no message is available on the stream head read queue.

ENODEV

Indicates that there is no such device.

ENOENT

Indicates that there is no such file or directory.

ENOEXEC

Indicates an exec format error.

ENOLCK

Indicates that there are no locks available.

ENOLINK

Indications that a link has been severed.

ENOMEM

Indicates that there is not enough space.

ENOMSG

Indicates that there is no message of the desired type.

ENOPROTOOPT

Indicates that a given protocol is not available.

ENOSPC

Indicates that there is no space available on the device.

ENOSR

Indicates that there are no stream resources available.

ENOSTR

Indicates that a given resource is not a stream.

ENOSYS

Indicates that a function has not been implemented.

ENOTCONN

Indicates that the socket is not connected.

ENOTDIR

Indicates that the path is not a directory.

ENOTEMPTY

Indicates that the directory is not empty.

ENOTSOCK

Indicates that the given item is not a socket.

ENOTSUP

Indicates that a given operation is not supported.

ENOTTY

Indicates an inappropriate I/O control operation.

ENXIO

Indicates no such device or address.

EOPNOTSUPP

Indicates that an operation is not supported on the socket. Although
ENOTSUP and EOPNOTSUPP have the same value on Linux, according to
POSIX.1 these error values should be distinct.)

EOVERFLOW

Indicates that a value is too large to be stored in a given data type.

EPERM

Indicates that the operation is not permitted.

EPIPE

Indicates a broken pipe.

EPROTO

Indicates a protocol error.

EPROTONOSUPPORT

Indicates that a protocol is not supported.

EPROTOTYPE

Indicates the wrong type of protocol for a socket.

ERANGE

Indicates that the results are too large.

EROFS

Indicates that the file system is read only.

ESPIPE

Indicates an invalid seek operation.

ESRCH

Indicates that there is no such process.

ESTALE

Indicates that the file handle is stale.

ETIME

Indicates an expired timer.

ETIMEDOUT

Indicates that the connection timed out.

ETXTBSY

Indicates that a text file is busy.

EWOULDBLOCK

Indicates that the operation would block.

EXDEV

Indicates an improper link.

\paragraph{Windows-specific error
constants}\label{windows-specific-error-constants}

The following error codes are specific to the Windows operating system.

Constant

Description

WSAEINTR

Indicates an interrupted function call.

WSAEBADF

Indicates an invalid file handle.

WSAEACCES

Indicates insufficient permissions to complete the operation.

WSAEFAULT

Indicates an invalid pointer address.

WSAEINVAL

Indicates that an invalid argument was passed.

WSAEMFILE

Indicates that there are too many open files.

WSAEWOULDBLOCK

Indicates that a resource is temporarily unavailable.

WSAEINPROGRESS

Indicates that an operation is currently in progress.

WSAEALREADY

Indicates that an operation is already in progress.

WSAENOTSOCK

Indicates that the resource is not a socket.

WSAEDESTADDRREQ

Indicates that a destination address is required.

WSAEMSGSIZE

Indicates that the message size is too long.

WSAEPROTOTYPE

Indicates the wrong protocol type for the socket.

WSAENOPROTOOPT

Indicates a bad protocol option.

WSAEPROTONOSUPPORT

Indicates that the protocol is not supported.

WSAESOCKTNOSUPPORT

Indicates that the socket type is not supported.

WSAEOPNOTSUPP

Indicates that the operation is not supported.

WSAEPFNOSUPPORT

Indicates that the protocol family is not supported.

WSAEAFNOSUPPORT

Indicates that the address family is not supported.

WSAEADDRINUSE

Indicates that the network address is already in use.

WSAEADDRNOTAVAIL

Indicates that the network address is not available.

WSAENETDOWN

Indicates that the network is down.

WSAENETUNREACH

Indicates that the network is unreachable.

WSAENETRESET

Indicates that the network connection has been reset.

WSAECONNABORTED

Indicates that the connection has been aborted.

WSAECONNRESET

Indicates that the connection has been reset by the peer.

WSAENOBUFS

Indicates that there is no buffer space available.

WSAEISCONN

Indicates that the socket is already connected.

WSAENOTCONN

Indicates that the socket is not connected.

WSAESHUTDOWN

Indicates that data cannot be sent after the socket has been shutdown.

WSAETOOMANYREFS

Indicates that there are too many references.

WSAETIMEDOUT

Indicates that the connection has timed out.

WSAECONNREFUSED

Indicates that the connection has been refused.

WSAELOOP

Indicates that a name cannot be translated.

WSAENAMETOOLONG

Indicates that a name was too long.

WSAEHOSTDOWN

Indicates that a network host is down.

WSAEHOSTUNREACH

Indicates that there is no route to a network host.

WSAENOTEMPTY

Indicates that the directory is not empty.

WSAEPROCLIM

Indicates that there are too many processes.

WSAEUSERS

Indicates that the user quota has been exceeded.

WSAEDQUOT

Indicates that the disk quota has been exceeded.

WSAESTALE

Indicates a stale file handle reference.

WSAEREMOTE

Indicates that the item is remote.

WSASYSNOTREADY

Indicates that the network subsystem is not ready.

WSAVERNOTSUPPORTED

Indicates that the winsock.dll version is out of range.

WSANOTINITIALISED

Indicates that successful WSAStartup has not yet been performed.

WSAEDISCON

Indicates that a graceful shutdown is in progress.

WSAENOMORE

Indicates that there are no more results.

WSAECANCELLED

Indicates that an operation has been canceled.

WSAEINVALIDPROCTABLE

Indicates that the procedure call table is invalid.

WSAEINVALIDPROVIDER

Indicates an invalid service provider.

WSAEPROVIDERFAILEDINIT

Indicates that the service provider failed to initialized.

WSASYSCALLFAILURE

Indicates a system call failure.

WSASERVICE\_NOT\_FOUND

Indicates that a service was not found.

WSATYPE\_NOT\_FOUND

Indicates that a class type was not found.

WSA\_E\_NO\_MORE

Indicates that there are no more results.

WSA\_E\_CANCELLED

Indicates that the call was canceled.

WSAEREFUSED

Indicates that a database query was refused.

\subsubsection{dlopen constants}\label{dlopen-constants}

If available on the operating system, the following constants are
exported in \texttt{os.constants.dlopen}. See dlopen(3) for detailed
information.

Constant

Description

RTLD\_LAZY

Perform lazy binding. Node.js sets this flag by default.

RTLD\_NOW

Resolve all undefined symbols in the library before dlopen(3) returns.

RTLD\_GLOBAL

Symbols defined by the library will be made available for symbol
resolution of subsequently loaded libraries.

RTLD\_LOCAL

The converse of RTLD\_GLOBAL. This is the default behavior if neither
flag is specified.

RTLD\_DEEPBIND

Make a self-contained library use its own symbols in preference to
symbols from previously loaded libraries.

\subsubsection{Priority constants}\label{priority-constants}

The following process scheduling constants are exported by
\texttt{os.constants.priority}.

Constant

Description

PRIORITY\_LOW

The lowest process scheduling priority. This corresponds to
IDLE\_PRIORITY\_CLASS on Windows, and a nice value of 19 on all other
platforms.

PRIORITY\_BELOW\_NORMAL

The process scheduling priority above PRIORITY\_LOW and below
PRIORITY\_NORMAL. This corresponds to BELOW\_NORMAL\_PRIORITY\_CLASS on
Windows, and a nice value of 10 on all other platforms.

PRIORITY\_NORMAL

The default process scheduling priority. This corresponds to
NORMAL\_PRIORITY\_CLASS on Windows, and a nice value of 0 on all other
platforms.

PRIORITY\_ABOVE\_NORMAL

The process scheduling priority above PRIORITY\_NORMAL and below
PRIORITY\_HIGH. This corresponds to ABOVE\_NORMAL\_PRIORITY\_CLASS on
Windows, and a nice value of -7 on all other platforms.

PRIORITY\_HIGH

The process scheduling priority above PRIORITY\_ABOVE\_NORMAL and below
PRIORITY\_HIGHEST. This corresponds to HIGH\_PRIORITY\_CLASS on Windows,
and a nice value of -14 on all other platforms.

PRIORITY\_HIGHEST

The highest process scheduling priority. This corresponds to
REALTIME\_PRIORITY\_CLASS on Windows, and a nice value of -20 on all
other platforms.

\subsubsection{libuv constants}\label{libuv-constants}

Constant

Description

UV\_UDP\_REUSEADDR
