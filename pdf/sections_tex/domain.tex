\section{Domain}\label{domain}

\begin{quote}
Stability: 0 - Deprecated
\end{quote}

\textbf{This module is pending deprecation.} Once a replacement API has
been finalized, this module will be fully deprecated. Most developers
should \textbf{not} have cause to use this module. Users who absolutely
must have the functionality that domains provide may rely on it for the
time being but should expect to have to migrate to a different solution
in the future.

Domains provide a way to handle multiple different IO operations as a
single group. If any of the event emitters or callbacks registered to a
domain emit an \texttt{\textquotesingle{}error\textquotesingle{}} event,
or throw an error, then the domain object will be notified, rather than
losing the context of the error in the
\texttt{process.on(\textquotesingle{}uncaughtException\textquotesingle{})}
handler, or causing the program to exit immediately with an error code.

\subsection{Warning: Don't ignore
errors!}\label{warning-dont-ignore-errors}

Domain error handlers are not a substitute for closing down a process
when an error occurs.

By the very nature of how
\href{https://developer.mozilla.org/en-US/docs/Web/JavaScript/Reference/Statements/throw}{\texttt{throw}}
works in JavaScript, there is almost never any way to safely ``pick up
where it left off'', without leaking references, or creating some other
sort of undefined brittle state.

The safest way to respond to a thrown error is to shut down the process.
Of course, in a normal web server, there may be many open connections,
and it is not reasonable to abruptly shut those down because an error
was triggered by someone else.

The better approach is to send an error response to the request that
triggered the error, while letting the others finish in their normal
time, and stop listening for new requests in that worker.

In this way, \texttt{domain} usage goes hand-in-hand with the cluster
module, since the primary process can fork a new worker when a worker
encounters an error. For Node.js programs that scale to multiple
machines, the terminating proxy or service registry can take note of the
failure, and react accordingly.

For example, this is not a good idea:

\begin{Shaded}
\begin{Highlighting}[]
\CommentTok{// XXX }\AlertTok{WARNING}\CommentTok{! BAD IDEA!}

\KeywordTok{const}\NormalTok{ d }\OperatorTok{=} \PreprocessorTok{require}\NormalTok{(}\StringTok{\textquotesingle{}node:domain\textquotesingle{}}\NormalTok{)}\OperatorTok{.}\FunctionTok{create}\NormalTok{()}\OperatorTok{;}
\NormalTok{d}\OperatorTok{.}\FunctionTok{on}\NormalTok{(}\StringTok{\textquotesingle{}error\textquotesingle{}}\OperatorTok{,}\NormalTok{ (er) }\KeywordTok{=\textgreater{}}\NormalTok{ \{}
  \CommentTok{// The error won\textquotesingle{}t crash the process, but what it does is worse!}
  \CommentTok{// Though we\textquotesingle{}ve prevented abrupt process restarting, we are leaking}
  \CommentTok{// a lot of resources if this ever happens.}
  \CommentTok{// This is no better than process.on(\textquotesingle{}uncaughtException\textquotesingle{})!}
  \BuiltInTok{console}\OperatorTok{.}\FunctionTok{log}\NormalTok{(}\VerbatimStringTok{\textasciigrave{}error, but oh well }\SpecialCharTok{$\{}\NormalTok{er}\OperatorTok{.}\AttributeTok{message}\SpecialCharTok{\}}\VerbatimStringTok{\textasciigrave{}}\NormalTok{)}\OperatorTok{;}
\NormalTok{\})}\OperatorTok{;}
\NormalTok{d}\OperatorTok{.}\FunctionTok{run}\NormalTok{(() }\KeywordTok{=\textgreater{}}\NormalTok{ \{}
  \PreprocessorTok{require}\NormalTok{(}\StringTok{\textquotesingle{}node:http\textquotesingle{}}\NormalTok{)}\OperatorTok{.}\FunctionTok{createServer}\NormalTok{((req}\OperatorTok{,}\NormalTok{ res) }\KeywordTok{=\textgreater{}}\NormalTok{ \{}
    \FunctionTok{handleRequest}\NormalTok{(req}\OperatorTok{,}\NormalTok{ res)}\OperatorTok{;}
\NormalTok{  \})}\OperatorTok{.}\FunctionTok{listen}\NormalTok{(PORT)}\OperatorTok{;}
\NormalTok{\})}\OperatorTok{;}
\end{Highlighting}
\end{Shaded}

By using the context of a domain, and the resilience of separating our
program into multiple worker processes, we can react more appropriately,
and handle errors with much greater safety.

\begin{Shaded}
\begin{Highlighting}[]
\CommentTok{// Much better!}

\KeywordTok{const}\NormalTok{ cluster }\OperatorTok{=} \PreprocessorTok{require}\NormalTok{(}\StringTok{\textquotesingle{}node:cluster\textquotesingle{}}\NormalTok{)}\OperatorTok{;}
\KeywordTok{const}\NormalTok{ PORT }\OperatorTok{=} \OperatorTok{+}\BuiltInTok{process}\OperatorTok{.}\AttributeTok{env}\OperatorTok{.}\AttributeTok{PORT} \OperatorTok{||} \DecValTok{1337}\OperatorTok{;}

\ControlFlowTok{if}\NormalTok{ (cluster}\OperatorTok{.}\AttributeTok{isPrimary}\NormalTok{) \{}
  \CommentTok{// A more realistic scenario would have more than 2 workers,}
  \CommentTok{// and perhaps not put the primary and worker in the same file.}
  \CommentTok{//}
  \CommentTok{// It is also possible to get a bit fancier about logging, and}
  \CommentTok{// implement whatever custom logic is needed to prevent DoS}
  \CommentTok{// attacks and other bad behavior.}
  \CommentTok{//}
  \CommentTok{// See the options in the cluster documentation.}
  \CommentTok{//}
  \CommentTok{// The important thing is that the primary does very little,}
  \CommentTok{// increasing our resilience to unexpected errors.}

\NormalTok{  cluster}\OperatorTok{.}\FunctionTok{fork}\NormalTok{()}\OperatorTok{;}
\NormalTok{  cluster}\OperatorTok{.}\FunctionTok{fork}\NormalTok{()}\OperatorTok{;}

\NormalTok{  cluster}\OperatorTok{.}\FunctionTok{on}\NormalTok{(}\StringTok{\textquotesingle{}disconnect\textquotesingle{}}\OperatorTok{,}\NormalTok{ (worker) }\KeywordTok{=\textgreater{}}\NormalTok{ \{}
    \BuiltInTok{console}\OperatorTok{.}\FunctionTok{error}\NormalTok{(}\StringTok{\textquotesingle{}disconnect!\textquotesingle{}}\NormalTok{)}\OperatorTok{;}
\NormalTok{    cluster}\OperatorTok{.}\FunctionTok{fork}\NormalTok{()}\OperatorTok{;}
\NormalTok{  \})}\OperatorTok{;}

\NormalTok{\} }\ControlFlowTok{else}\NormalTok{ \{}
  \CommentTok{// the worker}
  \CommentTok{//}
  \CommentTok{// This is where we put our bugs!}

  \KeywordTok{const}\NormalTok{ domain }\OperatorTok{=} \PreprocessorTok{require}\NormalTok{(}\StringTok{\textquotesingle{}node:domain\textquotesingle{}}\NormalTok{)}\OperatorTok{;}

  \CommentTok{// See the cluster documentation for more details about using}
  \CommentTok{// worker processes to serve requests. How it works, caveats, etc.}

  \KeywordTok{const}\NormalTok{ server }\OperatorTok{=} \PreprocessorTok{require}\NormalTok{(}\StringTok{\textquotesingle{}node:http\textquotesingle{}}\NormalTok{)}\OperatorTok{.}\FunctionTok{createServer}\NormalTok{((req}\OperatorTok{,}\NormalTok{ res) }\KeywordTok{=\textgreater{}}\NormalTok{ \{}
    \KeywordTok{const}\NormalTok{ d }\OperatorTok{=}\NormalTok{ domain}\OperatorTok{.}\FunctionTok{create}\NormalTok{()}\OperatorTok{;}
\NormalTok{    d}\OperatorTok{.}\FunctionTok{on}\NormalTok{(}\StringTok{\textquotesingle{}error\textquotesingle{}}\OperatorTok{,}\NormalTok{ (er) }\KeywordTok{=\textgreater{}}\NormalTok{ \{}
      \BuiltInTok{console}\OperatorTok{.}\FunctionTok{error}\NormalTok{(}\VerbatimStringTok{\textasciigrave{}error }\SpecialCharTok{$\{}\NormalTok{er}\OperatorTok{.}\AttributeTok{stack}\SpecialCharTok{\}}\VerbatimStringTok{\textasciigrave{}}\NormalTok{)}\OperatorTok{;}

      \CommentTok{// We\textquotesingle{}re in dangerous territory!}
      \CommentTok{// By definition, something unexpected occurred,}
      \CommentTok{// which we probably didn\textquotesingle{}t want.}
      \CommentTok{// Anything can happen now! Be very careful!}

      \ControlFlowTok{try}\NormalTok{ \{}
        \CommentTok{// Make sure we close down within 30 seconds}
        \KeywordTok{const}\NormalTok{ killtimer }\OperatorTok{=} \PreprocessorTok{setTimeout}\NormalTok{(() }\KeywordTok{=\textgreater{}}\NormalTok{ \{}
          \BuiltInTok{process}\OperatorTok{.}\FunctionTok{exit}\NormalTok{(}\DecValTok{1}\NormalTok{)}\OperatorTok{;}
\NormalTok{        \}}\OperatorTok{,} \DecValTok{30000}\NormalTok{)}\OperatorTok{;}
        \CommentTok{// But don\textquotesingle{}t keep the process open just for that!}
\NormalTok{        killtimer}\OperatorTok{.}\FunctionTok{unref}\NormalTok{()}\OperatorTok{;}

        \CommentTok{// Stop taking new requests.}
\NormalTok{        server}\OperatorTok{.}\FunctionTok{close}\NormalTok{()}\OperatorTok{;}

        \CommentTok{// Let the primary know we\textquotesingle{}re dead. This will trigger a}
        \CommentTok{// \textquotesingle{}disconnect\textquotesingle{} in the cluster primary, and then it will fork}
        \CommentTok{// a new worker.}
\NormalTok{        cluster}\OperatorTok{.}\AttributeTok{worker}\OperatorTok{.}\FunctionTok{disconnect}\NormalTok{()}\OperatorTok{;}

        \CommentTok{// Try to send an error to the request that triggered the problem}
\NormalTok{        res}\OperatorTok{.}\AttributeTok{statusCode} \OperatorTok{=} \DecValTok{500}\OperatorTok{;}
\NormalTok{        res}\OperatorTok{.}\FunctionTok{setHeader}\NormalTok{(}\StringTok{\textquotesingle{}content{-}type\textquotesingle{}}\OperatorTok{,} \StringTok{\textquotesingle{}text/plain\textquotesingle{}}\NormalTok{)}\OperatorTok{;}
\NormalTok{        res}\OperatorTok{.}\FunctionTok{end}\NormalTok{(}\StringTok{\textquotesingle{}Oops, there was a problem!}\SpecialCharTok{\textbackslash{}n}\StringTok{\textquotesingle{}}\NormalTok{)}\OperatorTok{;}
\NormalTok{      \} }\ControlFlowTok{catch}\NormalTok{ (er2) \{}
        \CommentTok{// Oh well, not much we can do at this point.}
        \BuiltInTok{console}\OperatorTok{.}\FunctionTok{error}\NormalTok{(}\VerbatimStringTok{\textasciigrave{}Error sending 500! }\SpecialCharTok{$\{}\NormalTok{er2}\OperatorTok{.}\AttributeTok{stack}\SpecialCharTok{\}}\VerbatimStringTok{\textasciigrave{}}\NormalTok{)}\OperatorTok{;}
\NormalTok{      \}}
\NormalTok{    \})}\OperatorTok{;}

    \CommentTok{// Because req and res were created before this domain existed,}
    \CommentTok{// we need to explicitly add them.}
    \CommentTok{// See the explanation of implicit vs explicit binding below.}
\NormalTok{    d}\OperatorTok{.}\FunctionTok{add}\NormalTok{(req)}\OperatorTok{;}
\NormalTok{    d}\OperatorTok{.}\FunctionTok{add}\NormalTok{(res)}\OperatorTok{;}

    \CommentTok{// Now run the handler function in the domain.}
\NormalTok{    d}\OperatorTok{.}\FunctionTok{run}\NormalTok{(() }\KeywordTok{=\textgreater{}}\NormalTok{ \{}
      \FunctionTok{handleRequest}\NormalTok{(req}\OperatorTok{,}\NormalTok{ res)}\OperatorTok{;}
\NormalTok{    \})}\OperatorTok{;}
\NormalTok{  \})}\OperatorTok{;}
\NormalTok{  server}\OperatorTok{.}\FunctionTok{listen}\NormalTok{(PORT)}\OperatorTok{;}
\NormalTok{\}}

\CommentTok{// This part is not important. Just an example routing thing.}
\CommentTok{// Put fancy application logic here.}
\KeywordTok{function} \FunctionTok{handleRequest}\NormalTok{(req}\OperatorTok{,}\NormalTok{ res) \{}
  \ControlFlowTok{switch}\NormalTok{ (req}\OperatorTok{.}\AttributeTok{url}\NormalTok{) \{}
    \ControlFlowTok{case} \StringTok{\textquotesingle{}/error\textquotesingle{}}\OperatorTok{:}
      \CommentTok{// We do some async stuff, and then...}
      \PreprocessorTok{setTimeout}\NormalTok{(() }\KeywordTok{=\textgreater{}}\NormalTok{ \{}
        \CommentTok{// Whoops!}
\NormalTok{        flerb}\OperatorTok{.}\FunctionTok{bark}\NormalTok{()}\OperatorTok{;}
\NormalTok{      \}}\OperatorTok{,}\NormalTok{ timeout)}\OperatorTok{;}
      \ControlFlowTok{break}\OperatorTok{;}
    \ControlFlowTok{default}\OperatorTok{:}
\NormalTok{      res}\OperatorTok{.}\FunctionTok{end}\NormalTok{(}\StringTok{\textquotesingle{}ok\textquotesingle{}}\NormalTok{)}\OperatorTok{;}
\NormalTok{  \}}
\NormalTok{\}}
\end{Highlighting}
\end{Shaded}

\subsection{\texorpdfstring{Additions to \texttt{Error}
objects}{Additions to Error objects}}\label{additions-to-error-objects}

Any time an \texttt{Error} object is routed through a domain, a few
extra fields are added to it.

\begin{itemize}
\tightlist
\item
  \texttt{error.domain} The domain that first handled the error.
\item
  \texttt{error.domainEmitter} The event emitter that emitted an
  \texttt{\textquotesingle{}error\textquotesingle{}} event with the
  error object.
\item
  \texttt{error.domainBound} The callback function which was bound to
  the domain, and passed an error as its first argument.
\item
  \texttt{error.domainThrown} A boolean indicating whether the error was
  thrown, emitted, or passed to a bound callback function.
\end{itemize}

\subsection{Implicit binding}\label{implicit-binding}

If domains are in use, then all \textbf{new} \texttt{EventEmitter}
objects (including Stream objects, requests, responses, etc.) will be
implicitly bound to the active domain at the time of their creation.

Additionally, callbacks passed to low-level event loop requests (such as
to \texttt{fs.open()}, or other callback-taking methods) will
automatically be bound to the active domain. If they throw, then the
domain will catch the error.

In order to prevent excessive memory usage, \texttt{Domain} objects
themselves are not implicitly added as children of the active domain. If
they were, then it would be too easy to prevent request and response
objects from being properly garbage collected.

To nest \texttt{Domain} objects as children of a parent \texttt{Domain}
they must be explicitly added.

Implicit binding routes thrown errors and
\texttt{\textquotesingle{}error\textquotesingle{}} events to the
\texttt{Domain}'s \texttt{\textquotesingle{}error\textquotesingle{}}
event, but does not register the \texttt{EventEmitter} on the
\texttt{Domain}. Implicit binding only takes care of thrown errors and
\texttt{\textquotesingle{}error\textquotesingle{}} events.

\subsection{Explicit binding}\label{explicit-binding}

Sometimes, the domain in use is not the one that ought to be used for a
specific event emitter. Or, the event emitter could have been created in
the context of one domain, but ought to instead be bound to some other
domain.

For example, there could be one domain in use for an HTTP server, but
perhaps we would like to have a separate domain to use for each request.

That is possible via explicit binding.

\begin{Shaded}
\begin{Highlighting}[]
\CommentTok{// Create a top{-}level domain for the server}
\KeywordTok{const}\NormalTok{ domain }\OperatorTok{=} \PreprocessorTok{require}\NormalTok{(}\StringTok{\textquotesingle{}node:domain\textquotesingle{}}\NormalTok{)}\OperatorTok{;}
\KeywordTok{const}\NormalTok{ http }\OperatorTok{=} \PreprocessorTok{require}\NormalTok{(}\StringTok{\textquotesingle{}node:http\textquotesingle{}}\NormalTok{)}\OperatorTok{;}
\KeywordTok{const}\NormalTok{ serverDomain }\OperatorTok{=}\NormalTok{ domain}\OperatorTok{.}\FunctionTok{create}\NormalTok{()}\OperatorTok{;}

\NormalTok{serverDomain}\OperatorTok{.}\FunctionTok{run}\NormalTok{(() }\KeywordTok{=\textgreater{}}\NormalTok{ \{}
  \CommentTok{// Server is created in the scope of serverDomain}
\NormalTok{  http}\OperatorTok{.}\FunctionTok{createServer}\NormalTok{((req}\OperatorTok{,}\NormalTok{ res) }\KeywordTok{=\textgreater{}}\NormalTok{ \{}
    \CommentTok{// Req and res are also created in the scope of serverDomain}
    \CommentTok{// however, we\textquotesingle{}d prefer to have a separate domain for each request.}
    \CommentTok{// create it first thing, and add req and res to it.}
    \KeywordTok{const}\NormalTok{ reqd }\OperatorTok{=}\NormalTok{ domain}\OperatorTok{.}\FunctionTok{create}\NormalTok{()}\OperatorTok{;}
\NormalTok{    reqd}\OperatorTok{.}\FunctionTok{add}\NormalTok{(req)}\OperatorTok{;}
\NormalTok{    reqd}\OperatorTok{.}\FunctionTok{add}\NormalTok{(res)}\OperatorTok{;}
\NormalTok{    reqd}\OperatorTok{.}\FunctionTok{on}\NormalTok{(}\StringTok{\textquotesingle{}error\textquotesingle{}}\OperatorTok{,}\NormalTok{ (er) }\KeywordTok{=\textgreater{}}\NormalTok{ \{}
      \BuiltInTok{console}\OperatorTok{.}\FunctionTok{error}\NormalTok{(}\StringTok{\textquotesingle{}Error\textquotesingle{}}\OperatorTok{,}\NormalTok{ er}\OperatorTok{,}\NormalTok{ req}\OperatorTok{.}\AttributeTok{url}\NormalTok{)}\OperatorTok{;}
      \ControlFlowTok{try}\NormalTok{ \{}
\NormalTok{        res}\OperatorTok{.}\FunctionTok{writeHead}\NormalTok{(}\DecValTok{500}\NormalTok{)}\OperatorTok{;}
\NormalTok{        res}\OperatorTok{.}\FunctionTok{end}\NormalTok{(}\StringTok{\textquotesingle{}Error occurred, sorry.\textquotesingle{}}\NormalTok{)}\OperatorTok{;}
\NormalTok{      \} }\ControlFlowTok{catch}\NormalTok{ (er2) \{}
        \BuiltInTok{console}\OperatorTok{.}\FunctionTok{error}\NormalTok{(}\StringTok{\textquotesingle{}Error sending 500\textquotesingle{}}\OperatorTok{,}\NormalTok{ er2}\OperatorTok{,}\NormalTok{ req}\OperatorTok{.}\AttributeTok{url}\NormalTok{)}\OperatorTok{;}
\NormalTok{      \}}
\NormalTok{    \})}\OperatorTok{;}
\NormalTok{  \})}\OperatorTok{.}\FunctionTok{listen}\NormalTok{(}\DecValTok{1337}\NormalTok{)}\OperatorTok{;}
\NormalTok{\})}\OperatorTok{;}
\end{Highlighting}
\end{Shaded}

\subsection{\texorpdfstring{\texttt{domain.create()}}{domain.create()}}\label{domain.create}

\begin{itemize}
\tightlist
\item
  Returns: \{Domain\}
\end{itemize}

\subsection{\texorpdfstring{Class:
\texttt{Domain}}{Class: Domain}}\label{class-domain}

\begin{itemize}
\tightlist
\item
  Extends: \{EventEmitter\}
\end{itemize}

The \texttt{Domain} class encapsulates the functionality of routing
errors and uncaught exceptions to the active \texttt{Domain} object.

To handle the errors that it catches, listen to its
\texttt{\textquotesingle{}error\textquotesingle{}} event.

\subsubsection{\texorpdfstring{\texttt{domain.members}}{domain.members}}\label{domain.members}

\begin{itemize}
\tightlist
\item
  \{Array\}
\end{itemize}

An array of timers and event emitters that have been explicitly added to
the domain.

\subsubsection{\texorpdfstring{\texttt{domain.add(emitter)}}{domain.add(emitter)}}\label{domain.addemitter}

\begin{itemize}
\tightlist
\item
  \texttt{emitter} \{EventEmitter\textbar Timer\} emitter or timer to be
  added to the domain
\end{itemize}

Explicitly adds an emitter to the domain. If any event handlers called
by the emitter throw an error, or if the emitter emits an
\texttt{\textquotesingle{}error\textquotesingle{}} event, it will be
routed to the domain's
\texttt{\textquotesingle{}error\textquotesingle{}} event, just like with
implicit binding.

This also works with timers that are returned from
\href{timers.md\#setintervalcallback-delay-args}{\texttt{setInterval()}}
and
\href{timers.md\#settimeoutcallback-delay-args}{\texttt{setTimeout()}}.
If their callback function throws, it will be caught by the domain
\texttt{\textquotesingle{}error\textquotesingle{}} handler.

If the Timer or \texttt{EventEmitter} was already bound to a domain, it
is removed from that one, and bound to this one instead.

\subsubsection{\texorpdfstring{\texttt{domain.bind(callback)}}{domain.bind(callback)}}\label{domain.bindcallback}

\begin{itemize}
\tightlist
\item
  \texttt{callback} \{Function\} The callback function
\item
  Returns: \{Function\} The bound function
\end{itemize}

The returned function will be a wrapper around the supplied callback
function. When the returned function is called, any errors that are
thrown will be routed to the domain's
\texttt{\textquotesingle{}error\textquotesingle{}} event.

\begin{Shaded}
\begin{Highlighting}[]
\KeywordTok{const}\NormalTok{ d }\OperatorTok{=}\NormalTok{ domain}\OperatorTok{.}\FunctionTok{create}\NormalTok{()}\OperatorTok{;}

\KeywordTok{function} \FunctionTok{readSomeFile}\NormalTok{(filename}\OperatorTok{,}\NormalTok{ cb) \{}
\NormalTok{  fs}\OperatorTok{.}\FunctionTok{readFile}\NormalTok{(filename}\OperatorTok{,} \StringTok{\textquotesingle{}utf8\textquotesingle{}}\OperatorTok{,}\NormalTok{ d}\OperatorTok{.}\FunctionTok{bind}\NormalTok{((er}\OperatorTok{,}\NormalTok{ data) }\KeywordTok{=\textgreater{}}\NormalTok{ \{}
    \CommentTok{// If this throws, it will also be passed to the domain.}
    \ControlFlowTok{return} \FunctionTok{cb}\NormalTok{(er}\OperatorTok{,}\NormalTok{ data }\OperatorTok{?} \BuiltInTok{JSON}\OperatorTok{.}\FunctionTok{parse}\NormalTok{(data) }\OperatorTok{:} \KeywordTok{null}\NormalTok{)}\OperatorTok{;}
\NormalTok{  \}))}\OperatorTok{;}
\NormalTok{\}}

\NormalTok{d}\OperatorTok{.}\FunctionTok{on}\NormalTok{(}\StringTok{\textquotesingle{}error\textquotesingle{}}\OperatorTok{,}\NormalTok{ (er) }\KeywordTok{=\textgreater{}}\NormalTok{ \{}
  \CommentTok{// An error occurred somewhere. If we throw it now, it will crash the program}
  \CommentTok{// with the normal line number and stack message.}
\NormalTok{\})}\OperatorTok{;}
\end{Highlighting}
\end{Shaded}

\subsubsection{\texorpdfstring{\texttt{domain.enter()}}{domain.enter()}}\label{domain.enter}

The \texttt{enter()} method is plumbing used by the \texttt{run()},
\texttt{bind()}, and \texttt{intercept()} methods to set the active
domain. It sets \texttt{domain.active} and \texttt{process.domain} to
the domain, and implicitly pushes the domain onto the domain stack
managed by the domain module (see
\hyperref[domainexit]{\texttt{domain.exit()}} for details on the domain
stack). The call to \texttt{enter()} delimits the beginning of a chain
of asynchronous calls and I/O operations bound to a domain.

Calling \texttt{enter()} changes only the active domain, and does not
alter the domain itself. \texttt{enter()} and \texttt{exit()} can be
called an arbitrary number of times on a single domain.

\subsubsection{\texorpdfstring{\texttt{domain.exit()}}{domain.exit()}}\label{domain.exit}

The \texttt{exit()} method exits the current domain, popping it off the
domain stack. Any time execution is going to switch to the context of a
different chain of asynchronous calls, it's important to ensure that the
current domain is exited. The call to \texttt{exit()} delimits either
the end of or an interruption to the chain of asynchronous calls and I/O
operations bound to a domain.

If there are multiple, nested domains bound to the current execution
context, \texttt{exit()} will exit any domains nested within this
domain.

Calling \texttt{exit()} changes only the active domain, and does not
alter the domain itself. \texttt{enter()} and \texttt{exit()} can be
called an arbitrary number of times on a single domain.

\subsubsection{\texorpdfstring{\texttt{domain.intercept(callback)}}{domain.intercept(callback)}}\label{domain.interceptcallback}

\begin{itemize}
\tightlist
\item
  \texttt{callback} \{Function\} The callback function
\item
  Returns: \{Function\} The intercepted function
\end{itemize}

This method is almost identical to
\hyperref[domainbindcallback]{\texttt{domain.bind(callback)}}. However,
in addition to catching thrown errors, it will also intercept
\href{errors.md\#class-error}{\texttt{Error}} objects sent as the first
argument to the function.

In this way, the common \texttt{if\ (err)\ return\ callback(err);}
pattern can be replaced with a single error handler in a single place.

\begin{Shaded}
\begin{Highlighting}[]
\KeywordTok{const}\NormalTok{ d }\OperatorTok{=}\NormalTok{ domain}\OperatorTok{.}\FunctionTok{create}\NormalTok{()}\OperatorTok{;}

\KeywordTok{function} \FunctionTok{readSomeFile}\NormalTok{(filename}\OperatorTok{,}\NormalTok{ cb) \{}
\NormalTok{  fs}\OperatorTok{.}\FunctionTok{readFile}\NormalTok{(filename}\OperatorTok{,} \StringTok{\textquotesingle{}utf8\textquotesingle{}}\OperatorTok{,}\NormalTok{ d}\OperatorTok{.}\FunctionTok{intercept}\NormalTok{((data) }\KeywordTok{=\textgreater{}}\NormalTok{ \{}
    \CommentTok{// Note, the first argument is never passed to the}
    \CommentTok{// callback since it is assumed to be the \textquotesingle{}Error\textquotesingle{} argument}
    \CommentTok{// and thus intercepted by the domain.}

    \CommentTok{// If this throws, it will also be passed to the domain}
    \CommentTok{// so the error{-}handling logic can be moved to the \textquotesingle{}error\textquotesingle{}}
    \CommentTok{// event on the domain instead of being repeated throughout}
    \CommentTok{// the program.}
    \ControlFlowTok{return} \FunctionTok{cb}\NormalTok{(}\KeywordTok{null}\OperatorTok{,} \BuiltInTok{JSON}\OperatorTok{.}\FunctionTok{parse}\NormalTok{(data))}\OperatorTok{;}
\NormalTok{  \}))}\OperatorTok{;}
\NormalTok{\}}

\NormalTok{d}\OperatorTok{.}\FunctionTok{on}\NormalTok{(}\StringTok{\textquotesingle{}error\textquotesingle{}}\OperatorTok{,}\NormalTok{ (er) }\KeywordTok{=\textgreater{}}\NormalTok{ \{}
  \CommentTok{// An error occurred somewhere. If we throw it now, it will crash the program}
  \CommentTok{// with the normal line number and stack message.}
\NormalTok{\})}\OperatorTok{;}
\end{Highlighting}
\end{Shaded}

\subsubsection{\texorpdfstring{\texttt{domain.remove(emitter)}}{domain.remove(emitter)}}\label{domain.removeemitter}

\begin{itemize}
\tightlist
\item
  \texttt{emitter} \{EventEmitter\textbar Timer\} emitter or timer to be
  removed from the domain
\end{itemize}

The opposite of
\hyperref[domainaddemitter]{\texttt{domain.add(emitter)}}. Removes
domain handling from the specified emitter.

\subsubsection{\texorpdfstring{\texttt{domain.run(fn{[},\ ...args{]})}}{domain.run(fn{[}, ...args{]})}}\label{domain.runfn-...args}

\begin{itemize}
\tightlist
\item
  \texttt{fn} \{Function\}
\item
  \texttt{...args} \{any\}
\end{itemize}

Run the supplied function in the context of the domain, implicitly
binding all event emitters, timers, and low-level requests that are
created in that context. Optionally, arguments can be passed to the
function.

This is the most basic way to use a domain.

\begin{Shaded}
\begin{Highlighting}[]
\KeywordTok{const}\NormalTok{ domain }\OperatorTok{=} \PreprocessorTok{require}\NormalTok{(}\StringTok{\textquotesingle{}node:domain\textquotesingle{}}\NormalTok{)}\OperatorTok{;}
\KeywordTok{const}\NormalTok{ fs }\OperatorTok{=} \PreprocessorTok{require}\NormalTok{(}\StringTok{\textquotesingle{}node:fs\textquotesingle{}}\NormalTok{)}\OperatorTok{;}
\KeywordTok{const}\NormalTok{ d }\OperatorTok{=}\NormalTok{ domain}\OperatorTok{.}\FunctionTok{create}\NormalTok{()}\OperatorTok{;}
\NormalTok{d}\OperatorTok{.}\FunctionTok{on}\NormalTok{(}\StringTok{\textquotesingle{}error\textquotesingle{}}\OperatorTok{,}\NormalTok{ (er) }\KeywordTok{=\textgreater{}}\NormalTok{ \{}
  \BuiltInTok{console}\OperatorTok{.}\FunctionTok{error}\NormalTok{(}\StringTok{\textquotesingle{}Caught error!\textquotesingle{}}\OperatorTok{,}\NormalTok{ er)}\OperatorTok{;}
\NormalTok{\})}\OperatorTok{;}
\NormalTok{d}\OperatorTok{.}\FunctionTok{run}\NormalTok{(() }\KeywordTok{=\textgreater{}}\NormalTok{ \{}
  \BuiltInTok{process}\OperatorTok{.}\FunctionTok{nextTick}\NormalTok{(() }\KeywordTok{=\textgreater{}}\NormalTok{ \{}
    \PreprocessorTok{setTimeout}\NormalTok{(() }\KeywordTok{=\textgreater{}}\NormalTok{ \{ }\CommentTok{// Simulating some various async stuff}
\NormalTok{      fs}\OperatorTok{.}\FunctionTok{open}\NormalTok{(}\StringTok{\textquotesingle{}non{-}existent file\textquotesingle{}}\OperatorTok{,} \StringTok{\textquotesingle{}r\textquotesingle{}}\OperatorTok{,}\NormalTok{ (er}\OperatorTok{,}\NormalTok{ fd) }\KeywordTok{=\textgreater{}}\NormalTok{ \{}
        \ControlFlowTok{if}\NormalTok{ (er) }\ControlFlowTok{throw}\NormalTok{ er}\OperatorTok{;}
        \CommentTok{// proceed...}
\NormalTok{      \})}\OperatorTok{;}
\NormalTok{    \}}\OperatorTok{,} \DecValTok{100}\NormalTok{)}\OperatorTok{;}
\NormalTok{  \})}\OperatorTok{;}
\NormalTok{\})}\OperatorTok{;}
\end{Highlighting}
\end{Shaded}

In this example, the
\texttt{d.on(\textquotesingle{}error\textquotesingle{})} handler will be
triggered, rather than crashing the program.

\subsection{Domains and promises}\label{domains-and-promises}

As of Node.js 8.0.0, the handlers of promises are run inside the domain
in which the call to \texttt{.then()} or \texttt{.catch()} itself was
made:

\begin{Shaded}
\begin{Highlighting}[]
\KeywordTok{const}\NormalTok{ d1 }\OperatorTok{=}\NormalTok{ domain}\OperatorTok{.}\FunctionTok{create}\NormalTok{()}\OperatorTok{;}
\KeywordTok{const}\NormalTok{ d2 }\OperatorTok{=}\NormalTok{ domain}\OperatorTok{.}\FunctionTok{create}\NormalTok{()}\OperatorTok{;}

\KeywordTok{let}\NormalTok{ p}\OperatorTok{;}
\NormalTok{d1}\OperatorTok{.}\FunctionTok{run}\NormalTok{(() }\KeywordTok{=\textgreater{}}\NormalTok{ \{}
\NormalTok{  p }\OperatorTok{=} \BuiltInTok{Promise}\OperatorTok{.}\FunctionTok{resolve}\NormalTok{(}\DecValTok{42}\NormalTok{)}\OperatorTok{;}
\NormalTok{\})}\OperatorTok{;}

\NormalTok{d2}\OperatorTok{.}\FunctionTok{run}\NormalTok{(() }\KeywordTok{=\textgreater{}}\NormalTok{ \{}
\NormalTok{  p}\OperatorTok{.}\FunctionTok{then}\NormalTok{((v) }\KeywordTok{=\textgreater{}}\NormalTok{ \{}
    \CommentTok{// running in d2}
\NormalTok{  \})}\OperatorTok{;}
\NormalTok{\})}\OperatorTok{;}
\end{Highlighting}
\end{Shaded}

A callback may be bound to a specific domain using
\hyperref[domainbindcallback]{\texttt{domain.bind(callback)}}:

\begin{Shaded}
\begin{Highlighting}[]
\KeywordTok{const}\NormalTok{ d1 }\OperatorTok{=}\NormalTok{ domain}\OperatorTok{.}\FunctionTok{create}\NormalTok{()}\OperatorTok{;}
\KeywordTok{const}\NormalTok{ d2 }\OperatorTok{=}\NormalTok{ domain}\OperatorTok{.}\FunctionTok{create}\NormalTok{()}\OperatorTok{;}

\KeywordTok{let}\NormalTok{ p}\OperatorTok{;}
\NormalTok{d1}\OperatorTok{.}\FunctionTok{run}\NormalTok{(() }\KeywordTok{=\textgreater{}}\NormalTok{ \{}
\NormalTok{  p }\OperatorTok{=} \BuiltInTok{Promise}\OperatorTok{.}\FunctionTok{resolve}\NormalTok{(}\DecValTok{42}\NormalTok{)}\OperatorTok{;}
\NormalTok{\})}\OperatorTok{;}

\NormalTok{d2}\OperatorTok{.}\FunctionTok{run}\NormalTok{(() }\KeywordTok{=\textgreater{}}\NormalTok{ \{}
\NormalTok{  p}\OperatorTok{.}\FunctionTok{then}\NormalTok{(p}\OperatorTok{.}\AttributeTok{domain}\OperatorTok{.}\FunctionTok{bind}\NormalTok{((v) }\KeywordTok{=\textgreater{}}\NormalTok{ \{}
    \CommentTok{// running in d1}
\NormalTok{  \}))}\OperatorTok{;}
\NormalTok{\})}\OperatorTok{;}
\end{Highlighting}
\end{Shaded}

Domains will not interfere with the error handling mechanisms for
promises. In other words, no
\texttt{\textquotesingle{}error\textquotesingle{}} event will be emitted
for unhandled \texttt{Promise} rejections.
