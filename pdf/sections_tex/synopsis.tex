\section{Usage and example}\label{usage-and-example}

\subsection{Usage}\label{usage}

\texttt{node\ {[}options{]}\ {[}V8\ options{]}\ {[}script.js\ \textbar{}\ -e\ "script"\ \textbar{}\ -\ {]}\ {[}arguments{]}}

Please see the \href{cli.md\#options}{Command-line options} document for
more information.

\subsection{Example}\label{example}

An example of a \href{http.md}{web server} written with Node.js which
responds with
\texttt{\textquotesingle{}Hello,\ World!\textquotesingle{}}:

Commands in this document start with \texttt{\$} or
\texttt{\textgreater{}} to replicate how they would appear in a user's
terminal. Do not include the \texttt{\$} and \texttt{\textgreater{}}
characters. They are there to show the start of each command.

Lines that don't start with \texttt{\$} or \texttt{\textgreater{}}
character show the output of the previous command.

First, make sure to have downloaded and installed Node.js. See
\href{https://nodejs.org/en/download/package-manager/}{Installing
Node.js via package manager} for further install information.

Now, create an empty project folder called \texttt{projects}, then
navigate into it.

Linux and Mac:

\begin{Shaded}
\begin{Highlighting}[]
\FunctionTok{mkdir}\NormalTok{ \textasciitilde{}/projects}
\BuiltInTok{cd}\NormalTok{ \textasciitilde{}/projects}
\end{Highlighting}
\end{Shaded}

Windows CMD:

\begin{Shaded}
\begin{Highlighting}[]
\NormalTok{mkdir }\OperatorTok{\%}\NormalTok{USERPROFILE}\OperatorTok{\%}\NormalTok{\textbackslash{}projects}
\FunctionTok{cd} \OperatorTok{\%}\NormalTok{USERPROFILE}\OperatorTok{\%}\NormalTok{\textbackslash{}projects}
\end{Highlighting}
\end{Shaded}

Windows PowerShell:

\begin{Shaded}
\begin{Highlighting}[]
\NormalTok{mkdir }\VariableTok{$env}\OperatorTok{:}\VariableTok{USERPROFILE}\NormalTok{\textbackslash{}projects}
\FunctionTok{cd} \VariableTok{$env}\OperatorTok{:}\VariableTok{USERPROFILE}\NormalTok{\textbackslash{}projects}
\end{Highlighting}
\end{Shaded}

Next, create a new source file in the \texttt{projects} folder and call
it \texttt{hello-world.js}.

Open \texttt{hello-world.js} in any preferred text editor and paste in
the following content:

\begin{Shaded}
\begin{Highlighting}[]
\KeywordTok{const}\NormalTok{ http }\OperatorTok{=} \PreprocessorTok{require}\NormalTok{(}\StringTok{\textquotesingle{}node:http\textquotesingle{}}\NormalTok{)}\OperatorTok{;}

\KeywordTok{const}\NormalTok{ hostname }\OperatorTok{=} \StringTok{\textquotesingle{}127.0.0.1\textquotesingle{}}\OperatorTok{;}
\KeywordTok{const}\NormalTok{ port }\OperatorTok{=} \DecValTok{3000}\OperatorTok{;}

\KeywordTok{const}\NormalTok{ server }\OperatorTok{=}\NormalTok{ http}\OperatorTok{.}\FunctionTok{createServer}\NormalTok{((req}\OperatorTok{,}\NormalTok{ res) }\KeywordTok{=\textgreater{}}\NormalTok{ \{}
\NormalTok{  res}\OperatorTok{.}\AttributeTok{statusCode} \OperatorTok{=} \DecValTok{200}\OperatorTok{;}
\NormalTok{  res}\OperatorTok{.}\FunctionTok{setHeader}\NormalTok{(}\StringTok{\textquotesingle{}Content{-}Type\textquotesingle{}}\OperatorTok{,} \StringTok{\textquotesingle{}text/plain\textquotesingle{}}\NormalTok{)}\OperatorTok{;}
\NormalTok{  res}\OperatorTok{.}\FunctionTok{end}\NormalTok{(}\StringTok{\textquotesingle{}Hello, World!}\SpecialCharTok{\textbackslash{}n}\StringTok{\textquotesingle{}}\NormalTok{)}\OperatorTok{;}
\NormalTok{\})}\OperatorTok{;}

\NormalTok{server}\OperatorTok{.}\FunctionTok{listen}\NormalTok{(port}\OperatorTok{,}\NormalTok{ hostname}\OperatorTok{,}\NormalTok{ () }\KeywordTok{=\textgreater{}}\NormalTok{ \{}
  \BuiltInTok{console}\OperatorTok{.}\FunctionTok{log}\NormalTok{(}\VerbatimStringTok{\textasciigrave{}Server running at http://}\SpecialCharTok{$\{}\NormalTok{hostname}\SpecialCharTok{\}}\VerbatimStringTok{:}\SpecialCharTok{$\{}\NormalTok{port}\SpecialCharTok{\}}\VerbatimStringTok{/\textasciigrave{}}\NormalTok{)}\OperatorTok{;}
\NormalTok{\})}\OperatorTok{;}
\end{Highlighting}
\end{Shaded}

Save the file. Then, in the terminal window, to run the
\texttt{hello-world.js} file, enter:

\begin{Shaded}
\begin{Highlighting}[]
\ExtensionTok{node}\NormalTok{ hello{-}world.js}
\end{Highlighting}
\end{Shaded}

Output like this should appear in the terminal:

\begin{Shaded}
\begin{Highlighting}[]
\NormalTok{Server running at http://127.0.0.1:3000/}
\end{Highlighting}
\end{Shaded}

Now, open any preferred web browser and visit
\texttt{http://127.0.0.1:3000}.

If the browser displays the string \texttt{Hello,\ World!}, that
indicates the server is working.
