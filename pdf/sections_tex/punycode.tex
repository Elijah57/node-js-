\section{Punycode}\label{punycode}

\begin{quote}
Stability: 0 - Deprecated
\end{quote}

\textbf{The version of the punycode module bundled in Node.js is being
deprecated.} In a future major version of Node.js this module will be
removed. Users currently depending on the \texttt{punycode} module
should switch to using the userland-provided
\href{https://github.com/bestiejs/punycode.js}{Punycode.js} module
instead. For punycode-based URL encoding, see
\href{url.md\#urldomaintoasciidomain}{\texttt{url.domainToASCII}} or,
more generally, the \href{url.md\#the-whatwg-url-api}{WHATWG URL API}.

The \texttt{punycode} module is a bundled version of the
\href{https://github.com/bestiejs/punycode.js}{Punycode.js} module. It
can be accessed using:

\begin{Shaded}
\begin{Highlighting}[]
\KeywordTok{const}\NormalTok{ punycode }\OperatorTok{=} \PreprocessorTok{require}\NormalTok{(}\StringTok{\textquotesingle{}punycode\textquotesingle{}}\NormalTok{)}\OperatorTok{;}
\end{Highlighting}
\end{Shaded}

\href{https://tools.ietf.org/html/rfc3492}{Punycode} is a character
encoding scheme defined by RFC 3492 that is primarily intended for use
in Internationalized Domain Names. Because host names in URLs are
limited to ASCII characters only, Domain Names that contain non-ASCII
characters must be converted into ASCII using the Punycode scheme. For
instance, the Japanese character that translates into the English word,
\texttt{\textquotesingle{}example\textquotesingle{}} is
\texttt{\textquotesingle{}例\textquotesingle{}}. The Internationalized
Domain Name, \texttt{\textquotesingle{}例.com\textquotesingle{}}
(equivalent to \texttt{\textquotesingle{}example.com\textquotesingle{}})
is represented by Punycode as the ASCII string
\texttt{\textquotesingle{}xn-\/-fsq.com\textquotesingle{}}.

The \texttt{punycode} module provides a simple implementation of the
Punycode standard.

The \texttt{punycode} module is a third-party dependency used by Node.js
and made available to developers as a convenience. Fixes or other
modifications to the module must be directed to the
\href{https://github.com/bestiejs/punycode.js}{Punycode.js} project.

\subsection{\texorpdfstring{\texttt{punycode.decode(string)}}{punycode.decode(string)}}\label{punycode.decodestring}

\begin{itemize}
\tightlist
\item
  \texttt{string} \{string\}
\end{itemize}

The \texttt{punycode.decode()} method converts a
\href{https://tools.ietf.org/html/rfc3492}{Punycode} string of
ASCII-only characters to the equivalent string of Unicode codepoints.

\begin{Shaded}
\begin{Highlighting}[]
\NormalTok{punycode}\OperatorTok{.}\FunctionTok{decode}\NormalTok{(}\StringTok{\textquotesingle{}maana{-}pta\textquotesingle{}}\NormalTok{)}\OperatorTok{;} \CommentTok{// \textquotesingle{}mañana\textquotesingle{}}
\NormalTok{punycode}\OperatorTok{.}\FunctionTok{decode}\NormalTok{(}\StringTok{\textquotesingle{}{-}{-}dqo34k\textquotesingle{}}\NormalTok{)}\OperatorTok{;} \CommentTok{// \textquotesingle{}☃{-}⌘\textquotesingle{}}
\end{Highlighting}
\end{Shaded}

\subsection{\texorpdfstring{\texttt{punycode.encode(string)}}{punycode.encode(string)}}\label{punycode.encodestring}

\begin{itemize}
\tightlist
\item
  \texttt{string} \{string\}
\end{itemize}

The \texttt{punycode.encode()} method converts a string of Unicode
codepoints to a \href{https://tools.ietf.org/html/rfc3492}{Punycode}
string of ASCII-only characters.

\begin{Shaded}
\begin{Highlighting}[]
\NormalTok{punycode}\OperatorTok{.}\FunctionTok{encode}\NormalTok{(}\StringTok{\textquotesingle{}mañana\textquotesingle{}}\NormalTok{)}\OperatorTok{;} \CommentTok{// \textquotesingle{}maana{-}pta\textquotesingle{}}
\NormalTok{punycode}\OperatorTok{.}\FunctionTok{encode}\NormalTok{(}\StringTok{\textquotesingle{}☃{-}⌘\textquotesingle{}}\NormalTok{)}\OperatorTok{;} \CommentTok{// \textquotesingle{}{-}{-}dqo34k\textquotesingle{}}
\end{Highlighting}
\end{Shaded}

\subsection{\texorpdfstring{\texttt{punycode.toASCII(domain)}}{punycode.toASCII(domain)}}\label{punycode.toasciidomain}

\begin{itemize}
\tightlist
\item
  \texttt{domain} \{string\}
\end{itemize}

The \texttt{punycode.toASCII()} method converts a Unicode string
representing an Internationalized Domain Name to
\href{https://tools.ietf.org/html/rfc3492}{Punycode}. Only the non-ASCII
parts of the domain name will be converted. Calling
\texttt{punycode.toASCII()} on a string that already only contains ASCII
characters will have no effect.

\begin{Shaded}
\begin{Highlighting}[]
\CommentTok{// encode domain names}
\NormalTok{punycode}\OperatorTok{.}\FunctionTok{toASCII}\NormalTok{(}\StringTok{\textquotesingle{}mañana.com\textquotesingle{}}\NormalTok{)}\OperatorTok{;}  \CommentTok{// \textquotesingle{}xn{-}{-}maana{-}pta.com\textquotesingle{}}
\NormalTok{punycode}\OperatorTok{.}\FunctionTok{toASCII}\NormalTok{(}\StringTok{\textquotesingle{}☃{-}⌘.com\textquotesingle{}}\NormalTok{)}\OperatorTok{;}   \CommentTok{// \textquotesingle{}xn{-}{-}{-}{-}dqo34k.com\textquotesingle{}}
\NormalTok{punycode}\OperatorTok{.}\FunctionTok{toASCII}\NormalTok{(}\StringTok{\textquotesingle{}example.com\textquotesingle{}}\NormalTok{)}\OperatorTok{;} \CommentTok{// \textquotesingle{}example.com\textquotesingle{}}
\end{Highlighting}
\end{Shaded}

\subsection{\texorpdfstring{\texttt{punycode.toUnicode(domain)}}{punycode.toUnicode(domain)}}\label{punycode.tounicodedomain}

\begin{itemize}
\tightlist
\item
  \texttt{domain} \{string\}
\end{itemize}

The \texttt{punycode.toUnicode()} method converts a string representing
a domain name containing
\href{https://tools.ietf.org/html/rfc3492}{Punycode} encoded characters
into Unicode. Only the
\href{https://tools.ietf.org/html/rfc3492}{Punycode} encoded parts of
the domain name are be converted.

\begin{Shaded}
\begin{Highlighting}[]
\CommentTok{// decode domain names}
\NormalTok{punycode}\OperatorTok{.}\FunctionTok{toUnicode}\NormalTok{(}\StringTok{\textquotesingle{}xn{-}{-}maana{-}pta.com\textquotesingle{}}\NormalTok{)}\OperatorTok{;} \CommentTok{// \textquotesingle{}mañana.com\textquotesingle{}}
\NormalTok{punycode}\OperatorTok{.}\FunctionTok{toUnicode}\NormalTok{(}\StringTok{\textquotesingle{}xn{-}{-}{-}{-}dqo34k.com\textquotesingle{}}\NormalTok{)}\OperatorTok{;}  \CommentTok{// \textquotesingle{}☃{-}⌘.com\textquotesingle{}}
\NormalTok{punycode}\OperatorTok{.}\FunctionTok{toUnicode}\NormalTok{(}\StringTok{\textquotesingle{}example.com\textquotesingle{}}\NormalTok{)}\OperatorTok{;}       \CommentTok{// \textquotesingle{}example.com\textquotesingle{}}
\end{Highlighting}
\end{Shaded}

\subsection{\texorpdfstring{\texttt{punycode.ucs2}}{punycode.ucs2}}\label{punycode.ucs2}

\subsubsection{\texorpdfstring{\texttt{punycode.ucs2.decode(string)}}{punycode.ucs2.decode(string)}}\label{punycode.ucs2.decodestring}

\begin{itemize}
\tightlist
\item
  \texttt{string} \{string\}
\end{itemize}

The \texttt{punycode.ucs2.decode()} method returns an array containing
the numeric codepoint values of each Unicode symbol in the string.

\begin{Shaded}
\begin{Highlighting}[]
\NormalTok{punycode}\OperatorTok{.}\AttributeTok{ucs2}\OperatorTok{.}\FunctionTok{decode}\NormalTok{(}\StringTok{\textquotesingle{}abc\textquotesingle{}}\NormalTok{)}\OperatorTok{;} \CommentTok{// [0x61, 0x62, 0x63]}
\CommentTok{// surrogate pair for U+1D306 tetragram for centre:}
\NormalTok{punycode}\OperatorTok{.}\AttributeTok{ucs2}\OperatorTok{.}\FunctionTok{decode}\NormalTok{(}\StringTok{\textquotesingle{}}\SpecialCharTok{\textbackslash{}uD834\textbackslash{}uDF06}\StringTok{\textquotesingle{}}\NormalTok{)}\OperatorTok{;} \CommentTok{// [0x1D306]}
\end{Highlighting}
\end{Shaded}

\subsubsection{\texorpdfstring{\texttt{punycode.ucs2.encode(codePoints)}}{punycode.ucs2.encode(codePoints)}}\label{punycode.ucs2.encodecodepoints}

\begin{itemize}
\tightlist
\item
  \texttt{codePoints} \{integer{[}{]}\}
\end{itemize}

The \texttt{punycode.ucs2.encode()} method returns a string based on an
array of numeric code point values.

\begin{Shaded}
\begin{Highlighting}[]
\NormalTok{punycode}\OperatorTok{.}\AttributeTok{ucs2}\OperatorTok{.}\FunctionTok{encode}\NormalTok{([}\BaseNTok{0x61}\OperatorTok{,} \BaseNTok{0x62}\OperatorTok{,} \BaseNTok{0x63}\NormalTok{])}\OperatorTok{;} \CommentTok{// \textquotesingle{}abc\textquotesingle{}}
\NormalTok{punycode}\OperatorTok{.}\AttributeTok{ucs2}\OperatorTok{.}\FunctionTok{encode}\NormalTok{([}\BaseNTok{0x1D306}\NormalTok{])}\OperatorTok{;} \CommentTok{// \textquotesingle{}\textbackslash{}uD834\textbackslash{}uDF06\textquotesingle{}}
\end{Highlighting}
\end{Shaded}

\subsection{\texorpdfstring{\texttt{punycode.version}}{punycode.version}}\label{punycode.version}

\begin{itemize}
\tightlist
\item
  \{string\}
\end{itemize}

Returns a string identifying the current
\href{https://github.com/bestiejs/punycode.js}{Punycode.js} version
number.
