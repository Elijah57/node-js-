\section{Cluster}\label{cluster}

\begin{quote}
Stability: 2 - Stable
\end{quote}

Clusters of Node.js processes can be used to run multiple instances of
Node.js that can distribute workloads among their application threads.
When process isolation is not needed, use the
\href{worker_threads.md}{\texttt{worker\_threads}} module instead, which
allows running multiple application threads within a single Node.js
instance.

The cluster module allows easy creation of child processes that all
share server ports.

\begin{Shaded}
\begin{Highlighting}[]
\ImportTok{import}\NormalTok{ cluster }\ImportTok{from} \StringTok{\textquotesingle{}node:cluster\textquotesingle{}}\OperatorTok{;}
\ImportTok{import}\NormalTok{ http }\ImportTok{from} \StringTok{\textquotesingle{}node:http\textquotesingle{}}\OperatorTok{;}
\ImportTok{import}\NormalTok{ \{ availableParallelism \} }\ImportTok{from} \StringTok{\textquotesingle{}node:os\textquotesingle{}}\OperatorTok{;}
\ImportTok{import} \BuiltInTok{process} \ImportTok{from} \StringTok{\textquotesingle{}node:process\textquotesingle{}}\OperatorTok{;}

\KeywordTok{const}\NormalTok{ numCPUs }\OperatorTok{=} \FunctionTok{availableParallelism}\NormalTok{()}\OperatorTok{;}

\ControlFlowTok{if}\NormalTok{ (cluster}\OperatorTok{.}\AttributeTok{isPrimary}\NormalTok{) \{}
  \BuiltInTok{console}\OperatorTok{.}\FunctionTok{log}\NormalTok{(}\VerbatimStringTok{\textasciigrave{}Primary }\SpecialCharTok{$\{}\BuiltInTok{process}\OperatorTok{.}\AttributeTok{pid}\SpecialCharTok{\}}\VerbatimStringTok{ is running\textasciigrave{}}\NormalTok{)}\OperatorTok{;}

  \CommentTok{// Fork workers.}
  \ControlFlowTok{for}\NormalTok{ (}\KeywordTok{let}\NormalTok{ i }\OperatorTok{=} \DecValTok{0}\OperatorTok{;}\NormalTok{ i }\OperatorTok{\textless{}}\NormalTok{ numCPUs}\OperatorTok{;}\NormalTok{ i}\OperatorTok{++}\NormalTok{) \{}
\NormalTok{    cluster}\OperatorTok{.}\FunctionTok{fork}\NormalTok{()}\OperatorTok{;}
\NormalTok{  \}}

\NormalTok{  cluster}\OperatorTok{.}\FunctionTok{on}\NormalTok{(}\StringTok{\textquotesingle{}exit\textquotesingle{}}\OperatorTok{,}\NormalTok{ (worker}\OperatorTok{,}\NormalTok{ code}\OperatorTok{,}\NormalTok{ signal) }\KeywordTok{=\textgreater{}}\NormalTok{ \{}
    \BuiltInTok{console}\OperatorTok{.}\FunctionTok{log}\NormalTok{(}\VerbatimStringTok{\textasciigrave{}worker }\SpecialCharTok{$\{}\NormalTok{worker}\OperatorTok{.}\AttributeTok{process}\OperatorTok{.}\AttributeTok{pid}\SpecialCharTok{\}}\VerbatimStringTok{ died\textasciigrave{}}\NormalTok{)}\OperatorTok{;}
\NormalTok{  \})}\OperatorTok{;}
\NormalTok{\} }\ControlFlowTok{else}\NormalTok{ \{}
  \CommentTok{// Workers can share any TCP connection}
  \CommentTok{// In this case it is an HTTP server}
\NormalTok{  http}\OperatorTok{.}\FunctionTok{createServer}\NormalTok{((req}\OperatorTok{,}\NormalTok{ res) }\KeywordTok{=\textgreater{}}\NormalTok{ \{}
\NormalTok{    res}\OperatorTok{.}\FunctionTok{writeHead}\NormalTok{(}\DecValTok{200}\NormalTok{)}\OperatorTok{;}
\NormalTok{    res}\OperatorTok{.}\FunctionTok{end}\NormalTok{(}\StringTok{\textquotesingle{}hello world}\SpecialCharTok{\textbackslash{}n}\StringTok{\textquotesingle{}}\NormalTok{)}\OperatorTok{;}
\NormalTok{  \})}\OperatorTok{.}\FunctionTok{listen}\NormalTok{(}\DecValTok{8000}\NormalTok{)}\OperatorTok{;}

  \BuiltInTok{console}\OperatorTok{.}\FunctionTok{log}\NormalTok{(}\VerbatimStringTok{\textasciigrave{}Worker }\SpecialCharTok{$\{}\BuiltInTok{process}\OperatorTok{.}\AttributeTok{pid}\SpecialCharTok{\}}\VerbatimStringTok{ started\textasciigrave{}}\NormalTok{)}\OperatorTok{;}
\NormalTok{\}}
\end{Highlighting}
\end{Shaded}

\begin{Shaded}
\begin{Highlighting}[]
\KeywordTok{const}\NormalTok{ cluster }\OperatorTok{=} \PreprocessorTok{require}\NormalTok{(}\StringTok{\textquotesingle{}node:cluster\textquotesingle{}}\NormalTok{)}\OperatorTok{;}
\KeywordTok{const}\NormalTok{ http }\OperatorTok{=} \PreprocessorTok{require}\NormalTok{(}\StringTok{\textquotesingle{}node:http\textquotesingle{}}\NormalTok{)}\OperatorTok{;}
\KeywordTok{const}\NormalTok{ numCPUs }\OperatorTok{=} \PreprocessorTok{require}\NormalTok{(}\StringTok{\textquotesingle{}node:os\textquotesingle{}}\NormalTok{)}\OperatorTok{.}\FunctionTok{availableParallelism}\NormalTok{()}\OperatorTok{;}
\KeywordTok{const} \BuiltInTok{process} \OperatorTok{=} \PreprocessorTok{require}\NormalTok{(}\StringTok{\textquotesingle{}node:process\textquotesingle{}}\NormalTok{)}\OperatorTok{;}

\ControlFlowTok{if}\NormalTok{ (cluster}\OperatorTok{.}\AttributeTok{isPrimary}\NormalTok{) \{}
  \BuiltInTok{console}\OperatorTok{.}\FunctionTok{log}\NormalTok{(}\VerbatimStringTok{\textasciigrave{}Primary }\SpecialCharTok{$\{}\BuiltInTok{process}\OperatorTok{.}\AttributeTok{pid}\SpecialCharTok{\}}\VerbatimStringTok{ is running\textasciigrave{}}\NormalTok{)}\OperatorTok{;}

  \CommentTok{// Fork workers.}
  \ControlFlowTok{for}\NormalTok{ (}\KeywordTok{let}\NormalTok{ i }\OperatorTok{=} \DecValTok{0}\OperatorTok{;}\NormalTok{ i }\OperatorTok{\textless{}}\NormalTok{ numCPUs}\OperatorTok{;}\NormalTok{ i}\OperatorTok{++}\NormalTok{) \{}
\NormalTok{    cluster}\OperatorTok{.}\FunctionTok{fork}\NormalTok{()}\OperatorTok{;}
\NormalTok{  \}}

\NormalTok{  cluster}\OperatorTok{.}\FunctionTok{on}\NormalTok{(}\StringTok{\textquotesingle{}exit\textquotesingle{}}\OperatorTok{,}\NormalTok{ (worker}\OperatorTok{,}\NormalTok{ code}\OperatorTok{,}\NormalTok{ signal) }\KeywordTok{=\textgreater{}}\NormalTok{ \{}
    \BuiltInTok{console}\OperatorTok{.}\FunctionTok{log}\NormalTok{(}\VerbatimStringTok{\textasciigrave{}worker }\SpecialCharTok{$\{}\NormalTok{worker}\OperatorTok{.}\AttributeTok{process}\OperatorTok{.}\AttributeTok{pid}\SpecialCharTok{\}}\VerbatimStringTok{ died\textasciigrave{}}\NormalTok{)}\OperatorTok{;}
\NormalTok{  \})}\OperatorTok{;}
\NormalTok{\} }\ControlFlowTok{else}\NormalTok{ \{}
  \CommentTok{// Workers can share any TCP connection}
  \CommentTok{// In this case it is an HTTP server}
\NormalTok{  http}\OperatorTok{.}\FunctionTok{createServer}\NormalTok{((req}\OperatorTok{,}\NormalTok{ res) }\KeywordTok{=\textgreater{}}\NormalTok{ \{}
\NormalTok{    res}\OperatorTok{.}\FunctionTok{writeHead}\NormalTok{(}\DecValTok{200}\NormalTok{)}\OperatorTok{;}
\NormalTok{    res}\OperatorTok{.}\FunctionTok{end}\NormalTok{(}\StringTok{\textquotesingle{}hello world}\SpecialCharTok{\textbackslash{}n}\StringTok{\textquotesingle{}}\NormalTok{)}\OperatorTok{;}
\NormalTok{  \})}\OperatorTok{.}\FunctionTok{listen}\NormalTok{(}\DecValTok{8000}\NormalTok{)}\OperatorTok{;}

  \BuiltInTok{console}\OperatorTok{.}\FunctionTok{log}\NormalTok{(}\VerbatimStringTok{\textasciigrave{}Worker }\SpecialCharTok{$\{}\BuiltInTok{process}\OperatorTok{.}\AttributeTok{pid}\SpecialCharTok{\}}\VerbatimStringTok{ started\textasciigrave{}}\NormalTok{)}\OperatorTok{;}
\NormalTok{\}}
\end{Highlighting}
\end{Shaded}

Running Node.js will now share port 8000 between the workers:

\begin{Shaded}
\begin{Highlighting}[]
\NormalTok{$ node server.js}
\NormalTok{Primary 3596 is running}
\NormalTok{Worker 4324 started}
\NormalTok{Worker 4520 started}
\NormalTok{Worker 6056 started}
\NormalTok{Worker 5644 started}
\end{Highlighting}
\end{Shaded}

On Windows, it is not yet possible to set up a named pipe server in a
worker.

\subsection{How it works}\label{how-it-works}

The worker processes are spawned using the
\href{child_process.md\#child_processforkmodulepath-args-options}{\texttt{child\_process.fork()}}
method, so that they can communicate with the parent via IPC and pass
server handles back and forth.

The cluster module supports two methods of distributing incoming
connections.

The first one (and the default one on all platforms except Windows) is
the round-robin approach, where the primary process listens on a port,
accepts new connections and distributes them across the workers in a
round-robin fashion, with some built-in smarts to avoid overloading a
worker process.

The second approach is where the primary process creates the listen
socket and sends it to interested workers. The workers then accept
incoming connections directly.

The second approach should, in theory, give the best performance. In
practice however, distribution tends to be very unbalanced due to
operating system scheduler vagaries. Loads have been observed where over
70\% of all connections ended up in just two processes, out of a total
of eight.

Because \texttt{server.listen()} hands off most of the work to the
primary process, there are three cases where the behavior between a
normal Node.js process and a cluster worker differs:

\begin{enumerate}
\def\labelenumi{\arabic{enumi}.}
\tightlist
\item
  \texttt{server.listen(\{fd:\ 7\})} Because the message is passed to
  the primary, file descriptor 7 \textbf{in the parent} will be listened
  on, and the handle passed to the worker, rather than listening to the
  worker's idea of what the number 7 file descriptor references.
\item
  \texttt{server.listen(handle)} Listening on handles explicitly will
  cause the worker to use the supplied handle, rather than talk to the
  primary process.
\item
  \texttt{server.listen(0)} Normally, this will cause servers to listen
  on a random port. However, in a cluster, each worker will receive the
  same ``random'' port each time they do \texttt{listen(0)}. In essence,
  the port is random the first time, but predictable thereafter. To
  listen on a unique port, generate a port number based on the cluster
  worker ID.
\end{enumerate}

Node.js does not provide routing logic. It is therefore important to
design an application such that it does not rely too heavily on
in-memory data objects for things like sessions and login.

Because workers are all separate processes, they can be killed or
re-spawned depending on a program's needs, without affecting other
workers. As long as there are some workers still alive, the server will
continue to accept connections. If no workers are alive, existing
connections will be dropped and new connections will be refused. Node.js
does not automatically manage the number of workers, however. It is the
application's responsibility to manage the worker pool based on its own
needs.

Although a primary use case for the \texttt{node:cluster} module is
networking, it can also be used for other use cases requiring worker
processes.

\subsection{\texorpdfstring{Class:
\texttt{Worker}}{Class: Worker}}\label{class-worker}

\begin{itemize}
\tightlist
\item
  Extends: \{EventEmitter\}
\end{itemize}

A \texttt{Worker} object contains all public information and method
about a worker. In the primary it can be obtained using
\texttt{cluster.workers}. In a worker it can be obtained using
\texttt{cluster.worker}.

\subsubsection{\texorpdfstring{Event:
\texttt{\textquotesingle{}disconnect\textquotesingle{}}}{Event: \textquotesingle disconnect\textquotesingle{}}}\label{event-disconnect}

Similar to the
\texttt{cluster.on(\textquotesingle{}disconnect\textquotesingle{})}
event, but specific to this worker.

\begin{Shaded}
\begin{Highlighting}[]
\NormalTok{cluster}\OperatorTok{.}\FunctionTok{fork}\NormalTok{()}\OperatorTok{.}\FunctionTok{on}\NormalTok{(}\StringTok{\textquotesingle{}disconnect\textquotesingle{}}\OperatorTok{,}\NormalTok{ () }\KeywordTok{=\textgreater{}}\NormalTok{ \{}
  \CommentTok{// Worker has disconnected}
\NormalTok{\})}\OperatorTok{;}
\end{Highlighting}
\end{Shaded}

\subsubsection{\texorpdfstring{Event:
\texttt{\textquotesingle{}error\textquotesingle{}}}{Event: \textquotesingle error\textquotesingle{}}}\label{event-error}

This event is the same as the one provided by
\href{child_process.md\#child_processforkmodulepath-args-options}{\texttt{child\_process.fork()}}.

Within a worker,
\texttt{process.on(\textquotesingle{}error\textquotesingle{})} may also
be used.

\subsubsection{\texorpdfstring{Event:
\texttt{\textquotesingle{}exit\textquotesingle{}}}{Event: \textquotesingle exit\textquotesingle{}}}\label{event-exit}

\begin{itemize}
\tightlist
\item
  \texttt{code} \{number\} The exit code, if it exited normally.
\item
  \texttt{signal} \{string\} The name of the signal
  (e.g.~\texttt{\textquotesingle{}SIGHUP\textquotesingle{}}) that caused
  the process to be killed.
\end{itemize}

Similar to the
\texttt{cluster.on(\textquotesingle{}exit\textquotesingle{})} event, but
specific to this worker.

\begin{Shaded}
\begin{Highlighting}[]
\ImportTok{import}\NormalTok{ cluster }\ImportTok{from} \StringTok{\textquotesingle{}node:cluster\textquotesingle{}}\OperatorTok{;}

\ControlFlowTok{if}\NormalTok{ (cluster}\OperatorTok{.}\AttributeTok{isPrimary}\NormalTok{) \{}
  \KeywordTok{const}\NormalTok{ worker }\OperatorTok{=}\NormalTok{ cluster}\OperatorTok{.}\FunctionTok{fork}\NormalTok{()}\OperatorTok{;}
\NormalTok{  worker}\OperatorTok{.}\FunctionTok{on}\NormalTok{(}\StringTok{\textquotesingle{}exit\textquotesingle{}}\OperatorTok{,}\NormalTok{ (code}\OperatorTok{,}\NormalTok{ signal) }\KeywordTok{=\textgreater{}}\NormalTok{ \{}
    \ControlFlowTok{if}\NormalTok{ (signal) \{}
      \BuiltInTok{console}\OperatorTok{.}\FunctionTok{log}\NormalTok{(}\VerbatimStringTok{\textasciigrave{}worker was killed by signal: }\SpecialCharTok{$\{}\NormalTok{signal}\SpecialCharTok{\}}\VerbatimStringTok{\textasciigrave{}}\NormalTok{)}\OperatorTok{;}
\NormalTok{    \} }\ControlFlowTok{else} \ControlFlowTok{if}\NormalTok{ (code }\OperatorTok{!==} \DecValTok{0}\NormalTok{) \{}
      \BuiltInTok{console}\OperatorTok{.}\FunctionTok{log}\NormalTok{(}\VerbatimStringTok{\textasciigrave{}worker exited with error code: }\SpecialCharTok{$\{}\NormalTok{code}\SpecialCharTok{\}}\VerbatimStringTok{\textasciigrave{}}\NormalTok{)}\OperatorTok{;}
\NormalTok{    \} }\ControlFlowTok{else}\NormalTok{ \{}
      \BuiltInTok{console}\OperatorTok{.}\FunctionTok{log}\NormalTok{(}\StringTok{\textquotesingle{}worker success!\textquotesingle{}}\NormalTok{)}\OperatorTok{;}
\NormalTok{    \}}
\NormalTok{  \})}\OperatorTok{;}
\NormalTok{\}}
\end{Highlighting}
\end{Shaded}

\begin{Shaded}
\begin{Highlighting}[]
\KeywordTok{const}\NormalTok{ cluster }\OperatorTok{=} \PreprocessorTok{require}\NormalTok{(}\StringTok{\textquotesingle{}node:cluster\textquotesingle{}}\NormalTok{)}\OperatorTok{;}

\ControlFlowTok{if}\NormalTok{ (cluster}\OperatorTok{.}\AttributeTok{isPrimary}\NormalTok{) \{}
  \KeywordTok{const}\NormalTok{ worker }\OperatorTok{=}\NormalTok{ cluster}\OperatorTok{.}\FunctionTok{fork}\NormalTok{()}\OperatorTok{;}
\NormalTok{  worker}\OperatorTok{.}\FunctionTok{on}\NormalTok{(}\StringTok{\textquotesingle{}exit\textquotesingle{}}\OperatorTok{,}\NormalTok{ (code}\OperatorTok{,}\NormalTok{ signal) }\KeywordTok{=\textgreater{}}\NormalTok{ \{}
    \ControlFlowTok{if}\NormalTok{ (signal) \{}
      \BuiltInTok{console}\OperatorTok{.}\FunctionTok{log}\NormalTok{(}\VerbatimStringTok{\textasciigrave{}worker was killed by signal: }\SpecialCharTok{$\{}\NormalTok{signal}\SpecialCharTok{\}}\VerbatimStringTok{\textasciigrave{}}\NormalTok{)}\OperatorTok{;}
\NormalTok{    \} }\ControlFlowTok{else} \ControlFlowTok{if}\NormalTok{ (code }\OperatorTok{!==} \DecValTok{0}\NormalTok{) \{}
      \BuiltInTok{console}\OperatorTok{.}\FunctionTok{log}\NormalTok{(}\VerbatimStringTok{\textasciigrave{}worker exited with error code: }\SpecialCharTok{$\{}\NormalTok{code}\SpecialCharTok{\}}\VerbatimStringTok{\textasciigrave{}}\NormalTok{)}\OperatorTok{;}
\NormalTok{    \} }\ControlFlowTok{else}\NormalTok{ \{}
      \BuiltInTok{console}\OperatorTok{.}\FunctionTok{log}\NormalTok{(}\StringTok{\textquotesingle{}worker success!\textquotesingle{}}\NormalTok{)}\OperatorTok{;}
\NormalTok{    \}}
\NormalTok{  \})}\OperatorTok{;}
\NormalTok{\}}
\end{Highlighting}
\end{Shaded}

\subsubsection{\texorpdfstring{Event:
\texttt{\textquotesingle{}listening\textquotesingle{}}}{Event: \textquotesingle listening\textquotesingle{}}}\label{event-listening}

\begin{itemize}
\tightlist
\item
  \texttt{address} \{Object\}
\end{itemize}

Similar to the
\texttt{cluster.on(\textquotesingle{}listening\textquotesingle{})}
event, but specific to this worker.

\begin{Shaded}
\begin{Highlighting}[]
\NormalTok{cluster}\OperatorTok{.}\FunctionTok{fork}\NormalTok{()}\OperatorTok{.}\FunctionTok{on}\NormalTok{(}\StringTok{\textquotesingle{}listening\textquotesingle{}}\OperatorTok{,}\NormalTok{ (address) }\KeywordTok{=\textgreater{}}\NormalTok{ \{}
  \CommentTok{// Worker is listening}
\NormalTok{\})}\OperatorTok{;}
\end{Highlighting}
\end{Shaded}

\begin{Shaded}
\begin{Highlighting}[]
\NormalTok{cluster}\OperatorTok{.}\FunctionTok{fork}\NormalTok{()}\OperatorTok{.}\FunctionTok{on}\NormalTok{(}\StringTok{\textquotesingle{}listening\textquotesingle{}}\OperatorTok{,}\NormalTok{ (address) }\KeywordTok{=\textgreater{}}\NormalTok{ \{}
  \CommentTok{// Worker is listening}
\NormalTok{\})}\OperatorTok{;}
\end{Highlighting}
\end{Shaded}

It is not emitted in the worker.

\subsubsection{\texorpdfstring{Event:
\texttt{\textquotesingle{}message\textquotesingle{}}}{Event: \textquotesingle message\textquotesingle{}}}\label{event-message}

\begin{itemize}
\tightlist
\item
  \texttt{message} \{Object\}
\item
  \texttt{handle} \{undefined\textbar Object\}
\end{itemize}

Similar to the \texttt{\textquotesingle{}message\textquotesingle{}}
event of \texttt{cluster}, but specific to this worker.

Within a worker,
\texttt{process.on(\textquotesingle{}message\textquotesingle{})} may
also be used.

See \href{process.md\#event-message}{\texttt{process} event:
\texttt{\textquotesingle{}message\textquotesingle{}}}.

Here is an example using the message system. It keeps a count in the
primary process of the number of HTTP requests received by the workers:

\begin{Shaded}
\begin{Highlighting}[]
\ImportTok{import}\NormalTok{ cluster }\ImportTok{from} \StringTok{\textquotesingle{}node:cluster\textquotesingle{}}\OperatorTok{;}
\ImportTok{import}\NormalTok{ http }\ImportTok{from} \StringTok{\textquotesingle{}node:http\textquotesingle{}}\OperatorTok{;}
\ImportTok{import}\NormalTok{ \{ availableParallelism \} }\ImportTok{from} \StringTok{\textquotesingle{}node:os\textquotesingle{}}\OperatorTok{;}
\ImportTok{import} \BuiltInTok{process} \ImportTok{from} \StringTok{\textquotesingle{}node:process\textquotesingle{}}\OperatorTok{;}

\ControlFlowTok{if}\NormalTok{ (cluster}\OperatorTok{.}\AttributeTok{isPrimary}\NormalTok{) \{}

  \CommentTok{// Keep track of http requests}
  \KeywordTok{let}\NormalTok{ numReqs }\OperatorTok{=} \DecValTok{0}\OperatorTok{;}
  \PreprocessorTok{setInterval}\NormalTok{(() }\KeywordTok{=\textgreater{}}\NormalTok{ \{}
    \BuiltInTok{console}\OperatorTok{.}\FunctionTok{log}\NormalTok{(}\VerbatimStringTok{\textasciigrave{}numReqs = }\SpecialCharTok{$\{}\NormalTok{numReqs}\SpecialCharTok{\}}\VerbatimStringTok{\textasciigrave{}}\NormalTok{)}\OperatorTok{;}
\NormalTok{  \}}\OperatorTok{,} \DecValTok{1000}\NormalTok{)}\OperatorTok{;}

  \CommentTok{// Count requests}
  \KeywordTok{function} \FunctionTok{messageHandler}\NormalTok{(msg) \{}
    \ControlFlowTok{if}\NormalTok{ (msg}\OperatorTok{.}\AttributeTok{cmd} \OperatorTok{\&\&}\NormalTok{ msg}\OperatorTok{.}\AttributeTok{cmd} \OperatorTok{===} \StringTok{\textquotesingle{}notifyRequest\textquotesingle{}}\NormalTok{) \{}
\NormalTok{      numReqs }\OperatorTok{+=} \DecValTok{1}\OperatorTok{;}
\NormalTok{    \}}
\NormalTok{  \}}

  \CommentTok{// Start workers and listen for messages containing notifyRequest}
  \KeywordTok{const}\NormalTok{ numCPUs }\OperatorTok{=} \FunctionTok{availableParallelism}\NormalTok{()}\OperatorTok{;}
  \ControlFlowTok{for}\NormalTok{ (}\KeywordTok{let}\NormalTok{ i }\OperatorTok{=} \DecValTok{0}\OperatorTok{;}\NormalTok{ i }\OperatorTok{\textless{}}\NormalTok{ numCPUs}\OperatorTok{;}\NormalTok{ i}\OperatorTok{++}\NormalTok{) \{}
\NormalTok{    cluster}\OperatorTok{.}\FunctionTok{fork}\NormalTok{()}\OperatorTok{;}
\NormalTok{  \}}

  \ControlFlowTok{for}\NormalTok{ (}\KeywordTok{const}\NormalTok{ id }\KeywordTok{in}\NormalTok{ cluster}\OperatorTok{.}\AttributeTok{workers}\NormalTok{) \{}
\NormalTok{    cluster}\OperatorTok{.}\AttributeTok{workers}\NormalTok{[id]}\OperatorTok{.}\FunctionTok{on}\NormalTok{(}\StringTok{\textquotesingle{}message\textquotesingle{}}\OperatorTok{,}\NormalTok{ messageHandler)}\OperatorTok{;}
\NormalTok{  \}}

\NormalTok{\} }\ControlFlowTok{else}\NormalTok{ \{}

  \CommentTok{// Worker processes have a http server.}
\NormalTok{  http}\OperatorTok{.}\FunctionTok{Server}\NormalTok{((req}\OperatorTok{,}\NormalTok{ res) }\KeywordTok{=\textgreater{}}\NormalTok{ \{}
\NormalTok{    res}\OperatorTok{.}\FunctionTok{writeHead}\NormalTok{(}\DecValTok{200}\NormalTok{)}\OperatorTok{;}
\NormalTok{    res}\OperatorTok{.}\FunctionTok{end}\NormalTok{(}\StringTok{\textquotesingle{}hello world}\SpecialCharTok{\textbackslash{}n}\StringTok{\textquotesingle{}}\NormalTok{)}\OperatorTok{;}

    \CommentTok{// Notify primary about the request}
    \BuiltInTok{process}\OperatorTok{.}\FunctionTok{send}\NormalTok{(\{ }\DataTypeTok{cmd}\OperatorTok{:} \StringTok{\textquotesingle{}notifyRequest\textquotesingle{}}\NormalTok{ \})}\OperatorTok{;}
\NormalTok{  \})}\OperatorTok{.}\FunctionTok{listen}\NormalTok{(}\DecValTok{8000}\NormalTok{)}\OperatorTok{;}
\NormalTok{\}}
\end{Highlighting}
\end{Shaded}

\begin{Shaded}
\begin{Highlighting}[]
\KeywordTok{const}\NormalTok{ cluster }\OperatorTok{=} \PreprocessorTok{require}\NormalTok{(}\StringTok{\textquotesingle{}node:cluster\textquotesingle{}}\NormalTok{)}\OperatorTok{;}
\KeywordTok{const}\NormalTok{ http }\OperatorTok{=} \PreprocessorTok{require}\NormalTok{(}\StringTok{\textquotesingle{}node:http\textquotesingle{}}\NormalTok{)}\OperatorTok{;}
\KeywordTok{const} \BuiltInTok{process} \OperatorTok{=} \PreprocessorTok{require}\NormalTok{(}\StringTok{\textquotesingle{}node:process\textquotesingle{}}\NormalTok{)}\OperatorTok{;}

\ControlFlowTok{if}\NormalTok{ (cluster}\OperatorTok{.}\AttributeTok{isPrimary}\NormalTok{) \{}

  \CommentTok{// Keep track of http requests}
  \KeywordTok{let}\NormalTok{ numReqs }\OperatorTok{=} \DecValTok{0}\OperatorTok{;}
  \PreprocessorTok{setInterval}\NormalTok{(() }\KeywordTok{=\textgreater{}}\NormalTok{ \{}
    \BuiltInTok{console}\OperatorTok{.}\FunctionTok{log}\NormalTok{(}\VerbatimStringTok{\textasciigrave{}numReqs = }\SpecialCharTok{$\{}\NormalTok{numReqs}\SpecialCharTok{\}}\VerbatimStringTok{\textasciigrave{}}\NormalTok{)}\OperatorTok{;}
\NormalTok{  \}}\OperatorTok{,} \DecValTok{1000}\NormalTok{)}\OperatorTok{;}

  \CommentTok{// Count requests}
  \KeywordTok{function} \FunctionTok{messageHandler}\NormalTok{(msg) \{}
    \ControlFlowTok{if}\NormalTok{ (msg}\OperatorTok{.}\AttributeTok{cmd} \OperatorTok{\&\&}\NormalTok{ msg}\OperatorTok{.}\AttributeTok{cmd} \OperatorTok{===} \StringTok{\textquotesingle{}notifyRequest\textquotesingle{}}\NormalTok{) \{}
\NormalTok{      numReqs }\OperatorTok{+=} \DecValTok{1}\OperatorTok{;}
\NormalTok{    \}}
\NormalTok{  \}}

  \CommentTok{// Start workers and listen for messages containing notifyRequest}
  \KeywordTok{const}\NormalTok{ numCPUs }\OperatorTok{=} \PreprocessorTok{require}\NormalTok{(}\StringTok{\textquotesingle{}node:os\textquotesingle{}}\NormalTok{)}\OperatorTok{.}\FunctionTok{availableParallelism}\NormalTok{()}\OperatorTok{;}
  \ControlFlowTok{for}\NormalTok{ (}\KeywordTok{let}\NormalTok{ i }\OperatorTok{=} \DecValTok{0}\OperatorTok{;}\NormalTok{ i }\OperatorTok{\textless{}}\NormalTok{ numCPUs}\OperatorTok{;}\NormalTok{ i}\OperatorTok{++}\NormalTok{) \{}
\NormalTok{    cluster}\OperatorTok{.}\FunctionTok{fork}\NormalTok{()}\OperatorTok{;}
\NormalTok{  \}}

  \ControlFlowTok{for}\NormalTok{ (}\KeywordTok{const}\NormalTok{ id }\KeywordTok{in}\NormalTok{ cluster}\OperatorTok{.}\AttributeTok{workers}\NormalTok{) \{}
\NormalTok{    cluster}\OperatorTok{.}\AttributeTok{workers}\NormalTok{[id]}\OperatorTok{.}\FunctionTok{on}\NormalTok{(}\StringTok{\textquotesingle{}message\textquotesingle{}}\OperatorTok{,}\NormalTok{ messageHandler)}\OperatorTok{;}
\NormalTok{  \}}

\NormalTok{\} }\ControlFlowTok{else}\NormalTok{ \{}

  \CommentTok{// Worker processes have a http server.}
\NormalTok{  http}\OperatorTok{.}\FunctionTok{Server}\NormalTok{((req}\OperatorTok{,}\NormalTok{ res) }\KeywordTok{=\textgreater{}}\NormalTok{ \{}
\NormalTok{    res}\OperatorTok{.}\FunctionTok{writeHead}\NormalTok{(}\DecValTok{200}\NormalTok{)}\OperatorTok{;}
\NormalTok{    res}\OperatorTok{.}\FunctionTok{end}\NormalTok{(}\StringTok{\textquotesingle{}hello world}\SpecialCharTok{\textbackslash{}n}\StringTok{\textquotesingle{}}\NormalTok{)}\OperatorTok{;}

    \CommentTok{// Notify primary about the request}
    \BuiltInTok{process}\OperatorTok{.}\FunctionTok{send}\NormalTok{(\{ }\DataTypeTok{cmd}\OperatorTok{:} \StringTok{\textquotesingle{}notifyRequest\textquotesingle{}}\NormalTok{ \})}\OperatorTok{;}
\NormalTok{  \})}\OperatorTok{.}\FunctionTok{listen}\NormalTok{(}\DecValTok{8000}\NormalTok{)}\OperatorTok{;}
\NormalTok{\}}
\end{Highlighting}
\end{Shaded}

\subsubsection{\texorpdfstring{Event:
\texttt{\textquotesingle{}online\textquotesingle{}}}{Event: \textquotesingle online\textquotesingle{}}}\label{event-online}

Similar to the
\texttt{cluster.on(\textquotesingle{}online\textquotesingle{})} event,
but specific to this worker.

\begin{Shaded}
\begin{Highlighting}[]
\NormalTok{cluster}\OperatorTok{.}\FunctionTok{fork}\NormalTok{()}\OperatorTok{.}\FunctionTok{on}\NormalTok{(}\StringTok{\textquotesingle{}online\textquotesingle{}}\OperatorTok{,}\NormalTok{ () }\KeywordTok{=\textgreater{}}\NormalTok{ \{}
  \CommentTok{// Worker is online}
\NormalTok{\})}\OperatorTok{;}
\end{Highlighting}
\end{Shaded}

It is not emitted in the worker.

\subsubsection{\texorpdfstring{\texttt{worker.disconnect()}}{worker.disconnect()}}\label{worker.disconnect}

\begin{itemize}
\tightlist
\item
  Returns: \{cluster.Worker\} A reference to \texttt{worker}.
\end{itemize}

In a worker, this function will close all servers, wait for the
\texttt{\textquotesingle{}close\textquotesingle{}} event on those
servers, and then disconnect the IPC channel.

In the primary, an internal message is sent to the worker causing it to
call \texttt{.disconnect()} on itself.

Causes \texttt{.exitedAfterDisconnect} to be set.

After a server is closed, it will no longer accept new connections, but
connections may be accepted by any other listening worker. Existing
connections will be allowed to close as usual. When no more connections
exist, see \href{net.md\#event-close}{\texttt{server.close()}}, the IPC
channel to the worker will close allowing it to die gracefully.

The above applies \emph{only} to server connections, client connections
are not automatically closed by workers, and disconnect does not wait
for them to close before exiting.

In a worker, \texttt{process.disconnect} exists, but it is not this
function; it is
\href{child_process.md\#subprocessdisconnect}{\texttt{disconnect()}}.

Because long living server connections may block workers from
disconnecting, it may be useful to send a message, so application
specific actions may be taken to close them. It also may be useful to
implement a timeout, killing a worker if the
\texttt{\textquotesingle{}disconnect\textquotesingle{}} event has not
been emitted after some time.

\begin{Shaded}
\begin{Highlighting}[]
\ControlFlowTok{if}\NormalTok{ (cluster}\OperatorTok{.}\AttributeTok{isPrimary}\NormalTok{) \{}
  \KeywordTok{const}\NormalTok{ worker }\OperatorTok{=}\NormalTok{ cluster}\OperatorTok{.}\FunctionTok{fork}\NormalTok{()}\OperatorTok{;}
  \KeywordTok{let}\NormalTok{ timeout}\OperatorTok{;}

\NormalTok{  worker}\OperatorTok{.}\FunctionTok{on}\NormalTok{(}\StringTok{\textquotesingle{}listening\textquotesingle{}}\OperatorTok{,}\NormalTok{ (address) }\KeywordTok{=\textgreater{}}\NormalTok{ \{}
\NormalTok{    worker}\OperatorTok{.}\FunctionTok{send}\NormalTok{(}\StringTok{\textquotesingle{}shutdown\textquotesingle{}}\NormalTok{)}\OperatorTok{;}
\NormalTok{    worker}\OperatorTok{.}\FunctionTok{disconnect}\NormalTok{()}\OperatorTok{;}
\NormalTok{    timeout }\OperatorTok{=} \PreprocessorTok{setTimeout}\NormalTok{(() }\KeywordTok{=\textgreater{}}\NormalTok{ \{}
\NormalTok{      worker}\OperatorTok{.}\FunctionTok{kill}\NormalTok{()}\OperatorTok{;}
\NormalTok{    \}}\OperatorTok{,} \DecValTok{2000}\NormalTok{)}\OperatorTok{;}
\NormalTok{  \})}\OperatorTok{;}

\NormalTok{  worker}\OperatorTok{.}\FunctionTok{on}\NormalTok{(}\StringTok{\textquotesingle{}disconnect\textquotesingle{}}\OperatorTok{,}\NormalTok{ () }\KeywordTok{=\textgreater{}}\NormalTok{ \{}
    \PreprocessorTok{clearTimeout}\NormalTok{(timeout)}\OperatorTok{;}
\NormalTok{  \})}\OperatorTok{;}

\NormalTok{\} }\ControlFlowTok{else} \ControlFlowTok{if}\NormalTok{ (cluster}\OperatorTok{.}\AttributeTok{isWorker}\NormalTok{) \{}
  \KeywordTok{const}\NormalTok{ net }\OperatorTok{=} \PreprocessorTok{require}\NormalTok{(}\StringTok{\textquotesingle{}node:net\textquotesingle{}}\NormalTok{)}\OperatorTok{;}
  \KeywordTok{const}\NormalTok{ server }\OperatorTok{=}\NormalTok{ net}\OperatorTok{.}\FunctionTok{createServer}\NormalTok{((socket) }\KeywordTok{=\textgreater{}}\NormalTok{ \{}
    \CommentTok{// Connections never end}
\NormalTok{  \})}\OperatorTok{;}

\NormalTok{  server}\OperatorTok{.}\FunctionTok{listen}\NormalTok{(}\DecValTok{8000}\NormalTok{)}\OperatorTok{;}

  \BuiltInTok{process}\OperatorTok{.}\FunctionTok{on}\NormalTok{(}\StringTok{\textquotesingle{}message\textquotesingle{}}\OperatorTok{,}\NormalTok{ (msg) }\KeywordTok{=\textgreater{}}\NormalTok{ \{}
    \ControlFlowTok{if}\NormalTok{ (msg }\OperatorTok{===} \StringTok{\textquotesingle{}shutdown\textquotesingle{}}\NormalTok{) \{}
      \CommentTok{// Initiate graceful close of any connections to server}
\NormalTok{    \}}
\NormalTok{  \})}\OperatorTok{;}
\NormalTok{\}}
\end{Highlighting}
\end{Shaded}

\subsubsection{\texorpdfstring{\texttt{worker.exitedAfterDisconnect}}{worker.exitedAfterDisconnect}}\label{worker.exitedafterdisconnect}

\begin{itemize}
\tightlist
\item
  \{boolean\}
\end{itemize}

This property is \texttt{true} if the worker exited due to
\texttt{.disconnect()}. If the worker exited any other way, it is
\texttt{false}. If the worker has not exited, it is \texttt{undefined}.

The boolean
\hyperref[workerexitedafterdisconnect]{\texttt{worker.exitedAfterDisconnect}}
allows distinguishing between voluntary and accidental exit, the primary
may choose not to respawn a worker based on this value.

\begin{Shaded}
\begin{Highlighting}[]
\NormalTok{cluster}\OperatorTok{.}\FunctionTok{on}\NormalTok{(}\StringTok{\textquotesingle{}exit\textquotesingle{}}\OperatorTok{,}\NormalTok{ (worker}\OperatorTok{,}\NormalTok{ code}\OperatorTok{,}\NormalTok{ signal) }\KeywordTok{=\textgreater{}}\NormalTok{ \{}
  \ControlFlowTok{if}\NormalTok{ (worker}\OperatorTok{.}\AttributeTok{exitedAfterDisconnect} \OperatorTok{===} \KeywordTok{true}\NormalTok{) \{}
    \BuiltInTok{console}\OperatorTok{.}\FunctionTok{log}\NormalTok{(}\StringTok{\textquotesingle{}Oh, it was just voluntary – no need to worry\textquotesingle{}}\NormalTok{)}\OperatorTok{;}
\NormalTok{  \}}
\NormalTok{\})}\OperatorTok{;}

\CommentTok{// kill worker}
\NormalTok{worker}\OperatorTok{.}\FunctionTok{kill}\NormalTok{()}\OperatorTok{;}
\end{Highlighting}
\end{Shaded}

\subsubsection{\texorpdfstring{\texttt{worker.id}}{worker.id}}\label{worker.id}

\begin{itemize}
\tightlist
\item
  \{integer\}
\end{itemize}

Each new worker is given its own unique id, this id is stored in the
\texttt{id}.

While a worker is alive, this is the key that indexes it in
\texttt{cluster.workers}.

\subsubsection{\texorpdfstring{\texttt{worker.isConnected()}}{worker.isConnected()}}\label{worker.isconnected}

This function returns \texttt{true} if the worker is connected to its
primary via its IPC channel, \texttt{false} otherwise. A worker is
connected to its primary after it has been created. It is disconnected
after the \texttt{\textquotesingle{}disconnect\textquotesingle{}} event
is emitted.

\subsubsection{\texorpdfstring{\texttt{worker.isDead()}}{worker.isDead()}}\label{worker.isdead}

This function returns \texttt{true} if the worker's process has
terminated (either because of exiting or being signaled). Otherwise, it
returns \texttt{false}.

\begin{Shaded}
\begin{Highlighting}[]
\ImportTok{import}\NormalTok{ cluster }\ImportTok{from} \StringTok{\textquotesingle{}node:cluster\textquotesingle{}}\OperatorTok{;}
\ImportTok{import}\NormalTok{ http }\ImportTok{from} \StringTok{\textquotesingle{}node:http\textquotesingle{}}\OperatorTok{;}
\ImportTok{import}\NormalTok{ \{ availableParallelism \} }\ImportTok{from} \StringTok{\textquotesingle{}node:os\textquotesingle{}}\OperatorTok{;}
\ImportTok{import} \BuiltInTok{process} \ImportTok{from} \StringTok{\textquotesingle{}node:process\textquotesingle{}}\OperatorTok{;}

\KeywordTok{const}\NormalTok{ numCPUs }\OperatorTok{=} \FunctionTok{availableParallelism}\NormalTok{()}\OperatorTok{;}

\ControlFlowTok{if}\NormalTok{ (cluster}\OperatorTok{.}\AttributeTok{isPrimary}\NormalTok{) \{}
  \BuiltInTok{console}\OperatorTok{.}\FunctionTok{log}\NormalTok{(}\VerbatimStringTok{\textasciigrave{}Primary }\SpecialCharTok{$\{}\BuiltInTok{process}\OperatorTok{.}\AttributeTok{pid}\SpecialCharTok{\}}\VerbatimStringTok{ is running\textasciigrave{}}\NormalTok{)}\OperatorTok{;}

  \CommentTok{// Fork workers.}
  \ControlFlowTok{for}\NormalTok{ (}\KeywordTok{let}\NormalTok{ i }\OperatorTok{=} \DecValTok{0}\OperatorTok{;}\NormalTok{ i }\OperatorTok{\textless{}}\NormalTok{ numCPUs}\OperatorTok{;}\NormalTok{ i}\OperatorTok{++}\NormalTok{) \{}
\NormalTok{    cluster}\OperatorTok{.}\FunctionTok{fork}\NormalTok{()}\OperatorTok{;}
\NormalTok{  \}}

\NormalTok{  cluster}\OperatorTok{.}\FunctionTok{on}\NormalTok{(}\StringTok{\textquotesingle{}fork\textquotesingle{}}\OperatorTok{,}\NormalTok{ (worker) }\KeywordTok{=\textgreater{}}\NormalTok{ \{}
    \BuiltInTok{console}\OperatorTok{.}\FunctionTok{log}\NormalTok{(}\StringTok{\textquotesingle{}worker is dead:\textquotesingle{}}\OperatorTok{,}\NormalTok{ worker}\OperatorTok{.}\FunctionTok{isDead}\NormalTok{())}\OperatorTok{;}
\NormalTok{  \})}\OperatorTok{;}

\NormalTok{  cluster}\OperatorTok{.}\FunctionTok{on}\NormalTok{(}\StringTok{\textquotesingle{}exit\textquotesingle{}}\OperatorTok{,}\NormalTok{ (worker}\OperatorTok{,}\NormalTok{ code}\OperatorTok{,}\NormalTok{ signal) }\KeywordTok{=\textgreater{}}\NormalTok{ \{}
    \BuiltInTok{console}\OperatorTok{.}\FunctionTok{log}\NormalTok{(}\StringTok{\textquotesingle{}worker is dead:\textquotesingle{}}\OperatorTok{,}\NormalTok{ worker}\OperatorTok{.}\FunctionTok{isDead}\NormalTok{())}\OperatorTok{;}
\NormalTok{  \})}\OperatorTok{;}
\NormalTok{\} }\ControlFlowTok{else}\NormalTok{ \{}
  \CommentTok{// Workers can share any TCP connection. In this case, it is an HTTP server.}
\NormalTok{  http}\OperatorTok{.}\FunctionTok{createServer}\NormalTok{((req}\OperatorTok{,}\NormalTok{ res) }\KeywordTok{=\textgreater{}}\NormalTok{ \{}
\NormalTok{    res}\OperatorTok{.}\FunctionTok{writeHead}\NormalTok{(}\DecValTok{200}\NormalTok{)}\OperatorTok{;}
\NormalTok{    res}\OperatorTok{.}\FunctionTok{end}\NormalTok{(}\VerbatimStringTok{\textasciigrave{}Current process}\SpecialCharTok{\textbackslash{}n}\VerbatimStringTok{ }\SpecialCharTok{$\{}\BuiltInTok{process}\OperatorTok{.}\AttributeTok{pid}\SpecialCharTok{\}}\VerbatimStringTok{\textasciigrave{}}\NormalTok{)}\OperatorTok{;}
    \BuiltInTok{process}\OperatorTok{.}\FunctionTok{kill}\NormalTok{(}\BuiltInTok{process}\OperatorTok{.}\AttributeTok{pid}\NormalTok{)}\OperatorTok{;}
\NormalTok{  \})}\OperatorTok{.}\FunctionTok{listen}\NormalTok{(}\DecValTok{8000}\NormalTok{)}\OperatorTok{;}
\NormalTok{\}}
\end{Highlighting}
\end{Shaded}

\begin{Shaded}
\begin{Highlighting}[]
\KeywordTok{const}\NormalTok{ cluster }\OperatorTok{=} \PreprocessorTok{require}\NormalTok{(}\StringTok{\textquotesingle{}node:cluster\textquotesingle{}}\NormalTok{)}\OperatorTok{;}
\KeywordTok{const}\NormalTok{ http }\OperatorTok{=} \PreprocessorTok{require}\NormalTok{(}\StringTok{\textquotesingle{}node:http\textquotesingle{}}\NormalTok{)}\OperatorTok{;}
\KeywordTok{const}\NormalTok{ numCPUs }\OperatorTok{=} \PreprocessorTok{require}\NormalTok{(}\StringTok{\textquotesingle{}node:os\textquotesingle{}}\NormalTok{)}\OperatorTok{.}\FunctionTok{availableParallelism}\NormalTok{()}\OperatorTok{;}
\KeywordTok{const} \BuiltInTok{process} \OperatorTok{=} \PreprocessorTok{require}\NormalTok{(}\StringTok{\textquotesingle{}node:process\textquotesingle{}}\NormalTok{)}\OperatorTok{;}

\ControlFlowTok{if}\NormalTok{ (cluster}\OperatorTok{.}\AttributeTok{isPrimary}\NormalTok{) \{}
  \BuiltInTok{console}\OperatorTok{.}\FunctionTok{log}\NormalTok{(}\VerbatimStringTok{\textasciigrave{}Primary }\SpecialCharTok{$\{}\BuiltInTok{process}\OperatorTok{.}\AttributeTok{pid}\SpecialCharTok{\}}\VerbatimStringTok{ is running\textasciigrave{}}\NormalTok{)}\OperatorTok{;}

  \CommentTok{// Fork workers.}
  \ControlFlowTok{for}\NormalTok{ (}\KeywordTok{let}\NormalTok{ i }\OperatorTok{=} \DecValTok{0}\OperatorTok{;}\NormalTok{ i }\OperatorTok{\textless{}}\NormalTok{ numCPUs}\OperatorTok{;}\NormalTok{ i}\OperatorTok{++}\NormalTok{) \{}
\NormalTok{    cluster}\OperatorTok{.}\FunctionTok{fork}\NormalTok{()}\OperatorTok{;}
\NormalTok{  \}}

\NormalTok{  cluster}\OperatorTok{.}\FunctionTok{on}\NormalTok{(}\StringTok{\textquotesingle{}fork\textquotesingle{}}\OperatorTok{,}\NormalTok{ (worker) }\KeywordTok{=\textgreater{}}\NormalTok{ \{}
    \BuiltInTok{console}\OperatorTok{.}\FunctionTok{log}\NormalTok{(}\StringTok{\textquotesingle{}worker is dead:\textquotesingle{}}\OperatorTok{,}\NormalTok{ worker}\OperatorTok{.}\FunctionTok{isDead}\NormalTok{())}\OperatorTok{;}
\NormalTok{  \})}\OperatorTok{;}

\NormalTok{  cluster}\OperatorTok{.}\FunctionTok{on}\NormalTok{(}\StringTok{\textquotesingle{}exit\textquotesingle{}}\OperatorTok{,}\NormalTok{ (worker}\OperatorTok{,}\NormalTok{ code}\OperatorTok{,}\NormalTok{ signal) }\KeywordTok{=\textgreater{}}\NormalTok{ \{}
    \BuiltInTok{console}\OperatorTok{.}\FunctionTok{log}\NormalTok{(}\StringTok{\textquotesingle{}worker is dead:\textquotesingle{}}\OperatorTok{,}\NormalTok{ worker}\OperatorTok{.}\FunctionTok{isDead}\NormalTok{())}\OperatorTok{;}
\NormalTok{  \})}\OperatorTok{;}
\NormalTok{\} }\ControlFlowTok{else}\NormalTok{ \{}
  \CommentTok{// Workers can share any TCP connection. In this case, it is an HTTP server.}
\NormalTok{  http}\OperatorTok{.}\FunctionTok{createServer}\NormalTok{((req}\OperatorTok{,}\NormalTok{ res) }\KeywordTok{=\textgreater{}}\NormalTok{ \{}
\NormalTok{    res}\OperatorTok{.}\FunctionTok{writeHead}\NormalTok{(}\DecValTok{200}\NormalTok{)}\OperatorTok{;}
\NormalTok{    res}\OperatorTok{.}\FunctionTok{end}\NormalTok{(}\VerbatimStringTok{\textasciigrave{}Current process}\SpecialCharTok{\textbackslash{}n}\VerbatimStringTok{ }\SpecialCharTok{$\{}\BuiltInTok{process}\OperatorTok{.}\AttributeTok{pid}\SpecialCharTok{\}}\VerbatimStringTok{\textasciigrave{}}\NormalTok{)}\OperatorTok{;}
    \BuiltInTok{process}\OperatorTok{.}\FunctionTok{kill}\NormalTok{(}\BuiltInTok{process}\OperatorTok{.}\AttributeTok{pid}\NormalTok{)}\OperatorTok{;}
\NormalTok{  \})}\OperatorTok{.}\FunctionTok{listen}\NormalTok{(}\DecValTok{8000}\NormalTok{)}\OperatorTok{;}
\NormalTok{\}}
\end{Highlighting}
\end{Shaded}

\subsubsection{\texorpdfstring{\texttt{worker.kill({[}signal{]})}}{worker.kill({[}signal{]})}}\label{worker.killsignal}

\begin{itemize}
\tightlist
\item
  \texttt{signal} \{string\} Name of the kill signal to send to the
  worker process. \textbf{Default:}
  \texttt{\textquotesingle{}SIGTERM\textquotesingle{}}
\end{itemize}

This function will kill the worker. In the primary worker, it does this
by disconnecting the \texttt{worker.process}, and once disconnected,
killing with \texttt{signal}. In the worker, it does it by killing the
process with \texttt{signal}.

The \texttt{kill()} function kills the worker process without waiting
for a graceful disconnect, it has the same behavior as
\texttt{worker.process.kill()}.

This method is aliased as \texttt{worker.destroy()} for backwards
compatibility.

In a worker, \texttt{process.kill()} exists, but it is not this
function; it is
\href{process.md\#processkillpid-signal}{\texttt{kill()}}.

\subsubsection{\texorpdfstring{\texttt{worker.process}}{worker.process}}\label{worker.process}

\begin{itemize}
\tightlist
\item
  \{ChildProcess\}
\end{itemize}

All workers are created using
\href{child_process.md\#child_processforkmodulepath-args-options}{\texttt{child\_process.fork()}},
the returned object from this function is stored as \texttt{.process}.
In a worker, the global \texttt{process} is stored.

See:
\href{child_process.md\#child_processforkmodulepath-args-options}{Child
Process module}.

Workers will call \texttt{process.exit(0)} if the
\texttt{\textquotesingle{}disconnect\textquotesingle{}} event occurs on
\texttt{process} and \texttt{.exitedAfterDisconnect} is not
\texttt{true}. This protects against accidental disconnection.

\subsubsection{\texorpdfstring{\texttt{worker.send(message{[},\ sendHandle{[},\ options{]}{]}{[},\ callback{]})}}{worker.send(message{[}, sendHandle{[}, options{]}{]}{[}, callback{]})}}\label{worker.sendmessage-sendhandle-options-callback}

\begin{itemize}
\tightlist
\item
  \texttt{message} \{Object\}
\item
  \texttt{sendHandle} \{Handle\}
\item
  \texttt{options} \{Object\} The \texttt{options} argument, if present,
  is an object used to parameterize the sending of certain types of
  handles. \texttt{options} supports the following properties:

  \begin{itemize}
  \tightlist
  \item
    \texttt{keepOpen} \{boolean\} A value that can be used when passing
    instances of \texttt{net.Socket}. When \texttt{true}, the socket is
    kept open in the sending process. \textbf{Default:} \texttt{false}.
  \end{itemize}
\item
  \texttt{callback} \{Function\}
\item
  Returns: \{boolean\}
\end{itemize}

Send a message to a worker or primary, optionally with a handle.

In the primary, this sends a message to a specific worker. It is
identical to
\href{child_process.md\#subprocesssendmessage-sendhandle-options-callback}{\texttt{ChildProcess.send()}}.

In a worker, this sends a message to the primary. It is identical to
\texttt{process.send()}.

This example will echo back all messages from the primary:

\begin{Shaded}
\begin{Highlighting}[]
\ControlFlowTok{if}\NormalTok{ (cluster}\OperatorTok{.}\AttributeTok{isPrimary}\NormalTok{) \{}
  \KeywordTok{const}\NormalTok{ worker }\OperatorTok{=}\NormalTok{ cluster}\OperatorTok{.}\FunctionTok{fork}\NormalTok{()}\OperatorTok{;}
\NormalTok{  worker}\OperatorTok{.}\FunctionTok{send}\NormalTok{(}\StringTok{\textquotesingle{}hi there\textquotesingle{}}\NormalTok{)}\OperatorTok{;}

\NormalTok{\} }\ControlFlowTok{else} \ControlFlowTok{if}\NormalTok{ (cluster}\OperatorTok{.}\AttributeTok{isWorker}\NormalTok{) \{}
  \BuiltInTok{process}\OperatorTok{.}\FunctionTok{on}\NormalTok{(}\StringTok{\textquotesingle{}message\textquotesingle{}}\OperatorTok{,}\NormalTok{ (msg) }\KeywordTok{=\textgreater{}}\NormalTok{ \{}
    \BuiltInTok{process}\OperatorTok{.}\FunctionTok{send}\NormalTok{(msg)}\OperatorTok{;}
\NormalTok{  \})}\OperatorTok{;}
\NormalTok{\}}
\end{Highlighting}
\end{Shaded}

\subsection{\texorpdfstring{Event:
\texttt{\textquotesingle{}disconnect\textquotesingle{}}}{Event: \textquotesingle disconnect\textquotesingle{}}}\label{event-disconnect-1}

\begin{itemize}
\tightlist
\item
  \texttt{worker} \{cluster.Worker\}
\end{itemize}

Emitted after the worker IPC channel has disconnected. This can occur
when a worker exits gracefully, is killed, or is disconnected manually
(such as with \texttt{worker.disconnect()}).

There may be a delay between the
\texttt{\textquotesingle{}disconnect\textquotesingle{}} and
\texttt{\textquotesingle{}exit\textquotesingle{}} events. These events
can be used to detect if the process is stuck in a cleanup or if there
are long-living connections.

\begin{Shaded}
\begin{Highlighting}[]
\NormalTok{cluster}\OperatorTok{.}\FunctionTok{on}\NormalTok{(}\StringTok{\textquotesingle{}disconnect\textquotesingle{}}\OperatorTok{,}\NormalTok{ (worker) }\KeywordTok{=\textgreater{}}\NormalTok{ \{}
  \BuiltInTok{console}\OperatorTok{.}\FunctionTok{log}\NormalTok{(}\VerbatimStringTok{\textasciigrave{}The worker \#}\SpecialCharTok{$\{}\NormalTok{worker}\OperatorTok{.}\AttributeTok{id}\SpecialCharTok{\}}\VerbatimStringTok{ has disconnected\textasciigrave{}}\NormalTok{)}\OperatorTok{;}
\NormalTok{\})}\OperatorTok{;}
\end{Highlighting}
\end{Shaded}

\subsection{\texorpdfstring{Event:
\texttt{\textquotesingle{}exit\textquotesingle{}}}{Event: \textquotesingle exit\textquotesingle{}}}\label{event-exit-1}

\begin{itemize}
\tightlist
\item
  \texttt{worker} \{cluster.Worker\}
\item
  \texttt{code} \{number\} The exit code, if it exited normally.
\item
  \texttt{signal} \{string\} The name of the signal
  (e.g.~\texttt{\textquotesingle{}SIGHUP\textquotesingle{}}) that caused
  the process to be killed.
\end{itemize}

When any of the workers die the cluster module will emit the
\texttt{\textquotesingle{}exit\textquotesingle{}} event.

This can be used to restart the worker by calling
\hyperref[clusterforkenv]{\texttt{.fork()}} again.

\begin{Shaded}
\begin{Highlighting}[]
\NormalTok{cluster}\OperatorTok{.}\FunctionTok{on}\NormalTok{(}\StringTok{\textquotesingle{}exit\textquotesingle{}}\OperatorTok{,}\NormalTok{ (worker}\OperatorTok{,}\NormalTok{ code}\OperatorTok{,}\NormalTok{ signal) }\KeywordTok{=\textgreater{}}\NormalTok{ \{}
  \BuiltInTok{console}\OperatorTok{.}\FunctionTok{log}\NormalTok{(}\StringTok{\textquotesingle{}worker \%d died (\%s). restarting...\textquotesingle{}}\OperatorTok{,}
\NormalTok{              worker}\OperatorTok{.}\AttributeTok{process}\OperatorTok{.}\AttributeTok{pid}\OperatorTok{,}\NormalTok{ signal }\OperatorTok{||}\NormalTok{ code)}\OperatorTok{;}
\NormalTok{  cluster}\OperatorTok{.}\FunctionTok{fork}\NormalTok{()}\OperatorTok{;}
\NormalTok{\})}\OperatorTok{;}
\end{Highlighting}
\end{Shaded}

See \href{child_process.md\#event-exit}{\texttt{child\_process} event:
\texttt{\textquotesingle{}exit\textquotesingle{}}}.

\subsection{\texorpdfstring{Event:
\texttt{\textquotesingle{}fork\textquotesingle{}}}{Event: \textquotesingle fork\textquotesingle{}}}\label{event-fork}

\begin{itemize}
\tightlist
\item
  \texttt{worker} \{cluster.Worker\}
\end{itemize}

When a new worker is forked the cluster module will emit a
\texttt{\textquotesingle{}fork\textquotesingle{}} event. This can be
used to log worker activity, and create a custom timeout.

\begin{Shaded}
\begin{Highlighting}[]
\KeywordTok{const}\NormalTok{ timeouts }\OperatorTok{=}\NormalTok{ []}\OperatorTok{;}
\KeywordTok{function} \FunctionTok{errorMsg}\NormalTok{() \{}
  \BuiltInTok{console}\OperatorTok{.}\FunctionTok{error}\NormalTok{(}\StringTok{\textquotesingle{}Something must be wrong with the connection ...\textquotesingle{}}\NormalTok{)}\OperatorTok{;}
\NormalTok{\}}

\NormalTok{cluster}\OperatorTok{.}\FunctionTok{on}\NormalTok{(}\StringTok{\textquotesingle{}fork\textquotesingle{}}\OperatorTok{,}\NormalTok{ (worker) }\KeywordTok{=\textgreater{}}\NormalTok{ \{}
\NormalTok{  timeouts[worker}\OperatorTok{.}\AttributeTok{id}\NormalTok{] }\OperatorTok{=} \PreprocessorTok{setTimeout}\NormalTok{(errorMsg}\OperatorTok{,} \DecValTok{2000}\NormalTok{)}\OperatorTok{;}
\NormalTok{\})}\OperatorTok{;}
\NormalTok{cluster}\OperatorTok{.}\FunctionTok{on}\NormalTok{(}\StringTok{\textquotesingle{}listening\textquotesingle{}}\OperatorTok{,}\NormalTok{ (worker}\OperatorTok{,}\NormalTok{ address) }\KeywordTok{=\textgreater{}}\NormalTok{ \{}
  \PreprocessorTok{clearTimeout}\NormalTok{(timeouts[worker}\OperatorTok{.}\AttributeTok{id}\NormalTok{])}\OperatorTok{;}
\NormalTok{\})}\OperatorTok{;}
\NormalTok{cluster}\OperatorTok{.}\FunctionTok{on}\NormalTok{(}\StringTok{\textquotesingle{}exit\textquotesingle{}}\OperatorTok{,}\NormalTok{ (worker}\OperatorTok{,}\NormalTok{ code}\OperatorTok{,}\NormalTok{ signal) }\KeywordTok{=\textgreater{}}\NormalTok{ \{}
  \PreprocessorTok{clearTimeout}\NormalTok{(timeouts[worker}\OperatorTok{.}\AttributeTok{id}\NormalTok{])}\OperatorTok{;}
  \FunctionTok{errorMsg}\NormalTok{()}\OperatorTok{;}
\NormalTok{\})}\OperatorTok{;}
\end{Highlighting}
\end{Shaded}

\subsection{\texorpdfstring{Event:
\texttt{\textquotesingle{}listening\textquotesingle{}}}{Event: \textquotesingle listening\textquotesingle{}}}\label{event-listening-1}

\begin{itemize}
\tightlist
\item
  \texttt{worker} \{cluster.Worker\}
\item
  \texttt{address} \{Object\}
\end{itemize}

After calling \texttt{listen()} from a worker, when the
\texttt{\textquotesingle{}listening\textquotesingle{}} event is emitted
on the server, a \texttt{\textquotesingle{}listening\textquotesingle{}}
event will also be emitted on \texttt{cluster} in the primary.

The event handler is executed with two arguments, the \texttt{worker}
contains the worker object and the \texttt{address} object contains the
following connection properties: \texttt{address}, \texttt{port}, and
\texttt{addressType}. This is very useful if the worker is listening on
more than one address.

\begin{Shaded}
\begin{Highlighting}[]
\NormalTok{cluster}\OperatorTok{.}\FunctionTok{on}\NormalTok{(}\StringTok{\textquotesingle{}listening\textquotesingle{}}\OperatorTok{,}\NormalTok{ (worker}\OperatorTok{,}\NormalTok{ address) }\KeywordTok{=\textgreater{}}\NormalTok{ \{}
  \BuiltInTok{console}\OperatorTok{.}\FunctionTok{log}\NormalTok{(}
    \VerbatimStringTok{\textasciigrave{}A worker is now connected to }\SpecialCharTok{$\{}\NormalTok{address}\OperatorTok{.}\AttributeTok{address}\SpecialCharTok{\}}\VerbatimStringTok{:}\SpecialCharTok{$\{}\NormalTok{address}\OperatorTok{.}\AttributeTok{port}\SpecialCharTok{\}}\VerbatimStringTok{\textasciigrave{}}\NormalTok{)}\OperatorTok{;}
\NormalTok{\})}\OperatorTok{;}
\end{Highlighting}
\end{Shaded}

The \texttt{addressType} is one of:

\begin{itemize}
\tightlist
\item
  \texttt{4} (TCPv4)
\item
  \texttt{6} (TCPv6)
\item
  \texttt{-1} (Unix domain socket)
\item
  \texttt{\textquotesingle{}udp4\textquotesingle{}} or
  \texttt{\textquotesingle{}udp6\textquotesingle{}} (UDPv4 or UDPv6)
\end{itemize}

\subsection{\texorpdfstring{Event:
\texttt{\textquotesingle{}message\textquotesingle{}}}{Event: \textquotesingle message\textquotesingle{}}}\label{event-message-1}

\begin{itemize}
\tightlist
\item
  \texttt{worker} \{cluster.Worker\}
\item
  \texttt{message} \{Object\}
\item
  \texttt{handle} \{undefined\textbar Object\}
\end{itemize}

Emitted when the cluster primary receives a message from any worker.

See \href{child_process.md\#event-message}{\texttt{child\_process}
event: \texttt{\textquotesingle{}message\textquotesingle{}}}.

\subsection{\texorpdfstring{Event:
\texttt{\textquotesingle{}online\textquotesingle{}}}{Event: \textquotesingle online\textquotesingle{}}}\label{event-online-1}

\begin{itemize}
\tightlist
\item
  \texttt{worker} \{cluster.Worker\}
\end{itemize}

After forking a new worker, the worker should respond with an online
message. When the primary receives an online message it will emit this
event. The difference between
\texttt{\textquotesingle{}fork\textquotesingle{}} and
\texttt{\textquotesingle{}online\textquotesingle{}} is that fork is
emitted when the primary forks a worker, and
\texttt{\textquotesingle{}online\textquotesingle{}} is emitted when the
worker is running.

\begin{Shaded}
\begin{Highlighting}[]
\NormalTok{cluster}\OperatorTok{.}\FunctionTok{on}\NormalTok{(}\StringTok{\textquotesingle{}online\textquotesingle{}}\OperatorTok{,}\NormalTok{ (worker) }\KeywordTok{=\textgreater{}}\NormalTok{ \{}
  \BuiltInTok{console}\OperatorTok{.}\FunctionTok{log}\NormalTok{(}\StringTok{\textquotesingle{}Yay, the worker responded after it was forked\textquotesingle{}}\NormalTok{)}\OperatorTok{;}
\NormalTok{\})}\OperatorTok{;}
\end{Highlighting}
\end{Shaded}

\subsection{\texorpdfstring{Event:
\texttt{\textquotesingle{}setup\textquotesingle{}}}{Event: \textquotesingle setup\textquotesingle{}}}\label{event-setup}

\begin{itemize}
\tightlist
\item
  \texttt{settings} \{Object\}
\end{itemize}

Emitted every time
\hyperref[clustersetupprimarysettings]{\texttt{.setupPrimary()}} is
called.

The \texttt{settings} object is the \texttt{cluster.settings} object at
the time
\hyperref[clustersetupprimarysettings]{\texttt{.setupPrimary()}} was
called and is advisory only, since multiple calls to
\hyperref[clustersetupprimarysettings]{\texttt{.setupPrimary()}} can be
made in a single tick.

If accuracy is important, use \texttt{cluster.settings}.

\subsection{\texorpdfstring{\texttt{cluster.disconnect({[}callback{]})}}{cluster.disconnect({[}callback{]})}}\label{cluster.disconnectcallback}

\begin{itemize}
\tightlist
\item
  \texttt{callback} \{Function\} Called when all workers are
  disconnected and handles are closed.
\end{itemize}

Calls \texttt{.disconnect()} on each worker in \texttt{cluster.workers}.

When they are disconnected all internal handles will be closed, allowing
the primary process to die gracefully if no other event is waiting.

The method takes an optional callback argument which will be called when
finished.

This can only be called from the primary process.

\subsection{\texorpdfstring{\texttt{cluster.fork({[}env{]})}}{cluster.fork({[}env{]})}}\label{cluster.forkenv}

\begin{itemize}
\tightlist
\item
  \texttt{env} \{Object\} Key/value pairs to add to worker process
  environment.
\item
  Returns: \{cluster.Worker\}
\end{itemize}

Spawn a new worker process.

This can only be called from the primary process.

\subsection{\texorpdfstring{\texttt{cluster.isMaster}}{cluster.isMaster}}\label{cluster.ismaster}

\begin{quote}
Stability: 0 - Deprecated
\end{quote}

Deprecated alias for
\hyperref[clusterisprimary]{\texttt{cluster.isPrimary}}.

\subsection{\texorpdfstring{\texttt{cluster.isPrimary}}{cluster.isPrimary}}\label{cluster.isprimary}

\begin{itemize}
\tightlist
\item
  \{boolean\}
\end{itemize}

True if the process is a primary. This is determined by the
\texttt{process.env.NODE\_UNIQUE\_ID}. If
\texttt{process.env.NODE\_UNIQUE\_ID} is undefined, then
\texttt{isPrimary} is \texttt{true}.

\subsection{\texorpdfstring{\texttt{cluster.isWorker}}{cluster.isWorker}}\label{cluster.isworker}

\begin{itemize}
\tightlist
\item
  \{boolean\}
\end{itemize}

True if the process is not a primary (it is the negation of
\texttt{cluster.isPrimary}).

\subsection{\texorpdfstring{\texttt{cluster.schedulingPolicy}}{cluster.schedulingPolicy}}\label{cluster.schedulingpolicy}

The scheduling policy, either \texttt{cluster.SCHED\_RR} for round-robin
or \texttt{cluster.SCHED\_NONE} to leave it to the operating system.
This is a global setting and effectively frozen once either the first
worker is spawned, or
\hyperref[clustersetupprimarysettings]{\texttt{.setupPrimary()}} is
called, whichever comes first.

\texttt{SCHED\_RR} is the default on all operating systems except
Windows. Windows will change to \texttt{SCHED\_RR} once libuv is able to
effectively distribute IOCP handles without incurring a large
performance hit.

\texttt{cluster.schedulingPolicy} can also be set through the
\texttt{NODE\_CLUSTER\_SCHED\_POLICY} environment variable. Valid values
are \texttt{\textquotesingle{}rr\textquotesingle{}} and
\texttt{\textquotesingle{}none\textquotesingle{}}.

\subsection{\texorpdfstring{\texttt{cluster.settings}}{cluster.settings}}\label{cluster.settings}

\begin{itemize}
\tightlist
\item
  \{Object\}

  \begin{itemize}
  \tightlist
  \item
    \texttt{execArgv} \{string{[}{]}\} List of string arguments passed
    to the Node.js executable. \textbf{Default:}
    \texttt{process.execArgv}.
  \item
    \texttt{exec} \{string\} File path to worker file. \textbf{Default:}
    \texttt{process.argv{[}1{]}}.
  \item
    \texttt{args} \{string{[}{]}\} String arguments passed to worker.
    \textbf{Default:} \texttt{process.argv.slice(2)}.
  \item
    \texttt{cwd} \{string\} Current working directory of the worker
    process. \textbf{Default:} \texttt{undefined} (inherits from parent
    process).
  \item
    \texttt{serialization} \{string\} Specify the kind of serialization
    used for sending messages between processes. Possible values are
    \texttt{\textquotesingle{}json\textquotesingle{}} and
    \texttt{\textquotesingle{}advanced\textquotesingle{}}. See
    \href{child_process.md\#advanced-serialization}{Advanced
    serialization for \texttt{child\_process}} for more details.
    \textbf{Default:} \texttt{false}.
  \item
    \texttt{silent} \{boolean\} Whether or not to send output to
    parent's stdio. \textbf{Default:} \texttt{false}.
  \item
    \texttt{stdio} \{Array\} Configures the stdio of forked processes.
    Because the cluster module relies on IPC to function, this
    configuration must contain an
    \texttt{\textquotesingle{}ipc\textquotesingle{}} entry. When this
    option is provided, it overrides \texttt{silent}. See
    \href{child_process.md\#child_processspawncommand-args-options}{\texttt{child\_process.spawn()}}'s
    \href{child_process.md\#optionsstdio}{\texttt{stdio}}.
  \item
    \texttt{uid} \{number\} Sets the user identity of the process. (See
    setuid(2).)
  \item
    \texttt{gid} \{number\} Sets the group identity of the process. (See
    setgid(2).)
  \item
    \texttt{inspectPort} \{number\textbar Function\} Sets inspector port
    of worker. This can be a number, or a function that takes no
    arguments and returns a number. By default each worker gets its own
    port, incremented from the primary's \texttt{process.debugPort}.
  \item
    \texttt{windowsHide} \{boolean\} Hide the forked processes console
    window that would normally be created on Windows systems.
    \textbf{Default:} \texttt{false}.
  \end{itemize}
\end{itemize}

After calling
\hyperref[clustersetupprimarysettings]{\texttt{.setupPrimary()}} (or
\hyperref[clusterforkenv]{\texttt{.fork()}}) this settings object will
contain the settings, including the default values.

This object is not intended to be changed or set manually.

\subsection{\texorpdfstring{\texttt{cluster.setupMaster({[}settings{]})}}{cluster.setupMaster({[}settings{]})}}\label{cluster.setupmastersettings}

\begin{quote}
Stability: 0 - Deprecated
\end{quote}

Deprecated alias for
\hyperref[clustersetupprimarysettings]{\texttt{.setupPrimary()}}.

\subsection{\texorpdfstring{\texttt{cluster.setupPrimary({[}settings{]})}}{cluster.setupPrimary({[}settings{]})}}\label{cluster.setupprimarysettings}

\begin{itemize}
\tightlist
\item
  \texttt{settings} \{Object\} See
  \hyperref[clustersettings]{\texttt{cluster.settings}}.
\end{itemize}

\texttt{setupPrimary} is used to change the default `fork' behavior.
Once called, the settings will be present in \texttt{cluster.settings}.

Any settings changes only affect future calls to
\hyperref[clusterforkenv]{\texttt{.fork()}} and have no effect on
workers that are already running.

The only attribute of a worker that cannot be set via
\texttt{.setupPrimary()} is the \texttt{env} passed to
\hyperref[clusterforkenv]{\texttt{.fork()}}.

The defaults above apply to the first call only; the defaults for later
calls are the current values at the time of
\texttt{cluster.setupPrimary()} is called.

\begin{Shaded}
\begin{Highlighting}[]
\ImportTok{import}\NormalTok{ cluster }\ImportTok{from} \StringTok{\textquotesingle{}node:cluster\textquotesingle{}}\OperatorTok{;}

\NormalTok{cluster}\OperatorTok{.}\FunctionTok{setupPrimary}\NormalTok{(\{}
  \DataTypeTok{exec}\OperatorTok{:} \StringTok{\textquotesingle{}worker.js\textquotesingle{}}\OperatorTok{,}
  \DataTypeTok{args}\OperatorTok{:}\NormalTok{ [}\StringTok{\textquotesingle{}{-}{-}use\textquotesingle{}}\OperatorTok{,} \StringTok{\textquotesingle{}https\textquotesingle{}}\NormalTok{]}\OperatorTok{,}
  \DataTypeTok{silent}\OperatorTok{:} \KeywordTok{true}\OperatorTok{,}
\NormalTok{\})}\OperatorTok{;}
\NormalTok{cluster}\OperatorTok{.}\FunctionTok{fork}\NormalTok{()}\OperatorTok{;} \CommentTok{// https worker}
\NormalTok{cluster}\OperatorTok{.}\FunctionTok{setupPrimary}\NormalTok{(\{}
  \DataTypeTok{exec}\OperatorTok{:} \StringTok{\textquotesingle{}worker.js\textquotesingle{}}\OperatorTok{,}
  \DataTypeTok{args}\OperatorTok{:}\NormalTok{ [}\StringTok{\textquotesingle{}{-}{-}use\textquotesingle{}}\OperatorTok{,} \StringTok{\textquotesingle{}http\textquotesingle{}}\NormalTok{]}\OperatorTok{,}
\NormalTok{\})}\OperatorTok{;}
\NormalTok{cluster}\OperatorTok{.}\FunctionTok{fork}\NormalTok{()}\OperatorTok{;} \CommentTok{// http worker}
\end{Highlighting}
\end{Shaded}

\begin{Shaded}
\begin{Highlighting}[]
\KeywordTok{const}\NormalTok{ cluster }\OperatorTok{=} \PreprocessorTok{require}\NormalTok{(}\StringTok{\textquotesingle{}node:cluster\textquotesingle{}}\NormalTok{)}\OperatorTok{;}

\NormalTok{cluster}\OperatorTok{.}\FunctionTok{setupPrimary}\NormalTok{(\{}
  \DataTypeTok{exec}\OperatorTok{:} \StringTok{\textquotesingle{}worker.js\textquotesingle{}}\OperatorTok{,}
  \DataTypeTok{args}\OperatorTok{:}\NormalTok{ [}\StringTok{\textquotesingle{}{-}{-}use\textquotesingle{}}\OperatorTok{,} \StringTok{\textquotesingle{}https\textquotesingle{}}\NormalTok{]}\OperatorTok{,}
  \DataTypeTok{silent}\OperatorTok{:} \KeywordTok{true}\OperatorTok{,}
\NormalTok{\})}\OperatorTok{;}
\NormalTok{cluster}\OperatorTok{.}\FunctionTok{fork}\NormalTok{()}\OperatorTok{;} \CommentTok{// https worker}
\NormalTok{cluster}\OperatorTok{.}\FunctionTok{setupPrimary}\NormalTok{(\{}
  \DataTypeTok{exec}\OperatorTok{:} \StringTok{\textquotesingle{}worker.js\textquotesingle{}}\OperatorTok{,}
  \DataTypeTok{args}\OperatorTok{:}\NormalTok{ [}\StringTok{\textquotesingle{}{-}{-}use\textquotesingle{}}\OperatorTok{,} \StringTok{\textquotesingle{}http\textquotesingle{}}\NormalTok{]}\OperatorTok{,}
\NormalTok{\})}\OperatorTok{;}
\NormalTok{cluster}\OperatorTok{.}\FunctionTok{fork}\NormalTok{()}\OperatorTok{;} \CommentTok{// http worker}
\end{Highlighting}
\end{Shaded}

This can only be called from the primary process.

\subsection{\texorpdfstring{\texttt{cluster.worker}}{cluster.worker}}\label{cluster.worker}

\begin{itemize}
\tightlist
\item
  \{Object\}
\end{itemize}

A reference to the current worker object. Not available in the primary
process.

\begin{Shaded}
\begin{Highlighting}[]
\ImportTok{import}\NormalTok{ cluster }\ImportTok{from} \StringTok{\textquotesingle{}node:cluster\textquotesingle{}}\OperatorTok{;}

\ControlFlowTok{if}\NormalTok{ (cluster}\OperatorTok{.}\AttributeTok{isPrimary}\NormalTok{) \{}
  \BuiltInTok{console}\OperatorTok{.}\FunctionTok{log}\NormalTok{(}\StringTok{\textquotesingle{}I am primary\textquotesingle{}}\NormalTok{)}\OperatorTok{;}
\NormalTok{  cluster}\OperatorTok{.}\FunctionTok{fork}\NormalTok{()}\OperatorTok{;}
\NormalTok{  cluster}\OperatorTok{.}\FunctionTok{fork}\NormalTok{()}\OperatorTok{;}
\NormalTok{\} }\ControlFlowTok{else} \ControlFlowTok{if}\NormalTok{ (cluster}\OperatorTok{.}\AttributeTok{isWorker}\NormalTok{) \{}
  \BuiltInTok{console}\OperatorTok{.}\FunctionTok{log}\NormalTok{(}\VerbatimStringTok{\textasciigrave{}I am worker \#}\SpecialCharTok{$\{}\NormalTok{cluster}\OperatorTok{.}\AttributeTok{worker}\OperatorTok{.}\AttributeTok{id}\SpecialCharTok{\}}\VerbatimStringTok{\textasciigrave{}}\NormalTok{)}\OperatorTok{;}
\NormalTok{\}}
\end{Highlighting}
\end{Shaded}

\begin{Shaded}
\begin{Highlighting}[]
\KeywordTok{const}\NormalTok{ cluster }\OperatorTok{=} \PreprocessorTok{require}\NormalTok{(}\StringTok{\textquotesingle{}node:cluster\textquotesingle{}}\NormalTok{)}\OperatorTok{;}

\ControlFlowTok{if}\NormalTok{ (cluster}\OperatorTok{.}\AttributeTok{isPrimary}\NormalTok{) \{}
  \BuiltInTok{console}\OperatorTok{.}\FunctionTok{log}\NormalTok{(}\StringTok{\textquotesingle{}I am primary\textquotesingle{}}\NormalTok{)}\OperatorTok{;}
\NormalTok{  cluster}\OperatorTok{.}\FunctionTok{fork}\NormalTok{()}\OperatorTok{;}
\NormalTok{  cluster}\OperatorTok{.}\FunctionTok{fork}\NormalTok{()}\OperatorTok{;}
\NormalTok{\} }\ControlFlowTok{else} \ControlFlowTok{if}\NormalTok{ (cluster}\OperatorTok{.}\AttributeTok{isWorker}\NormalTok{) \{}
  \BuiltInTok{console}\OperatorTok{.}\FunctionTok{log}\NormalTok{(}\VerbatimStringTok{\textasciigrave{}I am worker \#}\SpecialCharTok{$\{}\NormalTok{cluster}\OperatorTok{.}\AttributeTok{worker}\OperatorTok{.}\AttributeTok{id}\SpecialCharTok{\}}\VerbatimStringTok{\textasciigrave{}}\NormalTok{)}\OperatorTok{;}
\NormalTok{\}}
\end{Highlighting}
\end{Shaded}

\subsection{\texorpdfstring{\texttt{cluster.workers}}{cluster.workers}}\label{cluster.workers}

\begin{itemize}
\tightlist
\item
  \{Object\}
\end{itemize}

A hash that stores the active worker objects, keyed by \texttt{id}
field. This makes it easy to loop through all the workers. It is only
available in the primary process.

A worker is removed from \texttt{cluster.workers} after the worker has
disconnected \emph{and} exited. The order between these two events
cannot be determined in advance. However, it is guaranteed that the
removal from the \texttt{cluster.workers} list happens before the last
\texttt{\textquotesingle{}disconnect\textquotesingle{}} or
\texttt{\textquotesingle{}exit\textquotesingle{}} event is emitted.

\begin{Shaded}
\begin{Highlighting}[]
\ImportTok{import}\NormalTok{ cluster }\ImportTok{from} \StringTok{\textquotesingle{}node:cluster\textquotesingle{}}\OperatorTok{;}

\ControlFlowTok{for}\NormalTok{ (}\KeywordTok{const}\NormalTok{ worker }\KeywordTok{of} \BuiltInTok{Object}\OperatorTok{.}\FunctionTok{values}\NormalTok{(cluster}\OperatorTok{.}\AttributeTok{workers}\NormalTok{)) \{}
\NormalTok{  worker}\OperatorTok{.}\FunctionTok{send}\NormalTok{(}\StringTok{\textquotesingle{}big announcement to all workers\textquotesingle{}}\NormalTok{)}\OperatorTok{;}
\NormalTok{\}}
\end{Highlighting}
\end{Shaded}

\begin{Shaded}
\begin{Highlighting}[]
\KeywordTok{const}\NormalTok{ cluster }\OperatorTok{=} \PreprocessorTok{require}\NormalTok{(}\StringTok{\textquotesingle{}node:cluster\textquotesingle{}}\NormalTok{)}\OperatorTok{;}

\ControlFlowTok{for}\NormalTok{ (}\KeywordTok{const}\NormalTok{ worker }\KeywordTok{of} \BuiltInTok{Object}\OperatorTok{.}\FunctionTok{values}\NormalTok{(cluster}\OperatorTok{.}\AttributeTok{workers}\NormalTok{)) \{}
\NormalTok{  worker}\OperatorTok{.}\FunctionTok{send}\NormalTok{(}\StringTok{\textquotesingle{}big announcement to all workers\textquotesingle{}}\NormalTok{)}\OperatorTok{;}
\NormalTok{\}}
\end{Highlighting}
\end{Shaded}
