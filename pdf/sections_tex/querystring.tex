\section{Query string}\label{query-string}

\begin{quote}
Stability: 2 - Stable
\end{quote}

The \texttt{node:querystring} module provides utilities for parsing and
formatting URL query strings. It can be accessed using:

\begin{Shaded}
\begin{Highlighting}[]
\KeywordTok{const}\NormalTok{ querystring }\OperatorTok{=} \PreprocessorTok{require}\NormalTok{(}\StringTok{\textquotesingle{}node:querystring\textquotesingle{}}\NormalTok{)}\OperatorTok{;}
\end{Highlighting}
\end{Shaded}

\texttt{querystring} is more performant than \{URLSearchParams\} but is
not a standardized API. Use \{URLSearchParams\} when performance is not
critical or when compatibility with browser code is desirable.

\subsection{\texorpdfstring{\texttt{querystring.decode()}}{querystring.decode()}}\label{querystring.decode}

The \texttt{querystring.decode()} function is an alias for
\texttt{querystring.parse()}.

\subsection{\texorpdfstring{\texttt{querystring.encode()}}{querystring.encode()}}\label{querystring.encode}

The \texttt{querystring.encode()} function is an alias for
\texttt{querystring.stringify()}.

\subsection{\texorpdfstring{\texttt{querystring.escape(str)}}{querystring.escape(str)}}\label{querystring.escapestr}

\begin{itemize}
\tightlist
\item
  \texttt{str} \{string\}
\end{itemize}

The \texttt{querystring.escape()} method performs URL percent-encoding
on the given \texttt{str} in a manner that is optimized for the specific
requirements of URL query strings.

The \texttt{querystring.escape()} method is used by
\texttt{querystring.stringify()} and is generally not expected to be
used directly. It is exported primarily to allow application code to
provide a replacement percent-encoding implementation if necessary by
assigning \texttt{querystring.escape} to an alternative function.

\subsection{\texorpdfstring{\texttt{querystring.parse(str{[},\ sep{[},\ eq{[},\ options{]}{]}{]})}}{querystring.parse(str{[}, sep{[}, eq{[}, options{]}{]}{]})}}\label{querystring.parsestr-sep-eq-options}

\begin{itemize}
\tightlist
\item
  \texttt{str} \{string\} The URL query string to parse
\item
  \texttt{sep} \{string\} The substring used to delimit key and value
  pairs in the query string. \textbf{Default:}
  \texttt{\textquotesingle{}\&\textquotesingle{}}.
\item
  \texttt{eq} \{string\}. The substring used to delimit keys and values
  in the query string. \textbf{Default:}
  \texttt{\textquotesingle{}=\textquotesingle{}}.
\item
  \texttt{options} \{Object\}

  \begin{itemize}
  \tightlist
  \item
    \texttt{decodeURIComponent} \{Function\} The function to use when
    decoding percent-encoded characters in the query string.
    \textbf{Default:} \texttt{querystring.unescape()}.
  \item
    \texttt{maxKeys} \{number\} Specifies the maximum number of keys to
    parse. Specify \texttt{0} to remove key counting limitations.
    \textbf{Default:} \texttt{1000}.
  \end{itemize}
\end{itemize}

The \texttt{querystring.parse()} method parses a URL query string
(\texttt{str}) into a collection of key and value pairs.

For example, the query string
\texttt{\textquotesingle{}foo=bar\&abc=xyz\&abc=123\textquotesingle{}}
is parsed into:

\begin{Shaded}
\begin{Highlighting}[]
\FunctionTok{\{}
  \DataTypeTok{"foo"}\FunctionTok{:} \StringTok{"bar"}\FunctionTok{,}
  \DataTypeTok{"abc"}\FunctionTok{:} \OtherTok{[}\StringTok{"xyz"}\OtherTok{,} \StringTok{"123"}\OtherTok{]}
\FunctionTok{\}}
\end{Highlighting}
\end{Shaded}

The object returned by the \texttt{querystring.parse()} method
\emph{does not} prototypically inherit from the JavaScript
\texttt{Object}. This means that typical \texttt{Object} methods such as
\texttt{obj.toString()}, \texttt{obj.hasOwnProperty()}, and others are
not defined and \emph{will not work}.

By default, percent-encoded characters within the query string will be
assumed to use UTF-8 encoding. If an alternative character encoding is
used, then an alternative \texttt{decodeURIComponent} option will need
to be specified:

\begin{Shaded}
\begin{Highlighting}[]
\CommentTok{// Assuming gbkDecodeURIComponent function already exists...}

\NormalTok{querystring}\OperatorTok{.}\FunctionTok{parse}\NormalTok{(}\StringTok{\textquotesingle{}w=\%D6\%D0\%CE\%C4\&foo=bar\textquotesingle{}}\OperatorTok{,} \KeywordTok{null}\OperatorTok{,} \KeywordTok{null}\OperatorTok{,}
\NormalTok{                  \{ }\DataTypeTok{decodeURIComponent}\OperatorTok{:}\NormalTok{ gbkDecodeURIComponent \})}\OperatorTok{;}
\end{Highlighting}
\end{Shaded}

\subsection{\texorpdfstring{\texttt{querystring.stringify(obj{[},\ sep{[},\ eq{[},\ options{]}{]}{]})}}{querystring.stringify(obj{[}, sep{[}, eq{[}, options{]}{]}{]})}}\label{querystring.stringifyobj-sep-eq-options}

\begin{itemize}
\tightlist
\item
  \texttt{obj} \{Object\} The object to serialize into a URL query
  string
\item
  \texttt{sep} \{string\} The substring used to delimit key and value
  pairs in the query string. \textbf{Default:}
  \texttt{\textquotesingle{}\&\textquotesingle{}}.
\item
  \texttt{eq} \{string\}. The substring used to delimit keys and values
  in the query string. \textbf{Default:}
  \texttt{\textquotesingle{}=\textquotesingle{}}.
\item
  \texttt{options}

  \begin{itemize}
  \tightlist
  \item
    \texttt{encodeURIComponent} \{Function\} The function to use when
    converting URL-unsafe characters to percent-encoding in the query
    string. \textbf{Default:} \texttt{querystring.escape()}.
  \end{itemize}
\end{itemize}

The \texttt{querystring.stringify()} method produces a URL query string
from a given \texttt{obj} by iterating through the object's ``own
properties''.

It serializes the following types of values passed in \texttt{obj}:
\{string\textbar number\textbar bigint\textbar boolean\textbar string{[}{]}\textbar number{[}{]}\textbar bigint{[}{]}\textbar boolean{[}{]}\}
The numeric values must be finite. Any other input values will be
coerced to empty strings.

\begin{Shaded}
\begin{Highlighting}[]
\NormalTok{querystring}\OperatorTok{.}\FunctionTok{stringify}\NormalTok{(\{ }\DataTypeTok{foo}\OperatorTok{:} \StringTok{\textquotesingle{}bar\textquotesingle{}}\OperatorTok{,} \DataTypeTok{baz}\OperatorTok{:}\NormalTok{ [}\StringTok{\textquotesingle{}qux\textquotesingle{}}\OperatorTok{,} \StringTok{\textquotesingle{}quux\textquotesingle{}}\NormalTok{]}\OperatorTok{,} \DataTypeTok{corge}\OperatorTok{:} \StringTok{\textquotesingle{}\textquotesingle{}}\NormalTok{ \})}\OperatorTok{;}
\CommentTok{// Returns \textquotesingle{}foo=bar\&baz=qux\&baz=quux\&corge=\textquotesingle{}}

\NormalTok{querystring}\OperatorTok{.}\FunctionTok{stringify}\NormalTok{(\{ }\DataTypeTok{foo}\OperatorTok{:} \StringTok{\textquotesingle{}bar\textquotesingle{}}\OperatorTok{,} \DataTypeTok{baz}\OperatorTok{:} \StringTok{\textquotesingle{}qux\textquotesingle{}}\NormalTok{ \}}\OperatorTok{,} \StringTok{\textquotesingle{};\textquotesingle{}}\OperatorTok{,} \StringTok{\textquotesingle{}:\textquotesingle{}}\NormalTok{)}\OperatorTok{;}
\CommentTok{// Returns \textquotesingle{}foo:bar;baz:qux\textquotesingle{}}
\end{Highlighting}
\end{Shaded}

By default, characters requiring percent-encoding within the query
string will be encoded as UTF-8. If an alternative encoding is required,
then an alternative \texttt{encodeURIComponent} option will need to be
specified:

\begin{Shaded}
\begin{Highlighting}[]
\CommentTok{// Assuming gbkEncodeURIComponent function already exists,}

\NormalTok{querystring}\OperatorTok{.}\FunctionTok{stringify}\NormalTok{(\{ }\DataTypeTok{w}\OperatorTok{:} \StringTok{\textquotesingle{}中文\textquotesingle{}}\OperatorTok{,} \DataTypeTok{foo}\OperatorTok{:} \StringTok{\textquotesingle{}bar\textquotesingle{}}\NormalTok{ \}}\OperatorTok{,} \KeywordTok{null}\OperatorTok{,} \KeywordTok{null}\OperatorTok{,}
\NormalTok{                      \{ }\DataTypeTok{encodeURIComponent}\OperatorTok{:}\NormalTok{ gbkEncodeURIComponent \})}\OperatorTok{;}
\end{Highlighting}
\end{Shaded}

\subsection{\texorpdfstring{\texttt{querystring.unescape(str)}}{querystring.unescape(str)}}\label{querystring.unescapestr}

\begin{itemize}
\tightlist
\item
  \texttt{str} \{string\}
\end{itemize}

The \texttt{querystring.unescape()} method performs decoding of URL
percent-encoded characters on the given \texttt{str}.

The \texttt{querystring.unescape()} method is used by
\texttt{querystring.parse()} and is generally not expected to be used
directly. It is exported primarily to allow application code to provide
a replacement decoding implementation if necessary by assigning
\texttt{querystring.unescape} to an alternative function.

By default, the \texttt{querystring.unescape()} method will attempt to
use the JavaScript built-in \texttt{decodeURIComponent()} method to
decode. If that fails, a safer equivalent that does not throw on
malformed URLs will be used.
