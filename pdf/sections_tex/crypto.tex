\section{Crypto}\label{crypto}

\begin{quote}
Stability: 2 - Stable
\end{quote}

The \texttt{node:crypto} module provides cryptographic functionality
that includes a set of wrappers for OpenSSL's hash, HMAC, cipher,
decipher, sign, and verify functions.

\begin{Shaded}
\begin{Highlighting}[]
\KeywordTok{const}\NormalTok{ \{ createHmac \} }\OperatorTok{=} \ControlFlowTok{await} \ImportTok{import}\NormalTok{(}\StringTok{\textquotesingle{}node:crypto\textquotesingle{}}\NormalTok{)}\OperatorTok{;}

\KeywordTok{const}\NormalTok{ secret }\OperatorTok{=} \StringTok{\textquotesingle{}abcdefg\textquotesingle{}}\OperatorTok{;}
\KeywordTok{const}\NormalTok{ hash }\OperatorTok{=} \FunctionTok{createHmac}\NormalTok{(}\StringTok{\textquotesingle{}sha256\textquotesingle{}}\OperatorTok{,}\NormalTok{ secret)}
               \OperatorTok{.}\FunctionTok{update}\NormalTok{(}\StringTok{\textquotesingle{}I love cupcakes\textquotesingle{}}\NormalTok{)}
               \OperatorTok{.}\FunctionTok{digest}\NormalTok{(}\StringTok{\textquotesingle{}hex\textquotesingle{}}\NormalTok{)}\OperatorTok{;}
\BuiltInTok{console}\OperatorTok{.}\FunctionTok{log}\NormalTok{(hash)}\OperatorTok{;}
\CommentTok{// Prints:}
\CommentTok{//   c0fa1bc00531bd78ef38c628449c5102aeabd49b5dc3a2a516ea6ea959d6658e}
\end{Highlighting}
\end{Shaded}

\begin{Shaded}
\begin{Highlighting}[]
\KeywordTok{const}\NormalTok{ \{ createHmac \} }\OperatorTok{=} \PreprocessorTok{require}\NormalTok{(}\StringTok{\textquotesingle{}node:crypto\textquotesingle{}}\NormalTok{)}\OperatorTok{;}

\KeywordTok{const}\NormalTok{ secret }\OperatorTok{=} \StringTok{\textquotesingle{}abcdefg\textquotesingle{}}\OperatorTok{;}
\KeywordTok{const}\NormalTok{ hash }\OperatorTok{=} \FunctionTok{createHmac}\NormalTok{(}\StringTok{\textquotesingle{}sha256\textquotesingle{}}\OperatorTok{,}\NormalTok{ secret)}
               \OperatorTok{.}\FunctionTok{update}\NormalTok{(}\StringTok{\textquotesingle{}I love cupcakes\textquotesingle{}}\NormalTok{)}
               \OperatorTok{.}\FunctionTok{digest}\NormalTok{(}\StringTok{\textquotesingle{}hex\textquotesingle{}}\NormalTok{)}\OperatorTok{;}
\BuiltInTok{console}\OperatorTok{.}\FunctionTok{log}\NormalTok{(hash)}\OperatorTok{;}
\CommentTok{// Prints:}
\CommentTok{//   c0fa1bc00531bd78ef38c628449c5102aeabd49b5dc3a2a516ea6ea959d6658e}
\end{Highlighting}
\end{Shaded}

\subsection{Determining if crypto support is
unavailable}\label{determining-if-crypto-support-is-unavailable}

It is possible for Node.js to be built without including support for the
\texttt{node:crypto} module. In such cases, attempting to
\texttt{import} from \texttt{crypto} or calling
\texttt{require(\textquotesingle{}node:crypto\textquotesingle{})} will
result in an error being thrown.

When using CommonJS, the error thrown can be caught using try/catch:

\begin{Shaded}
\begin{Highlighting}[]
\KeywordTok{let}\NormalTok{ crypto}\OperatorTok{;}
\ControlFlowTok{try}\NormalTok{ \{}
\NormalTok{  crypto }\OperatorTok{=} \PreprocessorTok{require}\NormalTok{(}\StringTok{\textquotesingle{}node:crypto\textquotesingle{}}\NormalTok{)}\OperatorTok{;}
\NormalTok{\} }\ControlFlowTok{catch}\NormalTok{ (err) \{}
  \BuiltInTok{console}\OperatorTok{.}\FunctionTok{error}\NormalTok{(}\StringTok{\textquotesingle{}crypto support is disabled!\textquotesingle{}}\NormalTok{)}\OperatorTok{;}
\NormalTok{\}}
\end{Highlighting}
\end{Shaded}

When using the lexical ESM \texttt{import} keyword, the error can only
be caught if a handler for
\texttt{process.on(\textquotesingle{}uncaughtException\textquotesingle{})}
is registered \emph{before} any attempt to load the module is made
(using, for instance, a preload module).

When using ESM, if there is a chance that the code may be run on a build
of Node.js where crypto support is not enabled, consider using the
\href{https://developer.mozilla.org/en-US/docs/Web/JavaScript/Reference/Operators/import}{\texttt{import()}}
function instead of the lexical \texttt{import} keyword:

\begin{Shaded}
\begin{Highlighting}[]
\KeywordTok{let}\NormalTok{ crypto}\OperatorTok{;}
\ControlFlowTok{try}\NormalTok{ \{}
\NormalTok{  crypto }\OperatorTok{=} \ControlFlowTok{await} \ImportTok{import}\NormalTok{(}\StringTok{\textquotesingle{}node:crypto\textquotesingle{}}\NormalTok{)}\OperatorTok{;}
\NormalTok{\} }\ControlFlowTok{catch}\NormalTok{ (err) \{}
  \BuiltInTok{console}\OperatorTok{.}\FunctionTok{error}\NormalTok{(}\StringTok{\textquotesingle{}crypto support is disabled!\textquotesingle{}}\NormalTok{)}\OperatorTok{;}
\NormalTok{\}}
\end{Highlighting}
\end{Shaded}

\subsection{\texorpdfstring{Class:
\texttt{Certificate}}{Class: Certificate}}\label{class-certificate}

SPKAC is a Certificate Signing Request mechanism originally implemented
by Netscape and was specified formally as part of HTML5's
\texttt{keygen} element.

\texttt{\textless{}keygen\textgreater{}} is deprecated since
\href{https://www.w3.org/TR/html52/changes.html\#features-removed}{HTML
5.2} and new projects should not use this element anymore.

The \texttt{node:crypto} module provides the \texttt{Certificate} class
for working with SPKAC data. The most common usage is handling output
generated by the HTML5 \texttt{\textless{}keygen\textgreater{}} element.
Node.js uses
\href{https://www.openssl.org/docs/man3.0/man1/openssl-spkac.html}{OpenSSL's
SPKAC implementation} internally.

\subsubsection{\texorpdfstring{Static method:
\texttt{Certificate.exportChallenge(spkac{[},\ encoding{]})}}{Static method: Certificate.exportChallenge(spkac{[}, encoding{]})}}\label{static-method-certificate.exportchallengespkac-encoding}

\begin{itemize}
\tightlist
\item
  \texttt{spkac}
  \{string\textbar ArrayBuffer\textbar Buffer\textbar TypedArray\textbar DataView\}
\item
  \texttt{encoding} \{string\} The
  \href{buffer.md\#buffers-and-character-encodings}{encoding} of the
  \texttt{spkac} string.
\item
  Returns: \{Buffer\} The challenge component of the \texttt{spkac} data
  structure, which includes a public key and a challenge.
\end{itemize}

\begin{Shaded}
\begin{Highlighting}[]
\KeywordTok{const}\NormalTok{ \{ Certificate \} }\OperatorTok{=} \ControlFlowTok{await} \ImportTok{import}\NormalTok{(}\StringTok{\textquotesingle{}node:crypto\textquotesingle{}}\NormalTok{)}\OperatorTok{;}
\KeywordTok{const}\NormalTok{ spkac }\OperatorTok{=} \FunctionTok{getSpkacSomehow}\NormalTok{()}\OperatorTok{;}
\KeywordTok{const}\NormalTok{ challenge }\OperatorTok{=}\NormalTok{ Certificate}\OperatorTok{.}\FunctionTok{exportChallenge}\NormalTok{(spkac)}\OperatorTok{;}
\BuiltInTok{console}\OperatorTok{.}\FunctionTok{log}\NormalTok{(challenge}\OperatorTok{.}\FunctionTok{toString}\NormalTok{(}\StringTok{\textquotesingle{}utf8\textquotesingle{}}\NormalTok{))}\OperatorTok{;}
\CommentTok{// Prints: the challenge as a UTF8 string}
\end{Highlighting}
\end{Shaded}

\begin{Shaded}
\begin{Highlighting}[]
\KeywordTok{const}\NormalTok{ \{ Certificate \} }\OperatorTok{=} \PreprocessorTok{require}\NormalTok{(}\StringTok{\textquotesingle{}node:crypto\textquotesingle{}}\NormalTok{)}\OperatorTok{;}
\KeywordTok{const}\NormalTok{ spkac }\OperatorTok{=} \FunctionTok{getSpkacSomehow}\NormalTok{()}\OperatorTok{;}
\KeywordTok{const}\NormalTok{ challenge }\OperatorTok{=}\NormalTok{ Certificate}\OperatorTok{.}\FunctionTok{exportChallenge}\NormalTok{(spkac)}\OperatorTok{;}
\BuiltInTok{console}\OperatorTok{.}\FunctionTok{log}\NormalTok{(challenge}\OperatorTok{.}\FunctionTok{toString}\NormalTok{(}\StringTok{\textquotesingle{}utf8\textquotesingle{}}\NormalTok{))}\OperatorTok{;}
\CommentTok{// Prints: the challenge as a UTF8 string}
\end{Highlighting}
\end{Shaded}

\subsubsection{\texorpdfstring{Static method:
\texttt{Certificate.exportPublicKey(spkac{[},\ encoding{]})}}{Static method: Certificate.exportPublicKey(spkac{[}, encoding{]})}}\label{static-method-certificate.exportpublickeyspkac-encoding}

\begin{itemize}
\tightlist
\item
  \texttt{spkac}
  \{string\textbar ArrayBuffer\textbar Buffer\textbar TypedArray\textbar DataView\}
\item
  \texttt{encoding} \{string\} The
  \href{buffer.md\#buffers-and-character-encodings}{encoding} of the
  \texttt{spkac} string.
\item
  Returns: \{Buffer\} The public key component of the \texttt{spkac}
  data structure, which includes a public key and a challenge.
\end{itemize}

\begin{Shaded}
\begin{Highlighting}[]
\KeywordTok{const}\NormalTok{ \{ Certificate \} }\OperatorTok{=} \ControlFlowTok{await} \ImportTok{import}\NormalTok{(}\StringTok{\textquotesingle{}node:crypto\textquotesingle{}}\NormalTok{)}\OperatorTok{;}
\KeywordTok{const}\NormalTok{ spkac }\OperatorTok{=} \FunctionTok{getSpkacSomehow}\NormalTok{()}\OperatorTok{;}
\KeywordTok{const}\NormalTok{ publicKey }\OperatorTok{=}\NormalTok{ Certificate}\OperatorTok{.}\FunctionTok{exportPublicKey}\NormalTok{(spkac)}\OperatorTok{;}
\BuiltInTok{console}\OperatorTok{.}\FunctionTok{log}\NormalTok{(publicKey)}\OperatorTok{;}
\CommentTok{// Prints: the public key as \textless{}Buffer ...\textgreater{}}
\end{Highlighting}
\end{Shaded}

\begin{Shaded}
\begin{Highlighting}[]
\KeywordTok{const}\NormalTok{ \{ Certificate \} }\OperatorTok{=} \PreprocessorTok{require}\NormalTok{(}\StringTok{\textquotesingle{}node:crypto\textquotesingle{}}\NormalTok{)}\OperatorTok{;}
\KeywordTok{const}\NormalTok{ spkac }\OperatorTok{=} \FunctionTok{getSpkacSomehow}\NormalTok{()}\OperatorTok{;}
\KeywordTok{const}\NormalTok{ publicKey }\OperatorTok{=}\NormalTok{ Certificate}\OperatorTok{.}\FunctionTok{exportPublicKey}\NormalTok{(spkac)}\OperatorTok{;}
\BuiltInTok{console}\OperatorTok{.}\FunctionTok{log}\NormalTok{(publicKey)}\OperatorTok{;}
\CommentTok{// Prints: the public key as \textless{}Buffer ...\textgreater{}}
\end{Highlighting}
\end{Shaded}

\subsubsection{\texorpdfstring{Static method:
\texttt{Certificate.verifySpkac(spkac{[},\ encoding{]})}}{Static method: Certificate.verifySpkac(spkac{[}, encoding{]})}}\label{static-method-certificate.verifyspkacspkac-encoding}

\begin{itemize}
\tightlist
\item
  \texttt{spkac}
  \{string\textbar ArrayBuffer\textbar Buffer\textbar TypedArray\textbar DataView\}
\item
  \texttt{encoding} \{string\} The
  \href{buffer.md\#buffers-and-character-encodings}{encoding} of the
  \texttt{spkac} string.
\item
  Returns: \{boolean\} \texttt{true} if the given \texttt{spkac} data
  structure is valid, \texttt{false} otherwise.
\end{itemize}

\begin{Shaded}
\begin{Highlighting}[]
\ImportTok{import}\NormalTok{ \{ }\BuiltInTok{Buffer}\NormalTok{ \} }\ImportTok{from} \StringTok{\textquotesingle{}node:buffer\textquotesingle{}}\OperatorTok{;}
\KeywordTok{const}\NormalTok{ \{ Certificate \} }\OperatorTok{=} \ControlFlowTok{await} \ImportTok{import}\NormalTok{(}\StringTok{\textquotesingle{}node:crypto\textquotesingle{}}\NormalTok{)}\OperatorTok{;}

\KeywordTok{const}\NormalTok{ spkac }\OperatorTok{=} \FunctionTok{getSpkacSomehow}\NormalTok{()}\OperatorTok{;}
\BuiltInTok{console}\OperatorTok{.}\FunctionTok{log}\NormalTok{(Certificate}\OperatorTok{.}\FunctionTok{verifySpkac}\NormalTok{(}\BuiltInTok{Buffer}\OperatorTok{.}\FunctionTok{from}\NormalTok{(spkac)))}\OperatorTok{;}
\CommentTok{// Prints: true or false}
\end{Highlighting}
\end{Shaded}

\begin{Shaded}
\begin{Highlighting}[]
\KeywordTok{const}\NormalTok{ \{ }\BuiltInTok{Buffer}\NormalTok{ \} }\OperatorTok{=} \PreprocessorTok{require}\NormalTok{(}\StringTok{\textquotesingle{}node:buffer\textquotesingle{}}\NormalTok{)}\OperatorTok{;}
\KeywordTok{const}\NormalTok{ \{ Certificate \} }\OperatorTok{=} \PreprocessorTok{require}\NormalTok{(}\StringTok{\textquotesingle{}node:crypto\textquotesingle{}}\NormalTok{)}\OperatorTok{;}

\KeywordTok{const}\NormalTok{ spkac }\OperatorTok{=} \FunctionTok{getSpkacSomehow}\NormalTok{()}\OperatorTok{;}
\BuiltInTok{console}\OperatorTok{.}\FunctionTok{log}\NormalTok{(Certificate}\OperatorTok{.}\FunctionTok{verifySpkac}\NormalTok{(}\BuiltInTok{Buffer}\OperatorTok{.}\FunctionTok{from}\NormalTok{(spkac)))}\OperatorTok{;}
\CommentTok{// Prints: true or false}
\end{Highlighting}
\end{Shaded}

\subsubsection{Legacy API}\label{legacy-api}

\begin{quote}
Stability: 0 - Deprecated
\end{quote}

As a legacy interface, it is possible to create new instances of the
\texttt{crypto.Certificate} class as illustrated in the examples below.

\paragraph{\texorpdfstring{\texttt{new\ crypto.Certificate()}}{new crypto.Certificate()}}\label{new-crypto.certificate}

Instances of the \texttt{Certificate} class can be created using the
\texttt{new} keyword or by calling \texttt{crypto.Certificate()} as a
function:

\begin{Shaded}
\begin{Highlighting}[]
\KeywordTok{const}\NormalTok{ \{ Certificate \} }\OperatorTok{=} \ControlFlowTok{await} \ImportTok{import}\NormalTok{(}\StringTok{\textquotesingle{}node:crypto\textquotesingle{}}\NormalTok{)}\OperatorTok{;}

\KeywordTok{const}\NormalTok{ cert1 }\OperatorTok{=} \KeywordTok{new} \FunctionTok{Certificate}\NormalTok{()}\OperatorTok{;}
\KeywordTok{const}\NormalTok{ cert2 }\OperatorTok{=} \FunctionTok{Certificate}\NormalTok{()}\OperatorTok{;}
\end{Highlighting}
\end{Shaded}

\begin{Shaded}
\begin{Highlighting}[]
\KeywordTok{const}\NormalTok{ \{ Certificate \} }\OperatorTok{=} \PreprocessorTok{require}\NormalTok{(}\StringTok{\textquotesingle{}node:crypto\textquotesingle{}}\NormalTok{)}\OperatorTok{;}

\KeywordTok{const}\NormalTok{ cert1 }\OperatorTok{=} \KeywordTok{new} \FunctionTok{Certificate}\NormalTok{()}\OperatorTok{;}
\KeywordTok{const}\NormalTok{ cert2 }\OperatorTok{=} \FunctionTok{Certificate}\NormalTok{()}\OperatorTok{;}
\end{Highlighting}
\end{Shaded}

\paragraph{\texorpdfstring{\texttt{certificate.exportChallenge(spkac{[},\ encoding{]})}}{certificate.exportChallenge(spkac{[}, encoding{]})}}\label{certificate.exportchallengespkac-encoding}

\begin{itemize}
\tightlist
\item
  \texttt{spkac}
  \{string\textbar ArrayBuffer\textbar Buffer\textbar TypedArray\textbar DataView\}
\item
  \texttt{encoding} \{string\} The
  \href{buffer.md\#buffers-and-character-encodings}{encoding} of the
  \texttt{spkac} string.
\item
  Returns: \{Buffer\} The challenge component of the \texttt{spkac} data
  structure, which includes a public key and a challenge.
\end{itemize}

\begin{Shaded}
\begin{Highlighting}[]
\KeywordTok{const}\NormalTok{ \{ Certificate \} }\OperatorTok{=} \ControlFlowTok{await} \ImportTok{import}\NormalTok{(}\StringTok{\textquotesingle{}node:crypto\textquotesingle{}}\NormalTok{)}\OperatorTok{;}
\KeywordTok{const}\NormalTok{ cert }\OperatorTok{=} \FunctionTok{Certificate}\NormalTok{()}\OperatorTok{;}
\KeywordTok{const}\NormalTok{ spkac }\OperatorTok{=} \FunctionTok{getSpkacSomehow}\NormalTok{()}\OperatorTok{;}
\KeywordTok{const}\NormalTok{ challenge }\OperatorTok{=}\NormalTok{ cert}\OperatorTok{.}\FunctionTok{exportChallenge}\NormalTok{(spkac)}\OperatorTok{;}
\BuiltInTok{console}\OperatorTok{.}\FunctionTok{log}\NormalTok{(challenge}\OperatorTok{.}\FunctionTok{toString}\NormalTok{(}\StringTok{\textquotesingle{}utf8\textquotesingle{}}\NormalTok{))}\OperatorTok{;}
\CommentTok{// Prints: the challenge as a UTF8 string}
\end{Highlighting}
\end{Shaded}

\begin{Shaded}
\begin{Highlighting}[]
\KeywordTok{const}\NormalTok{ \{ Certificate \} }\OperatorTok{=} \PreprocessorTok{require}\NormalTok{(}\StringTok{\textquotesingle{}node:crypto\textquotesingle{}}\NormalTok{)}\OperatorTok{;}
\KeywordTok{const}\NormalTok{ cert }\OperatorTok{=} \FunctionTok{Certificate}\NormalTok{()}\OperatorTok{;}
\KeywordTok{const}\NormalTok{ spkac }\OperatorTok{=} \FunctionTok{getSpkacSomehow}\NormalTok{()}\OperatorTok{;}
\KeywordTok{const}\NormalTok{ challenge }\OperatorTok{=}\NormalTok{ cert}\OperatorTok{.}\FunctionTok{exportChallenge}\NormalTok{(spkac)}\OperatorTok{;}
\BuiltInTok{console}\OperatorTok{.}\FunctionTok{log}\NormalTok{(challenge}\OperatorTok{.}\FunctionTok{toString}\NormalTok{(}\StringTok{\textquotesingle{}utf8\textquotesingle{}}\NormalTok{))}\OperatorTok{;}
\CommentTok{// Prints: the challenge as a UTF8 string}
\end{Highlighting}
\end{Shaded}

\paragraph{\texorpdfstring{\texttt{certificate.exportPublicKey(spkac{[},\ encoding{]})}}{certificate.exportPublicKey(spkac{[}, encoding{]})}}\label{certificate.exportpublickeyspkac-encoding}

\begin{itemize}
\tightlist
\item
  \texttt{spkac}
  \{string\textbar ArrayBuffer\textbar Buffer\textbar TypedArray\textbar DataView\}
\item
  \texttt{encoding} \{string\} The
  \href{buffer.md\#buffers-and-character-encodings}{encoding} of the
  \texttt{spkac} string.
\item
  Returns: \{Buffer\} The public key component of the \texttt{spkac}
  data structure, which includes a public key and a challenge.
\end{itemize}

\begin{Shaded}
\begin{Highlighting}[]
\KeywordTok{const}\NormalTok{ \{ Certificate \} }\OperatorTok{=} \ControlFlowTok{await} \ImportTok{import}\NormalTok{(}\StringTok{\textquotesingle{}node:crypto\textquotesingle{}}\NormalTok{)}\OperatorTok{;}
\KeywordTok{const}\NormalTok{ cert }\OperatorTok{=} \FunctionTok{Certificate}\NormalTok{()}\OperatorTok{;}
\KeywordTok{const}\NormalTok{ spkac }\OperatorTok{=} \FunctionTok{getSpkacSomehow}\NormalTok{()}\OperatorTok{;}
\KeywordTok{const}\NormalTok{ publicKey }\OperatorTok{=}\NormalTok{ cert}\OperatorTok{.}\FunctionTok{exportPublicKey}\NormalTok{(spkac)}\OperatorTok{;}
\BuiltInTok{console}\OperatorTok{.}\FunctionTok{log}\NormalTok{(publicKey)}\OperatorTok{;}
\CommentTok{// Prints: the public key as \textless{}Buffer ...\textgreater{}}
\end{Highlighting}
\end{Shaded}

\begin{Shaded}
\begin{Highlighting}[]
\KeywordTok{const}\NormalTok{ \{ Certificate \} }\OperatorTok{=} \PreprocessorTok{require}\NormalTok{(}\StringTok{\textquotesingle{}node:crypto\textquotesingle{}}\NormalTok{)}\OperatorTok{;}
\KeywordTok{const}\NormalTok{ cert }\OperatorTok{=} \FunctionTok{Certificate}\NormalTok{()}\OperatorTok{;}
\KeywordTok{const}\NormalTok{ spkac }\OperatorTok{=} \FunctionTok{getSpkacSomehow}\NormalTok{()}\OperatorTok{;}
\KeywordTok{const}\NormalTok{ publicKey }\OperatorTok{=}\NormalTok{ cert}\OperatorTok{.}\FunctionTok{exportPublicKey}\NormalTok{(spkac)}\OperatorTok{;}
\BuiltInTok{console}\OperatorTok{.}\FunctionTok{log}\NormalTok{(publicKey)}\OperatorTok{;}
\CommentTok{// Prints: the public key as \textless{}Buffer ...\textgreater{}}
\end{Highlighting}
\end{Shaded}

\paragraph{\texorpdfstring{\texttt{certificate.verifySpkac(spkac{[},\ encoding{]})}}{certificate.verifySpkac(spkac{[}, encoding{]})}}\label{certificate.verifyspkacspkac-encoding}

\begin{itemize}
\tightlist
\item
  \texttt{spkac}
  \{string\textbar ArrayBuffer\textbar Buffer\textbar TypedArray\textbar DataView\}
\item
  \texttt{encoding} \{string\} The
  \href{buffer.md\#buffers-and-character-encodings}{encoding} of the
  \texttt{spkac} string.
\item
  Returns: \{boolean\} \texttt{true} if the given \texttt{spkac} data
  structure is valid, \texttt{false} otherwise.
\end{itemize}

\begin{Shaded}
\begin{Highlighting}[]
\ImportTok{import}\NormalTok{ \{ }\BuiltInTok{Buffer}\NormalTok{ \} }\ImportTok{from} \StringTok{\textquotesingle{}node:buffer\textquotesingle{}}\OperatorTok{;}
\KeywordTok{const}\NormalTok{ \{ Certificate \} }\OperatorTok{=} \ControlFlowTok{await} \ImportTok{import}\NormalTok{(}\StringTok{\textquotesingle{}node:crypto\textquotesingle{}}\NormalTok{)}\OperatorTok{;}

\KeywordTok{const}\NormalTok{ cert }\OperatorTok{=} \FunctionTok{Certificate}\NormalTok{()}\OperatorTok{;}
\KeywordTok{const}\NormalTok{ spkac }\OperatorTok{=} \FunctionTok{getSpkacSomehow}\NormalTok{()}\OperatorTok{;}
\BuiltInTok{console}\OperatorTok{.}\FunctionTok{log}\NormalTok{(cert}\OperatorTok{.}\FunctionTok{verifySpkac}\NormalTok{(}\BuiltInTok{Buffer}\OperatorTok{.}\FunctionTok{from}\NormalTok{(spkac)))}\OperatorTok{;}
\CommentTok{// Prints: true or false}
\end{Highlighting}
\end{Shaded}

\begin{Shaded}
\begin{Highlighting}[]
\KeywordTok{const}\NormalTok{ \{ }\BuiltInTok{Buffer}\NormalTok{ \} }\OperatorTok{=} \PreprocessorTok{require}\NormalTok{(}\StringTok{\textquotesingle{}node:buffer\textquotesingle{}}\NormalTok{)}\OperatorTok{;}
\KeywordTok{const}\NormalTok{ \{ Certificate \} }\OperatorTok{=} \PreprocessorTok{require}\NormalTok{(}\StringTok{\textquotesingle{}node:crypto\textquotesingle{}}\NormalTok{)}\OperatorTok{;}

\KeywordTok{const}\NormalTok{ cert }\OperatorTok{=} \FunctionTok{Certificate}\NormalTok{()}\OperatorTok{;}
\KeywordTok{const}\NormalTok{ spkac }\OperatorTok{=} \FunctionTok{getSpkacSomehow}\NormalTok{()}\OperatorTok{;}
\BuiltInTok{console}\OperatorTok{.}\FunctionTok{log}\NormalTok{(cert}\OperatorTok{.}\FunctionTok{verifySpkac}\NormalTok{(}\BuiltInTok{Buffer}\OperatorTok{.}\FunctionTok{from}\NormalTok{(spkac)))}\OperatorTok{;}
\CommentTok{// Prints: true or false}
\end{Highlighting}
\end{Shaded}

\subsection{\texorpdfstring{Class:
\texttt{Cipher}}{Class: Cipher}}\label{class-cipher}

\begin{itemize}
\tightlist
\item
  Extends: \{stream.Transform\}
\end{itemize}

Instances of the \texttt{Cipher} class are used to encrypt data. The
class can be used in one of two ways:

\begin{itemize}
\tightlist
\item
  As a \href{stream.md}{stream} that is both readable and writable,
  where plain unencrypted data is written to produce encrypted data on
  the readable side, or
\item
  Using the
  \hyperref[cipherupdatedata-inputencoding-outputencoding]{\texttt{cipher.update()}}
  and \hyperref[cipherfinaloutputencoding]{\texttt{cipher.final()}}
  methods to produce the encrypted data.
\end{itemize}

The
\hyperref[cryptocreatecipherivalgorithm-key-iv-options]{\texttt{crypto.createCipheriv()}}
method is used to create \texttt{Cipher} instances. \texttt{Cipher}
objects are not to be created directly using the \texttt{new} keyword.

Example: Using \texttt{Cipher} objects as streams:

\begin{Shaded}
\begin{Highlighting}[]
\KeywordTok{const}\NormalTok{ \{}
\NormalTok{  scrypt}\OperatorTok{,}
\NormalTok{  randomFill}\OperatorTok{,}
\NormalTok{  createCipheriv}\OperatorTok{,}
\NormalTok{\} }\OperatorTok{=} \ControlFlowTok{await} \ImportTok{import}\NormalTok{(}\StringTok{\textquotesingle{}node:crypto\textquotesingle{}}\NormalTok{)}\OperatorTok{;}

\KeywordTok{const}\NormalTok{ algorithm }\OperatorTok{=} \StringTok{\textquotesingle{}aes{-}192{-}cbc\textquotesingle{}}\OperatorTok{;}
\KeywordTok{const}\NormalTok{ password }\OperatorTok{=} \StringTok{\textquotesingle{}Password used to generate key\textquotesingle{}}\OperatorTok{;}

\CommentTok{// First, we\textquotesingle{}ll generate the key. The key length is dependent on the algorithm.}
\CommentTok{// In this case for aes192, it is 24 bytes (192 bits).}
\FunctionTok{scrypt}\NormalTok{(password}\OperatorTok{,} \StringTok{\textquotesingle{}salt\textquotesingle{}}\OperatorTok{,} \DecValTok{24}\OperatorTok{,}\NormalTok{ (err}\OperatorTok{,}\NormalTok{ key) }\KeywordTok{=\textgreater{}}\NormalTok{ \{}
  \ControlFlowTok{if}\NormalTok{ (err) }\ControlFlowTok{throw}\NormalTok{ err}\OperatorTok{;}
  \CommentTok{// Then, we\textquotesingle{}ll generate a random initialization vector}
  \FunctionTok{randomFill}\NormalTok{(}\KeywordTok{new} \BuiltInTok{Uint8Array}\NormalTok{(}\DecValTok{16}\NormalTok{)}\OperatorTok{,}\NormalTok{ (err}\OperatorTok{,}\NormalTok{ iv) }\KeywordTok{=\textgreater{}}\NormalTok{ \{}
    \ControlFlowTok{if}\NormalTok{ (err) }\ControlFlowTok{throw}\NormalTok{ err}\OperatorTok{;}

    \CommentTok{// Once we have the key and iv, we can create and use the cipher...}
    \KeywordTok{const}\NormalTok{ cipher }\OperatorTok{=} \FunctionTok{createCipheriv}\NormalTok{(algorithm}\OperatorTok{,}\NormalTok{ key}\OperatorTok{,}\NormalTok{ iv)}\OperatorTok{;}

    \KeywordTok{let}\NormalTok{ encrypted }\OperatorTok{=} \StringTok{\textquotesingle{}\textquotesingle{}}\OperatorTok{;}
\NormalTok{    cipher}\OperatorTok{.}\FunctionTok{setEncoding}\NormalTok{(}\StringTok{\textquotesingle{}hex\textquotesingle{}}\NormalTok{)}\OperatorTok{;}

\NormalTok{    cipher}\OperatorTok{.}\FunctionTok{on}\NormalTok{(}\StringTok{\textquotesingle{}data\textquotesingle{}}\OperatorTok{,}\NormalTok{ (chunk) }\KeywordTok{=\textgreater{}}\NormalTok{ encrypted }\OperatorTok{+=}\NormalTok{ chunk)}\OperatorTok{;}
\NormalTok{    cipher}\OperatorTok{.}\FunctionTok{on}\NormalTok{(}\StringTok{\textquotesingle{}end\textquotesingle{}}\OperatorTok{,}\NormalTok{ () }\KeywordTok{=\textgreater{}} \BuiltInTok{console}\OperatorTok{.}\FunctionTok{log}\NormalTok{(encrypted))}\OperatorTok{;}

\NormalTok{    cipher}\OperatorTok{.}\FunctionTok{write}\NormalTok{(}\StringTok{\textquotesingle{}some clear text data\textquotesingle{}}\NormalTok{)}\OperatorTok{;}
\NormalTok{    cipher}\OperatorTok{.}\FunctionTok{end}\NormalTok{()}\OperatorTok{;}
\NormalTok{  \})}\OperatorTok{;}
\NormalTok{\})}\OperatorTok{;}
\end{Highlighting}
\end{Shaded}

\begin{Shaded}
\begin{Highlighting}[]
\KeywordTok{const}\NormalTok{ \{}
\NormalTok{  scrypt}\OperatorTok{,}
\NormalTok{  randomFill}\OperatorTok{,}
\NormalTok{  createCipheriv}\OperatorTok{,}
\NormalTok{\} }\OperatorTok{=} \PreprocessorTok{require}\NormalTok{(}\StringTok{\textquotesingle{}node:crypto\textquotesingle{}}\NormalTok{)}\OperatorTok{;}

\KeywordTok{const}\NormalTok{ algorithm }\OperatorTok{=} \StringTok{\textquotesingle{}aes{-}192{-}cbc\textquotesingle{}}\OperatorTok{;}
\KeywordTok{const}\NormalTok{ password }\OperatorTok{=} \StringTok{\textquotesingle{}Password used to generate key\textquotesingle{}}\OperatorTok{;}

\CommentTok{// First, we\textquotesingle{}ll generate the key. The key length is dependent on the algorithm.}
\CommentTok{// In this case for aes192, it is 24 bytes (192 bits).}
\FunctionTok{scrypt}\NormalTok{(password}\OperatorTok{,} \StringTok{\textquotesingle{}salt\textquotesingle{}}\OperatorTok{,} \DecValTok{24}\OperatorTok{,}\NormalTok{ (err}\OperatorTok{,}\NormalTok{ key) }\KeywordTok{=\textgreater{}}\NormalTok{ \{}
  \ControlFlowTok{if}\NormalTok{ (err) }\ControlFlowTok{throw}\NormalTok{ err}\OperatorTok{;}
  \CommentTok{// Then, we\textquotesingle{}ll generate a random initialization vector}
  \FunctionTok{randomFill}\NormalTok{(}\KeywordTok{new} \BuiltInTok{Uint8Array}\NormalTok{(}\DecValTok{16}\NormalTok{)}\OperatorTok{,}\NormalTok{ (err}\OperatorTok{,}\NormalTok{ iv) }\KeywordTok{=\textgreater{}}\NormalTok{ \{}
    \ControlFlowTok{if}\NormalTok{ (err) }\ControlFlowTok{throw}\NormalTok{ err}\OperatorTok{;}

    \CommentTok{// Once we have the key and iv, we can create and use the cipher...}
    \KeywordTok{const}\NormalTok{ cipher }\OperatorTok{=} \FunctionTok{createCipheriv}\NormalTok{(algorithm}\OperatorTok{,}\NormalTok{ key}\OperatorTok{,}\NormalTok{ iv)}\OperatorTok{;}

    \KeywordTok{let}\NormalTok{ encrypted }\OperatorTok{=} \StringTok{\textquotesingle{}\textquotesingle{}}\OperatorTok{;}
\NormalTok{    cipher}\OperatorTok{.}\FunctionTok{setEncoding}\NormalTok{(}\StringTok{\textquotesingle{}hex\textquotesingle{}}\NormalTok{)}\OperatorTok{;}

\NormalTok{    cipher}\OperatorTok{.}\FunctionTok{on}\NormalTok{(}\StringTok{\textquotesingle{}data\textquotesingle{}}\OperatorTok{,}\NormalTok{ (chunk) }\KeywordTok{=\textgreater{}}\NormalTok{ encrypted }\OperatorTok{+=}\NormalTok{ chunk)}\OperatorTok{;}
\NormalTok{    cipher}\OperatorTok{.}\FunctionTok{on}\NormalTok{(}\StringTok{\textquotesingle{}end\textquotesingle{}}\OperatorTok{,}\NormalTok{ () }\KeywordTok{=\textgreater{}} \BuiltInTok{console}\OperatorTok{.}\FunctionTok{log}\NormalTok{(encrypted))}\OperatorTok{;}

\NormalTok{    cipher}\OperatorTok{.}\FunctionTok{write}\NormalTok{(}\StringTok{\textquotesingle{}some clear text data\textquotesingle{}}\NormalTok{)}\OperatorTok{;}
\NormalTok{    cipher}\OperatorTok{.}\FunctionTok{end}\NormalTok{()}\OperatorTok{;}
\NormalTok{  \})}\OperatorTok{;}
\NormalTok{\})}\OperatorTok{;}
\end{Highlighting}
\end{Shaded}

Example: Using \texttt{Cipher} and piped streams:

\begin{Shaded}
\begin{Highlighting}[]
\ImportTok{import}\NormalTok{ \{}
\NormalTok{  createReadStream}\OperatorTok{,}
\NormalTok{  createWriteStream}\OperatorTok{,}
\NormalTok{\} }\ImportTok{from} \StringTok{\textquotesingle{}node:fs\textquotesingle{}}\OperatorTok{;}

\ImportTok{import}\NormalTok{ \{}
\NormalTok{  pipeline}\OperatorTok{,}
\NormalTok{\} }\ImportTok{from} \StringTok{\textquotesingle{}node:stream\textquotesingle{}}\OperatorTok{;}

\KeywordTok{const}\NormalTok{ \{}
\NormalTok{  scrypt}\OperatorTok{,}
\NormalTok{  randomFill}\OperatorTok{,}
\NormalTok{  createCipheriv}\OperatorTok{,}
\NormalTok{\} }\OperatorTok{=} \ControlFlowTok{await} \ImportTok{import}\NormalTok{(}\StringTok{\textquotesingle{}node:crypto\textquotesingle{}}\NormalTok{)}\OperatorTok{;}

\KeywordTok{const}\NormalTok{ algorithm }\OperatorTok{=} \StringTok{\textquotesingle{}aes{-}192{-}cbc\textquotesingle{}}\OperatorTok{;}
\KeywordTok{const}\NormalTok{ password }\OperatorTok{=} \StringTok{\textquotesingle{}Password used to generate key\textquotesingle{}}\OperatorTok{;}

\CommentTok{// First, we\textquotesingle{}ll generate the key. The key length is dependent on the algorithm.}
\CommentTok{// In this case for aes192, it is 24 bytes (192 bits).}
\FunctionTok{scrypt}\NormalTok{(password}\OperatorTok{,} \StringTok{\textquotesingle{}salt\textquotesingle{}}\OperatorTok{,} \DecValTok{24}\OperatorTok{,}\NormalTok{ (err}\OperatorTok{,}\NormalTok{ key) }\KeywordTok{=\textgreater{}}\NormalTok{ \{}
  \ControlFlowTok{if}\NormalTok{ (err) }\ControlFlowTok{throw}\NormalTok{ err}\OperatorTok{;}
  \CommentTok{// Then, we\textquotesingle{}ll generate a random initialization vector}
  \FunctionTok{randomFill}\NormalTok{(}\KeywordTok{new} \BuiltInTok{Uint8Array}\NormalTok{(}\DecValTok{16}\NormalTok{)}\OperatorTok{,}\NormalTok{ (err}\OperatorTok{,}\NormalTok{ iv) }\KeywordTok{=\textgreater{}}\NormalTok{ \{}
    \ControlFlowTok{if}\NormalTok{ (err) }\ControlFlowTok{throw}\NormalTok{ err}\OperatorTok{;}

    \KeywordTok{const}\NormalTok{ cipher }\OperatorTok{=} \FunctionTok{createCipheriv}\NormalTok{(algorithm}\OperatorTok{,}\NormalTok{ key}\OperatorTok{,}\NormalTok{ iv)}\OperatorTok{;}

    \KeywordTok{const}\NormalTok{ input }\OperatorTok{=} \FunctionTok{createReadStream}\NormalTok{(}\StringTok{\textquotesingle{}test.js\textquotesingle{}}\NormalTok{)}\OperatorTok{;}
    \KeywordTok{const}\NormalTok{ output }\OperatorTok{=} \FunctionTok{createWriteStream}\NormalTok{(}\StringTok{\textquotesingle{}test.enc\textquotesingle{}}\NormalTok{)}\OperatorTok{;}

    \FunctionTok{pipeline}\NormalTok{(input}\OperatorTok{,}\NormalTok{ cipher}\OperatorTok{,}\NormalTok{ output}\OperatorTok{,}\NormalTok{ (err) }\KeywordTok{=\textgreater{}}\NormalTok{ \{}
      \ControlFlowTok{if}\NormalTok{ (err) }\ControlFlowTok{throw}\NormalTok{ err}\OperatorTok{;}
\NormalTok{    \})}\OperatorTok{;}
\NormalTok{  \})}\OperatorTok{;}
\NormalTok{\})}\OperatorTok{;}
\end{Highlighting}
\end{Shaded}

\begin{Shaded}
\begin{Highlighting}[]
\KeywordTok{const}\NormalTok{ \{}
\NormalTok{  createReadStream}\OperatorTok{,}
\NormalTok{  createWriteStream}\OperatorTok{,}
\NormalTok{\} }\OperatorTok{=} \PreprocessorTok{require}\NormalTok{(}\StringTok{\textquotesingle{}node:fs\textquotesingle{}}\NormalTok{)}\OperatorTok{;}

\KeywordTok{const}\NormalTok{ \{}
\NormalTok{  pipeline}\OperatorTok{,}
\NormalTok{\} }\OperatorTok{=} \PreprocessorTok{require}\NormalTok{(}\StringTok{\textquotesingle{}node:stream\textquotesingle{}}\NormalTok{)}\OperatorTok{;}

\KeywordTok{const}\NormalTok{ \{}
\NormalTok{  scrypt}\OperatorTok{,}
\NormalTok{  randomFill}\OperatorTok{,}
\NormalTok{  createCipheriv}\OperatorTok{,}
\NormalTok{\} }\OperatorTok{=} \PreprocessorTok{require}\NormalTok{(}\StringTok{\textquotesingle{}node:crypto\textquotesingle{}}\NormalTok{)}\OperatorTok{;}

\KeywordTok{const}\NormalTok{ algorithm }\OperatorTok{=} \StringTok{\textquotesingle{}aes{-}192{-}cbc\textquotesingle{}}\OperatorTok{;}
\KeywordTok{const}\NormalTok{ password }\OperatorTok{=} \StringTok{\textquotesingle{}Password used to generate key\textquotesingle{}}\OperatorTok{;}

\CommentTok{// First, we\textquotesingle{}ll generate the key. The key length is dependent on the algorithm.}
\CommentTok{// In this case for aes192, it is 24 bytes (192 bits).}
\FunctionTok{scrypt}\NormalTok{(password}\OperatorTok{,} \StringTok{\textquotesingle{}salt\textquotesingle{}}\OperatorTok{,} \DecValTok{24}\OperatorTok{,}\NormalTok{ (err}\OperatorTok{,}\NormalTok{ key) }\KeywordTok{=\textgreater{}}\NormalTok{ \{}
  \ControlFlowTok{if}\NormalTok{ (err) }\ControlFlowTok{throw}\NormalTok{ err}\OperatorTok{;}
  \CommentTok{// Then, we\textquotesingle{}ll generate a random initialization vector}
  \FunctionTok{randomFill}\NormalTok{(}\KeywordTok{new} \BuiltInTok{Uint8Array}\NormalTok{(}\DecValTok{16}\NormalTok{)}\OperatorTok{,}\NormalTok{ (err}\OperatorTok{,}\NormalTok{ iv) }\KeywordTok{=\textgreater{}}\NormalTok{ \{}
    \ControlFlowTok{if}\NormalTok{ (err) }\ControlFlowTok{throw}\NormalTok{ err}\OperatorTok{;}

    \KeywordTok{const}\NormalTok{ cipher }\OperatorTok{=} \FunctionTok{createCipheriv}\NormalTok{(algorithm}\OperatorTok{,}\NormalTok{ key}\OperatorTok{,}\NormalTok{ iv)}\OperatorTok{;}

    \KeywordTok{const}\NormalTok{ input }\OperatorTok{=} \FunctionTok{createReadStream}\NormalTok{(}\StringTok{\textquotesingle{}test.js\textquotesingle{}}\NormalTok{)}\OperatorTok{;}
    \KeywordTok{const}\NormalTok{ output }\OperatorTok{=} \FunctionTok{createWriteStream}\NormalTok{(}\StringTok{\textquotesingle{}test.enc\textquotesingle{}}\NormalTok{)}\OperatorTok{;}

    \FunctionTok{pipeline}\NormalTok{(input}\OperatorTok{,}\NormalTok{ cipher}\OperatorTok{,}\NormalTok{ output}\OperatorTok{,}\NormalTok{ (err) }\KeywordTok{=\textgreater{}}\NormalTok{ \{}
      \ControlFlowTok{if}\NormalTok{ (err) }\ControlFlowTok{throw}\NormalTok{ err}\OperatorTok{;}
\NormalTok{    \})}\OperatorTok{;}
\NormalTok{  \})}\OperatorTok{;}
\NormalTok{\})}\OperatorTok{;}
\end{Highlighting}
\end{Shaded}

Example: Using the
\hyperref[cipherupdatedata-inputencoding-outputencoding]{\texttt{cipher.update()}}
and \hyperref[cipherfinaloutputencoding]{\texttt{cipher.final()}}
methods:

\begin{Shaded}
\begin{Highlighting}[]
\KeywordTok{const}\NormalTok{ \{}
\NormalTok{  scrypt}\OperatorTok{,}
\NormalTok{  randomFill}\OperatorTok{,}
\NormalTok{  createCipheriv}\OperatorTok{,}
\NormalTok{\} }\OperatorTok{=} \ControlFlowTok{await} \ImportTok{import}\NormalTok{(}\StringTok{\textquotesingle{}node:crypto\textquotesingle{}}\NormalTok{)}\OperatorTok{;}

\KeywordTok{const}\NormalTok{ algorithm }\OperatorTok{=} \StringTok{\textquotesingle{}aes{-}192{-}cbc\textquotesingle{}}\OperatorTok{;}
\KeywordTok{const}\NormalTok{ password }\OperatorTok{=} \StringTok{\textquotesingle{}Password used to generate key\textquotesingle{}}\OperatorTok{;}

\CommentTok{// First, we\textquotesingle{}ll generate the key. The key length is dependent on the algorithm.}
\CommentTok{// In this case for aes192, it is 24 bytes (192 bits).}
\FunctionTok{scrypt}\NormalTok{(password}\OperatorTok{,} \StringTok{\textquotesingle{}salt\textquotesingle{}}\OperatorTok{,} \DecValTok{24}\OperatorTok{,}\NormalTok{ (err}\OperatorTok{,}\NormalTok{ key) }\KeywordTok{=\textgreater{}}\NormalTok{ \{}
  \ControlFlowTok{if}\NormalTok{ (err) }\ControlFlowTok{throw}\NormalTok{ err}\OperatorTok{;}
  \CommentTok{// Then, we\textquotesingle{}ll generate a random initialization vector}
  \FunctionTok{randomFill}\NormalTok{(}\KeywordTok{new} \BuiltInTok{Uint8Array}\NormalTok{(}\DecValTok{16}\NormalTok{)}\OperatorTok{,}\NormalTok{ (err}\OperatorTok{,}\NormalTok{ iv) }\KeywordTok{=\textgreater{}}\NormalTok{ \{}
    \ControlFlowTok{if}\NormalTok{ (err) }\ControlFlowTok{throw}\NormalTok{ err}\OperatorTok{;}

    \KeywordTok{const}\NormalTok{ cipher }\OperatorTok{=} \FunctionTok{createCipheriv}\NormalTok{(algorithm}\OperatorTok{,}\NormalTok{ key}\OperatorTok{,}\NormalTok{ iv)}\OperatorTok{;}

    \KeywordTok{let}\NormalTok{ encrypted }\OperatorTok{=}\NormalTok{ cipher}\OperatorTok{.}\FunctionTok{update}\NormalTok{(}\StringTok{\textquotesingle{}some clear text data\textquotesingle{}}\OperatorTok{,} \StringTok{\textquotesingle{}utf8\textquotesingle{}}\OperatorTok{,} \StringTok{\textquotesingle{}hex\textquotesingle{}}\NormalTok{)}\OperatorTok{;}
\NormalTok{    encrypted }\OperatorTok{+=}\NormalTok{ cipher}\OperatorTok{.}\FunctionTok{final}\NormalTok{(}\StringTok{\textquotesingle{}hex\textquotesingle{}}\NormalTok{)}\OperatorTok{;}
    \BuiltInTok{console}\OperatorTok{.}\FunctionTok{log}\NormalTok{(encrypted)}\OperatorTok{;}
\NormalTok{  \})}\OperatorTok{;}
\NormalTok{\})}\OperatorTok{;}
\end{Highlighting}
\end{Shaded}

\begin{Shaded}
\begin{Highlighting}[]
\KeywordTok{const}\NormalTok{ \{}
\NormalTok{  scrypt}\OperatorTok{,}
\NormalTok{  randomFill}\OperatorTok{,}
\NormalTok{  createCipheriv}\OperatorTok{,}
\NormalTok{\} }\OperatorTok{=} \PreprocessorTok{require}\NormalTok{(}\StringTok{\textquotesingle{}node:crypto\textquotesingle{}}\NormalTok{)}\OperatorTok{;}

\KeywordTok{const}\NormalTok{ algorithm }\OperatorTok{=} \StringTok{\textquotesingle{}aes{-}192{-}cbc\textquotesingle{}}\OperatorTok{;}
\KeywordTok{const}\NormalTok{ password }\OperatorTok{=} \StringTok{\textquotesingle{}Password used to generate key\textquotesingle{}}\OperatorTok{;}

\CommentTok{// First, we\textquotesingle{}ll generate the key. The key length is dependent on the algorithm.}
\CommentTok{// In this case for aes192, it is 24 bytes (192 bits).}
\FunctionTok{scrypt}\NormalTok{(password}\OperatorTok{,} \StringTok{\textquotesingle{}salt\textquotesingle{}}\OperatorTok{,} \DecValTok{24}\OperatorTok{,}\NormalTok{ (err}\OperatorTok{,}\NormalTok{ key) }\KeywordTok{=\textgreater{}}\NormalTok{ \{}
  \ControlFlowTok{if}\NormalTok{ (err) }\ControlFlowTok{throw}\NormalTok{ err}\OperatorTok{;}
  \CommentTok{// Then, we\textquotesingle{}ll generate a random initialization vector}
  \FunctionTok{randomFill}\NormalTok{(}\KeywordTok{new} \BuiltInTok{Uint8Array}\NormalTok{(}\DecValTok{16}\NormalTok{)}\OperatorTok{,}\NormalTok{ (err}\OperatorTok{,}\NormalTok{ iv) }\KeywordTok{=\textgreater{}}\NormalTok{ \{}
    \ControlFlowTok{if}\NormalTok{ (err) }\ControlFlowTok{throw}\NormalTok{ err}\OperatorTok{;}

    \KeywordTok{const}\NormalTok{ cipher }\OperatorTok{=} \FunctionTok{createCipheriv}\NormalTok{(algorithm}\OperatorTok{,}\NormalTok{ key}\OperatorTok{,}\NormalTok{ iv)}\OperatorTok{;}

    \KeywordTok{let}\NormalTok{ encrypted }\OperatorTok{=}\NormalTok{ cipher}\OperatorTok{.}\FunctionTok{update}\NormalTok{(}\StringTok{\textquotesingle{}some clear text data\textquotesingle{}}\OperatorTok{,} \StringTok{\textquotesingle{}utf8\textquotesingle{}}\OperatorTok{,} \StringTok{\textquotesingle{}hex\textquotesingle{}}\NormalTok{)}\OperatorTok{;}
\NormalTok{    encrypted }\OperatorTok{+=}\NormalTok{ cipher}\OperatorTok{.}\FunctionTok{final}\NormalTok{(}\StringTok{\textquotesingle{}hex\textquotesingle{}}\NormalTok{)}\OperatorTok{;}
    \BuiltInTok{console}\OperatorTok{.}\FunctionTok{log}\NormalTok{(encrypted)}\OperatorTok{;}
\NormalTok{  \})}\OperatorTok{;}
\NormalTok{\})}\OperatorTok{;}
\end{Highlighting}
\end{Shaded}

\subsubsection{\texorpdfstring{\texttt{cipher.final({[}outputEncoding{]})}}{cipher.final({[}outputEncoding{]})}}\label{cipher.finaloutputencoding}

\begin{itemize}
\tightlist
\item
  \texttt{outputEncoding} \{string\} The
  \href{buffer.md\#buffers-and-character-encodings}{encoding} of the
  return value.
\item
  Returns: \{Buffer \textbar{} string\} Any remaining enciphered
  contents. If \texttt{outputEncoding} is specified, a string is
  returned. If an \texttt{outputEncoding} is not provided, a
  \href{buffer.md}{\texttt{Buffer}} is returned.
\end{itemize}

Once the \texttt{cipher.final()} method has been called, the
\texttt{Cipher} object can no longer be used to encrypt data. Attempts
to call \texttt{cipher.final()} more than once will result in an error
being thrown.

\subsubsection{\texorpdfstring{\texttt{cipher.getAuthTag()}}{cipher.getAuthTag()}}\label{cipher.getauthtag}

\begin{itemize}
\tightlist
\item
  Returns: \{Buffer\} When using an authenticated encryption mode
  (\texttt{GCM}, \texttt{CCM}, \texttt{OCB}, and
  \texttt{chacha20-poly1305} are currently supported), the
  \texttt{cipher.getAuthTag()} method returns a
  \href{buffer.md}{\texttt{Buffer}} containing the \emph{authentication
  tag} that has been computed from the given data.
\end{itemize}

The \texttt{cipher.getAuthTag()} method should only be called after
encryption has been completed using the
\hyperref[cipherfinaloutputencoding]{\texttt{cipher.final()}} method.

If the \texttt{authTagLength} option was set during the \texttt{cipher}
instance's creation, this function will return exactly
\texttt{authTagLength} bytes.

\subsubsection{\texorpdfstring{\texttt{cipher.setAAD(buffer{[},\ options{]})}}{cipher.setAAD(buffer{[}, options{]})}}\label{cipher.setaadbuffer-options}

\begin{itemize}
\tightlist
\item
  \texttt{buffer}
  \{string\textbar ArrayBuffer\textbar Buffer\textbar TypedArray\textbar DataView\}
\item
  \texttt{options} \{Object\}
  \href{stream.md\#new-streamtransformoptions}{\texttt{stream.transform}
  options}

  \begin{itemize}
  \tightlist
  \item
    \texttt{plaintextLength} \{number\}
  \item
    \texttt{encoding} \{string\} The string encoding to use when
    \texttt{buffer} is a string.
  \end{itemize}
\item
  Returns: \{Cipher\} for method chaining.
\end{itemize}

When using an authenticated encryption mode (\texttt{GCM}, \texttt{CCM},
\texttt{OCB}, and \texttt{chacha20-poly1305} are currently supported),
the \texttt{cipher.setAAD()} method sets the value used for the
\emph{additional authenticated data} (AAD) input parameter.

The \texttt{plaintextLength} option is optional for \texttt{GCM} and
\texttt{OCB}. When using \texttt{CCM}, the \texttt{plaintextLength}
option must be specified and its value must match the length of the
plaintext in bytes. See \hyperref[ccm-mode]{CCM mode}.

The \texttt{cipher.setAAD()} method must be called before
\hyperref[cipherupdatedata-inputencoding-outputencoding]{\texttt{cipher.update()}}.

\subsubsection{\texorpdfstring{\texttt{cipher.setAutoPadding({[}autoPadding{]})}}{cipher.setAutoPadding({[}autoPadding{]})}}\label{cipher.setautopaddingautopadding}

\begin{itemize}
\tightlist
\item
  \texttt{autoPadding} \{boolean\} \textbf{Default:} \texttt{true}
\item
  Returns: \{Cipher\} for method chaining.
\end{itemize}

When using block encryption algorithms, the \texttt{Cipher} class will
automatically add padding to the input data to the appropriate block
size. To disable the default padding call
\texttt{cipher.setAutoPadding(false)}.

When \texttt{autoPadding} is \texttt{false}, the length of the entire
input data must be a multiple of the cipher's block size or
\hyperref[cipherfinaloutputencoding]{\texttt{cipher.final()}} will throw
an error. Disabling automatic padding is useful for non-standard
padding, for instance using \texttt{0x0} instead of PKCS padding.

The \texttt{cipher.setAutoPadding()} method must be called before
\hyperref[cipherfinaloutputencoding]{\texttt{cipher.final()}}.

\subsubsection{\texorpdfstring{\texttt{cipher.update(data{[},\ inputEncoding{]}{[},\ outputEncoding{]})}}{cipher.update(data{[}, inputEncoding{]}{[}, outputEncoding{]})}}\label{cipher.updatedata-inputencoding-outputencoding}

\begin{itemize}
\tightlist
\item
  \texttt{data}
  \{string\textbar Buffer\textbar TypedArray\textbar DataView\}
\item
  \texttt{inputEncoding} \{string\} The
  \href{buffer.md\#buffers-and-character-encodings}{encoding} of the
  data.
\item
  \texttt{outputEncoding} \{string\} The
  \href{buffer.md\#buffers-and-character-encodings}{encoding} of the
  return value.
\item
  Returns: \{Buffer \textbar{} string\}
\end{itemize}

Updates the cipher with \texttt{data}. If the \texttt{inputEncoding}
argument is given, the \texttt{data} argument is a string using the
specified encoding. If the \texttt{inputEncoding} argument is not given,
\texttt{data} must be a \href{buffer.md}{\texttt{Buffer}},
\texttt{TypedArray}, or \texttt{DataView}. If \texttt{data} is a
\href{buffer.md}{\texttt{Buffer}}, \texttt{TypedArray}, or
\texttt{DataView}, then \texttt{inputEncoding} is ignored.

The \texttt{outputEncoding} specifies the output format of the
enciphered data. If the \texttt{outputEncoding} is specified, a string
using the specified encoding is returned. If no \texttt{outputEncoding}
is provided, a \href{buffer.md}{\texttt{Buffer}} is returned.

The \texttt{cipher.update()} method can be called multiple times with
new data until
\hyperref[cipherfinaloutputencoding]{\texttt{cipher.final()}} is called.
Calling \texttt{cipher.update()} after
\hyperref[cipherfinaloutputencoding]{\texttt{cipher.final()}} will
result in an error being thrown.

\subsection{\texorpdfstring{Class:
\texttt{Decipher}}{Class: Decipher}}\label{class-decipher}

\begin{itemize}
\tightlist
\item
  Extends: \{stream.Transform\}
\end{itemize}

Instances of the \texttt{Decipher} class are used to decrypt data. The
class can be used in one of two ways:

\begin{itemize}
\tightlist
\item
  As a \href{stream.md}{stream} that is both readable and writable,
  where plain encrypted data is written to produce unencrypted data on
  the readable side, or
\item
  Using the
  \hyperref[decipherupdatedata-inputencoding-outputencoding]{\texttt{decipher.update()}}
  and \hyperref[decipherfinaloutputencoding]{\texttt{decipher.final()}}
  methods to produce the unencrypted data.
\end{itemize}

The
\hyperref[cryptocreatedecipherivalgorithm-key-iv-options]{\texttt{crypto.createDecipheriv()}}
method is used to create \texttt{Decipher} instances. \texttt{Decipher}
objects are not to be created directly using the \texttt{new} keyword.

Example: Using \texttt{Decipher} objects as streams:

\begin{Shaded}
\begin{Highlighting}[]
\ImportTok{import}\NormalTok{ \{ }\BuiltInTok{Buffer}\NormalTok{ \} }\ImportTok{from} \StringTok{\textquotesingle{}node:buffer\textquotesingle{}}\OperatorTok{;}
\KeywordTok{const}\NormalTok{ \{}
\NormalTok{  scryptSync}\OperatorTok{,}
\NormalTok{  createDecipheriv}\OperatorTok{,}
\NormalTok{\} }\OperatorTok{=} \ControlFlowTok{await} \ImportTok{import}\NormalTok{(}\StringTok{\textquotesingle{}node:crypto\textquotesingle{}}\NormalTok{)}\OperatorTok{;}

\KeywordTok{const}\NormalTok{ algorithm }\OperatorTok{=} \StringTok{\textquotesingle{}aes{-}192{-}cbc\textquotesingle{}}\OperatorTok{;}
\KeywordTok{const}\NormalTok{ password }\OperatorTok{=} \StringTok{\textquotesingle{}Password used to generate key\textquotesingle{}}\OperatorTok{;}
\CommentTok{// Key length is dependent on the algorithm. In this case for aes192, it is}
\CommentTok{// 24 bytes (192 bits).}
\CommentTok{// Use the async \textasciigrave{}crypto.scrypt()\textasciigrave{} instead.}
\KeywordTok{const}\NormalTok{ key }\OperatorTok{=} \FunctionTok{scryptSync}\NormalTok{(password}\OperatorTok{,} \StringTok{\textquotesingle{}salt\textquotesingle{}}\OperatorTok{,} \DecValTok{24}\NormalTok{)}\OperatorTok{;}
\CommentTok{// The IV is usually passed along with the ciphertext.}
\KeywordTok{const}\NormalTok{ iv }\OperatorTok{=} \BuiltInTok{Buffer}\OperatorTok{.}\FunctionTok{alloc}\NormalTok{(}\DecValTok{16}\OperatorTok{,} \DecValTok{0}\NormalTok{)}\OperatorTok{;} \CommentTok{// Initialization vector.}

\KeywordTok{const}\NormalTok{ decipher }\OperatorTok{=} \FunctionTok{createDecipheriv}\NormalTok{(algorithm}\OperatorTok{,}\NormalTok{ key}\OperatorTok{,}\NormalTok{ iv)}\OperatorTok{;}

\KeywordTok{let}\NormalTok{ decrypted }\OperatorTok{=} \StringTok{\textquotesingle{}\textquotesingle{}}\OperatorTok{;}
\NormalTok{decipher}\OperatorTok{.}\FunctionTok{on}\NormalTok{(}\StringTok{\textquotesingle{}readable\textquotesingle{}}\OperatorTok{,}\NormalTok{ () }\KeywordTok{=\textgreater{}}\NormalTok{ \{}
  \KeywordTok{let}\NormalTok{ chunk}\OperatorTok{;}
  \ControlFlowTok{while}\NormalTok{ (}\KeywordTok{null} \OperatorTok{!==}\NormalTok{ (chunk }\OperatorTok{=}\NormalTok{ decipher}\OperatorTok{.}\FunctionTok{read}\NormalTok{())) \{}
\NormalTok{    decrypted }\OperatorTok{+=}\NormalTok{ chunk}\OperatorTok{.}\FunctionTok{toString}\NormalTok{(}\StringTok{\textquotesingle{}utf8\textquotesingle{}}\NormalTok{)}\OperatorTok{;}
\NormalTok{  \}}
\NormalTok{\})}\OperatorTok{;}
\NormalTok{decipher}\OperatorTok{.}\FunctionTok{on}\NormalTok{(}\StringTok{\textquotesingle{}end\textquotesingle{}}\OperatorTok{,}\NormalTok{ () }\KeywordTok{=\textgreater{}}\NormalTok{ \{}
  \BuiltInTok{console}\OperatorTok{.}\FunctionTok{log}\NormalTok{(decrypted)}\OperatorTok{;}
  \CommentTok{// Prints: some clear text data}
\NormalTok{\})}\OperatorTok{;}

\CommentTok{// Encrypted with same algorithm, key and iv.}
\KeywordTok{const}\NormalTok{ encrypted }\OperatorTok{=}
  \StringTok{\textquotesingle{}e5f79c5915c02171eec6b212d5520d44480993d7d622a7c4c2da32f6efda0ffa\textquotesingle{}}\OperatorTok{;}
\NormalTok{decipher}\OperatorTok{.}\FunctionTok{write}\NormalTok{(encrypted}\OperatorTok{,} \StringTok{\textquotesingle{}hex\textquotesingle{}}\NormalTok{)}\OperatorTok{;}
\NormalTok{decipher}\OperatorTok{.}\FunctionTok{end}\NormalTok{()}\OperatorTok{;}
\end{Highlighting}
\end{Shaded}

\begin{Shaded}
\begin{Highlighting}[]
\KeywordTok{const}\NormalTok{ \{}
\NormalTok{  scryptSync}\OperatorTok{,}
\NormalTok{  createDecipheriv}\OperatorTok{,}
\NormalTok{\} }\OperatorTok{=} \PreprocessorTok{require}\NormalTok{(}\StringTok{\textquotesingle{}node:crypto\textquotesingle{}}\NormalTok{)}\OperatorTok{;}
\KeywordTok{const}\NormalTok{ \{ }\BuiltInTok{Buffer}\NormalTok{ \} }\OperatorTok{=} \PreprocessorTok{require}\NormalTok{(}\StringTok{\textquotesingle{}node:buffer\textquotesingle{}}\NormalTok{)}\OperatorTok{;}

\KeywordTok{const}\NormalTok{ algorithm }\OperatorTok{=} \StringTok{\textquotesingle{}aes{-}192{-}cbc\textquotesingle{}}\OperatorTok{;}
\KeywordTok{const}\NormalTok{ password }\OperatorTok{=} \StringTok{\textquotesingle{}Password used to generate key\textquotesingle{}}\OperatorTok{;}
\CommentTok{// Key length is dependent on the algorithm. In this case for aes192, it is}
\CommentTok{// 24 bytes (192 bits).}
\CommentTok{// Use the async \textasciigrave{}crypto.scrypt()\textasciigrave{} instead.}
\KeywordTok{const}\NormalTok{ key }\OperatorTok{=} \FunctionTok{scryptSync}\NormalTok{(password}\OperatorTok{,} \StringTok{\textquotesingle{}salt\textquotesingle{}}\OperatorTok{,} \DecValTok{24}\NormalTok{)}\OperatorTok{;}
\CommentTok{// The IV is usually passed along with the ciphertext.}
\KeywordTok{const}\NormalTok{ iv }\OperatorTok{=} \BuiltInTok{Buffer}\OperatorTok{.}\FunctionTok{alloc}\NormalTok{(}\DecValTok{16}\OperatorTok{,} \DecValTok{0}\NormalTok{)}\OperatorTok{;} \CommentTok{// Initialization vector.}

\KeywordTok{const}\NormalTok{ decipher }\OperatorTok{=} \FunctionTok{createDecipheriv}\NormalTok{(algorithm}\OperatorTok{,}\NormalTok{ key}\OperatorTok{,}\NormalTok{ iv)}\OperatorTok{;}

\KeywordTok{let}\NormalTok{ decrypted }\OperatorTok{=} \StringTok{\textquotesingle{}\textquotesingle{}}\OperatorTok{;}
\NormalTok{decipher}\OperatorTok{.}\FunctionTok{on}\NormalTok{(}\StringTok{\textquotesingle{}readable\textquotesingle{}}\OperatorTok{,}\NormalTok{ () }\KeywordTok{=\textgreater{}}\NormalTok{ \{}
  \KeywordTok{let}\NormalTok{ chunk}\OperatorTok{;}
  \ControlFlowTok{while}\NormalTok{ (}\KeywordTok{null} \OperatorTok{!==}\NormalTok{ (chunk }\OperatorTok{=}\NormalTok{ decipher}\OperatorTok{.}\FunctionTok{read}\NormalTok{())) \{}
\NormalTok{    decrypted }\OperatorTok{+=}\NormalTok{ chunk}\OperatorTok{.}\FunctionTok{toString}\NormalTok{(}\StringTok{\textquotesingle{}utf8\textquotesingle{}}\NormalTok{)}\OperatorTok{;}
\NormalTok{  \}}
\NormalTok{\})}\OperatorTok{;}
\NormalTok{decipher}\OperatorTok{.}\FunctionTok{on}\NormalTok{(}\StringTok{\textquotesingle{}end\textquotesingle{}}\OperatorTok{,}\NormalTok{ () }\KeywordTok{=\textgreater{}}\NormalTok{ \{}
  \BuiltInTok{console}\OperatorTok{.}\FunctionTok{log}\NormalTok{(decrypted)}\OperatorTok{;}
  \CommentTok{// Prints: some clear text data}
\NormalTok{\})}\OperatorTok{;}

\CommentTok{// Encrypted with same algorithm, key and iv.}
\KeywordTok{const}\NormalTok{ encrypted }\OperatorTok{=}
  \StringTok{\textquotesingle{}e5f79c5915c02171eec6b212d5520d44480993d7d622a7c4c2da32f6efda0ffa\textquotesingle{}}\OperatorTok{;}
\NormalTok{decipher}\OperatorTok{.}\FunctionTok{write}\NormalTok{(encrypted}\OperatorTok{,} \StringTok{\textquotesingle{}hex\textquotesingle{}}\NormalTok{)}\OperatorTok{;}
\NormalTok{decipher}\OperatorTok{.}\FunctionTok{end}\NormalTok{()}\OperatorTok{;}
\end{Highlighting}
\end{Shaded}

Example: Using \texttt{Decipher} and piped streams:

\begin{Shaded}
\begin{Highlighting}[]
\ImportTok{import}\NormalTok{ \{}
\NormalTok{  createReadStream}\OperatorTok{,}
\NormalTok{  createWriteStream}\OperatorTok{,}
\NormalTok{\} }\ImportTok{from} \StringTok{\textquotesingle{}node:fs\textquotesingle{}}\OperatorTok{;}
\ImportTok{import}\NormalTok{ \{ }\BuiltInTok{Buffer}\NormalTok{ \} }\ImportTok{from} \StringTok{\textquotesingle{}node:buffer\textquotesingle{}}\OperatorTok{;}
\KeywordTok{const}\NormalTok{ \{}
\NormalTok{  scryptSync}\OperatorTok{,}
\NormalTok{  createDecipheriv}\OperatorTok{,}
\NormalTok{\} }\OperatorTok{=} \ControlFlowTok{await} \ImportTok{import}\NormalTok{(}\StringTok{\textquotesingle{}node:crypto\textquotesingle{}}\NormalTok{)}\OperatorTok{;}

\KeywordTok{const}\NormalTok{ algorithm }\OperatorTok{=} \StringTok{\textquotesingle{}aes{-}192{-}cbc\textquotesingle{}}\OperatorTok{;}
\KeywordTok{const}\NormalTok{ password }\OperatorTok{=} \StringTok{\textquotesingle{}Password used to generate key\textquotesingle{}}\OperatorTok{;}
\CommentTok{// Use the async \textasciigrave{}crypto.scrypt()\textasciigrave{} instead.}
\KeywordTok{const}\NormalTok{ key }\OperatorTok{=} \FunctionTok{scryptSync}\NormalTok{(password}\OperatorTok{,} \StringTok{\textquotesingle{}salt\textquotesingle{}}\OperatorTok{,} \DecValTok{24}\NormalTok{)}\OperatorTok{;}
\CommentTok{// The IV is usually passed along with the ciphertext.}
\KeywordTok{const}\NormalTok{ iv }\OperatorTok{=} \BuiltInTok{Buffer}\OperatorTok{.}\FunctionTok{alloc}\NormalTok{(}\DecValTok{16}\OperatorTok{,} \DecValTok{0}\NormalTok{)}\OperatorTok{;} \CommentTok{// Initialization vector.}

\KeywordTok{const}\NormalTok{ decipher }\OperatorTok{=} \FunctionTok{createDecipheriv}\NormalTok{(algorithm}\OperatorTok{,}\NormalTok{ key}\OperatorTok{,}\NormalTok{ iv)}\OperatorTok{;}

\KeywordTok{const}\NormalTok{ input }\OperatorTok{=} \FunctionTok{createReadStream}\NormalTok{(}\StringTok{\textquotesingle{}test.enc\textquotesingle{}}\NormalTok{)}\OperatorTok{;}
\KeywordTok{const}\NormalTok{ output }\OperatorTok{=} \FunctionTok{createWriteStream}\NormalTok{(}\StringTok{\textquotesingle{}test.js\textquotesingle{}}\NormalTok{)}\OperatorTok{;}

\NormalTok{input}\OperatorTok{.}\FunctionTok{pipe}\NormalTok{(decipher)}\OperatorTok{.}\FunctionTok{pipe}\NormalTok{(output)}\OperatorTok{;}
\end{Highlighting}
\end{Shaded}

\begin{Shaded}
\begin{Highlighting}[]
\KeywordTok{const}\NormalTok{ \{}
\NormalTok{  createReadStream}\OperatorTok{,}
\NormalTok{  createWriteStream}\OperatorTok{,}
\NormalTok{\} }\OperatorTok{=} \PreprocessorTok{require}\NormalTok{(}\StringTok{\textquotesingle{}node:fs\textquotesingle{}}\NormalTok{)}\OperatorTok{;}
\KeywordTok{const}\NormalTok{ \{}
\NormalTok{  scryptSync}\OperatorTok{,}
\NormalTok{  createDecipheriv}\OperatorTok{,}
\NormalTok{\} }\OperatorTok{=} \PreprocessorTok{require}\NormalTok{(}\StringTok{\textquotesingle{}node:crypto\textquotesingle{}}\NormalTok{)}\OperatorTok{;}
\KeywordTok{const}\NormalTok{ \{ }\BuiltInTok{Buffer}\NormalTok{ \} }\OperatorTok{=} \PreprocessorTok{require}\NormalTok{(}\StringTok{\textquotesingle{}node:buffer\textquotesingle{}}\NormalTok{)}\OperatorTok{;}

\KeywordTok{const}\NormalTok{ algorithm }\OperatorTok{=} \StringTok{\textquotesingle{}aes{-}192{-}cbc\textquotesingle{}}\OperatorTok{;}
\KeywordTok{const}\NormalTok{ password }\OperatorTok{=} \StringTok{\textquotesingle{}Password used to generate key\textquotesingle{}}\OperatorTok{;}
\CommentTok{// Use the async \textasciigrave{}crypto.scrypt()\textasciigrave{} instead.}
\KeywordTok{const}\NormalTok{ key }\OperatorTok{=} \FunctionTok{scryptSync}\NormalTok{(password}\OperatorTok{,} \StringTok{\textquotesingle{}salt\textquotesingle{}}\OperatorTok{,} \DecValTok{24}\NormalTok{)}\OperatorTok{;}
\CommentTok{// The IV is usually passed along with the ciphertext.}
\KeywordTok{const}\NormalTok{ iv }\OperatorTok{=} \BuiltInTok{Buffer}\OperatorTok{.}\FunctionTok{alloc}\NormalTok{(}\DecValTok{16}\OperatorTok{,} \DecValTok{0}\NormalTok{)}\OperatorTok{;} \CommentTok{// Initialization vector.}

\KeywordTok{const}\NormalTok{ decipher }\OperatorTok{=} \FunctionTok{createDecipheriv}\NormalTok{(algorithm}\OperatorTok{,}\NormalTok{ key}\OperatorTok{,}\NormalTok{ iv)}\OperatorTok{;}

\KeywordTok{const}\NormalTok{ input }\OperatorTok{=} \FunctionTok{createReadStream}\NormalTok{(}\StringTok{\textquotesingle{}test.enc\textquotesingle{}}\NormalTok{)}\OperatorTok{;}
\KeywordTok{const}\NormalTok{ output }\OperatorTok{=} \FunctionTok{createWriteStream}\NormalTok{(}\StringTok{\textquotesingle{}test.js\textquotesingle{}}\NormalTok{)}\OperatorTok{;}

\NormalTok{input}\OperatorTok{.}\FunctionTok{pipe}\NormalTok{(decipher)}\OperatorTok{.}\FunctionTok{pipe}\NormalTok{(output)}\OperatorTok{;}
\end{Highlighting}
\end{Shaded}

Example: Using the
\hyperref[decipherupdatedata-inputencoding-outputencoding]{\texttt{decipher.update()}}
and \hyperref[decipherfinaloutputencoding]{\texttt{decipher.final()}}
methods:

\begin{Shaded}
\begin{Highlighting}[]
\ImportTok{import}\NormalTok{ \{ }\BuiltInTok{Buffer}\NormalTok{ \} }\ImportTok{from} \StringTok{\textquotesingle{}node:buffer\textquotesingle{}}\OperatorTok{;}
\KeywordTok{const}\NormalTok{ \{}
\NormalTok{  scryptSync}\OperatorTok{,}
\NormalTok{  createDecipheriv}\OperatorTok{,}
\NormalTok{\} }\OperatorTok{=} \ControlFlowTok{await} \ImportTok{import}\NormalTok{(}\StringTok{\textquotesingle{}node:crypto\textquotesingle{}}\NormalTok{)}\OperatorTok{;}

\KeywordTok{const}\NormalTok{ algorithm }\OperatorTok{=} \StringTok{\textquotesingle{}aes{-}192{-}cbc\textquotesingle{}}\OperatorTok{;}
\KeywordTok{const}\NormalTok{ password }\OperatorTok{=} \StringTok{\textquotesingle{}Password used to generate key\textquotesingle{}}\OperatorTok{;}
\CommentTok{// Use the async \textasciigrave{}crypto.scrypt()\textasciigrave{} instead.}
\KeywordTok{const}\NormalTok{ key }\OperatorTok{=} \FunctionTok{scryptSync}\NormalTok{(password}\OperatorTok{,} \StringTok{\textquotesingle{}salt\textquotesingle{}}\OperatorTok{,} \DecValTok{24}\NormalTok{)}\OperatorTok{;}
\CommentTok{// The IV is usually passed along with the ciphertext.}
\KeywordTok{const}\NormalTok{ iv }\OperatorTok{=} \BuiltInTok{Buffer}\OperatorTok{.}\FunctionTok{alloc}\NormalTok{(}\DecValTok{16}\OperatorTok{,} \DecValTok{0}\NormalTok{)}\OperatorTok{;} \CommentTok{// Initialization vector.}

\KeywordTok{const}\NormalTok{ decipher }\OperatorTok{=} \FunctionTok{createDecipheriv}\NormalTok{(algorithm}\OperatorTok{,}\NormalTok{ key}\OperatorTok{,}\NormalTok{ iv)}\OperatorTok{;}

\CommentTok{// Encrypted using same algorithm, key and iv.}
\KeywordTok{const}\NormalTok{ encrypted }\OperatorTok{=}
  \StringTok{\textquotesingle{}e5f79c5915c02171eec6b212d5520d44480993d7d622a7c4c2da32f6efda0ffa\textquotesingle{}}\OperatorTok{;}
\KeywordTok{let}\NormalTok{ decrypted }\OperatorTok{=}\NormalTok{ decipher}\OperatorTok{.}\FunctionTok{update}\NormalTok{(encrypted}\OperatorTok{,} \StringTok{\textquotesingle{}hex\textquotesingle{}}\OperatorTok{,} \StringTok{\textquotesingle{}utf8\textquotesingle{}}\NormalTok{)}\OperatorTok{;}
\NormalTok{decrypted }\OperatorTok{+=}\NormalTok{ decipher}\OperatorTok{.}\FunctionTok{final}\NormalTok{(}\StringTok{\textquotesingle{}utf8\textquotesingle{}}\NormalTok{)}\OperatorTok{;}
\BuiltInTok{console}\OperatorTok{.}\FunctionTok{log}\NormalTok{(decrypted)}\OperatorTok{;}
\CommentTok{// Prints: some clear text data}
\end{Highlighting}
\end{Shaded}

\begin{Shaded}
\begin{Highlighting}[]
\KeywordTok{const}\NormalTok{ \{}
\NormalTok{  scryptSync}\OperatorTok{,}
\NormalTok{  createDecipheriv}\OperatorTok{,}
\NormalTok{\} }\OperatorTok{=} \PreprocessorTok{require}\NormalTok{(}\StringTok{\textquotesingle{}node:crypto\textquotesingle{}}\NormalTok{)}\OperatorTok{;}
\KeywordTok{const}\NormalTok{ \{ }\BuiltInTok{Buffer}\NormalTok{ \} }\OperatorTok{=} \PreprocessorTok{require}\NormalTok{(}\StringTok{\textquotesingle{}node:buffer\textquotesingle{}}\NormalTok{)}\OperatorTok{;}

\KeywordTok{const}\NormalTok{ algorithm }\OperatorTok{=} \StringTok{\textquotesingle{}aes{-}192{-}cbc\textquotesingle{}}\OperatorTok{;}
\KeywordTok{const}\NormalTok{ password }\OperatorTok{=} \StringTok{\textquotesingle{}Password used to generate key\textquotesingle{}}\OperatorTok{;}
\CommentTok{// Use the async \textasciigrave{}crypto.scrypt()\textasciigrave{} instead.}
\KeywordTok{const}\NormalTok{ key }\OperatorTok{=} \FunctionTok{scryptSync}\NormalTok{(password}\OperatorTok{,} \StringTok{\textquotesingle{}salt\textquotesingle{}}\OperatorTok{,} \DecValTok{24}\NormalTok{)}\OperatorTok{;}
\CommentTok{// The IV is usually passed along with the ciphertext.}
\KeywordTok{const}\NormalTok{ iv }\OperatorTok{=} \BuiltInTok{Buffer}\OperatorTok{.}\FunctionTok{alloc}\NormalTok{(}\DecValTok{16}\OperatorTok{,} \DecValTok{0}\NormalTok{)}\OperatorTok{;} \CommentTok{// Initialization vector.}

\KeywordTok{const}\NormalTok{ decipher }\OperatorTok{=} \FunctionTok{createDecipheriv}\NormalTok{(algorithm}\OperatorTok{,}\NormalTok{ key}\OperatorTok{,}\NormalTok{ iv)}\OperatorTok{;}

\CommentTok{// Encrypted using same algorithm, key and iv.}
\KeywordTok{const}\NormalTok{ encrypted }\OperatorTok{=}
  \StringTok{\textquotesingle{}e5f79c5915c02171eec6b212d5520d44480993d7d622a7c4c2da32f6efda0ffa\textquotesingle{}}\OperatorTok{;}
\KeywordTok{let}\NormalTok{ decrypted }\OperatorTok{=}\NormalTok{ decipher}\OperatorTok{.}\FunctionTok{update}\NormalTok{(encrypted}\OperatorTok{,} \StringTok{\textquotesingle{}hex\textquotesingle{}}\OperatorTok{,} \StringTok{\textquotesingle{}utf8\textquotesingle{}}\NormalTok{)}\OperatorTok{;}
\NormalTok{decrypted }\OperatorTok{+=}\NormalTok{ decipher}\OperatorTok{.}\FunctionTok{final}\NormalTok{(}\StringTok{\textquotesingle{}utf8\textquotesingle{}}\NormalTok{)}\OperatorTok{;}
\BuiltInTok{console}\OperatorTok{.}\FunctionTok{log}\NormalTok{(decrypted)}\OperatorTok{;}
\CommentTok{// Prints: some clear text data}
\end{Highlighting}
\end{Shaded}

\subsubsection{\texorpdfstring{\texttt{decipher.final({[}outputEncoding{]})}}{decipher.final({[}outputEncoding{]})}}\label{decipher.finaloutputencoding}

\begin{itemize}
\tightlist
\item
  \texttt{outputEncoding} \{string\} The
  \href{buffer.md\#buffers-and-character-encodings}{encoding} of the
  return value.
\item
  Returns: \{Buffer \textbar{} string\} Any remaining deciphered
  contents. If \texttt{outputEncoding} is specified, a string is
  returned. If an \texttt{outputEncoding} is not provided, a
  \href{buffer.md}{\texttt{Buffer}} is returned.
\end{itemize}

Once the \texttt{decipher.final()} method has been called, the
\texttt{Decipher} object can no longer be used to decrypt data. Attempts
to call \texttt{decipher.final()} more than once will result in an error
being thrown.

\subsubsection{\texorpdfstring{\texttt{decipher.setAAD(buffer{[},\ options{]})}}{decipher.setAAD(buffer{[}, options{]})}}\label{decipher.setaadbuffer-options}

\begin{itemize}
\tightlist
\item
  \texttt{buffer}
  \{string\textbar ArrayBuffer\textbar Buffer\textbar TypedArray\textbar DataView\}
\item
  \texttt{options} \{Object\}
  \href{stream.md\#new-streamtransformoptions}{\texttt{stream.transform}
  options}

  \begin{itemize}
  \tightlist
  \item
    \texttt{plaintextLength} \{number\}
  \item
    \texttt{encoding} \{string\} String encoding to use when
    \texttt{buffer} is a string.
  \end{itemize}
\item
  Returns: \{Decipher\} for method chaining.
\end{itemize}

When using an authenticated encryption mode (\texttt{GCM}, \texttt{CCM},
\texttt{OCB}, and \texttt{chacha20-poly1305} are currently supported),
the \texttt{decipher.setAAD()} method sets the value used for the
\emph{additional authenticated data} (AAD) input parameter.

The \texttt{options} argument is optional for \texttt{GCM}. When using
\texttt{CCM}, the \texttt{plaintextLength} option must be specified and
its value must match the length of the ciphertext in bytes. See
\hyperref[ccm-mode]{CCM mode}.

The \texttt{decipher.setAAD()} method must be called before
\hyperref[decipherupdatedata-inputencoding-outputencoding]{\texttt{decipher.update()}}.

When passing a string as the \texttt{buffer}, please consider
\hyperref[using-strings-as-inputs-to-cryptographic-apis]{caveats when
using strings as inputs to cryptographic APIs}.

\subsubsection{\texorpdfstring{\texttt{decipher.setAuthTag(buffer{[},\ encoding{]})}}{decipher.setAuthTag(buffer{[}, encoding{]})}}\label{decipher.setauthtagbuffer-encoding}

\begin{itemize}
\tightlist
\item
  \texttt{buffer}
  \{string\textbar Buffer\textbar ArrayBuffer\textbar TypedArray\textbar DataView\}
\item
  \texttt{encoding} \{string\} String encoding to use when
  \texttt{buffer} is a string.
\item
  Returns: \{Decipher\} for method chaining.
\end{itemize}

When using an authenticated encryption mode (\texttt{GCM}, \texttt{CCM},
\texttt{OCB}, and \texttt{chacha20-poly1305} are currently supported),
the \texttt{decipher.setAuthTag()} method is used to pass in the
received \emph{authentication tag}. If no tag is provided, or if the
cipher text has been tampered with,
\hyperref[decipherfinaloutputencoding]{\texttt{decipher.final()}} will
throw, indicating that the cipher text should be discarded due to failed
authentication. If the tag length is invalid according to
\href{https://nvlpubs.nist.gov/nistpubs/Legacy/SP/nistspecialpublication800-38d.pdf}{NIST
SP 800-38D} or does not match the value of the \texttt{authTagLength}
option, \texttt{decipher.setAuthTag()} will throw an error.

The \texttt{decipher.setAuthTag()} method must be called before
\hyperref[decipherupdatedata-inputencoding-outputencoding]{\texttt{decipher.update()}}
for \texttt{CCM} mode or before
\hyperref[decipherfinaloutputencoding]{\texttt{decipher.final()}} for
\texttt{GCM} and \texttt{OCB} modes and \texttt{chacha20-poly1305}.
\texttt{decipher.setAuthTag()} can only be called once.

When passing a string as the authentication tag, please consider
\hyperref[using-strings-as-inputs-to-cryptographic-apis]{caveats when
using strings as inputs to cryptographic APIs}.

\subsubsection{\texorpdfstring{\texttt{decipher.setAutoPadding({[}autoPadding{]})}}{decipher.setAutoPadding({[}autoPadding{]})}}\label{decipher.setautopaddingautopadding}

\begin{itemize}
\tightlist
\item
  \texttt{autoPadding} \{boolean\} \textbf{Default:} \texttt{true}
\item
  Returns: \{Decipher\} for method chaining.
\end{itemize}

When data has been encrypted without standard block padding, calling
\texttt{decipher.setAutoPadding(false)} will disable automatic padding
to prevent
\hyperref[decipherfinaloutputencoding]{\texttt{decipher.final()}} from
checking for and removing padding.

Turning auto padding off will only work if the input data's length is a
multiple of the ciphers block size.

The \texttt{decipher.setAutoPadding()} method must be called before
\hyperref[decipherfinaloutputencoding]{\texttt{decipher.final()}}.

\subsubsection{\texorpdfstring{\texttt{decipher.update(data{[},\ inputEncoding{]}{[},\ outputEncoding{]})}}{decipher.update(data{[}, inputEncoding{]}{[}, outputEncoding{]})}}\label{decipher.updatedata-inputencoding-outputencoding}

\begin{itemize}
\tightlist
\item
  \texttt{data}
  \{string\textbar Buffer\textbar TypedArray\textbar DataView\}
\item
  \texttt{inputEncoding} \{string\} The
  \href{buffer.md\#buffers-and-character-encodings}{encoding} of the
  \texttt{data} string.
\item
  \texttt{outputEncoding} \{string\} The
  \href{buffer.md\#buffers-and-character-encodings}{encoding} of the
  return value.
\item
  Returns: \{Buffer \textbar{} string\}
\end{itemize}

Updates the decipher with \texttt{data}. If the \texttt{inputEncoding}
argument is given, the \texttt{data} argument is a string using the
specified encoding. If the \texttt{inputEncoding} argument is not given,
\texttt{data} must be a \href{buffer.md}{\texttt{Buffer}}. If
\texttt{data} is a \href{buffer.md}{\texttt{Buffer}} then
\texttt{inputEncoding} is ignored.

The \texttt{outputEncoding} specifies the output format of the
enciphered data. If the \texttt{outputEncoding} is specified, a string
using the specified encoding is returned. If no \texttt{outputEncoding}
is provided, a \href{buffer.md}{\texttt{Buffer}} is returned.

The \texttt{decipher.update()} method can be called multiple times with
new data until
\hyperref[decipherfinaloutputencoding]{\texttt{decipher.final()}} is
called. Calling \texttt{decipher.update()} after
\hyperref[decipherfinaloutputencoding]{\texttt{decipher.final()}} will
result in an error being thrown.

\subsection{\texorpdfstring{Class:
\texttt{DiffieHellman}}{Class: DiffieHellman}}\label{class-diffiehellman}

The \texttt{DiffieHellman} class is a utility for creating
Diffie-Hellman key exchanges.

Instances of the \texttt{DiffieHellman} class can be created using the
\hyperref[cryptocreatediffiehellmanprime-primeencoding-generator-generatorencoding]{\texttt{crypto.createDiffieHellman()}}
function.

\begin{Shaded}
\begin{Highlighting}[]
\ImportTok{import}\NormalTok{ assert }\ImportTok{from} \StringTok{\textquotesingle{}node:assert\textquotesingle{}}\OperatorTok{;}

\KeywordTok{const}\NormalTok{ \{}
\NormalTok{  createDiffieHellman}\OperatorTok{,}
\NormalTok{\} }\OperatorTok{=} \ControlFlowTok{await} \ImportTok{import}\NormalTok{(}\StringTok{\textquotesingle{}node:crypto\textquotesingle{}}\NormalTok{)}\OperatorTok{;}

\CommentTok{// Generate Alice\textquotesingle{}s keys...}
\KeywordTok{const}\NormalTok{ alice }\OperatorTok{=} \FunctionTok{createDiffieHellman}\NormalTok{(}\DecValTok{2048}\NormalTok{)}\OperatorTok{;}
\KeywordTok{const}\NormalTok{ aliceKey }\OperatorTok{=}\NormalTok{ alice}\OperatorTok{.}\FunctionTok{generateKeys}\NormalTok{()}\OperatorTok{;}

\CommentTok{// Generate Bob\textquotesingle{}s keys...}
\KeywordTok{const}\NormalTok{ bob }\OperatorTok{=} \FunctionTok{createDiffieHellman}\NormalTok{(alice}\OperatorTok{.}\FunctionTok{getPrime}\NormalTok{()}\OperatorTok{,}\NormalTok{ alice}\OperatorTok{.}\FunctionTok{getGenerator}\NormalTok{())}\OperatorTok{;}
\KeywordTok{const}\NormalTok{ bobKey }\OperatorTok{=}\NormalTok{ bob}\OperatorTok{.}\FunctionTok{generateKeys}\NormalTok{()}\OperatorTok{;}

\CommentTok{// Exchange and generate the secret...}
\KeywordTok{const}\NormalTok{ aliceSecret }\OperatorTok{=}\NormalTok{ alice}\OperatorTok{.}\FunctionTok{computeSecret}\NormalTok{(bobKey)}\OperatorTok{;}
\KeywordTok{const}\NormalTok{ bobSecret }\OperatorTok{=}\NormalTok{ bob}\OperatorTok{.}\FunctionTok{computeSecret}\NormalTok{(aliceKey)}\OperatorTok{;}

\CommentTok{// OK}
\NormalTok{assert}\OperatorTok{.}\FunctionTok{strictEqual}\NormalTok{(aliceSecret}\OperatorTok{.}\FunctionTok{toString}\NormalTok{(}\StringTok{\textquotesingle{}hex\textquotesingle{}}\NormalTok{)}\OperatorTok{,}\NormalTok{ bobSecret}\OperatorTok{.}\FunctionTok{toString}\NormalTok{(}\StringTok{\textquotesingle{}hex\textquotesingle{}}\NormalTok{))}\OperatorTok{;}
\end{Highlighting}
\end{Shaded}

\begin{Shaded}
\begin{Highlighting}[]
\KeywordTok{const}\NormalTok{ assert }\OperatorTok{=} \PreprocessorTok{require}\NormalTok{(}\StringTok{\textquotesingle{}node:assert\textquotesingle{}}\NormalTok{)}\OperatorTok{;}

\KeywordTok{const}\NormalTok{ \{}
\NormalTok{  createDiffieHellman}\OperatorTok{,}
\NormalTok{\} }\OperatorTok{=} \PreprocessorTok{require}\NormalTok{(}\StringTok{\textquotesingle{}node:crypto\textquotesingle{}}\NormalTok{)}\OperatorTok{;}

\CommentTok{// Generate Alice\textquotesingle{}s keys...}
\KeywordTok{const}\NormalTok{ alice }\OperatorTok{=} \FunctionTok{createDiffieHellman}\NormalTok{(}\DecValTok{2048}\NormalTok{)}\OperatorTok{;}
\KeywordTok{const}\NormalTok{ aliceKey }\OperatorTok{=}\NormalTok{ alice}\OperatorTok{.}\FunctionTok{generateKeys}\NormalTok{()}\OperatorTok{;}

\CommentTok{// Generate Bob\textquotesingle{}s keys...}
\KeywordTok{const}\NormalTok{ bob }\OperatorTok{=} \FunctionTok{createDiffieHellman}\NormalTok{(alice}\OperatorTok{.}\FunctionTok{getPrime}\NormalTok{()}\OperatorTok{,}\NormalTok{ alice}\OperatorTok{.}\FunctionTok{getGenerator}\NormalTok{())}\OperatorTok{;}
\KeywordTok{const}\NormalTok{ bobKey }\OperatorTok{=}\NormalTok{ bob}\OperatorTok{.}\FunctionTok{generateKeys}\NormalTok{()}\OperatorTok{;}

\CommentTok{// Exchange and generate the secret...}
\KeywordTok{const}\NormalTok{ aliceSecret }\OperatorTok{=}\NormalTok{ alice}\OperatorTok{.}\FunctionTok{computeSecret}\NormalTok{(bobKey)}\OperatorTok{;}
\KeywordTok{const}\NormalTok{ bobSecret }\OperatorTok{=}\NormalTok{ bob}\OperatorTok{.}\FunctionTok{computeSecret}\NormalTok{(aliceKey)}\OperatorTok{;}

\CommentTok{// OK}
\NormalTok{assert}\OperatorTok{.}\FunctionTok{strictEqual}\NormalTok{(aliceSecret}\OperatorTok{.}\FunctionTok{toString}\NormalTok{(}\StringTok{\textquotesingle{}hex\textquotesingle{}}\NormalTok{)}\OperatorTok{,}\NormalTok{ bobSecret}\OperatorTok{.}\FunctionTok{toString}\NormalTok{(}\StringTok{\textquotesingle{}hex\textquotesingle{}}\NormalTok{))}\OperatorTok{;}
\end{Highlighting}
\end{Shaded}

\subsubsection{\texorpdfstring{\texttt{diffieHellman.computeSecret(otherPublicKey{[},\ inputEncoding{]}{[},\ outputEncoding{]})}}{diffieHellman.computeSecret(otherPublicKey{[}, inputEncoding{]}{[}, outputEncoding{]})}}\label{diffiehellman.computesecretotherpublickey-inputencoding-outputencoding}

\begin{itemize}
\tightlist
\item
  \texttt{otherPublicKey}
  \{string\textbar ArrayBuffer\textbar Buffer\textbar TypedArray\textbar DataView\}
\item
  \texttt{inputEncoding} \{string\} The
  \href{buffer.md\#buffers-and-character-encodings}{encoding} of an
  \texttt{otherPublicKey} string.
\item
  \texttt{outputEncoding} \{string\} The
  \href{buffer.md\#buffers-and-character-encodings}{encoding} of the
  return value.
\item
  Returns: \{Buffer \textbar{} string\}
\end{itemize}

Computes the shared secret using \texttt{otherPublicKey} as the other
party's public key and returns the computed shared secret. The supplied
key is interpreted using the specified \texttt{inputEncoding}, and
secret is encoded using specified \texttt{outputEncoding}. If the
\texttt{inputEncoding} is not provided, \texttt{otherPublicKey} is
expected to be a \href{buffer.md}{\texttt{Buffer}}, \texttt{TypedArray},
or \texttt{DataView}.

If \texttt{outputEncoding} is given a string is returned; otherwise, a
\href{buffer.md}{\texttt{Buffer}} is returned.

\subsubsection{\texorpdfstring{\texttt{diffieHellman.generateKeys({[}encoding{]})}}{diffieHellman.generateKeys({[}encoding{]})}}\label{diffiehellman.generatekeysencoding}

\begin{itemize}
\tightlist
\item
  \texttt{encoding} \{string\} The
  \href{buffer.md\#buffers-and-character-encodings}{encoding} of the
  return value.
\item
  Returns: \{Buffer \textbar{} string\}
\end{itemize}

Generates private and public Diffie-Hellman key values unless they have
been generated or computed already, and returns the public key in the
specified \texttt{encoding}. This key should be transferred to the other
party. If \texttt{encoding} is provided a string is returned; otherwise
a \href{buffer.md}{\texttt{Buffer}} is returned.

This function is a thin wrapper around
\href{https://www.openssl.org/docs/man3.0/man3/DH_generate_key.html}{\texttt{DH\_generate\_key()}}.
In particular, once a private key has been generated or set, calling
this function only updates the public key but does not generate a new
private key.

\subsubsection{\texorpdfstring{\texttt{diffieHellman.getGenerator({[}encoding{]})}}{diffieHellman.getGenerator({[}encoding{]})}}\label{diffiehellman.getgeneratorencoding}

\begin{itemize}
\tightlist
\item
  \texttt{encoding} \{string\} The
  \href{buffer.md\#buffers-and-character-encodings}{encoding} of the
  return value.
\item
  Returns: \{Buffer \textbar{} string\}
\end{itemize}

Returns the Diffie-Hellman generator in the specified \texttt{encoding}.
If \texttt{encoding} is provided a string is returned; otherwise a
\href{buffer.md}{\texttt{Buffer}} is returned.

\subsubsection{\texorpdfstring{\texttt{diffieHellman.getPrime({[}encoding{]})}}{diffieHellman.getPrime({[}encoding{]})}}\label{diffiehellman.getprimeencoding}

\begin{itemize}
\tightlist
\item
  \texttt{encoding} \{string\} The
  \href{buffer.md\#buffers-and-character-encodings}{encoding} of the
  return value.
\item
  Returns: \{Buffer \textbar{} string\}
\end{itemize}

Returns the Diffie-Hellman prime in the specified \texttt{encoding}. If
\texttt{encoding} is provided a string is returned; otherwise a
\href{buffer.md}{\texttt{Buffer}} is returned.

\subsubsection{\texorpdfstring{\texttt{diffieHellman.getPrivateKey({[}encoding{]})}}{diffieHellman.getPrivateKey({[}encoding{]})}}\label{diffiehellman.getprivatekeyencoding}

\begin{itemize}
\tightlist
\item
  \texttt{encoding} \{string\} The
  \href{buffer.md\#buffers-and-character-encodings}{encoding} of the
  return value.
\item
  Returns: \{Buffer \textbar{} string\}
\end{itemize}

Returns the Diffie-Hellman private key in the specified
\texttt{encoding}. If \texttt{encoding} is provided a string is
returned; otherwise a \href{buffer.md}{\texttt{Buffer}} is returned.

\subsubsection{\texorpdfstring{\texttt{diffieHellman.getPublicKey({[}encoding{]})}}{diffieHellman.getPublicKey({[}encoding{]})}}\label{diffiehellman.getpublickeyencoding}

\begin{itemize}
\tightlist
\item
  \texttt{encoding} \{string\} The
  \href{buffer.md\#buffers-and-character-encodings}{encoding} of the
  return value.
\item
  Returns: \{Buffer \textbar{} string\}
\end{itemize}

Returns the Diffie-Hellman public key in the specified
\texttt{encoding}. If \texttt{encoding} is provided a string is
returned; otherwise a \href{buffer.md}{\texttt{Buffer}} is returned.

\subsubsection{\texorpdfstring{\texttt{diffieHellman.setPrivateKey(privateKey{[},\ encoding{]})}}{diffieHellman.setPrivateKey(privateKey{[}, encoding{]})}}\label{diffiehellman.setprivatekeyprivatekey-encoding}

\begin{itemize}
\tightlist
\item
  \texttt{privateKey}
  \{string\textbar ArrayBuffer\textbar Buffer\textbar TypedArray\textbar DataView\}
\item
  \texttt{encoding} \{string\} The
  \href{buffer.md\#buffers-and-character-encodings}{encoding} of the
  \texttt{privateKey} string.
\end{itemize}

Sets the Diffie-Hellman private key. If the \texttt{encoding} argument
is provided, \texttt{privateKey} is expected to be a string. If no
\texttt{encoding} is provided, \texttt{privateKey} is expected to be a
\href{buffer.md}{\texttt{Buffer}}, \texttt{TypedArray}, or
\texttt{DataView}.

This function does not automatically compute the associated public key.
Either
\hyperref[diffiehellmansetpublickeypublickey-encoding]{\texttt{diffieHellman.setPublicKey()}}
or
\hyperref[diffiehellmangeneratekeysencoding]{\texttt{diffieHellman.generateKeys()}}
can be used to manually provide the public key or to automatically
derive it.

\subsubsection{\texorpdfstring{\texttt{diffieHellman.setPublicKey(publicKey{[},\ encoding{]})}}{diffieHellman.setPublicKey(publicKey{[}, encoding{]})}}\label{diffiehellman.setpublickeypublickey-encoding}

\begin{itemize}
\tightlist
\item
  \texttt{publicKey}
  \{string\textbar ArrayBuffer\textbar Buffer\textbar TypedArray\textbar DataView\}
\item
  \texttt{encoding} \{string\} The
  \href{buffer.md\#buffers-and-character-encodings}{encoding} of the
  \texttt{publicKey} string.
\end{itemize}

Sets the Diffie-Hellman public key. If the \texttt{encoding} argument is
provided, \texttt{publicKey} is expected to be a string. If no
\texttt{encoding} is provided, \texttt{publicKey} is expected to be a
\href{buffer.md}{\texttt{Buffer}}, \texttt{TypedArray}, or
\texttt{DataView}.

\subsubsection{\texorpdfstring{\texttt{diffieHellman.verifyError}}{diffieHellman.verifyError}}\label{diffiehellman.verifyerror}

A bit field containing any warnings and/or errors resulting from a check
performed during initialization of the \texttt{DiffieHellman} object.

The following values are valid for this property (as defined in
\texttt{node:constants} module):

\begin{itemize}
\tightlist
\item
  \texttt{DH\_CHECK\_P\_NOT\_SAFE\_PRIME}
\item
  \texttt{DH\_CHECK\_P\_NOT\_PRIME}
\item
  \texttt{DH\_UNABLE\_TO\_CHECK\_GENERATOR}
\item
  \texttt{DH\_NOT\_SUITABLE\_GENERATOR}
\end{itemize}

\subsection{\texorpdfstring{Class:
\texttt{DiffieHellmanGroup}}{Class: DiffieHellmanGroup}}\label{class-diffiehellmangroup}

The \texttt{DiffieHellmanGroup} class takes a well-known modp group as
its argument. It works the same as \texttt{DiffieHellman}, except that
it does not allow changing its keys after creation. In other words, it
does not implement \texttt{setPublicKey()} or \texttt{setPrivateKey()}
methods.

\begin{Shaded}
\begin{Highlighting}[]
\KeywordTok{const}\NormalTok{ \{ createDiffieHellmanGroup \} }\OperatorTok{=} \ControlFlowTok{await} \ImportTok{import}\NormalTok{(}\StringTok{\textquotesingle{}node:crypto\textquotesingle{}}\NormalTok{)}\OperatorTok{;}
\KeywordTok{const}\NormalTok{ dh }\OperatorTok{=} \FunctionTok{createDiffieHellmanGroup}\NormalTok{(}\StringTok{\textquotesingle{}modp16\textquotesingle{}}\NormalTok{)}\OperatorTok{;}
\end{Highlighting}
\end{Shaded}

\begin{Shaded}
\begin{Highlighting}[]
\KeywordTok{const}\NormalTok{ \{ createDiffieHellmanGroup \} }\OperatorTok{=} \PreprocessorTok{require}\NormalTok{(}\StringTok{\textquotesingle{}node:crypto\textquotesingle{}}\NormalTok{)}\OperatorTok{;}
\KeywordTok{const}\NormalTok{ dh }\OperatorTok{=} \FunctionTok{createDiffieHellmanGroup}\NormalTok{(}\StringTok{\textquotesingle{}modp16\textquotesingle{}}\NormalTok{)}\OperatorTok{;}
\end{Highlighting}
\end{Shaded}

The following groups are supported:

\begin{itemize}
\tightlist
\item
  \texttt{\textquotesingle{}modp14\textquotesingle{}} (2048 bits,
  \href{https://www.rfc-editor.org/rfc/rfc3526.txt}{RFC 3526} Section 3)
\item
  \texttt{\textquotesingle{}modp15\textquotesingle{}} (3072 bits,
  \href{https://www.rfc-editor.org/rfc/rfc3526.txt}{RFC 3526} Section 4)
\item
  \texttt{\textquotesingle{}modp16\textquotesingle{}} (4096 bits,
  \href{https://www.rfc-editor.org/rfc/rfc3526.txt}{RFC 3526} Section 5)
\item
  \texttt{\textquotesingle{}modp17\textquotesingle{}} (6144 bits,
  \href{https://www.rfc-editor.org/rfc/rfc3526.txt}{RFC 3526} Section 6)
\item
  \texttt{\textquotesingle{}modp18\textquotesingle{}} (8192 bits,
  \href{https://www.rfc-editor.org/rfc/rfc3526.txt}{RFC 3526} Section 7)
\end{itemize}

The following groups are still supported but deprecated (see
\hyperref[support-for-weak-or-compromised-algorithms]{Caveats}):

\begin{itemize}
\tightlist
\item
  \texttt{\textquotesingle{}modp1\textquotesingle{}} (768 bits,
  \href{https://www.rfc-editor.org/rfc/rfc2409.txt}{RFC 2409} Section
  6.1) {}
\item
  \texttt{\textquotesingle{}modp2\textquotesingle{}} (1024 bits,
  \href{https://www.rfc-editor.org/rfc/rfc2409.txt}{RFC 2409} Section
  6.2) {}
\item
  \texttt{\textquotesingle{}modp5\textquotesingle{}} (1536 bits,
  \href{https://www.rfc-editor.org/rfc/rfc3526.txt}{RFC 3526} Section 2)
  {}
\end{itemize}

These deprecated groups might be removed in future versions of Node.js.

\subsection{\texorpdfstring{Class:
\texttt{ECDH}}{Class: ECDH}}\label{class-ecdh}

The \texttt{ECDH} class is a utility for creating Elliptic Curve
Diffie-Hellman (ECDH) key exchanges.

Instances of the \texttt{ECDH} class can be created using the
\hyperref[cryptocreateecdhcurvename]{\texttt{crypto.createECDH()}}
function.

\begin{Shaded}
\begin{Highlighting}[]
\ImportTok{import}\NormalTok{ assert }\ImportTok{from} \StringTok{\textquotesingle{}node:assert\textquotesingle{}}\OperatorTok{;}

\KeywordTok{const}\NormalTok{ \{}
\NormalTok{  createECDH}\OperatorTok{,}
\NormalTok{\} }\OperatorTok{=} \ControlFlowTok{await} \ImportTok{import}\NormalTok{(}\StringTok{\textquotesingle{}node:crypto\textquotesingle{}}\NormalTok{)}\OperatorTok{;}

\CommentTok{// Generate Alice\textquotesingle{}s keys...}
\KeywordTok{const}\NormalTok{ alice }\OperatorTok{=} \FunctionTok{createECDH}\NormalTok{(}\StringTok{\textquotesingle{}secp521r1\textquotesingle{}}\NormalTok{)}\OperatorTok{;}
\KeywordTok{const}\NormalTok{ aliceKey }\OperatorTok{=}\NormalTok{ alice}\OperatorTok{.}\FunctionTok{generateKeys}\NormalTok{()}\OperatorTok{;}

\CommentTok{// Generate Bob\textquotesingle{}s keys...}
\KeywordTok{const}\NormalTok{ bob }\OperatorTok{=} \FunctionTok{createECDH}\NormalTok{(}\StringTok{\textquotesingle{}secp521r1\textquotesingle{}}\NormalTok{)}\OperatorTok{;}
\KeywordTok{const}\NormalTok{ bobKey }\OperatorTok{=}\NormalTok{ bob}\OperatorTok{.}\FunctionTok{generateKeys}\NormalTok{()}\OperatorTok{;}

\CommentTok{// Exchange and generate the secret...}
\KeywordTok{const}\NormalTok{ aliceSecret }\OperatorTok{=}\NormalTok{ alice}\OperatorTok{.}\FunctionTok{computeSecret}\NormalTok{(bobKey)}\OperatorTok{;}
\KeywordTok{const}\NormalTok{ bobSecret }\OperatorTok{=}\NormalTok{ bob}\OperatorTok{.}\FunctionTok{computeSecret}\NormalTok{(aliceKey)}\OperatorTok{;}

\NormalTok{assert}\OperatorTok{.}\FunctionTok{strictEqual}\NormalTok{(aliceSecret}\OperatorTok{.}\FunctionTok{toString}\NormalTok{(}\StringTok{\textquotesingle{}hex\textquotesingle{}}\NormalTok{)}\OperatorTok{,}\NormalTok{ bobSecret}\OperatorTok{.}\FunctionTok{toString}\NormalTok{(}\StringTok{\textquotesingle{}hex\textquotesingle{}}\NormalTok{))}\OperatorTok{;}
\CommentTok{// OK}
\end{Highlighting}
\end{Shaded}

\begin{Shaded}
\begin{Highlighting}[]
\KeywordTok{const}\NormalTok{ assert }\OperatorTok{=} \PreprocessorTok{require}\NormalTok{(}\StringTok{\textquotesingle{}node:assert\textquotesingle{}}\NormalTok{)}\OperatorTok{;}

\KeywordTok{const}\NormalTok{ \{}
\NormalTok{  createECDH}\OperatorTok{,}
\NormalTok{\} }\OperatorTok{=} \PreprocessorTok{require}\NormalTok{(}\StringTok{\textquotesingle{}node:crypto\textquotesingle{}}\NormalTok{)}\OperatorTok{;}

\CommentTok{// Generate Alice\textquotesingle{}s keys...}
\KeywordTok{const}\NormalTok{ alice }\OperatorTok{=} \FunctionTok{createECDH}\NormalTok{(}\StringTok{\textquotesingle{}secp521r1\textquotesingle{}}\NormalTok{)}\OperatorTok{;}
\KeywordTok{const}\NormalTok{ aliceKey }\OperatorTok{=}\NormalTok{ alice}\OperatorTok{.}\FunctionTok{generateKeys}\NormalTok{()}\OperatorTok{;}

\CommentTok{// Generate Bob\textquotesingle{}s keys...}
\KeywordTok{const}\NormalTok{ bob }\OperatorTok{=} \FunctionTok{createECDH}\NormalTok{(}\StringTok{\textquotesingle{}secp521r1\textquotesingle{}}\NormalTok{)}\OperatorTok{;}
\KeywordTok{const}\NormalTok{ bobKey }\OperatorTok{=}\NormalTok{ bob}\OperatorTok{.}\FunctionTok{generateKeys}\NormalTok{()}\OperatorTok{;}

\CommentTok{// Exchange and generate the secret...}
\KeywordTok{const}\NormalTok{ aliceSecret }\OperatorTok{=}\NormalTok{ alice}\OperatorTok{.}\FunctionTok{computeSecret}\NormalTok{(bobKey)}\OperatorTok{;}
\KeywordTok{const}\NormalTok{ bobSecret }\OperatorTok{=}\NormalTok{ bob}\OperatorTok{.}\FunctionTok{computeSecret}\NormalTok{(aliceKey)}\OperatorTok{;}

\NormalTok{assert}\OperatorTok{.}\FunctionTok{strictEqual}\NormalTok{(aliceSecret}\OperatorTok{.}\FunctionTok{toString}\NormalTok{(}\StringTok{\textquotesingle{}hex\textquotesingle{}}\NormalTok{)}\OperatorTok{,}\NormalTok{ bobSecret}\OperatorTok{.}\FunctionTok{toString}\NormalTok{(}\StringTok{\textquotesingle{}hex\textquotesingle{}}\NormalTok{))}\OperatorTok{;}
\CommentTok{// OK}
\end{Highlighting}
\end{Shaded}

\subsubsection{\texorpdfstring{Static method:
\texttt{ECDH.convertKey(key,\ curve{[},\ inputEncoding{[},\ outputEncoding{[},\ format{]}{]}{]})}}{Static method: ECDH.convertKey(key, curve{[}, inputEncoding{[}, outputEncoding{[}, format{]}{]}{]})}}\label{static-method-ecdh.convertkeykey-curve-inputencoding-outputencoding-format}

\begin{itemize}
\tightlist
\item
  \texttt{key}
  \{string\textbar ArrayBuffer\textbar Buffer\textbar TypedArray\textbar DataView\}
\item
  \texttt{curve} \{string\}
\item
  \texttt{inputEncoding} \{string\} The
  \href{buffer.md\#buffers-and-character-encodings}{encoding} of the
  \texttt{key} string.
\item
  \texttt{outputEncoding} \{string\} The
  \href{buffer.md\#buffers-and-character-encodings}{encoding} of the
  return value.
\item
  \texttt{format} \{string\} \textbf{Default:}
  \texttt{\textquotesingle{}uncompressed\textquotesingle{}}
\item
  Returns: \{Buffer \textbar{} string\}
\end{itemize}

Converts the EC Diffie-Hellman public key specified by \texttt{key} and
\texttt{curve} to the format specified by \texttt{format}. The
\texttt{format} argument specifies point encoding and can be
\texttt{\textquotesingle{}compressed\textquotesingle{}},
\texttt{\textquotesingle{}uncompressed\textquotesingle{}} or
\texttt{\textquotesingle{}hybrid\textquotesingle{}}. The supplied key is
interpreted using the specified \texttt{inputEncoding}, and the returned
key is encoded using the specified \texttt{outputEncoding}.

Use \hyperref[cryptogetcurves]{\texttt{crypto.getCurves()}} to obtain a
list of available curve names. On recent OpenSSL releases,
\texttt{openssl\ ecparam\ -list\_curves} will also display the name and
description of each available elliptic curve.

If \texttt{format} is not specified the point will be returned in
\texttt{\textquotesingle{}uncompressed\textquotesingle{}} format.

If the \texttt{inputEncoding} is not provided, \texttt{key} is expected
to be a \href{buffer.md}{\texttt{Buffer}}, \texttt{TypedArray}, or
\texttt{DataView}.

Example (uncompressing a key):

\begin{Shaded}
\begin{Highlighting}[]
\KeywordTok{const}\NormalTok{ \{}
\NormalTok{  createECDH}\OperatorTok{,}
\NormalTok{  ECDH}\OperatorTok{,}
\NormalTok{\} }\OperatorTok{=} \ControlFlowTok{await} \ImportTok{import}\NormalTok{(}\StringTok{\textquotesingle{}node:crypto\textquotesingle{}}\NormalTok{)}\OperatorTok{;}

\KeywordTok{const}\NormalTok{ ecdh }\OperatorTok{=} \FunctionTok{createECDH}\NormalTok{(}\StringTok{\textquotesingle{}secp256k1\textquotesingle{}}\NormalTok{)}\OperatorTok{;}
\NormalTok{ecdh}\OperatorTok{.}\FunctionTok{generateKeys}\NormalTok{()}\OperatorTok{;}

\KeywordTok{const}\NormalTok{ compressedKey }\OperatorTok{=}\NormalTok{ ecdh}\OperatorTok{.}\FunctionTok{getPublicKey}\NormalTok{(}\StringTok{\textquotesingle{}hex\textquotesingle{}}\OperatorTok{,} \StringTok{\textquotesingle{}compressed\textquotesingle{}}\NormalTok{)}\OperatorTok{;}

\KeywordTok{const}\NormalTok{ uncompressedKey }\OperatorTok{=}\NormalTok{ ECDH}\OperatorTok{.}\FunctionTok{convertKey}\NormalTok{(compressedKey}\OperatorTok{,}
                                        \StringTok{\textquotesingle{}secp256k1\textquotesingle{}}\OperatorTok{,}
                                        \StringTok{\textquotesingle{}hex\textquotesingle{}}\OperatorTok{,}
                                        \StringTok{\textquotesingle{}hex\textquotesingle{}}\OperatorTok{,}
                                        \StringTok{\textquotesingle{}uncompressed\textquotesingle{}}\NormalTok{)}\OperatorTok{;}

\CommentTok{// The converted key and the uncompressed public key should be the same}
\BuiltInTok{console}\OperatorTok{.}\FunctionTok{log}\NormalTok{(uncompressedKey }\OperatorTok{===}\NormalTok{ ecdh}\OperatorTok{.}\FunctionTok{getPublicKey}\NormalTok{(}\StringTok{\textquotesingle{}hex\textquotesingle{}}\NormalTok{))}\OperatorTok{;}
\end{Highlighting}
\end{Shaded}

\begin{Shaded}
\begin{Highlighting}[]
\KeywordTok{const}\NormalTok{ \{}
\NormalTok{  createECDH}\OperatorTok{,}
\NormalTok{  ECDH}\OperatorTok{,}
\NormalTok{\} }\OperatorTok{=} \PreprocessorTok{require}\NormalTok{(}\StringTok{\textquotesingle{}node:crypto\textquotesingle{}}\NormalTok{)}\OperatorTok{;}

\KeywordTok{const}\NormalTok{ ecdh }\OperatorTok{=} \FunctionTok{createECDH}\NormalTok{(}\StringTok{\textquotesingle{}secp256k1\textquotesingle{}}\NormalTok{)}\OperatorTok{;}
\NormalTok{ecdh}\OperatorTok{.}\FunctionTok{generateKeys}\NormalTok{()}\OperatorTok{;}

\KeywordTok{const}\NormalTok{ compressedKey }\OperatorTok{=}\NormalTok{ ecdh}\OperatorTok{.}\FunctionTok{getPublicKey}\NormalTok{(}\StringTok{\textquotesingle{}hex\textquotesingle{}}\OperatorTok{,} \StringTok{\textquotesingle{}compressed\textquotesingle{}}\NormalTok{)}\OperatorTok{;}

\KeywordTok{const}\NormalTok{ uncompressedKey }\OperatorTok{=}\NormalTok{ ECDH}\OperatorTok{.}\FunctionTok{convertKey}\NormalTok{(compressedKey}\OperatorTok{,}
                                        \StringTok{\textquotesingle{}secp256k1\textquotesingle{}}\OperatorTok{,}
                                        \StringTok{\textquotesingle{}hex\textquotesingle{}}\OperatorTok{,}
                                        \StringTok{\textquotesingle{}hex\textquotesingle{}}\OperatorTok{,}
                                        \StringTok{\textquotesingle{}uncompressed\textquotesingle{}}\NormalTok{)}\OperatorTok{;}

\CommentTok{// The converted key and the uncompressed public key should be the same}
\BuiltInTok{console}\OperatorTok{.}\FunctionTok{log}\NormalTok{(uncompressedKey }\OperatorTok{===}\NormalTok{ ecdh}\OperatorTok{.}\FunctionTok{getPublicKey}\NormalTok{(}\StringTok{\textquotesingle{}hex\textquotesingle{}}\NormalTok{))}\OperatorTok{;}
\end{Highlighting}
\end{Shaded}

\subsubsection{\texorpdfstring{\texttt{ecdh.computeSecret(otherPublicKey{[},\ inputEncoding{]}{[},\ outputEncoding{]})}}{ecdh.computeSecret(otherPublicKey{[}, inputEncoding{]}{[}, outputEncoding{]})}}\label{ecdh.computesecretotherpublickey-inputencoding-outputencoding}

\begin{itemize}
\tightlist
\item
  \texttt{otherPublicKey}
  \{string\textbar ArrayBuffer\textbar Buffer\textbar TypedArray\textbar DataView\}
\item
  \texttt{inputEncoding} \{string\} The
  \href{buffer.md\#buffers-and-character-encodings}{encoding} of the
  \texttt{otherPublicKey} string.
\item
  \texttt{outputEncoding} \{string\} The
  \href{buffer.md\#buffers-and-character-encodings}{encoding} of the
  return value.
\item
  Returns: \{Buffer \textbar{} string\}
\end{itemize}

Computes the shared secret using \texttt{otherPublicKey} as the other
party's public key and returns the computed shared secret. The supplied
key is interpreted using specified \texttt{inputEncoding}, and the
returned secret is encoded using the specified \texttt{outputEncoding}.
If the \texttt{inputEncoding} is not provided, \texttt{otherPublicKey}
is expected to be a \href{buffer.md}{\texttt{Buffer}},
\texttt{TypedArray}, or \texttt{DataView}.

If \texttt{outputEncoding} is given a string will be returned; otherwise
a \href{buffer.md}{\texttt{Buffer}} is returned.

\texttt{ecdh.computeSecret} will throw an
\texttt{ERR\_CRYPTO\_ECDH\_INVALID\_PUBLIC\_KEY} error when
\texttt{otherPublicKey} lies outside of the elliptic curve. Since
\texttt{otherPublicKey} is usually supplied from a remote user over an
insecure network, be sure to handle this exception accordingly.

\subsubsection{\texorpdfstring{\texttt{ecdh.generateKeys({[}encoding{[},\ format{]}{]})}}{ecdh.generateKeys({[}encoding{[}, format{]}{]})}}\label{ecdh.generatekeysencoding-format}

\begin{itemize}
\tightlist
\item
  \texttt{encoding} \{string\} The
  \href{buffer.md\#buffers-and-character-encodings}{encoding} of the
  return value.
\item
  \texttt{format} \{string\} \textbf{Default:}
  \texttt{\textquotesingle{}uncompressed\textquotesingle{}}
\item
  Returns: \{Buffer \textbar{} string\}
\end{itemize}

Generates private and public EC Diffie-Hellman key values, and returns
the public key in the specified \texttt{format} and \texttt{encoding}.
This key should be transferred to the other party.

The \texttt{format} argument specifies point encoding and can be
\texttt{\textquotesingle{}compressed\textquotesingle{}} or
\texttt{\textquotesingle{}uncompressed\textquotesingle{}}. If
\texttt{format} is not specified, the point will be returned in
\texttt{\textquotesingle{}uncompressed\textquotesingle{}} format.

If \texttt{encoding} is provided a string is returned; otherwise a
\href{buffer.md}{\texttt{Buffer}} is returned.

\subsubsection{\texorpdfstring{\texttt{ecdh.getPrivateKey({[}encoding{]})}}{ecdh.getPrivateKey({[}encoding{]})}}\label{ecdh.getprivatekeyencoding}

\begin{itemize}
\tightlist
\item
  \texttt{encoding} \{string\} The
  \href{buffer.md\#buffers-and-character-encodings}{encoding} of the
  return value.
\item
  Returns: \{Buffer \textbar{} string\} The EC Diffie-Hellman in the
  specified \texttt{encoding}.
\end{itemize}

If \texttt{encoding} is specified, a string is returned; otherwise a
\href{buffer.md}{\texttt{Buffer}} is returned.

\subsubsection{\texorpdfstring{\texttt{ecdh.getPublicKey({[}encoding{]}{[},\ format{]})}}{ecdh.getPublicKey({[}encoding{]}{[}, format{]})}}\label{ecdh.getpublickeyencoding-format}

\begin{itemize}
\tightlist
\item
  \texttt{encoding} \{string\} The
  \href{buffer.md\#buffers-and-character-encodings}{encoding} of the
  return value.
\item
  \texttt{format} \{string\} \textbf{Default:}
  \texttt{\textquotesingle{}uncompressed\textquotesingle{}}
\item
  Returns: \{Buffer \textbar{} string\} The EC Diffie-Hellman public key
  in the specified \texttt{encoding} and \texttt{format}.
\end{itemize}

The \texttt{format} argument specifies point encoding and can be
\texttt{\textquotesingle{}compressed\textquotesingle{}} or
\texttt{\textquotesingle{}uncompressed\textquotesingle{}}. If
\texttt{format} is not specified the point will be returned in
\texttt{\textquotesingle{}uncompressed\textquotesingle{}} format.

If \texttt{encoding} is specified, a string is returned; otherwise a
\href{buffer.md}{\texttt{Buffer}} is returned.

\subsubsection{\texorpdfstring{\texttt{ecdh.setPrivateKey(privateKey{[},\ encoding{]})}}{ecdh.setPrivateKey(privateKey{[}, encoding{]})}}\label{ecdh.setprivatekeyprivatekey-encoding}

\begin{itemize}
\tightlist
\item
  \texttt{privateKey}
  \{string\textbar ArrayBuffer\textbar Buffer\textbar TypedArray\textbar DataView\}
\item
  \texttt{encoding} \{string\} The
  \href{buffer.md\#buffers-and-character-encodings}{encoding} of the
  \texttt{privateKey} string.
\end{itemize}

Sets the EC Diffie-Hellman private key. If \texttt{encoding} is
provided, \texttt{privateKey} is expected to be a string; otherwise
\texttt{privateKey} is expected to be a
\href{buffer.md}{\texttt{Buffer}}, \texttt{TypedArray}, or
\texttt{DataView}.

If \texttt{privateKey} is not valid for the curve specified when the
\texttt{ECDH} object was created, an error is thrown. Upon setting the
private key, the associated public point (key) is also generated and set
in the \texttt{ECDH} object.

\subsubsection{\texorpdfstring{\texttt{ecdh.setPublicKey(publicKey{[},\ encoding{]})}}{ecdh.setPublicKey(publicKey{[}, encoding{]})}}\label{ecdh.setpublickeypublickey-encoding}

\begin{quote}
Stability: 0 - Deprecated
\end{quote}

\begin{itemize}
\tightlist
\item
  \texttt{publicKey}
  \{string\textbar ArrayBuffer\textbar Buffer\textbar TypedArray\textbar DataView\}
\item
  \texttt{encoding} \{string\} The
  \href{buffer.md\#buffers-and-character-encodings}{encoding} of the
  \texttt{publicKey} string.
\end{itemize}

Sets the EC Diffie-Hellman public key. If \texttt{encoding} is provided
\texttt{publicKey} is expected to be a string; otherwise a
\href{buffer.md}{\texttt{Buffer}}, \texttt{TypedArray}, or
\texttt{DataView} is expected.

There is not normally a reason to call this method because \texttt{ECDH}
only requires a private key and the other party's public key to compute
the shared secret. Typically either
\hyperref[ecdhgeneratekeysencoding-format]{\texttt{ecdh.generateKeys()}}
or
\hyperref[ecdhsetprivatekeyprivatekey-encoding]{\texttt{ecdh.setPrivateKey()}}
will be called. The
\hyperref[ecdhsetprivatekeyprivatekey-encoding]{\texttt{ecdh.setPrivateKey()}}
method attempts to generate the public point/key associated with the
private key being set.

Example (obtaining a shared secret):

\begin{Shaded}
\begin{Highlighting}[]
\KeywordTok{const}\NormalTok{ \{}
\NormalTok{  createECDH}\OperatorTok{,}
\NormalTok{  createHash}\OperatorTok{,}
\NormalTok{\} }\OperatorTok{=} \ControlFlowTok{await} \ImportTok{import}\NormalTok{(}\StringTok{\textquotesingle{}node:crypto\textquotesingle{}}\NormalTok{)}\OperatorTok{;}

\KeywordTok{const}\NormalTok{ alice }\OperatorTok{=} \FunctionTok{createECDH}\NormalTok{(}\StringTok{\textquotesingle{}secp256k1\textquotesingle{}}\NormalTok{)}\OperatorTok{;}
\KeywordTok{const}\NormalTok{ bob }\OperatorTok{=} \FunctionTok{createECDH}\NormalTok{(}\StringTok{\textquotesingle{}secp256k1\textquotesingle{}}\NormalTok{)}\OperatorTok{;}

\CommentTok{// This is a shortcut way of specifying one of Alice\textquotesingle{}s previous private}
\CommentTok{// keys. It would be unwise to use such a predictable private key in a real}
\CommentTok{// application.}
\NormalTok{alice}\OperatorTok{.}\FunctionTok{setPrivateKey}\NormalTok{(}
  \FunctionTok{createHash}\NormalTok{(}\StringTok{\textquotesingle{}sha256\textquotesingle{}}\NormalTok{)}\OperatorTok{.}\FunctionTok{update}\NormalTok{(}\StringTok{\textquotesingle{}alice\textquotesingle{}}\OperatorTok{,} \StringTok{\textquotesingle{}utf8\textquotesingle{}}\NormalTok{)}\OperatorTok{.}\FunctionTok{digest}\NormalTok{()}\OperatorTok{,}
\NormalTok{)}\OperatorTok{;}

\CommentTok{// Bob uses a newly generated cryptographically strong}
\CommentTok{// pseudorandom key pair}
\NormalTok{bob}\OperatorTok{.}\FunctionTok{generateKeys}\NormalTok{()}\OperatorTok{;}

\KeywordTok{const}\NormalTok{ aliceSecret }\OperatorTok{=}\NormalTok{ alice}\OperatorTok{.}\FunctionTok{computeSecret}\NormalTok{(bob}\OperatorTok{.}\FunctionTok{getPublicKey}\NormalTok{()}\OperatorTok{,} \KeywordTok{null}\OperatorTok{,} \StringTok{\textquotesingle{}hex\textquotesingle{}}\NormalTok{)}\OperatorTok{;}
\KeywordTok{const}\NormalTok{ bobSecret }\OperatorTok{=}\NormalTok{ bob}\OperatorTok{.}\FunctionTok{computeSecret}\NormalTok{(alice}\OperatorTok{.}\FunctionTok{getPublicKey}\NormalTok{()}\OperatorTok{,} \KeywordTok{null}\OperatorTok{,} \StringTok{\textquotesingle{}hex\textquotesingle{}}\NormalTok{)}\OperatorTok{;}

\CommentTok{// aliceSecret and bobSecret should be the same shared secret value}
\BuiltInTok{console}\OperatorTok{.}\FunctionTok{log}\NormalTok{(aliceSecret }\OperatorTok{===}\NormalTok{ bobSecret)}\OperatorTok{;}
\end{Highlighting}
\end{Shaded}

\begin{Shaded}
\begin{Highlighting}[]
\KeywordTok{const}\NormalTok{ \{}
\NormalTok{  createECDH}\OperatorTok{,}
\NormalTok{  createHash}\OperatorTok{,}
\NormalTok{\} }\OperatorTok{=} \PreprocessorTok{require}\NormalTok{(}\StringTok{\textquotesingle{}node:crypto\textquotesingle{}}\NormalTok{)}\OperatorTok{;}

\KeywordTok{const}\NormalTok{ alice }\OperatorTok{=} \FunctionTok{createECDH}\NormalTok{(}\StringTok{\textquotesingle{}secp256k1\textquotesingle{}}\NormalTok{)}\OperatorTok{;}
\KeywordTok{const}\NormalTok{ bob }\OperatorTok{=} \FunctionTok{createECDH}\NormalTok{(}\StringTok{\textquotesingle{}secp256k1\textquotesingle{}}\NormalTok{)}\OperatorTok{;}

\CommentTok{// This is a shortcut way of specifying one of Alice\textquotesingle{}s previous private}
\CommentTok{// keys. It would be unwise to use such a predictable private key in a real}
\CommentTok{// application.}
\NormalTok{alice}\OperatorTok{.}\FunctionTok{setPrivateKey}\NormalTok{(}
  \FunctionTok{createHash}\NormalTok{(}\StringTok{\textquotesingle{}sha256\textquotesingle{}}\NormalTok{)}\OperatorTok{.}\FunctionTok{update}\NormalTok{(}\StringTok{\textquotesingle{}alice\textquotesingle{}}\OperatorTok{,} \StringTok{\textquotesingle{}utf8\textquotesingle{}}\NormalTok{)}\OperatorTok{.}\FunctionTok{digest}\NormalTok{()}\OperatorTok{,}
\NormalTok{)}\OperatorTok{;}

\CommentTok{// Bob uses a newly generated cryptographically strong}
\CommentTok{// pseudorandom key pair}
\NormalTok{bob}\OperatorTok{.}\FunctionTok{generateKeys}\NormalTok{()}\OperatorTok{;}

\KeywordTok{const}\NormalTok{ aliceSecret }\OperatorTok{=}\NormalTok{ alice}\OperatorTok{.}\FunctionTok{computeSecret}\NormalTok{(bob}\OperatorTok{.}\FunctionTok{getPublicKey}\NormalTok{()}\OperatorTok{,} \KeywordTok{null}\OperatorTok{,} \StringTok{\textquotesingle{}hex\textquotesingle{}}\NormalTok{)}\OperatorTok{;}
\KeywordTok{const}\NormalTok{ bobSecret }\OperatorTok{=}\NormalTok{ bob}\OperatorTok{.}\FunctionTok{computeSecret}\NormalTok{(alice}\OperatorTok{.}\FunctionTok{getPublicKey}\NormalTok{()}\OperatorTok{,} \KeywordTok{null}\OperatorTok{,} \StringTok{\textquotesingle{}hex\textquotesingle{}}\NormalTok{)}\OperatorTok{;}

\CommentTok{// aliceSecret and bobSecret should be the same shared secret value}
\BuiltInTok{console}\OperatorTok{.}\FunctionTok{log}\NormalTok{(aliceSecret }\OperatorTok{===}\NormalTok{ bobSecret)}\OperatorTok{;}
\end{Highlighting}
\end{Shaded}

\subsection{\texorpdfstring{Class:
\texttt{Hash}}{Class: Hash}}\label{class-hash}

\begin{itemize}
\tightlist
\item
  Extends: \{stream.Transform\}
\end{itemize}

The \texttt{Hash} class is a utility for creating hash digests of data.
It can be used in one of two ways:

\begin{itemize}
\tightlist
\item
  As a \href{stream.md}{stream} that is both readable and writable,
  where data is written to produce a computed hash digest on the
  readable side, or
\item
  Using the
  \hyperref[hashupdatedata-inputencoding]{\texttt{hash.update()}} and
  \hyperref[hashdigestencoding]{\texttt{hash.digest()}} methods to
  produce the computed hash.
\end{itemize}

The
\hyperref[cryptocreatehashalgorithm-options]{\texttt{crypto.createHash()}}
method is used to create \texttt{Hash} instances. \texttt{Hash} objects
are not to be created directly using the \texttt{new} keyword.

Example: Using \texttt{Hash} objects as streams:

\begin{Shaded}
\begin{Highlighting}[]
\KeywordTok{const}\NormalTok{ \{}
\NormalTok{  createHash}\OperatorTok{,}
\NormalTok{\} }\OperatorTok{=} \ControlFlowTok{await} \ImportTok{import}\NormalTok{(}\StringTok{\textquotesingle{}node:crypto\textquotesingle{}}\NormalTok{)}\OperatorTok{;}

\KeywordTok{const}\NormalTok{ hash }\OperatorTok{=} \FunctionTok{createHash}\NormalTok{(}\StringTok{\textquotesingle{}sha256\textquotesingle{}}\NormalTok{)}\OperatorTok{;}

\NormalTok{hash}\OperatorTok{.}\FunctionTok{on}\NormalTok{(}\StringTok{\textquotesingle{}readable\textquotesingle{}}\OperatorTok{,}\NormalTok{ () }\KeywordTok{=\textgreater{}}\NormalTok{ \{}
  \CommentTok{// Only one element is going to be produced by the}
  \CommentTok{// hash stream.}
  \KeywordTok{const}\NormalTok{ data }\OperatorTok{=}\NormalTok{ hash}\OperatorTok{.}\FunctionTok{read}\NormalTok{()}\OperatorTok{;}
  \ControlFlowTok{if}\NormalTok{ (data) \{}
    \BuiltInTok{console}\OperatorTok{.}\FunctionTok{log}\NormalTok{(data}\OperatorTok{.}\FunctionTok{toString}\NormalTok{(}\StringTok{\textquotesingle{}hex\textquotesingle{}}\NormalTok{))}\OperatorTok{;}
    \CommentTok{// Prints:}
    \CommentTok{//   6a2da20943931e9834fc12cfe5bb47bbd9ae43489a30726962b576f4e3993e50}
\NormalTok{  \}}
\NormalTok{\})}\OperatorTok{;}

\NormalTok{hash}\OperatorTok{.}\FunctionTok{write}\NormalTok{(}\StringTok{\textquotesingle{}some data to hash\textquotesingle{}}\NormalTok{)}\OperatorTok{;}
\NormalTok{hash}\OperatorTok{.}\FunctionTok{end}\NormalTok{()}\OperatorTok{;}
\end{Highlighting}
\end{Shaded}

\begin{Shaded}
\begin{Highlighting}[]
\KeywordTok{const}\NormalTok{ \{}
\NormalTok{  createHash}\OperatorTok{,}
\NormalTok{\} }\OperatorTok{=} \PreprocessorTok{require}\NormalTok{(}\StringTok{\textquotesingle{}node:crypto\textquotesingle{}}\NormalTok{)}\OperatorTok{;}

\KeywordTok{const}\NormalTok{ hash }\OperatorTok{=} \FunctionTok{createHash}\NormalTok{(}\StringTok{\textquotesingle{}sha256\textquotesingle{}}\NormalTok{)}\OperatorTok{;}

\NormalTok{hash}\OperatorTok{.}\FunctionTok{on}\NormalTok{(}\StringTok{\textquotesingle{}readable\textquotesingle{}}\OperatorTok{,}\NormalTok{ () }\KeywordTok{=\textgreater{}}\NormalTok{ \{}
  \CommentTok{// Only one element is going to be produced by the}
  \CommentTok{// hash stream.}
  \KeywordTok{const}\NormalTok{ data }\OperatorTok{=}\NormalTok{ hash}\OperatorTok{.}\FunctionTok{read}\NormalTok{()}\OperatorTok{;}
  \ControlFlowTok{if}\NormalTok{ (data) \{}
    \BuiltInTok{console}\OperatorTok{.}\FunctionTok{log}\NormalTok{(data}\OperatorTok{.}\FunctionTok{toString}\NormalTok{(}\StringTok{\textquotesingle{}hex\textquotesingle{}}\NormalTok{))}\OperatorTok{;}
    \CommentTok{// Prints:}
    \CommentTok{//   6a2da20943931e9834fc12cfe5bb47bbd9ae43489a30726962b576f4e3993e50}
\NormalTok{  \}}
\NormalTok{\})}\OperatorTok{;}

\NormalTok{hash}\OperatorTok{.}\FunctionTok{write}\NormalTok{(}\StringTok{\textquotesingle{}some data to hash\textquotesingle{}}\NormalTok{)}\OperatorTok{;}
\NormalTok{hash}\OperatorTok{.}\FunctionTok{end}\NormalTok{()}\OperatorTok{;}
\end{Highlighting}
\end{Shaded}

Example: Using \texttt{Hash} and piped streams:

\begin{Shaded}
\begin{Highlighting}[]
\ImportTok{import}\NormalTok{ \{ createReadStream \} }\ImportTok{from} \StringTok{\textquotesingle{}node:fs\textquotesingle{}}\OperatorTok{;}
\ImportTok{import}\NormalTok{ \{ stdout \} }\ImportTok{from} \StringTok{\textquotesingle{}node:process\textquotesingle{}}\OperatorTok{;}
\KeywordTok{const}\NormalTok{ \{ createHash \} }\OperatorTok{=} \ControlFlowTok{await} \ImportTok{import}\NormalTok{(}\StringTok{\textquotesingle{}node:crypto\textquotesingle{}}\NormalTok{)}\OperatorTok{;}

\KeywordTok{const}\NormalTok{ hash }\OperatorTok{=} \FunctionTok{createHash}\NormalTok{(}\StringTok{\textquotesingle{}sha256\textquotesingle{}}\NormalTok{)}\OperatorTok{;}

\KeywordTok{const}\NormalTok{ input }\OperatorTok{=} \FunctionTok{createReadStream}\NormalTok{(}\StringTok{\textquotesingle{}test.js\textquotesingle{}}\NormalTok{)}\OperatorTok{;}
\NormalTok{input}\OperatorTok{.}\FunctionTok{pipe}\NormalTok{(hash)}\OperatorTok{.}\FunctionTok{setEncoding}\NormalTok{(}\StringTok{\textquotesingle{}hex\textquotesingle{}}\NormalTok{)}\OperatorTok{.}\FunctionTok{pipe}\NormalTok{(stdout)}\OperatorTok{;}
\end{Highlighting}
\end{Shaded}

\begin{Shaded}
\begin{Highlighting}[]
\KeywordTok{const}\NormalTok{ \{ createReadStream \} }\OperatorTok{=} \PreprocessorTok{require}\NormalTok{(}\StringTok{\textquotesingle{}node:fs\textquotesingle{}}\NormalTok{)}\OperatorTok{;}
\KeywordTok{const}\NormalTok{ \{ createHash \} }\OperatorTok{=} \PreprocessorTok{require}\NormalTok{(}\StringTok{\textquotesingle{}node:crypto\textquotesingle{}}\NormalTok{)}\OperatorTok{;}
\KeywordTok{const}\NormalTok{ \{ stdout \} }\OperatorTok{=} \PreprocessorTok{require}\NormalTok{(}\StringTok{\textquotesingle{}node:process\textquotesingle{}}\NormalTok{)}\OperatorTok{;}

\KeywordTok{const}\NormalTok{ hash }\OperatorTok{=} \FunctionTok{createHash}\NormalTok{(}\StringTok{\textquotesingle{}sha256\textquotesingle{}}\NormalTok{)}\OperatorTok{;}

\KeywordTok{const}\NormalTok{ input }\OperatorTok{=} \FunctionTok{createReadStream}\NormalTok{(}\StringTok{\textquotesingle{}test.js\textquotesingle{}}\NormalTok{)}\OperatorTok{;}
\NormalTok{input}\OperatorTok{.}\FunctionTok{pipe}\NormalTok{(hash)}\OperatorTok{.}\FunctionTok{setEncoding}\NormalTok{(}\StringTok{\textquotesingle{}hex\textquotesingle{}}\NormalTok{)}\OperatorTok{.}\FunctionTok{pipe}\NormalTok{(stdout)}\OperatorTok{;}
\end{Highlighting}
\end{Shaded}

Example: Using the
\hyperref[hashupdatedata-inputencoding]{\texttt{hash.update()}} and
\hyperref[hashdigestencoding]{\texttt{hash.digest()}} methods:

\begin{Shaded}
\begin{Highlighting}[]
\KeywordTok{const}\NormalTok{ \{}
\NormalTok{  createHash}\OperatorTok{,}
\NormalTok{\} }\OperatorTok{=} \ControlFlowTok{await} \ImportTok{import}\NormalTok{(}\StringTok{\textquotesingle{}node:crypto\textquotesingle{}}\NormalTok{)}\OperatorTok{;}

\KeywordTok{const}\NormalTok{ hash }\OperatorTok{=} \FunctionTok{createHash}\NormalTok{(}\StringTok{\textquotesingle{}sha256\textquotesingle{}}\NormalTok{)}\OperatorTok{;}

\NormalTok{hash}\OperatorTok{.}\FunctionTok{update}\NormalTok{(}\StringTok{\textquotesingle{}some data to hash\textquotesingle{}}\NormalTok{)}\OperatorTok{;}
\BuiltInTok{console}\OperatorTok{.}\FunctionTok{log}\NormalTok{(hash}\OperatorTok{.}\FunctionTok{digest}\NormalTok{(}\StringTok{\textquotesingle{}hex\textquotesingle{}}\NormalTok{))}\OperatorTok{;}
\CommentTok{// Prints:}
\CommentTok{//   6a2da20943931e9834fc12cfe5bb47bbd9ae43489a30726962b576f4e3993e50}
\end{Highlighting}
\end{Shaded}

\begin{Shaded}
\begin{Highlighting}[]
\KeywordTok{const}\NormalTok{ \{}
\NormalTok{  createHash}\OperatorTok{,}
\NormalTok{\} }\OperatorTok{=} \PreprocessorTok{require}\NormalTok{(}\StringTok{\textquotesingle{}node:crypto\textquotesingle{}}\NormalTok{)}\OperatorTok{;}

\KeywordTok{const}\NormalTok{ hash }\OperatorTok{=} \FunctionTok{createHash}\NormalTok{(}\StringTok{\textquotesingle{}sha256\textquotesingle{}}\NormalTok{)}\OperatorTok{;}

\NormalTok{hash}\OperatorTok{.}\FunctionTok{update}\NormalTok{(}\StringTok{\textquotesingle{}some data to hash\textquotesingle{}}\NormalTok{)}\OperatorTok{;}
\BuiltInTok{console}\OperatorTok{.}\FunctionTok{log}\NormalTok{(hash}\OperatorTok{.}\FunctionTok{digest}\NormalTok{(}\StringTok{\textquotesingle{}hex\textquotesingle{}}\NormalTok{))}\OperatorTok{;}
\CommentTok{// Prints:}
\CommentTok{//   6a2da20943931e9834fc12cfe5bb47bbd9ae43489a30726962b576f4e3993e50}
\end{Highlighting}
\end{Shaded}

\subsubsection{\texorpdfstring{\texttt{hash.copy({[}options{]})}}{hash.copy({[}options{]})}}\label{hash.copyoptions}

\begin{itemize}
\tightlist
\item
  \texttt{options} \{Object\}
  \href{stream.md\#new-streamtransformoptions}{\texttt{stream.transform}
  options}
\item
  Returns: \{Hash\}
\end{itemize}

Creates a new \texttt{Hash} object that contains a deep copy of the
internal state of the current \texttt{Hash} object.

The optional \texttt{options} argument controls stream behavior. For XOF
hash functions such as
\texttt{\textquotesingle{}shake256\textquotesingle{}}, the
\texttt{outputLength} option can be used to specify the desired output
length in bytes.

An error is thrown when an attempt is made to copy the \texttt{Hash}
object after its \hyperref[hashdigestencoding]{\texttt{hash.digest()}}
method has been called.

\begin{Shaded}
\begin{Highlighting}[]
\CommentTok{// Calculate a rolling hash.}
\KeywordTok{const}\NormalTok{ \{}
\NormalTok{  createHash}\OperatorTok{,}
\NormalTok{\} }\OperatorTok{=} \ControlFlowTok{await} \ImportTok{import}\NormalTok{(}\StringTok{\textquotesingle{}node:crypto\textquotesingle{}}\NormalTok{)}\OperatorTok{;}

\KeywordTok{const}\NormalTok{ hash }\OperatorTok{=} \FunctionTok{createHash}\NormalTok{(}\StringTok{\textquotesingle{}sha256\textquotesingle{}}\NormalTok{)}\OperatorTok{;}

\NormalTok{hash}\OperatorTok{.}\FunctionTok{update}\NormalTok{(}\StringTok{\textquotesingle{}one\textquotesingle{}}\NormalTok{)}\OperatorTok{;}
\BuiltInTok{console}\OperatorTok{.}\FunctionTok{log}\NormalTok{(hash}\OperatorTok{.}\FunctionTok{copy}\NormalTok{()}\OperatorTok{.}\FunctionTok{digest}\NormalTok{(}\StringTok{\textquotesingle{}hex\textquotesingle{}}\NormalTok{))}\OperatorTok{;}

\NormalTok{hash}\OperatorTok{.}\FunctionTok{update}\NormalTok{(}\StringTok{\textquotesingle{}two\textquotesingle{}}\NormalTok{)}\OperatorTok{;}
\BuiltInTok{console}\OperatorTok{.}\FunctionTok{log}\NormalTok{(hash}\OperatorTok{.}\FunctionTok{copy}\NormalTok{()}\OperatorTok{.}\FunctionTok{digest}\NormalTok{(}\StringTok{\textquotesingle{}hex\textquotesingle{}}\NormalTok{))}\OperatorTok{;}

\NormalTok{hash}\OperatorTok{.}\FunctionTok{update}\NormalTok{(}\StringTok{\textquotesingle{}three\textquotesingle{}}\NormalTok{)}\OperatorTok{;}
\BuiltInTok{console}\OperatorTok{.}\FunctionTok{log}\NormalTok{(hash}\OperatorTok{.}\FunctionTok{copy}\NormalTok{()}\OperatorTok{.}\FunctionTok{digest}\NormalTok{(}\StringTok{\textquotesingle{}hex\textquotesingle{}}\NormalTok{))}\OperatorTok{;}

\CommentTok{// Etc.}
\end{Highlighting}
\end{Shaded}

\begin{Shaded}
\begin{Highlighting}[]
\CommentTok{// Calculate a rolling hash.}
\KeywordTok{const}\NormalTok{ \{}
\NormalTok{  createHash}\OperatorTok{,}
\NormalTok{\} }\OperatorTok{=} \PreprocessorTok{require}\NormalTok{(}\StringTok{\textquotesingle{}node:crypto\textquotesingle{}}\NormalTok{)}\OperatorTok{;}

\KeywordTok{const}\NormalTok{ hash }\OperatorTok{=} \FunctionTok{createHash}\NormalTok{(}\StringTok{\textquotesingle{}sha256\textquotesingle{}}\NormalTok{)}\OperatorTok{;}

\NormalTok{hash}\OperatorTok{.}\FunctionTok{update}\NormalTok{(}\StringTok{\textquotesingle{}one\textquotesingle{}}\NormalTok{)}\OperatorTok{;}
\BuiltInTok{console}\OperatorTok{.}\FunctionTok{log}\NormalTok{(hash}\OperatorTok{.}\FunctionTok{copy}\NormalTok{()}\OperatorTok{.}\FunctionTok{digest}\NormalTok{(}\StringTok{\textquotesingle{}hex\textquotesingle{}}\NormalTok{))}\OperatorTok{;}

\NormalTok{hash}\OperatorTok{.}\FunctionTok{update}\NormalTok{(}\StringTok{\textquotesingle{}two\textquotesingle{}}\NormalTok{)}\OperatorTok{;}
\BuiltInTok{console}\OperatorTok{.}\FunctionTok{log}\NormalTok{(hash}\OperatorTok{.}\FunctionTok{copy}\NormalTok{()}\OperatorTok{.}\FunctionTok{digest}\NormalTok{(}\StringTok{\textquotesingle{}hex\textquotesingle{}}\NormalTok{))}\OperatorTok{;}

\NormalTok{hash}\OperatorTok{.}\FunctionTok{update}\NormalTok{(}\StringTok{\textquotesingle{}three\textquotesingle{}}\NormalTok{)}\OperatorTok{;}
\BuiltInTok{console}\OperatorTok{.}\FunctionTok{log}\NormalTok{(hash}\OperatorTok{.}\FunctionTok{copy}\NormalTok{()}\OperatorTok{.}\FunctionTok{digest}\NormalTok{(}\StringTok{\textquotesingle{}hex\textquotesingle{}}\NormalTok{))}\OperatorTok{;}

\CommentTok{// Etc.}
\end{Highlighting}
\end{Shaded}

\subsubsection{\texorpdfstring{\texttt{hash.digest({[}encoding{]})}}{hash.digest({[}encoding{]})}}\label{hash.digestencoding}

\begin{itemize}
\tightlist
\item
  \texttt{encoding} \{string\} The
  \href{buffer.md\#buffers-and-character-encodings}{encoding} of the
  return value.
\item
  Returns: \{Buffer \textbar{} string\}
\end{itemize}

Calculates the digest of all of the data passed to be hashed (using the
\hyperref[hashupdatedata-inputencoding]{\texttt{hash.update()}} method).
If \texttt{encoding} is provided a string will be returned; otherwise a
\href{buffer.md}{\texttt{Buffer}} is returned.

The \texttt{Hash} object can not be used again after
\texttt{hash.digest()} method has been called. Multiple calls will cause
an error to be thrown.

\subsubsection{\texorpdfstring{\texttt{hash.update(data{[},\ inputEncoding{]})}}{hash.update(data{[}, inputEncoding{]})}}\label{hash.updatedata-inputencoding}

\begin{itemize}
\tightlist
\item
  \texttt{data}
  \{string\textbar Buffer\textbar TypedArray\textbar DataView\}
\item
  \texttt{inputEncoding} \{string\} The
  \href{buffer.md\#buffers-and-character-encodings}{encoding} of the
  \texttt{data} string.
\end{itemize}

Updates the hash content with the given \texttt{data}, the encoding of
which is given in \texttt{inputEncoding}. If \texttt{encoding} is not
provided, and the \texttt{data} is a string, an encoding of
\texttt{\textquotesingle{}utf8\textquotesingle{}} is enforced. If
\texttt{data} is a \href{buffer.md}{\texttt{Buffer}},
\texttt{TypedArray}, or \texttt{DataView}, then \texttt{inputEncoding}
is ignored.

This can be called many times with new data as it is streamed.

\subsection{\texorpdfstring{Class:
\texttt{Hmac}}{Class: Hmac}}\label{class-hmac}

\begin{itemize}
\tightlist
\item
  Extends: \{stream.Transform\}
\end{itemize}

The \texttt{Hmac} class is a utility for creating cryptographic HMAC
digests. It can be used in one of two ways:

\begin{itemize}
\tightlist
\item
  As a \href{stream.md}{stream} that is both readable and writable,
  where data is written to produce a computed HMAC digest on the
  readable side, or
\item
  Using the
  \hyperref[hmacupdatedata-inputencoding]{\texttt{hmac.update()}} and
  \hyperref[hmacdigestencoding]{\texttt{hmac.digest()}} methods to
  produce the computed HMAC digest.
\end{itemize}

The
\hyperref[cryptocreatehmacalgorithm-key-options]{\texttt{crypto.createHmac()}}
method is used to create \texttt{Hmac} instances. \texttt{Hmac} objects
are not to be created directly using the \texttt{new} keyword.

Example: Using \texttt{Hmac} objects as streams:

\begin{Shaded}
\begin{Highlighting}[]
\KeywordTok{const}\NormalTok{ \{}
\NormalTok{  createHmac}\OperatorTok{,}
\NormalTok{\} }\OperatorTok{=} \ControlFlowTok{await} \ImportTok{import}\NormalTok{(}\StringTok{\textquotesingle{}node:crypto\textquotesingle{}}\NormalTok{)}\OperatorTok{;}

\KeywordTok{const}\NormalTok{ hmac }\OperatorTok{=} \FunctionTok{createHmac}\NormalTok{(}\StringTok{\textquotesingle{}sha256\textquotesingle{}}\OperatorTok{,} \StringTok{\textquotesingle{}a secret\textquotesingle{}}\NormalTok{)}\OperatorTok{;}

\NormalTok{hmac}\OperatorTok{.}\FunctionTok{on}\NormalTok{(}\StringTok{\textquotesingle{}readable\textquotesingle{}}\OperatorTok{,}\NormalTok{ () }\KeywordTok{=\textgreater{}}\NormalTok{ \{}
  \CommentTok{// Only one element is going to be produced by the}
  \CommentTok{// hash stream.}
  \KeywordTok{const}\NormalTok{ data }\OperatorTok{=}\NormalTok{ hmac}\OperatorTok{.}\FunctionTok{read}\NormalTok{()}\OperatorTok{;}
  \ControlFlowTok{if}\NormalTok{ (data) \{}
    \BuiltInTok{console}\OperatorTok{.}\FunctionTok{log}\NormalTok{(data}\OperatorTok{.}\FunctionTok{toString}\NormalTok{(}\StringTok{\textquotesingle{}hex\textquotesingle{}}\NormalTok{))}\OperatorTok{;}
    \CommentTok{// Prints:}
    \CommentTok{//   7fd04df92f636fd450bc841c9418e5825c17f33ad9c87c518115a45971f7f77e}
\NormalTok{  \}}
\NormalTok{\})}\OperatorTok{;}

\NormalTok{hmac}\OperatorTok{.}\FunctionTok{write}\NormalTok{(}\StringTok{\textquotesingle{}some data to hash\textquotesingle{}}\NormalTok{)}\OperatorTok{;}
\NormalTok{hmac}\OperatorTok{.}\FunctionTok{end}\NormalTok{()}\OperatorTok{;}
\end{Highlighting}
\end{Shaded}

\begin{Shaded}
\begin{Highlighting}[]
\KeywordTok{const}\NormalTok{ \{}
\NormalTok{  createHmac}\OperatorTok{,}
\NormalTok{\} }\OperatorTok{=} \PreprocessorTok{require}\NormalTok{(}\StringTok{\textquotesingle{}node:crypto\textquotesingle{}}\NormalTok{)}\OperatorTok{;}

\KeywordTok{const}\NormalTok{ hmac }\OperatorTok{=} \FunctionTok{createHmac}\NormalTok{(}\StringTok{\textquotesingle{}sha256\textquotesingle{}}\OperatorTok{,} \StringTok{\textquotesingle{}a secret\textquotesingle{}}\NormalTok{)}\OperatorTok{;}

\NormalTok{hmac}\OperatorTok{.}\FunctionTok{on}\NormalTok{(}\StringTok{\textquotesingle{}readable\textquotesingle{}}\OperatorTok{,}\NormalTok{ () }\KeywordTok{=\textgreater{}}\NormalTok{ \{}
  \CommentTok{// Only one element is going to be produced by the}
  \CommentTok{// hash stream.}
  \KeywordTok{const}\NormalTok{ data }\OperatorTok{=}\NormalTok{ hmac}\OperatorTok{.}\FunctionTok{read}\NormalTok{()}\OperatorTok{;}
  \ControlFlowTok{if}\NormalTok{ (data) \{}
    \BuiltInTok{console}\OperatorTok{.}\FunctionTok{log}\NormalTok{(data}\OperatorTok{.}\FunctionTok{toString}\NormalTok{(}\StringTok{\textquotesingle{}hex\textquotesingle{}}\NormalTok{))}\OperatorTok{;}
    \CommentTok{// Prints:}
    \CommentTok{//   7fd04df92f636fd450bc841c9418e5825c17f33ad9c87c518115a45971f7f77e}
\NormalTok{  \}}
\NormalTok{\})}\OperatorTok{;}

\NormalTok{hmac}\OperatorTok{.}\FunctionTok{write}\NormalTok{(}\StringTok{\textquotesingle{}some data to hash\textquotesingle{}}\NormalTok{)}\OperatorTok{;}
\NormalTok{hmac}\OperatorTok{.}\FunctionTok{end}\NormalTok{()}\OperatorTok{;}
\end{Highlighting}
\end{Shaded}

Example: Using \texttt{Hmac} and piped streams:

\begin{Shaded}
\begin{Highlighting}[]
\ImportTok{import}\NormalTok{ \{ createReadStream \} }\ImportTok{from} \StringTok{\textquotesingle{}node:fs\textquotesingle{}}\OperatorTok{;}
\ImportTok{import}\NormalTok{ \{ stdout \} }\ImportTok{from} \StringTok{\textquotesingle{}node:process\textquotesingle{}}\OperatorTok{;}
\KeywordTok{const}\NormalTok{ \{}
\NormalTok{  createHmac}\OperatorTok{,}
\NormalTok{\} }\OperatorTok{=} \ControlFlowTok{await} \ImportTok{import}\NormalTok{(}\StringTok{\textquotesingle{}node:crypto\textquotesingle{}}\NormalTok{)}\OperatorTok{;}

\KeywordTok{const}\NormalTok{ hmac }\OperatorTok{=} \FunctionTok{createHmac}\NormalTok{(}\StringTok{\textquotesingle{}sha256\textquotesingle{}}\OperatorTok{,} \StringTok{\textquotesingle{}a secret\textquotesingle{}}\NormalTok{)}\OperatorTok{;}

\KeywordTok{const}\NormalTok{ input }\OperatorTok{=} \FunctionTok{createReadStream}\NormalTok{(}\StringTok{\textquotesingle{}test.js\textquotesingle{}}\NormalTok{)}\OperatorTok{;}
\NormalTok{input}\OperatorTok{.}\FunctionTok{pipe}\NormalTok{(hmac)}\OperatorTok{.}\FunctionTok{pipe}\NormalTok{(stdout)}\OperatorTok{;}
\end{Highlighting}
\end{Shaded}

\begin{Shaded}
\begin{Highlighting}[]
\KeywordTok{const}\NormalTok{ \{}
\NormalTok{  createReadStream}\OperatorTok{,}
\NormalTok{\} }\OperatorTok{=} \PreprocessorTok{require}\NormalTok{(}\StringTok{\textquotesingle{}node:fs\textquotesingle{}}\NormalTok{)}\OperatorTok{;}
\KeywordTok{const}\NormalTok{ \{}
\NormalTok{  createHmac}\OperatorTok{,}
\NormalTok{\} }\OperatorTok{=} \PreprocessorTok{require}\NormalTok{(}\StringTok{\textquotesingle{}node:crypto\textquotesingle{}}\NormalTok{)}\OperatorTok{;}
\KeywordTok{const}\NormalTok{ \{ stdout \} }\OperatorTok{=} \PreprocessorTok{require}\NormalTok{(}\StringTok{\textquotesingle{}node:process\textquotesingle{}}\NormalTok{)}\OperatorTok{;}

\KeywordTok{const}\NormalTok{ hmac }\OperatorTok{=} \FunctionTok{createHmac}\NormalTok{(}\StringTok{\textquotesingle{}sha256\textquotesingle{}}\OperatorTok{,} \StringTok{\textquotesingle{}a secret\textquotesingle{}}\NormalTok{)}\OperatorTok{;}

\KeywordTok{const}\NormalTok{ input }\OperatorTok{=} \FunctionTok{createReadStream}\NormalTok{(}\StringTok{\textquotesingle{}test.js\textquotesingle{}}\NormalTok{)}\OperatorTok{;}
\NormalTok{input}\OperatorTok{.}\FunctionTok{pipe}\NormalTok{(hmac)}\OperatorTok{.}\FunctionTok{pipe}\NormalTok{(stdout)}\OperatorTok{;}
\end{Highlighting}
\end{Shaded}

Example: Using the
\hyperref[hmacupdatedata-inputencoding]{\texttt{hmac.update()}} and
\hyperref[hmacdigestencoding]{\texttt{hmac.digest()}} methods:

\begin{Shaded}
\begin{Highlighting}[]
\KeywordTok{const}\NormalTok{ \{}
\NormalTok{  createHmac}\OperatorTok{,}
\NormalTok{\} }\OperatorTok{=} \ControlFlowTok{await} \ImportTok{import}\NormalTok{(}\StringTok{\textquotesingle{}node:crypto\textquotesingle{}}\NormalTok{)}\OperatorTok{;}

\KeywordTok{const}\NormalTok{ hmac }\OperatorTok{=} \FunctionTok{createHmac}\NormalTok{(}\StringTok{\textquotesingle{}sha256\textquotesingle{}}\OperatorTok{,} \StringTok{\textquotesingle{}a secret\textquotesingle{}}\NormalTok{)}\OperatorTok{;}

\NormalTok{hmac}\OperatorTok{.}\FunctionTok{update}\NormalTok{(}\StringTok{\textquotesingle{}some data to hash\textquotesingle{}}\NormalTok{)}\OperatorTok{;}
\BuiltInTok{console}\OperatorTok{.}\FunctionTok{log}\NormalTok{(hmac}\OperatorTok{.}\FunctionTok{digest}\NormalTok{(}\StringTok{\textquotesingle{}hex\textquotesingle{}}\NormalTok{))}\OperatorTok{;}
\CommentTok{// Prints:}
\CommentTok{//   7fd04df92f636fd450bc841c9418e5825c17f33ad9c87c518115a45971f7f77e}
\end{Highlighting}
\end{Shaded}

\begin{Shaded}
\begin{Highlighting}[]
\KeywordTok{const}\NormalTok{ \{}
\NormalTok{  createHmac}\OperatorTok{,}
\NormalTok{\} }\OperatorTok{=} \PreprocessorTok{require}\NormalTok{(}\StringTok{\textquotesingle{}node:crypto\textquotesingle{}}\NormalTok{)}\OperatorTok{;}

\KeywordTok{const}\NormalTok{ hmac }\OperatorTok{=} \FunctionTok{createHmac}\NormalTok{(}\StringTok{\textquotesingle{}sha256\textquotesingle{}}\OperatorTok{,} \StringTok{\textquotesingle{}a secret\textquotesingle{}}\NormalTok{)}\OperatorTok{;}

\NormalTok{hmac}\OperatorTok{.}\FunctionTok{update}\NormalTok{(}\StringTok{\textquotesingle{}some data to hash\textquotesingle{}}\NormalTok{)}\OperatorTok{;}
\BuiltInTok{console}\OperatorTok{.}\FunctionTok{log}\NormalTok{(hmac}\OperatorTok{.}\FunctionTok{digest}\NormalTok{(}\StringTok{\textquotesingle{}hex\textquotesingle{}}\NormalTok{))}\OperatorTok{;}
\CommentTok{// Prints:}
\CommentTok{//   7fd04df92f636fd450bc841c9418e5825c17f33ad9c87c518115a45971f7f77e}
\end{Highlighting}
\end{Shaded}

\subsubsection{\texorpdfstring{\texttt{hmac.digest({[}encoding{]})}}{hmac.digest({[}encoding{]})}}\label{hmac.digestencoding}

\begin{itemize}
\tightlist
\item
  \texttt{encoding} \{string\} The
  \href{buffer.md\#buffers-and-character-encodings}{encoding} of the
  return value.
\item
  Returns: \{Buffer \textbar{} string\}
\end{itemize}

Calculates the HMAC digest of all of the data passed using
\hyperref[hmacupdatedata-inputencoding]{\texttt{hmac.update()}}. If
\texttt{encoding} is provided a string is returned; otherwise a
\href{buffer.md}{\texttt{Buffer}} is returned;

The \texttt{Hmac} object can not be used again after
\texttt{hmac.digest()} has been called. Multiple calls to
\texttt{hmac.digest()} will result in an error being thrown.

\subsubsection{\texorpdfstring{\texttt{hmac.update(data{[},\ inputEncoding{]})}}{hmac.update(data{[}, inputEncoding{]})}}\label{hmac.updatedata-inputencoding}

\begin{itemize}
\tightlist
\item
  \texttt{data}
  \{string\textbar Buffer\textbar TypedArray\textbar DataView\}
\item
  \texttt{inputEncoding} \{string\} The
  \href{buffer.md\#buffers-and-character-encodings}{encoding} of the
  \texttt{data} string.
\end{itemize}

Updates the \texttt{Hmac} content with the given \texttt{data}, the
encoding of which is given in \texttt{inputEncoding}. If
\texttt{encoding} is not provided, and the \texttt{data} is a string, an
encoding of \texttt{\textquotesingle{}utf8\textquotesingle{}} is
enforced. If \texttt{data} is a \href{buffer.md}{\texttt{Buffer}},
\texttt{TypedArray}, or \texttt{DataView}, then \texttt{inputEncoding}
is ignored.

This can be called many times with new data as it is streamed.

\subsection{\texorpdfstring{Class:
\texttt{KeyObject}}{Class: KeyObject}}\label{class-keyobject}

Node.js uses a \texttt{KeyObject} class to represent a symmetric or
asymmetric key, and each kind of key exposes different functions. The
\hyperref[cryptocreatesecretkeykey-encoding]{\texttt{crypto.createSecretKey()}},
\hyperref[cryptocreatepublickeykey]{\texttt{crypto.createPublicKey()}}
and
\hyperref[cryptocreateprivatekeykey]{\texttt{crypto.createPrivateKey()}}
methods are used to create \texttt{KeyObject} instances.
\texttt{KeyObject} objects are not to be created directly using the
\texttt{new} keyword.

Most applications should consider using the new \texttt{KeyObject} API
instead of passing keys as strings or \texttt{Buffer}s due to improved
security features.

\texttt{KeyObject} instances can be passed to other threads via
\href{worker_threads.md\#portpostmessagevalue-transferlist}{\texttt{postMessage()}}.
The receiver obtains a cloned \texttt{KeyObject}, and the
\texttt{KeyObject} does not need to be listed in the
\texttt{transferList} argument.

\subsubsection{\texorpdfstring{Static method:
\texttt{KeyObject.from(key)}}{Static method: KeyObject.from(key)}}\label{static-method-keyobject.fromkey}

\begin{itemize}
\tightlist
\item
  \texttt{key} \{CryptoKey\}
\item
  Returns: \{KeyObject\}
\end{itemize}

Example: Converting a \texttt{CryptoKey} instance to a
\texttt{KeyObject}:

\begin{Shaded}
\begin{Highlighting}[]
\KeywordTok{const}\NormalTok{ \{ KeyObject \} }\OperatorTok{=} \ControlFlowTok{await} \ImportTok{import}\NormalTok{(}\StringTok{\textquotesingle{}node:crypto\textquotesingle{}}\NormalTok{)}\OperatorTok{;}
\KeywordTok{const}\NormalTok{ \{ subtle \} }\OperatorTok{=}\NormalTok{ globalThis}\OperatorTok{.}\AttributeTok{crypto}\OperatorTok{;}

\KeywordTok{const}\NormalTok{ key }\OperatorTok{=} \ControlFlowTok{await}\NormalTok{ subtle}\OperatorTok{.}\FunctionTok{generateKey}\NormalTok{(\{}
  \DataTypeTok{name}\OperatorTok{:} \StringTok{\textquotesingle{}HMAC\textquotesingle{}}\OperatorTok{,}
  \DataTypeTok{hash}\OperatorTok{:} \StringTok{\textquotesingle{}SHA{-}256\textquotesingle{}}\OperatorTok{,}
  \DataTypeTok{length}\OperatorTok{:} \DecValTok{256}\OperatorTok{,}
\NormalTok{\}}\OperatorTok{,} \KeywordTok{true}\OperatorTok{,}\NormalTok{ [}\StringTok{\textquotesingle{}sign\textquotesingle{}}\OperatorTok{,} \StringTok{\textquotesingle{}verify\textquotesingle{}}\NormalTok{])}\OperatorTok{;}

\KeywordTok{const}\NormalTok{ keyObject }\OperatorTok{=}\NormalTok{ KeyObject}\OperatorTok{.}\FunctionTok{from}\NormalTok{(key)}\OperatorTok{;}
\BuiltInTok{console}\OperatorTok{.}\FunctionTok{log}\NormalTok{(keyObject}\OperatorTok{.}\AttributeTok{symmetricKeySize}\NormalTok{)}\OperatorTok{;}
\CommentTok{// Prints: 32 (symmetric key size in bytes)}
\end{Highlighting}
\end{Shaded}

\begin{Shaded}
\begin{Highlighting}[]
\KeywordTok{const}\NormalTok{ \{ KeyObject \} }\OperatorTok{=} \PreprocessorTok{require}\NormalTok{(}\StringTok{\textquotesingle{}node:crypto\textquotesingle{}}\NormalTok{)}\OperatorTok{;}
\KeywordTok{const}\NormalTok{ \{ subtle \} }\OperatorTok{=}\NormalTok{ globalThis}\OperatorTok{.}\AttributeTok{crypto}\OperatorTok{;}

\NormalTok{(}\KeywordTok{async} \KeywordTok{function}\NormalTok{() \{}
  \KeywordTok{const}\NormalTok{ key }\OperatorTok{=} \ControlFlowTok{await}\NormalTok{ subtle}\OperatorTok{.}\FunctionTok{generateKey}\NormalTok{(\{}
    \DataTypeTok{name}\OperatorTok{:} \StringTok{\textquotesingle{}HMAC\textquotesingle{}}\OperatorTok{,}
    \DataTypeTok{hash}\OperatorTok{:} \StringTok{\textquotesingle{}SHA{-}256\textquotesingle{}}\OperatorTok{,}
    \DataTypeTok{length}\OperatorTok{:} \DecValTok{256}\OperatorTok{,}
\NormalTok{  \}}\OperatorTok{,} \KeywordTok{true}\OperatorTok{,}\NormalTok{ [}\StringTok{\textquotesingle{}sign\textquotesingle{}}\OperatorTok{,} \StringTok{\textquotesingle{}verify\textquotesingle{}}\NormalTok{])}\OperatorTok{;}

  \KeywordTok{const}\NormalTok{ keyObject }\OperatorTok{=}\NormalTok{ KeyObject}\OperatorTok{.}\FunctionTok{from}\NormalTok{(key)}\OperatorTok{;}
  \BuiltInTok{console}\OperatorTok{.}\FunctionTok{log}\NormalTok{(keyObject}\OperatorTok{.}\AttributeTok{symmetricKeySize}\NormalTok{)}\OperatorTok{;}
  \CommentTok{// Prints: 32 (symmetric key size in bytes)}
\NormalTok{\})()}\OperatorTok{;}
\end{Highlighting}
\end{Shaded}

\subsubsection{\texorpdfstring{\texttt{keyObject.asymmetricKeyDetails}}{keyObject.asymmetricKeyDetails}}\label{keyobject.asymmetrickeydetails}

\begin{itemize}
\tightlist
\item
  \{Object\}

  \begin{itemize}
  \tightlist
  \item
    \texttt{modulusLength}: \{number\} Key size in bits (RSA, DSA).
  \item
    \texttt{publicExponent}: \{bigint\} Public exponent (RSA).
  \item
    \texttt{hashAlgorithm}: \{string\} Name of the message digest
    (RSA-PSS).
  \item
    \texttt{mgf1HashAlgorithm}: \{string\} Name of the message digest
    used by MGF1 (RSA-PSS).
  \item
    \texttt{saltLength}: \{number\} Minimal salt length in bytes
    (RSA-PSS).
  \item
    \texttt{divisorLength}: \{number\} Size of \texttt{q} in bits (DSA).
  \item
    \texttt{namedCurve}: \{string\} Name of the curve (EC).
  \end{itemize}
\end{itemize}

This property exists only on asymmetric keys. Depending on the type of
the key, this object contains information about the key. None of the
information obtained through this property can be used to uniquely
identify a key or to compromise the security of the key.

For RSA-PSS keys, if the key material contains a
\texttt{RSASSA-PSS-params} sequence, the \texttt{hashAlgorithm},
\texttt{mgf1HashAlgorithm}, and \texttt{saltLength} properties will be
set.

Other key details might be exposed via this API using additional
attributes.

\subsubsection{\texorpdfstring{\texttt{keyObject.asymmetricKeyType}}{keyObject.asymmetricKeyType}}\label{keyobject.asymmetrickeytype}

\begin{itemize}
\tightlist
\item
  \{string\}
\end{itemize}

For asymmetric keys, this property represents the type of the key.
Supported key types are:

\begin{itemize}
\tightlist
\item
  \texttt{\textquotesingle{}rsa\textquotesingle{}} (OID
  1.2.840.113549.1.1.1)
\item
  \texttt{\textquotesingle{}rsa-pss\textquotesingle{}} (OID
  1.2.840.113549.1.1.10)
\item
  \texttt{\textquotesingle{}dsa\textquotesingle{}} (OID
  1.2.840.10040.4.1)
\item
  \texttt{\textquotesingle{}ec\textquotesingle{}} (OID
  1.2.840.10045.2.1)
\item
  \texttt{\textquotesingle{}x25519\textquotesingle{}} (OID 1.3.101.110)
\item
  \texttt{\textquotesingle{}x448\textquotesingle{}} (OID 1.3.101.111)
\item
  \texttt{\textquotesingle{}ed25519\textquotesingle{}} (OID 1.3.101.112)
\item
  \texttt{\textquotesingle{}ed448\textquotesingle{}} (OID 1.3.101.113)
\item
  \texttt{\textquotesingle{}dh\textquotesingle{}} (OID
  1.2.840.113549.1.3.1)
\end{itemize}

This property is \texttt{undefined} for unrecognized \texttt{KeyObject}
types and symmetric keys.

\subsubsection{\texorpdfstring{\texttt{keyObject.export({[}options{]})}}{keyObject.export({[}options{]})}}\label{keyobject.exportoptions}

\begin{itemize}
\tightlist
\item
  \texttt{options}: \{Object\}
\item
  Returns: \{string \textbar{} Buffer \textbar{} Object\}
\end{itemize}

For symmetric keys, the following encoding options can be used:

\begin{itemize}
\tightlist
\item
  \texttt{format}: \{string\} Must be
  \texttt{\textquotesingle{}buffer\textquotesingle{}} (default) or
  \texttt{\textquotesingle{}jwk\textquotesingle{}}.
\end{itemize}

For public keys, the following encoding options can be used:

\begin{itemize}
\tightlist
\item
  \texttt{type}: \{string\} Must be one of
  \texttt{\textquotesingle{}pkcs1\textquotesingle{}} (RSA only) or
  \texttt{\textquotesingle{}spki\textquotesingle{}}.
\item
  \texttt{format}: \{string\} Must be
  \texttt{\textquotesingle{}pem\textquotesingle{}},
  \texttt{\textquotesingle{}der\textquotesingle{}}, or
  \texttt{\textquotesingle{}jwk\textquotesingle{}}.
\end{itemize}

For private keys, the following encoding options can be used:

\begin{itemize}
\tightlist
\item
  \texttt{type}: \{string\} Must be one of
  \texttt{\textquotesingle{}pkcs1\textquotesingle{}} (RSA only),
  \texttt{\textquotesingle{}pkcs8\textquotesingle{}} or
  \texttt{\textquotesingle{}sec1\textquotesingle{}} (EC only).
\item
  \texttt{format}: \{string\} Must be
  \texttt{\textquotesingle{}pem\textquotesingle{}},
  \texttt{\textquotesingle{}der\textquotesingle{}}, or
  \texttt{\textquotesingle{}jwk\textquotesingle{}}.
\item
  \texttt{cipher}: \{string\} If specified, the private key will be
  encrypted with the given \texttt{cipher} and \texttt{passphrase} using
  PKCS\#5 v2.0 password based encryption.
\item
  \texttt{passphrase}: \{string \textbar{} Buffer\} The passphrase to
  use for encryption, see \texttt{cipher}.
\end{itemize}

The result type depends on the selected encoding format, when PEM the
result is a string, when DER it will be a buffer containing the data
encoded as DER, when \href{https://tools.ietf.org/html/rfc7517}{JWK} it
will be an object.

When \href{https://tools.ietf.org/html/rfc7517}{JWK} encoding format was
selected, all other encoding options are ignored.

PKCS\#1, SEC1, and PKCS\#8 type keys can be encrypted by using a
combination of the \texttt{cipher} and \texttt{format} options. The
PKCS\#8 \texttt{type} can be used with any \texttt{format} to encrypt
any key algorithm (RSA, EC, or DH) by specifying a \texttt{cipher}.
PKCS\#1 and SEC1 can only be encrypted by specifying a \texttt{cipher}
when the PEM \texttt{format} is used. For maximum compatibility, use
PKCS\#8 for encrypted private keys. Since PKCS\#8 defines its own
encryption mechanism, PEM-level encryption is not supported when
encrypting a PKCS\#8 key. See
\href{https://www.rfc-editor.org/rfc/rfc5208.txt}{RFC 5208} for PKCS\#8
encryption and \href{https://www.rfc-editor.org/rfc/rfc1421.txt}{RFC
1421} for PKCS\#1 and SEC1 encryption.

\subsubsection{\texorpdfstring{\texttt{keyObject.equals(otherKeyObject)}}{keyObject.equals(otherKeyObject)}}\label{keyobject.equalsotherkeyobject}

\begin{itemize}
\tightlist
\item
  \texttt{otherKeyObject}: \{KeyObject\} A \texttt{KeyObject} with which
  to compare \texttt{keyObject}.
\item
  Returns: \{boolean\}
\end{itemize}

Returns \texttt{true} or \texttt{false} depending on whether the keys
have exactly the same type, value, and parameters. This method is not
\href{https://en.wikipedia.org/wiki/Timing_attack}{constant time}.

\subsubsection{\texorpdfstring{\texttt{keyObject.symmetricKeySize}}{keyObject.symmetricKeySize}}\label{keyobject.symmetrickeysize}

\begin{itemize}
\tightlist
\item
  \{number\}
\end{itemize}

For secret keys, this property represents the size of the key in bytes.
This property is \texttt{undefined} for asymmetric keys.

\subsubsection{\texorpdfstring{\texttt{keyObject.type}}{keyObject.type}}\label{keyobject.type}

\begin{itemize}
\tightlist
\item
  \{string\}
\end{itemize}

Depending on the type of this \texttt{KeyObject}, this property is
either \texttt{\textquotesingle{}secret\textquotesingle{}} for secret
(symmetric) keys, \texttt{\textquotesingle{}public\textquotesingle{}}
for public (asymmetric) keys or
\texttt{\textquotesingle{}private\textquotesingle{}} for private
(asymmetric) keys.

\subsection{\texorpdfstring{Class:
\texttt{Sign}}{Class: Sign}}\label{class-sign}

\begin{itemize}
\tightlist
\item
  Extends: \{stream.Writable\}
\end{itemize}

The \texttt{Sign} class is a utility for generating signatures. It can
be used in one of two ways:

\begin{itemize}
\tightlist
\item
  As a writable \href{stream.md}{stream}, where data to be signed is
  written and the
  \hyperref[signsignprivatekey-outputencoding]{\texttt{sign.sign()}}
  method is used to generate and return the signature, or
\item
  Using the
  \hyperref[signupdatedata-inputencoding]{\texttt{sign.update()}} and
  \hyperref[signsignprivatekey-outputencoding]{\texttt{sign.sign()}}
  methods to produce the signature.
\end{itemize}

The
\hyperref[cryptocreatesignalgorithm-options]{\texttt{crypto.createSign()}}
method is used to create \texttt{Sign} instances. The argument is the
string name of the hash function to use. \texttt{Sign} objects are not
to be created directly using the \texttt{new} keyword.

Example: Using \texttt{Sign} and
\hyperref[class-verify]{\texttt{Verify}} objects as streams:

\begin{Shaded}
\begin{Highlighting}[]
\KeywordTok{const}\NormalTok{ \{}
\NormalTok{  generateKeyPairSync}\OperatorTok{,}
\NormalTok{  createSign}\OperatorTok{,}
\NormalTok{  createVerify}\OperatorTok{,}
\NormalTok{\} }\OperatorTok{=} \ControlFlowTok{await} \ImportTok{import}\NormalTok{(}\StringTok{\textquotesingle{}node:crypto\textquotesingle{}}\NormalTok{)}\OperatorTok{;}

\KeywordTok{const}\NormalTok{ \{ privateKey}\OperatorTok{,}\NormalTok{ publicKey \} }\OperatorTok{=} \FunctionTok{generateKeyPairSync}\NormalTok{(}\StringTok{\textquotesingle{}ec\textquotesingle{}}\OperatorTok{,}\NormalTok{ \{}
  \DataTypeTok{namedCurve}\OperatorTok{:} \StringTok{\textquotesingle{}sect239k1\textquotesingle{}}\OperatorTok{,}
\NormalTok{\})}\OperatorTok{;}

\KeywordTok{const}\NormalTok{ sign }\OperatorTok{=} \FunctionTok{createSign}\NormalTok{(}\StringTok{\textquotesingle{}SHA256\textquotesingle{}}\NormalTok{)}\OperatorTok{;}
\NormalTok{sign}\OperatorTok{.}\FunctionTok{write}\NormalTok{(}\StringTok{\textquotesingle{}some data to sign\textquotesingle{}}\NormalTok{)}\OperatorTok{;}
\NormalTok{sign}\OperatorTok{.}\FunctionTok{end}\NormalTok{()}\OperatorTok{;}
\KeywordTok{const}\NormalTok{ signature }\OperatorTok{=}\NormalTok{ sign}\OperatorTok{.}\FunctionTok{sign}\NormalTok{(privateKey}\OperatorTok{,} \StringTok{\textquotesingle{}hex\textquotesingle{}}\NormalTok{)}\OperatorTok{;}

\KeywordTok{const}\NormalTok{ verify }\OperatorTok{=} \FunctionTok{createVerify}\NormalTok{(}\StringTok{\textquotesingle{}SHA256\textquotesingle{}}\NormalTok{)}\OperatorTok{;}
\NormalTok{verify}\OperatorTok{.}\FunctionTok{write}\NormalTok{(}\StringTok{\textquotesingle{}some data to sign\textquotesingle{}}\NormalTok{)}\OperatorTok{;}
\NormalTok{verify}\OperatorTok{.}\FunctionTok{end}\NormalTok{()}\OperatorTok{;}
\BuiltInTok{console}\OperatorTok{.}\FunctionTok{log}\NormalTok{(verify}\OperatorTok{.}\FunctionTok{verify}\NormalTok{(publicKey}\OperatorTok{,}\NormalTok{ signature}\OperatorTok{,} \StringTok{\textquotesingle{}hex\textquotesingle{}}\NormalTok{))}\OperatorTok{;}
\CommentTok{// Prints: true}
\end{Highlighting}
\end{Shaded}

\begin{Shaded}
\begin{Highlighting}[]
\KeywordTok{const}\NormalTok{ \{}
\NormalTok{  generateKeyPairSync}\OperatorTok{,}
\NormalTok{  createSign}\OperatorTok{,}
\NormalTok{  createVerify}\OperatorTok{,}
\NormalTok{\} }\OperatorTok{=} \PreprocessorTok{require}\NormalTok{(}\StringTok{\textquotesingle{}node:crypto\textquotesingle{}}\NormalTok{)}\OperatorTok{;}

\KeywordTok{const}\NormalTok{ \{ privateKey}\OperatorTok{,}\NormalTok{ publicKey \} }\OperatorTok{=} \FunctionTok{generateKeyPairSync}\NormalTok{(}\StringTok{\textquotesingle{}ec\textquotesingle{}}\OperatorTok{,}\NormalTok{ \{}
  \DataTypeTok{namedCurve}\OperatorTok{:} \StringTok{\textquotesingle{}sect239k1\textquotesingle{}}\OperatorTok{,}
\NormalTok{\})}\OperatorTok{;}

\KeywordTok{const}\NormalTok{ sign }\OperatorTok{=} \FunctionTok{createSign}\NormalTok{(}\StringTok{\textquotesingle{}SHA256\textquotesingle{}}\NormalTok{)}\OperatorTok{;}
\NormalTok{sign}\OperatorTok{.}\FunctionTok{write}\NormalTok{(}\StringTok{\textquotesingle{}some data to sign\textquotesingle{}}\NormalTok{)}\OperatorTok{;}
\NormalTok{sign}\OperatorTok{.}\FunctionTok{end}\NormalTok{()}\OperatorTok{;}
\KeywordTok{const}\NormalTok{ signature }\OperatorTok{=}\NormalTok{ sign}\OperatorTok{.}\FunctionTok{sign}\NormalTok{(privateKey}\OperatorTok{,} \StringTok{\textquotesingle{}hex\textquotesingle{}}\NormalTok{)}\OperatorTok{;}

\KeywordTok{const}\NormalTok{ verify }\OperatorTok{=} \FunctionTok{createVerify}\NormalTok{(}\StringTok{\textquotesingle{}SHA256\textquotesingle{}}\NormalTok{)}\OperatorTok{;}
\NormalTok{verify}\OperatorTok{.}\FunctionTok{write}\NormalTok{(}\StringTok{\textquotesingle{}some data to sign\textquotesingle{}}\NormalTok{)}\OperatorTok{;}
\NormalTok{verify}\OperatorTok{.}\FunctionTok{end}\NormalTok{()}\OperatorTok{;}
\BuiltInTok{console}\OperatorTok{.}\FunctionTok{log}\NormalTok{(verify}\OperatorTok{.}\FunctionTok{verify}\NormalTok{(publicKey}\OperatorTok{,}\NormalTok{ signature}\OperatorTok{,} \StringTok{\textquotesingle{}hex\textquotesingle{}}\NormalTok{))}\OperatorTok{;}
\CommentTok{// Prints: true}
\end{Highlighting}
\end{Shaded}

Example: Using the
\hyperref[signupdatedata-inputencoding]{\texttt{sign.update()}} and
\hyperref[verifyupdatedata-inputencoding]{\texttt{verify.update()}}
methods:

\begin{Shaded}
\begin{Highlighting}[]
\KeywordTok{const}\NormalTok{ \{}
\NormalTok{  generateKeyPairSync}\OperatorTok{,}
\NormalTok{  createSign}\OperatorTok{,}
\NormalTok{  createVerify}\OperatorTok{,}
\NormalTok{\} }\OperatorTok{=} \ControlFlowTok{await} \ImportTok{import}\NormalTok{(}\StringTok{\textquotesingle{}node:crypto\textquotesingle{}}\NormalTok{)}\OperatorTok{;}

\KeywordTok{const}\NormalTok{ \{ privateKey}\OperatorTok{,}\NormalTok{ publicKey \} }\OperatorTok{=} \FunctionTok{generateKeyPairSync}\NormalTok{(}\StringTok{\textquotesingle{}rsa\textquotesingle{}}\OperatorTok{,}\NormalTok{ \{}
  \DataTypeTok{modulusLength}\OperatorTok{:} \DecValTok{2048}\OperatorTok{,}
\NormalTok{\})}\OperatorTok{;}

\KeywordTok{const}\NormalTok{ sign }\OperatorTok{=} \FunctionTok{createSign}\NormalTok{(}\StringTok{\textquotesingle{}SHA256\textquotesingle{}}\NormalTok{)}\OperatorTok{;}
\NormalTok{sign}\OperatorTok{.}\FunctionTok{update}\NormalTok{(}\StringTok{\textquotesingle{}some data to sign\textquotesingle{}}\NormalTok{)}\OperatorTok{;}
\NormalTok{sign}\OperatorTok{.}\FunctionTok{end}\NormalTok{()}\OperatorTok{;}
\KeywordTok{const}\NormalTok{ signature }\OperatorTok{=}\NormalTok{ sign}\OperatorTok{.}\FunctionTok{sign}\NormalTok{(privateKey)}\OperatorTok{;}

\KeywordTok{const}\NormalTok{ verify }\OperatorTok{=} \FunctionTok{createVerify}\NormalTok{(}\StringTok{\textquotesingle{}SHA256\textquotesingle{}}\NormalTok{)}\OperatorTok{;}
\NormalTok{verify}\OperatorTok{.}\FunctionTok{update}\NormalTok{(}\StringTok{\textquotesingle{}some data to sign\textquotesingle{}}\NormalTok{)}\OperatorTok{;}
\NormalTok{verify}\OperatorTok{.}\FunctionTok{end}\NormalTok{()}\OperatorTok{;}
\BuiltInTok{console}\OperatorTok{.}\FunctionTok{log}\NormalTok{(verify}\OperatorTok{.}\FunctionTok{verify}\NormalTok{(publicKey}\OperatorTok{,}\NormalTok{ signature))}\OperatorTok{;}
\CommentTok{// Prints: true}
\end{Highlighting}
\end{Shaded}

\begin{Shaded}
\begin{Highlighting}[]
\KeywordTok{const}\NormalTok{ \{}
\NormalTok{  generateKeyPairSync}\OperatorTok{,}
\NormalTok{  createSign}\OperatorTok{,}
\NormalTok{  createVerify}\OperatorTok{,}
\NormalTok{\} }\OperatorTok{=} \PreprocessorTok{require}\NormalTok{(}\StringTok{\textquotesingle{}node:crypto\textquotesingle{}}\NormalTok{)}\OperatorTok{;}

\KeywordTok{const}\NormalTok{ \{ privateKey}\OperatorTok{,}\NormalTok{ publicKey \} }\OperatorTok{=} \FunctionTok{generateKeyPairSync}\NormalTok{(}\StringTok{\textquotesingle{}rsa\textquotesingle{}}\OperatorTok{,}\NormalTok{ \{}
  \DataTypeTok{modulusLength}\OperatorTok{:} \DecValTok{2048}\OperatorTok{,}
\NormalTok{\})}\OperatorTok{;}

\KeywordTok{const}\NormalTok{ sign }\OperatorTok{=} \FunctionTok{createSign}\NormalTok{(}\StringTok{\textquotesingle{}SHA256\textquotesingle{}}\NormalTok{)}\OperatorTok{;}
\NormalTok{sign}\OperatorTok{.}\FunctionTok{update}\NormalTok{(}\StringTok{\textquotesingle{}some data to sign\textquotesingle{}}\NormalTok{)}\OperatorTok{;}
\NormalTok{sign}\OperatorTok{.}\FunctionTok{end}\NormalTok{()}\OperatorTok{;}
\KeywordTok{const}\NormalTok{ signature }\OperatorTok{=}\NormalTok{ sign}\OperatorTok{.}\FunctionTok{sign}\NormalTok{(privateKey)}\OperatorTok{;}

\KeywordTok{const}\NormalTok{ verify }\OperatorTok{=} \FunctionTok{createVerify}\NormalTok{(}\StringTok{\textquotesingle{}SHA256\textquotesingle{}}\NormalTok{)}\OperatorTok{;}
\NormalTok{verify}\OperatorTok{.}\FunctionTok{update}\NormalTok{(}\StringTok{\textquotesingle{}some data to sign\textquotesingle{}}\NormalTok{)}\OperatorTok{;}
\NormalTok{verify}\OperatorTok{.}\FunctionTok{end}\NormalTok{()}\OperatorTok{;}
\BuiltInTok{console}\OperatorTok{.}\FunctionTok{log}\NormalTok{(verify}\OperatorTok{.}\FunctionTok{verify}\NormalTok{(publicKey}\OperatorTok{,}\NormalTok{ signature))}\OperatorTok{;}
\CommentTok{// Prints: true}
\end{Highlighting}
\end{Shaded}

\subsubsection{\texorpdfstring{\texttt{sign.sign(privateKey{[},\ outputEncoding{]})}}{sign.sign(privateKey{[}, outputEncoding{]})}}\label{sign.signprivatekey-outputencoding}

\begin{itemize}
\tightlist
\item
  \texttt{privateKey}
  \{Object\textbar string\textbar ArrayBuffer\textbar Buffer\textbar TypedArray\textbar DataView\textbar KeyObject\textbar CryptoKey\}

  \begin{itemize}
  \tightlist
  \item
    \texttt{dsaEncoding} \{string\}
  \item
    \texttt{padding} \{integer\}
  \item
    \texttt{saltLength} \{integer\}
  \end{itemize}
\item
  \texttt{outputEncoding} \{string\} The
  \href{buffer.md\#buffers-and-character-encodings}{encoding} of the
  return value.
\item
  Returns: \{Buffer \textbar{} string\}
\end{itemize}

Calculates the signature on all the data passed through using either
\hyperref[signupdatedata-inputencoding]{\texttt{sign.update()}} or
\href{stream.md\#writablewritechunk-encoding-callback}{\texttt{sign.write()}}.

If \texttt{privateKey} is not a
\hyperref[class-keyobject]{\texttt{KeyObject}}, this function behaves as
if \texttt{privateKey} had been passed to
\hyperref[cryptocreateprivatekeykey]{\texttt{crypto.createPrivateKey()}}.
If it is an object, the following additional properties can be passed:

\begin{itemize}
\item
  \texttt{dsaEncoding} \{string\} For DSA and ECDSA, this option
  specifies the format of the generated signature. It can be one of the
  following:

  \begin{itemize}
  \tightlist
  \item
    \texttt{\textquotesingle{}der\textquotesingle{}} (default):
    DER-encoded ASN.1 signature structure encoding \texttt{(r,\ s)}.
  \item
    \texttt{\textquotesingle{}ieee-p1363\textquotesingle{}}: Signature
    format \texttt{r\ \textbar{}\textbar{}\ s} as proposed in
    IEEE-P1363.
  \end{itemize}
\item
  \texttt{padding} \{integer\} Optional padding value for RSA, one of
  the following:

  \begin{itemize}
  \tightlist
  \item
    \texttt{crypto.constants.RSA\_PKCS1\_PADDING} (default)
  \item
    \texttt{crypto.constants.RSA\_PKCS1\_PSS\_PADDING}
  \end{itemize}

  \texttt{RSA\_PKCS1\_PSS\_PADDING} will use MGF1 with the same hash
  function used to sign the message as specified in section 3.1 of
  \href{https://www.rfc-editor.org/rfc/rfc4055.txt}{RFC 4055}, unless an
  MGF1 hash function has been specified as part of the key in compliance
  with section 3.3 of
  \href{https://www.rfc-editor.org/rfc/rfc4055.txt}{RFC 4055}.
\item
  \texttt{saltLength} \{integer\} Salt length for when padding is
  \texttt{RSA\_PKCS1\_PSS\_PADDING}. The special value
  \texttt{crypto.constants.RSA\_PSS\_SALTLEN\_DIGEST} sets the salt
  length to the digest size,
  \texttt{crypto.constants.RSA\_PSS\_SALTLEN\_MAX\_SIGN} (default) sets
  it to the maximum permissible value.
\end{itemize}

If \texttt{outputEncoding} is provided a string is returned; otherwise a
\href{buffer.md}{\texttt{Buffer}} is returned.

The \texttt{Sign} object can not be again used after
\texttt{sign.sign()} method has been called. Multiple calls to
\texttt{sign.sign()} will result in an error being thrown.

\subsubsection{\texorpdfstring{\texttt{sign.update(data{[},\ inputEncoding{]})}}{sign.update(data{[}, inputEncoding{]})}}\label{sign.updatedata-inputencoding}

\begin{itemize}
\tightlist
\item
  \texttt{data}
  \{string\textbar Buffer\textbar TypedArray\textbar DataView\}
\item
  \texttt{inputEncoding} \{string\} The
  \href{buffer.md\#buffers-and-character-encodings}{encoding} of the
  \texttt{data} string.
\end{itemize}

Updates the \texttt{Sign} content with the given \texttt{data}, the
encoding of which is given in \texttt{inputEncoding}. If
\texttt{encoding} is not provided, and the \texttt{data} is a string, an
encoding of \texttt{\textquotesingle{}utf8\textquotesingle{}} is
enforced. If \texttt{data} is a \href{buffer.md}{\texttt{Buffer}},
\texttt{TypedArray}, or \texttt{DataView}, then \texttt{inputEncoding}
is ignored.

This can be called many times with new data as it is streamed.

\subsection{\texorpdfstring{Class:
\texttt{Verify}}{Class: Verify}}\label{class-verify}

\begin{itemize}
\tightlist
\item
  Extends: \{stream.Writable\}
\end{itemize}

The \texttt{Verify} class is a utility for verifying signatures. It can
be used in one of two ways:

\begin{itemize}
\tightlist
\item
  As a writable \href{stream.md}{stream} where written data is used to
  validate against the supplied signature, or
\item
  Using the
  \hyperref[verifyupdatedata-inputencoding]{\texttt{verify.update()}}
  and
  \hyperref[verifyverifyobject-signature-signatureencoding]{\texttt{verify.verify()}}
  methods to verify the signature.
\end{itemize}

The
\hyperref[cryptocreateverifyalgorithm-options]{\texttt{crypto.createVerify()}}
method is used to create \texttt{Verify} instances. \texttt{Verify}
objects are not to be created directly using the \texttt{new} keyword.

See \hyperref[class-sign]{\texttt{Sign}} for examples.

\subsubsection{\texorpdfstring{\texttt{verify.update(data{[},\ inputEncoding{]})}}{verify.update(data{[}, inputEncoding{]})}}\label{verify.updatedata-inputencoding}

\begin{itemize}
\tightlist
\item
  \texttt{data}
  \{string\textbar Buffer\textbar TypedArray\textbar DataView\}
\item
  \texttt{inputEncoding} \{string\} The
  \href{buffer.md\#buffers-and-character-encodings}{encoding} of the
  \texttt{data} string.
\end{itemize}

Updates the \texttt{Verify} content with the given \texttt{data}, the
encoding of which is given in \texttt{inputEncoding}. If
\texttt{inputEncoding} is not provided, and the \texttt{data} is a
string, an encoding of \texttt{\textquotesingle{}utf8\textquotesingle{}}
is enforced. If \texttt{data} is a \href{buffer.md}{\texttt{Buffer}},
\texttt{TypedArray}, or \texttt{DataView}, then \texttt{inputEncoding}
is ignored.

This can be called many times with new data as it is streamed.

\subsubsection{\texorpdfstring{\texttt{verify.verify(object,\ signature{[},\ signatureEncoding{]})}}{verify.verify(object, signature{[}, signatureEncoding{]})}}\label{verify.verifyobject-signature-signatureencoding}

\begin{itemize}
\tightlist
\item
  \texttt{object}
  \{Object\textbar string\textbar ArrayBuffer\textbar Buffer\textbar TypedArray\textbar DataView\textbar KeyObject\textbar CryptoKey\}

  \begin{itemize}
  \tightlist
  \item
    \texttt{dsaEncoding} \{string\}
  \item
    \texttt{padding} \{integer\}
  \item
    \texttt{saltLength} \{integer\}
  \end{itemize}
\item
  \texttt{signature}
  \{string\textbar ArrayBuffer\textbar Buffer\textbar TypedArray\textbar DataView\}
\item
  \texttt{signatureEncoding} \{string\} The
  \href{buffer.md\#buffers-and-character-encodings}{encoding} of the
  \texttt{signature} string.
\item
  Returns: \{boolean\} \texttt{true} or \texttt{false} depending on the
  validity of the signature for the data and public key.
\end{itemize}

Verifies the provided data using the given \texttt{object} and
\texttt{signature}.

If \texttt{object} is not a
\hyperref[class-keyobject]{\texttt{KeyObject}}, this function behaves as
if \texttt{object} had been passed to
\hyperref[cryptocreatepublickeykey]{\texttt{crypto.createPublicKey()}}.
If it is an object, the following additional properties can be passed:

\begin{itemize}
\item
  \texttt{dsaEncoding} \{string\} For DSA and ECDSA, this option
  specifies the format of the signature. It can be one of the following:

  \begin{itemize}
  \tightlist
  \item
    \texttt{\textquotesingle{}der\textquotesingle{}} (default):
    DER-encoded ASN.1 signature structure encoding \texttt{(r,\ s)}.
  \item
    \texttt{\textquotesingle{}ieee-p1363\textquotesingle{}}: Signature
    format \texttt{r\ \textbar{}\textbar{}\ s} as proposed in
    IEEE-P1363.
  \end{itemize}
\item
  \texttt{padding} \{integer\} Optional padding value for RSA, one of
  the following:

  \begin{itemize}
  \tightlist
  \item
    \texttt{crypto.constants.RSA\_PKCS1\_PADDING} (default)
  \item
    \texttt{crypto.constants.RSA\_PKCS1\_PSS\_PADDING}
  \end{itemize}

  \texttt{RSA\_PKCS1\_PSS\_PADDING} will use MGF1 with the same hash
  function used to verify the message as specified in section 3.1 of
  \href{https://www.rfc-editor.org/rfc/rfc4055.txt}{RFC 4055}, unless an
  MGF1 hash function has been specified as part of the key in compliance
  with section 3.3 of
  \href{https://www.rfc-editor.org/rfc/rfc4055.txt}{RFC 4055}.
\item
  \texttt{saltLength} \{integer\} Salt length for when padding is
  \texttt{RSA\_PKCS1\_PSS\_PADDING}. The special value
  \texttt{crypto.constants.RSA\_PSS\_SALTLEN\_DIGEST} sets the salt
  length to the digest size,
  \texttt{crypto.constants.RSA\_PSS\_SALTLEN\_AUTO} (default) causes it
  to be determined automatically.
\end{itemize}

The \texttt{signature} argument is the previously calculated signature
for the data, in the \texttt{signatureEncoding}. If a
\texttt{signatureEncoding} is specified, the \texttt{signature} is
expected to be a string; otherwise \texttt{signature} is expected to be
a \href{buffer.md}{\texttt{Buffer}}, \texttt{TypedArray}, or
\texttt{DataView}.

The \texttt{verify} object can not be used again after
\texttt{verify.verify()} has been called. Multiple calls to
\texttt{verify.verify()} will result in an error being thrown.

Because public keys can be derived from private keys, a private key may
be passed instead of a public key.

\subsection{\texorpdfstring{Class:
\texttt{X509Certificate}}{Class: X509Certificate}}\label{class-x509certificate}

Encapsulates an X509 certificate and provides read-only access to its
information.

\begin{Shaded}
\begin{Highlighting}[]
\KeywordTok{const}\NormalTok{ \{ X509Certificate \} }\OperatorTok{=} \ControlFlowTok{await} \ImportTok{import}\NormalTok{(}\StringTok{\textquotesingle{}node:crypto\textquotesingle{}}\NormalTok{)}\OperatorTok{;}

\KeywordTok{const}\NormalTok{ x509 }\OperatorTok{=} \KeywordTok{new} \FunctionTok{X509Certificate}\NormalTok{(}\StringTok{\textquotesingle{}\{... pem encoded cert ...\}\textquotesingle{}}\NormalTok{)}\OperatorTok{;}

\BuiltInTok{console}\OperatorTok{.}\FunctionTok{log}\NormalTok{(x509}\OperatorTok{.}\AttributeTok{subject}\NormalTok{)}\OperatorTok{;}
\end{Highlighting}
\end{Shaded}

\begin{Shaded}
\begin{Highlighting}[]
\KeywordTok{const}\NormalTok{ \{ X509Certificate \} }\OperatorTok{=} \PreprocessorTok{require}\NormalTok{(}\StringTok{\textquotesingle{}node:crypto\textquotesingle{}}\NormalTok{)}\OperatorTok{;}

\KeywordTok{const}\NormalTok{ x509 }\OperatorTok{=} \KeywordTok{new} \FunctionTok{X509Certificate}\NormalTok{(}\StringTok{\textquotesingle{}\{... pem encoded cert ...\}\textquotesingle{}}\NormalTok{)}\OperatorTok{;}

\BuiltInTok{console}\OperatorTok{.}\FunctionTok{log}\NormalTok{(x509}\OperatorTok{.}\AttributeTok{subject}\NormalTok{)}\OperatorTok{;}
\end{Highlighting}
\end{Shaded}

\subsubsection{\texorpdfstring{\texttt{new\ X509Certificate(buffer)}}{new X509Certificate(buffer)}}\label{new-x509certificatebuffer}

\begin{itemize}
\tightlist
\item
  \texttt{buffer}
  \{string\textbar TypedArray\textbar Buffer\textbar DataView\} A PEM or
  DER encoded X509 Certificate.
\end{itemize}

\subsubsection{\texorpdfstring{\texttt{x509.ca}}{x509.ca}}\label{x509.ca}

\begin{itemize}
\tightlist
\item
  Type: \{boolean\} Will be \texttt{true} if this is a Certificate
  Authority (CA) certificate.
\end{itemize}

\subsubsection{\texorpdfstring{\texttt{x509.checkEmail(email{[},\ options{]})}}{x509.checkEmail(email{[}, options{]})}}\label{x509.checkemailemail-options}

\begin{itemize}
\tightlist
\item
  \texttt{email} \{string\}
\item
  \texttt{options} \{Object\}

  \begin{itemize}
  \tightlist
  \item
    \texttt{subject} \{string\}
    \texttt{\textquotesingle{}default\textquotesingle{}},
    \texttt{\textquotesingle{}always\textquotesingle{}}, or
    \texttt{\textquotesingle{}never\textquotesingle{}}.
    \textbf{Default:}
    \texttt{\textquotesingle{}default\textquotesingle{}}.
  \end{itemize}
\item
  Returns: \{string\textbar undefined\} Returns \texttt{email} if the
  certificate matches, \texttt{undefined} if it does not.
\end{itemize}

Checks whether the certificate matches the given email address.

If the \texttt{\textquotesingle{}subject\textquotesingle{}} option is
undefined or set to
\texttt{\textquotesingle{}default\textquotesingle{}}, the certificate
subject is only considered if the subject alternative name extension
either does not exist or does not contain any email addresses.

If the \texttt{\textquotesingle{}subject\textquotesingle{}} option is
set to \texttt{\textquotesingle{}always\textquotesingle{}} and if the
subject alternative name extension either does not exist or does not
contain a matching email address, the certificate subject is considered.

If the \texttt{\textquotesingle{}subject\textquotesingle{}} option is
set to \texttt{\textquotesingle{}never\textquotesingle{}}, the
certificate subject is never considered, even if the certificate
contains no subject alternative names.

\subsubsection{\texorpdfstring{\texttt{x509.checkHost(name{[},\ options{]})}}{x509.checkHost(name{[}, options{]})}}\label{x509.checkhostname-options}

\begin{itemize}
\tightlist
\item
  \texttt{name} \{string\}
\item
  \texttt{options} \{Object\}

  \begin{itemize}
  \tightlist
  \item
    \texttt{subject} \{string\}
    \texttt{\textquotesingle{}default\textquotesingle{}},
    \texttt{\textquotesingle{}always\textquotesingle{}}, or
    \texttt{\textquotesingle{}never\textquotesingle{}}.
    \textbf{Default:}
    \texttt{\textquotesingle{}default\textquotesingle{}}.
  \item
    \texttt{wildcards} \{boolean\} \textbf{Default:} \texttt{true}.
  \item
    \texttt{partialWildcards} \{boolean\} \textbf{Default:}
    \texttt{true}.
  \item
    \texttt{multiLabelWildcards} \{boolean\} \textbf{Default:}
    \texttt{false}.
  \item
    \texttt{singleLabelSubdomains} \{boolean\} \textbf{Default:}
    \texttt{false}.
  \end{itemize}
\item
  Returns: \{string\textbar undefined\} Returns a subject name that
  matches \texttt{name}, or \texttt{undefined} if no subject name
  matches \texttt{name}.
\end{itemize}

Checks whether the certificate matches the given host name.

If the certificate matches the given host name, the matching subject
name is returned. The returned name might be an exact match (e.g.,
\texttt{foo.example.com}) or it might contain wildcards (e.g.,
\texttt{*.example.com}). Because host name comparisons are
case-insensitive, the returned subject name might also differ from the
given \texttt{name} in capitalization.

If the \texttt{\textquotesingle{}subject\textquotesingle{}} option is
undefined or set to
\texttt{\textquotesingle{}default\textquotesingle{}}, the certificate
subject is only considered if the subject alternative name extension
either does not exist or does not contain any DNS names. This behavior
is consistent with \href{https://www.rfc-editor.org/rfc/rfc2818.txt}{RFC
2818} (``HTTP Over TLS'').

If the \texttt{\textquotesingle{}subject\textquotesingle{}} option is
set to \texttt{\textquotesingle{}always\textquotesingle{}} and if the
subject alternative name extension either does not exist or does not
contain a matching DNS name, the certificate subject is considered.

If the \texttt{\textquotesingle{}subject\textquotesingle{}} option is
set to \texttt{\textquotesingle{}never\textquotesingle{}}, the
certificate subject is never considered, even if the certificate
contains no subject alternative names.

\subsubsection{\texorpdfstring{\texttt{x509.checkIP(ip)}}{x509.checkIP(ip)}}\label{x509.checkipip}

\begin{itemize}
\tightlist
\item
  \texttt{ip} \{string\}
\item
  Returns: \{string\textbar undefined\} Returns \texttt{ip} if the
  certificate matches, \texttt{undefined} if it does not.
\end{itemize}

Checks whether the certificate matches the given IP address (IPv4 or
IPv6).

Only \href{https://www.rfc-editor.org/rfc/rfc5280.txt}{RFC 5280}
\texttt{iPAddress} subject alternative names are considered, and they
must match the given \texttt{ip} address exactly. Other subject
alternative names as well as the subject field of the certificate are
ignored.

\subsubsection{\texorpdfstring{\texttt{x509.checkIssued(otherCert)}}{x509.checkIssued(otherCert)}}\label{x509.checkissuedothercert}

\begin{itemize}
\tightlist
\item
  \texttt{otherCert} \{X509Certificate\}
\item
  Returns: \{boolean\}
\end{itemize}

Checks whether this certificate was issued by the given
\texttt{otherCert}.

\subsubsection{\texorpdfstring{\texttt{x509.checkPrivateKey(privateKey)}}{x509.checkPrivateKey(privateKey)}}\label{x509.checkprivatekeyprivatekey}

\begin{itemize}
\tightlist
\item
  \texttt{privateKey} \{KeyObject\} A private key.
\item
  Returns: \{boolean\}
\end{itemize}

Checks whether the public key for this certificate is consistent with
the given private key.

\subsubsection{\texorpdfstring{\texttt{x509.fingerprint}}{x509.fingerprint}}\label{x509.fingerprint}

\begin{itemize}
\tightlist
\item
  Type: \{string\}
\end{itemize}

The SHA-1 fingerprint of this certificate.

Because SHA-1 is cryptographically broken and because the security of
SHA-1 is significantly worse than that of algorithms that are commonly
used to sign certificates, consider using
\hyperref[x509fingerprint256]{\texttt{x509.fingerprint256}} instead.

\subsubsection{\texorpdfstring{\texttt{x509.fingerprint256}}{x509.fingerprint256}}\label{x509.fingerprint256}

\begin{itemize}
\tightlist
\item
  Type: \{string\}
\end{itemize}

The SHA-256 fingerprint of this certificate.

\subsubsection{\texorpdfstring{\texttt{x509.fingerprint512}}{x509.fingerprint512}}\label{x509.fingerprint512}

\begin{itemize}
\tightlist
\item
  Type: \{string\}
\end{itemize}

The SHA-512 fingerprint of this certificate.

Because computing the SHA-256 fingerprint is usually faster and because
it is only half the size of the SHA-512 fingerprint,
\hyperref[x509fingerprint256]{\texttt{x509.fingerprint256}} may be a
better choice. While SHA-512 presumably provides a higher level of
security in general, the security of SHA-256 matches that of most
algorithms that are commonly used to sign certificates.

\subsubsection{\texorpdfstring{\texttt{x509.infoAccess}}{x509.infoAccess}}\label{x509.infoaccess}

\begin{itemize}
\tightlist
\item
  Type: \{string\}
\end{itemize}

A textual representation of the certificate's authority information
access extension.

This is a line feed separated list of access descriptions. Each line
begins with the access method and the kind of the access location,
followed by a colon and the value associated with the access location.

After the prefix denoting the access method and the kind of the access
location, the remainder of each line might be enclosed in quotes to
indicate that the value is a JSON string literal. For backward
compatibility, Node.js only uses JSON string literals within this
property when necessary to avoid ambiguity. Third-party code should be
prepared to handle both possible entry formats.

\subsubsection{\texorpdfstring{\texttt{x509.issuer}}{x509.issuer}}\label{x509.issuer}

\begin{itemize}
\tightlist
\item
  Type: \{string\}
\end{itemize}

The issuer identification included in this certificate.

\subsubsection{\texorpdfstring{\texttt{x509.issuerCertificate}}{x509.issuerCertificate}}\label{x509.issuercertificate}

\begin{itemize}
\tightlist
\item
  Type: \{X509Certificate\}
\end{itemize}

The issuer certificate or \texttt{undefined} if the issuer certificate
is not available.

\subsubsection{\texorpdfstring{\texttt{x509.extKeyUsage}}{x509.extKeyUsage}}\label{x509.extkeyusage}

\begin{itemize}
\tightlist
\item
  Type: \{string{[}{]}\}
\end{itemize}

An array detailing the key extended usages for this certificate.

\subsubsection{\texorpdfstring{\texttt{x509.publicKey}}{x509.publicKey}}\label{x509.publickey}

\begin{itemize}
\tightlist
\item
  Type: \{KeyObject\}
\end{itemize}

The public key \{KeyObject\} for this certificate.

\subsubsection{\texorpdfstring{\texttt{x509.raw}}{x509.raw}}\label{x509.raw}

\begin{itemize}
\tightlist
\item
  Type: \{Buffer\}
\end{itemize}

A \texttt{Buffer} containing the DER encoding of this certificate.

\subsubsection{\texorpdfstring{\texttt{x509.serialNumber}}{x509.serialNumber}}\label{x509.serialnumber}

\begin{itemize}
\tightlist
\item
  Type: \{string\}
\end{itemize}

The serial number of this certificate.

Serial numbers are assigned by certificate authorities and do not
uniquely identify certificates. Consider using
\hyperref[x509fingerprint256]{\texttt{x509.fingerprint256}} as a unique
identifier instead.

\subsubsection{\texorpdfstring{\texttt{x509.subject}}{x509.subject}}\label{x509.subject}

\begin{itemize}
\tightlist
\item
  Type: \{string\}
\end{itemize}

The complete subject of this certificate.

\subsubsection{\texorpdfstring{\texttt{x509.subjectAltName}}{x509.subjectAltName}}\label{x509.subjectaltname}

\begin{itemize}
\tightlist
\item
  Type: \{string\}
\end{itemize}

The subject alternative name specified for this certificate.

This is a comma-separated list of subject alternative names. Each entry
begins with a string identifying the kind of the subject alternative
name followed by a colon and the value associated with the entry.

Earlier versions of Node.js incorrectly assumed that it is safe to split
this property at the two-character sequence
\texttt{\textquotesingle{},\ \textquotesingle{}} (see
\href{https://cve.mitre.org/cgi-bin/cvename.cgi?name=CVE-2021-44532}{CVE-2021-44532}).
However, both malicious and legitimate certificates can contain subject
alternative names that include this sequence when represented as a
string.

After the prefix denoting the type of the entry, the remainder of each
entry might be enclosed in quotes to indicate that the value is a JSON
string literal. For backward compatibility, Node.js only uses JSON
string literals within this property when necessary to avoid ambiguity.
Third-party code should be prepared to handle both possible entry
formats.

\subsubsection{\texorpdfstring{\texttt{x509.toJSON()}}{x509.toJSON()}}\label{x509.tojson}

\begin{itemize}
\tightlist
\item
  Type: \{string\}
\end{itemize}

There is no standard JSON encoding for X509 certificates. The
\texttt{toJSON()} method returns a string containing the PEM encoded
certificate.

\subsubsection{\texorpdfstring{\texttt{x509.toLegacyObject()}}{x509.toLegacyObject()}}\label{x509.tolegacyobject}

\begin{itemize}
\tightlist
\item
  Type: \{Object\}
\end{itemize}

Returns information about this certificate using the legacy
\href{tls.md\#certificate-object}{certificate object} encoding.

\subsubsection{\texorpdfstring{\texttt{x509.toString()}}{x509.toString()}}\label{x509.tostring}

\begin{itemize}
\tightlist
\item
  Type: \{string\}
\end{itemize}

Returns the PEM-encoded certificate.

\subsubsection{\texorpdfstring{\texttt{x509.validFrom}}{x509.validFrom}}\label{x509.validfrom}

\begin{itemize}
\tightlist
\item
  Type: \{string\}
\end{itemize}

The date/time from which this certificate is considered valid.

\subsubsection{\texorpdfstring{\texttt{x509.validTo}}{x509.validTo}}\label{x509.validto}

\begin{itemize}
\tightlist
\item
  Type: \{string\}
\end{itemize}

The date/time until which this certificate is considered valid.

\subsubsection{\texorpdfstring{\texttt{x509.verify(publicKey)}}{x509.verify(publicKey)}}\label{x509.verifypublickey}

\begin{itemize}
\tightlist
\item
  \texttt{publicKey} \{KeyObject\} A public key.
\item
  Returns: \{boolean\}
\end{itemize}

Verifies that this certificate was signed by the given public key. Does
not perform any other validation checks on the certificate.

\subsection{\texorpdfstring{\texttt{node:crypto} module methods and
properties}{node:crypto module methods and properties}}\label{nodecrypto-module-methods-and-properties}

\subsubsection{\texorpdfstring{\texttt{crypto.constants}}{crypto.constants}}\label{crypto.constants}

\begin{itemize}
\tightlist
\item
  \{Object\}
\end{itemize}

An object containing commonly used constants for crypto and security
related operations. The specific constants currently defined are
described in \hyperref[crypto-constants]{Crypto constants}.

\subsubsection{\texorpdfstring{\texttt{crypto.fips}}{crypto.fips}}\label{crypto.fips}

\begin{quote}
Stability: 0 - Deprecated
\end{quote}

Property for checking and controlling whether a FIPS compliant crypto
provider is currently in use. Setting to true requires a FIPS build of
Node.js.

This property is deprecated. Please use \texttt{crypto.setFips()} and
\texttt{crypto.getFips()} instead.

\subsubsection{\texorpdfstring{\texttt{crypto.checkPrime(candidate{[},\ options{]},\ callback)}}{crypto.checkPrime(candidate{[}, options{]}, callback)}}\label{crypto.checkprimecandidate-options-callback}

\begin{itemize}
\tightlist
\item
  \texttt{candidate}
  \{ArrayBuffer\textbar SharedArrayBuffer\textbar TypedArray\textbar Buffer\textbar DataView\textbar bigint\}
  A possible prime encoded as a sequence of big endian octets of
  arbitrary length.
\item
  \texttt{options} \{Object\}

  \begin{itemize}
  \tightlist
  \item
    \texttt{checks} \{number\} The number of Miller-Rabin probabilistic
    primality iterations to perform. When the value is \texttt{0}
    (zero), a number of checks is used that yields a false positive rate
    of at most 2-64 for random input. Care must be used when selecting a
    number of checks. Refer to the OpenSSL documentation for the
    \href{https://www.openssl.org/docs/man1.1.1/man3/BN_is_prime_ex.html}{\texttt{BN\_is\_prime\_ex}}
    function \texttt{nchecks} options for more details.
    \textbf{Default:} \texttt{0}
  \end{itemize}
\item
  \texttt{callback} \{Function\}

  \begin{itemize}
  \tightlist
  \item
    \texttt{err} \{Error\} Set to an \{Error\} object if an error
    occurred during check.
  \item
    \texttt{result} \{boolean\} \texttt{true} if the candidate is a
    prime with an error probability less than
    \texttt{0.25\ **\ options.checks}.
  \end{itemize}
\end{itemize}

Checks the primality of the \texttt{candidate}.

\subsubsection{\texorpdfstring{\texttt{crypto.checkPrimeSync(candidate{[},\ options{]})}}{crypto.checkPrimeSync(candidate{[}, options{]})}}\label{crypto.checkprimesynccandidate-options}

\begin{itemize}
\tightlist
\item
  \texttt{candidate}
  \{ArrayBuffer\textbar SharedArrayBuffer\textbar TypedArray\textbar Buffer\textbar DataView\textbar bigint\}
  A possible prime encoded as a sequence of big endian octets of
  arbitrary length.
\item
  \texttt{options} \{Object\}

  \begin{itemize}
  \tightlist
  \item
    \texttt{checks} \{number\} The number of Miller-Rabin probabilistic
    primality iterations to perform. When the value is \texttt{0}
    (zero), a number of checks is used that yields a false positive rate
    of at most 2-64 for random input. Care must be used when selecting a
    number of checks. Refer to the OpenSSL documentation for the
    \href{https://www.openssl.org/docs/man1.1.1/man3/BN_is_prime_ex.html}{\texttt{BN\_is\_prime\_ex}}
    function \texttt{nchecks} options for more details.
    \textbf{Default:} \texttt{0}
  \end{itemize}
\item
  Returns: \{boolean\} \texttt{true} if the candidate is a prime with an
  error probability less than \texttt{0.25\ **\ options.checks}.
\end{itemize}

Checks the primality of the \texttt{candidate}.

\subsubsection{\texorpdfstring{\texttt{crypto.createCipheriv(algorithm,\ key,\ iv{[},\ options{]})}}{crypto.createCipheriv(algorithm, key, iv{[}, options{]})}}\label{crypto.createcipherivalgorithm-key-iv-options}

\begin{itemize}
\tightlist
\item
  \texttt{algorithm} \{string\}
\item
  \texttt{key}
  \{string\textbar ArrayBuffer\textbar Buffer\textbar TypedArray\textbar DataView\textbar KeyObject\textbar CryptoKey\}
\item
  \texttt{iv}
  \{string\textbar ArrayBuffer\textbar Buffer\textbar TypedArray\textbar DataView\textbar null\}
\item
  \texttt{options} \{Object\}
  \href{stream.md\#new-streamtransformoptions}{\texttt{stream.transform}
  options}
\item
  Returns: \{Cipher\}
\end{itemize}

Creates and returns a \texttt{Cipher} object, with the given
\texttt{algorithm}, \texttt{key} and initialization vector
(\texttt{iv}).

The \texttt{options} argument controls stream behavior and is optional
except when a cipher in CCM or OCB mode
(e.g.~\texttt{\textquotesingle{}aes-128-ccm\textquotesingle{}}) is used.
In that case, the \texttt{authTagLength} option is required and
specifies the length of the authentication tag in bytes, see
\hyperref[ccm-mode]{CCM mode}. In GCM mode, the \texttt{authTagLength}
option is not required but can be used to set the length of the
authentication tag that will be returned by \texttt{getAuthTag()} and
defaults to 16 bytes. For \texttt{chacha20-poly1305}, the
\texttt{authTagLength} option defaults to 16 bytes.

The \texttt{algorithm} is dependent on OpenSSL, examples are
\texttt{\textquotesingle{}aes192\textquotesingle{}}, etc. On recent
OpenSSL releases, \texttt{openssl\ list\ -cipher-algorithms} will
display the available cipher algorithms.

The \texttt{key} is the raw key used by the \texttt{algorithm} and
\texttt{iv} is an
\href{https://en.wikipedia.org/wiki/Initialization_vector}{initialization
vector}. Both arguments must be
\texttt{\textquotesingle{}utf8\textquotesingle{}} encoded strings,
\href{buffer.md}{Buffers}, \texttt{TypedArray}, or \texttt{DataView}s.
The \texttt{key} may optionally be a
\hyperref[class-keyobject]{\texttt{KeyObject}} of type \texttt{secret}.
If the cipher does not need an initialization vector, \texttt{iv} may be
\texttt{null}.

When passing strings for \texttt{key} or \texttt{iv}, please consider
\hyperref[using-strings-as-inputs-to-cryptographic-apis]{caveats when
using strings as inputs to cryptographic APIs}.

Initialization vectors should be unpredictable and unique; ideally, they
will be cryptographically random. They do not have to be secret: IVs are
typically just added to ciphertext messages unencrypted. It may sound
contradictory that something has to be unpredictable and unique, but
does not have to be secret; remember that an attacker must not be able
to predict ahead of time what a given IV will be.

\subsubsection{\texorpdfstring{\texttt{crypto.createDecipheriv(algorithm,\ key,\ iv{[},\ options{]})}}{crypto.createDecipheriv(algorithm, key, iv{[}, options{]})}}\label{crypto.createdecipherivalgorithm-key-iv-options}

\begin{itemize}
\tightlist
\item
  \texttt{algorithm} \{string\}
\item
  \texttt{key}
  \{string\textbar ArrayBuffer\textbar Buffer\textbar TypedArray\textbar DataView\textbar KeyObject\textbar CryptoKey\}
\item
  \texttt{iv}
  \{string\textbar ArrayBuffer\textbar Buffer\textbar TypedArray\textbar DataView\textbar null\}
\item
  \texttt{options} \{Object\}
  \href{stream.md\#new-streamtransformoptions}{\texttt{stream.transform}
  options}
\item
  Returns: \{Decipher\}
\end{itemize}

Creates and returns a \texttt{Decipher} object that uses the given
\texttt{algorithm}, \texttt{key} and initialization vector
(\texttt{iv}).

The \texttt{options} argument controls stream behavior and is optional
except when a cipher in CCM or OCB mode
(e.g.~\texttt{\textquotesingle{}aes-128-ccm\textquotesingle{}}) is used.
In that case, the \texttt{authTagLength} option is required and
specifies the length of the authentication tag in bytes, see
\hyperref[ccm-mode]{CCM mode}. In GCM mode, the \texttt{authTagLength}
option is not required but can be used to restrict accepted
authentication tags to those with the specified length. For
\texttt{chacha20-poly1305}, the \texttt{authTagLength} option defaults
to 16 bytes.

The \texttt{algorithm} is dependent on OpenSSL, examples are
\texttt{\textquotesingle{}aes192\textquotesingle{}}, etc. On recent
OpenSSL releases, \texttt{openssl\ list\ -cipher-algorithms} will
display the available cipher algorithms.

The \texttt{key} is the raw key used by the \texttt{algorithm} and
\texttt{iv} is an
\href{https://en.wikipedia.org/wiki/Initialization_vector}{initialization
vector}. Both arguments must be
\texttt{\textquotesingle{}utf8\textquotesingle{}} encoded strings,
\href{buffer.md}{Buffers}, \texttt{TypedArray}, or \texttt{DataView}s.
The \texttt{key} may optionally be a
\hyperref[class-keyobject]{\texttt{KeyObject}} of type \texttt{secret}.
If the cipher does not need an initialization vector, \texttt{iv} may be
\texttt{null}.

When passing strings for \texttt{key} or \texttt{iv}, please consider
\hyperref[using-strings-as-inputs-to-cryptographic-apis]{caveats when
using strings as inputs to cryptographic APIs}.

Initialization vectors should be unpredictable and unique; ideally, they
will be cryptographically random. They do not have to be secret: IVs are
typically just added to ciphertext messages unencrypted. It may sound
contradictory that something has to be unpredictable and unique, but
does not have to be secret; remember that an attacker must not be able
to predict ahead of time what a given IV will be.

\subsubsection{\texorpdfstring{\texttt{crypto.createDiffieHellman(prime{[},\ primeEncoding{]}{[},\ generator{]}{[},\ generatorEncoding{]})}}{crypto.createDiffieHellman(prime{[}, primeEncoding{]}{[}, generator{]}{[}, generatorEncoding{]})}}\label{crypto.creatediffiehellmanprime-primeencoding-generator-generatorencoding}

\begin{itemize}
\tightlist
\item
  \texttt{prime}
  \{string\textbar ArrayBuffer\textbar Buffer\textbar TypedArray\textbar DataView\}
\item
  \texttt{primeEncoding} \{string\} The
  \href{buffer.md\#buffers-and-character-encodings}{encoding} of the
  \texttt{prime} string.
\item
  \texttt{generator}
  \{number\textbar string\textbar ArrayBuffer\textbar Buffer\textbar TypedArray\textbar DataView\}
  \textbf{Default:} \texttt{2}
\item
  \texttt{generatorEncoding} \{string\} The
  \href{buffer.md\#buffers-and-character-encodings}{encoding} of the
  \texttt{generator} string.
\item
  Returns: \{DiffieHellman\}
\end{itemize}

Creates a \texttt{DiffieHellman} key exchange object using the supplied
\texttt{prime} and an optional specific \texttt{generator}.

The \texttt{generator} argument can be a number, string, or
\href{buffer.md}{\texttt{Buffer}}. If \texttt{generator} is not
specified, the value \texttt{2} is used.

If \texttt{primeEncoding} is specified, \texttt{prime} is expected to be
a string; otherwise a \href{buffer.md}{\texttt{Buffer}},
\texttt{TypedArray}, or \texttt{DataView} is expected.

If \texttt{generatorEncoding} is specified, \texttt{generator} is
expected to be a string; otherwise a number,
\href{buffer.md}{\texttt{Buffer}}, \texttt{TypedArray}, or
\texttt{DataView} is expected.

\subsubsection{\texorpdfstring{\texttt{crypto.createDiffieHellman(primeLength{[},\ generator{]})}}{crypto.createDiffieHellman(primeLength{[}, generator{]})}}\label{crypto.creatediffiehellmanprimelength-generator}

\begin{itemize}
\tightlist
\item
  \texttt{primeLength} \{number\}
\item
  \texttt{generator} \{number\} \textbf{Default:} \texttt{2}
\item
  Returns: \{DiffieHellman\}
\end{itemize}

Creates a \texttt{DiffieHellman} key exchange object and generates a
prime of \texttt{primeLength} bits using an optional specific numeric
\texttt{generator}. If \texttt{generator} is not specified, the value
\texttt{2} is used.

\subsubsection{\texorpdfstring{\texttt{crypto.createDiffieHellmanGroup(name)}}{crypto.createDiffieHellmanGroup(name)}}\label{crypto.creatediffiehellmangroupname}

\begin{itemize}
\tightlist
\item
  \texttt{name} \{string\}
\item
  Returns: \{DiffieHellmanGroup\}
\end{itemize}

An alias for
\hyperref[cryptogetdiffiehellmangroupname]{\texttt{crypto.getDiffieHellman()}}

\subsubsection{\texorpdfstring{\texttt{crypto.createECDH(curveName)}}{crypto.createECDH(curveName)}}\label{crypto.createecdhcurvename}

\begin{itemize}
\tightlist
\item
  \texttt{curveName} \{string\}
\item
  Returns: \{ECDH\}
\end{itemize}

Creates an Elliptic Curve Diffie-Hellman (\texttt{ECDH}) key exchange
object using a predefined curve specified by the \texttt{curveName}
string. Use \hyperref[cryptogetcurves]{\texttt{crypto.getCurves()}} to
obtain a list of available curve names. On recent OpenSSL releases,
\texttt{openssl\ ecparam\ -list\_curves} will also display the name and
description of each available elliptic curve.

\subsubsection{\texorpdfstring{\texttt{crypto.createHash(algorithm{[},\ options{]})}}{crypto.createHash(algorithm{[}, options{]})}}\label{crypto.createhashalgorithm-options}

\begin{itemize}
\tightlist
\item
  \texttt{algorithm} \{string\}
\item
  \texttt{options} \{Object\}
  \href{stream.md\#new-streamtransformoptions}{\texttt{stream.transform}
  options}
\item
  Returns: \{Hash\}
\end{itemize}

Creates and returns a \texttt{Hash} object that can be used to generate
hash digests using the given \texttt{algorithm}. Optional
\texttt{options} argument controls stream behavior. For XOF hash
functions such as \texttt{\textquotesingle{}shake256\textquotesingle{}},
the \texttt{outputLength} option can be used to specify the desired
output length in bytes.

The \texttt{algorithm} is dependent on the available algorithms
supported by the version of OpenSSL on the platform. Examples are
\texttt{\textquotesingle{}sha256\textquotesingle{}},
\texttt{\textquotesingle{}sha512\textquotesingle{}}, etc. On recent
releases of OpenSSL, \texttt{openssl\ list\ -digest-algorithms} will
display the available digest algorithms.

Example: generating the sha256 sum of a file

\begin{Shaded}
\begin{Highlighting}[]
\ImportTok{import}\NormalTok{ \{}
\NormalTok{  createReadStream}\OperatorTok{,}
\NormalTok{\} }\ImportTok{from} \StringTok{\textquotesingle{}node:fs\textquotesingle{}}\OperatorTok{;}
\ImportTok{import}\NormalTok{ \{ argv \} }\ImportTok{from} \StringTok{\textquotesingle{}node:process\textquotesingle{}}\OperatorTok{;}
\KeywordTok{const}\NormalTok{ \{}
\NormalTok{  createHash}\OperatorTok{,}
\NormalTok{\} }\OperatorTok{=} \ControlFlowTok{await} \ImportTok{import}\NormalTok{(}\StringTok{\textquotesingle{}node:crypto\textquotesingle{}}\NormalTok{)}\OperatorTok{;}

\KeywordTok{const}\NormalTok{ filename }\OperatorTok{=}\NormalTok{ argv[}\DecValTok{2}\NormalTok{]}\OperatorTok{;}

\KeywordTok{const}\NormalTok{ hash }\OperatorTok{=} \FunctionTok{createHash}\NormalTok{(}\StringTok{\textquotesingle{}sha256\textquotesingle{}}\NormalTok{)}\OperatorTok{;}

\KeywordTok{const}\NormalTok{ input }\OperatorTok{=} \FunctionTok{createReadStream}\NormalTok{(filename)}\OperatorTok{;}
\NormalTok{input}\OperatorTok{.}\FunctionTok{on}\NormalTok{(}\StringTok{\textquotesingle{}readable\textquotesingle{}}\OperatorTok{,}\NormalTok{ () }\KeywordTok{=\textgreater{}}\NormalTok{ \{}
  \CommentTok{// Only one element is going to be produced by the}
  \CommentTok{// hash stream.}
  \KeywordTok{const}\NormalTok{ data }\OperatorTok{=}\NormalTok{ input}\OperatorTok{.}\FunctionTok{read}\NormalTok{()}\OperatorTok{;}
  \ControlFlowTok{if}\NormalTok{ (data)}
\NormalTok{    hash}\OperatorTok{.}\FunctionTok{update}\NormalTok{(data)}\OperatorTok{;}
  \ControlFlowTok{else}\NormalTok{ \{}
    \BuiltInTok{console}\OperatorTok{.}\FunctionTok{log}\NormalTok{(}\VerbatimStringTok{\textasciigrave{}}\SpecialCharTok{$\{}\NormalTok{hash}\OperatorTok{.}\FunctionTok{digest}\NormalTok{(}\StringTok{\textquotesingle{}hex\textquotesingle{}}\NormalTok{)}\SpecialCharTok{\}}\VerbatimStringTok{ }\SpecialCharTok{$\{}\NormalTok{filename}\SpecialCharTok{\}}\VerbatimStringTok{\textasciigrave{}}\NormalTok{)}\OperatorTok{;}
\NormalTok{  \}}
\NormalTok{\})}\OperatorTok{;}
\end{Highlighting}
\end{Shaded}

\begin{Shaded}
\begin{Highlighting}[]
\KeywordTok{const}\NormalTok{ \{}
\NormalTok{  createReadStream}\OperatorTok{,}
\NormalTok{\} }\OperatorTok{=} \PreprocessorTok{require}\NormalTok{(}\StringTok{\textquotesingle{}node:fs\textquotesingle{}}\NormalTok{)}\OperatorTok{;}
\KeywordTok{const}\NormalTok{ \{}
\NormalTok{  createHash}\OperatorTok{,}
\NormalTok{\} }\OperatorTok{=} \PreprocessorTok{require}\NormalTok{(}\StringTok{\textquotesingle{}node:crypto\textquotesingle{}}\NormalTok{)}\OperatorTok{;}
\KeywordTok{const}\NormalTok{ \{ argv \} }\OperatorTok{=} \PreprocessorTok{require}\NormalTok{(}\StringTok{\textquotesingle{}node:process\textquotesingle{}}\NormalTok{)}\OperatorTok{;}

\KeywordTok{const}\NormalTok{ filename }\OperatorTok{=}\NormalTok{ argv[}\DecValTok{2}\NormalTok{]}\OperatorTok{;}

\KeywordTok{const}\NormalTok{ hash }\OperatorTok{=} \FunctionTok{createHash}\NormalTok{(}\StringTok{\textquotesingle{}sha256\textquotesingle{}}\NormalTok{)}\OperatorTok{;}

\KeywordTok{const}\NormalTok{ input }\OperatorTok{=} \FunctionTok{createReadStream}\NormalTok{(filename)}\OperatorTok{;}
\NormalTok{input}\OperatorTok{.}\FunctionTok{on}\NormalTok{(}\StringTok{\textquotesingle{}readable\textquotesingle{}}\OperatorTok{,}\NormalTok{ () }\KeywordTok{=\textgreater{}}\NormalTok{ \{}
  \CommentTok{// Only one element is going to be produced by the}
  \CommentTok{// hash stream.}
  \KeywordTok{const}\NormalTok{ data }\OperatorTok{=}\NormalTok{ input}\OperatorTok{.}\FunctionTok{read}\NormalTok{()}\OperatorTok{;}
  \ControlFlowTok{if}\NormalTok{ (data)}
\NormalTok{    hash}\OperatorTok{.}\FunctionTok{update}\NormalTok{(data)}\OperatorTok{;}
  \ControlFlowTok{else}\NormalTok{ \{}
    \BuiltInTok{console}\OperatorTok{.}\FunctionTok{log}\NormalTok{(}\VerbatimStringTok{\textasciigrave{}}\SpecialCharTok{$\{}\NormalTok{hash}\OperatorTok{.}\FunctionTok{digest}\NormalTok{(}\StringTok{\textquotesingle{}hex\textquotesingle{}}\NormalTok{)}\SpecialCharTok{\}}\VerbatimStringTok{ }\SpecialCharTok{$\{}\NormalTok{filename}\SpecialCharTok{\}}\VerbatimStringTok{\textasciigrave{}}\NormalTok{)}\OperatorTok{;}
\NormalTok{  \}}
\NormalTok{\})}\OperatorTok{;}
\end{Highlighting}
\end{Shaded}

\subsubsection{\texorpdfstring{\texttt{crypto.createHmac(algorithm,\ key{[},\ options{]})}}{crypto.createHmac(algorithm, key{[}, options{]})}}\label{crypto.createhmacalgorithm-key-options}

\begin{itemize}
\tightlist
\item
  \texttt{algorithm} \{string\}
\item
  \texttt{key}
  \{string\textbar ArrayBuffer\textbar Buffer\textbar TypedArray\textbar DataView\textbar KeyObject\textbar CryptoKey\}
\item
  \texttt{options} \{Object\}
  \href{stream.md\#new-streamtransformoptions}{\texttt{stream.transform}
  options}

  \begin{itemize}
  \tightlist
  \item
    \texttt{encoding} \{string\} The string encoding to use when
    \texttt{key} is a string.
  \end{itemize}
\item
  Returns: \{Hmac\}
\end{itemize}

Creates and returns an \texttt{Hmac} object that uses the given
\texttt{algorithm} and \texttt{key}. Optional \texttt{options} argument
controls stream behavior.

The \texttt{algorithm} is dependent on the available algorithms
supported by the version of OpenSSL on the platform. Examples are
\texttt{\textquotesingle{}sha256\textquotesingle{}},
\texttt{\textquotesingle{}sha512\textquotesingle{}}, etc. On recent
releases of OpenSSL, \texttt{openssl\ list\ -digest-algorithms} will
display the available digest algorithms.

The \texttt{key} is the HMAC key used to generate the cryptographic HMAC
hash. If it is a \hyperref[class-keyobject]{\texttt{KeyObject}}, its
type must be \texttt{secret}. If it is a string, please consider
\hyperref[using-strings-as-inputs-to-cryptographic-apis]{caveats when
using strings as inputs to cryptographic APIs}. If it was obtained from
a cryptographically secure source of entropy, such as
\hyperref[cryptorandombytessize-callback]{\texttt{crypto.randomBytes()}}
or
\hyperref[cryptogeneratekeytype-options-callback]{\texttt{crypto.generateKey()}},
its length should not exceed the block size of \texttt{algorithm} (e.g.,
512 bits for SHA-256).

Example: generating the sha256 HMAC of a file

\begin{Shaded}
\begin{Highlighting}[]
\ImportTok{import}\NormalTok{ \{}
\NormalTok{  createReadStream}\OperatorTok{,}
\NormalTok{\} }\ImportTok{from} \StringTok{\textquotesingle{}node:fs\textquotesingle{}}\OperatorTok{;}
\ImportTok{import}\NormalTok{ \{ argv \} }\ImportTok{from} \StringTok{\textquotesingle{}node:process\textquotesingle{}}\OperatorTok{;}
\KeywordTok{const}\NormalTok{ \{}
\NormalTok{  createHmac}\OperatorTok{,}
\NormalTok{\} }\OperatorTok{=} \ControlFlowTok{await} \ImportTok{import}\NormalTok{(}\StringTok{\textquotesingle{}node:crypto\textquotesingle{}}\NormalTok{)}\OperatorTok{;}

\KeywordTok{const}\NormalTok{ filename }\OperatorTok{=}\NormalTok{ argv[}\DecValTok{2}\NormalTok{]}\OperatorTok{;}

\KeywordTok{const}\NormalTok{ hmac }\OperatorTok{=} \FunctionTok{createHmac}\NormalTok{(}\StringTok{\textquotesingle{}sha256\textquotesingle{}}\OperatorTok{,} \StringTok{\textquotesingle{}a secret\textquotesingle{}}\NormalTok{)}\OperatorTok{;}

\KeywordTok{const}\NormalTok{ input }\OperatorTok{=} \FunctionTok{createReadStream}\NormalTok{(filename)}\OperatorTok{;}
\NormalTok{input}\OperatorTok{.}\FunctionTok{on}\NormalTok{(}\StringTok{\textquotesingle{}readable\textquotesingle{}}\OperatorTok{,}\NormalTok{ () }\KeywordTok{=\textgreater{}}\NormalTok{ \{}
  \CommentTok{// Only one element is going to be produced by the}
  \CommentTok{// hash stream.}
  \KeywordTok{const}\NormalTok{ data }\OperatorTok{=}\NormalTok{ input}\OperatorTok{.}\FunctionTok{read}\NormalTok{()}\OperatorTok{;}
  \ControlFlowTok{if}\NormalTok{ (data)}
\NormalTok{    hmac}\OperatorTok{.}\FunctionTok{update}\NormalTok{(data)}\OperatorTok{;}
  \ControlFlowTok{else}\NormalTok{ \{}
    \BuiltInTok{console}\OperatorTok{.}\FunctionTok{log}\NormalTok{(}\VerbatimStringTok{\textasciigrave{}}\SpecialCharTok{$\{}\NormalTok{hmac}\OperatorTok{.}\FunctionTok{digest}\NormalTok{(}\StringTok{\textquotesingle{}hex\textquotesingle{}}\NormalTok{)}\SpecialCharTok{\}}\VerbatimStringTok{ }\SpecialCharTok{$\{}\NormalTok{filename}\SpecialCharTok{\}}\VerbatimStringTok{\textasciigrave{}}\NormalTok{)}\OperatorTok{;}
\NormalTok{  \}}
\NormalTok{\})}\OperatorTok{;}
\end{Highlighting}
\end{Shaded}

\begin{Shaded}
\begin{Highlighting}[]
\KeywordTok{const}\NormalTok{ \{}
\NormalTok{  createReadStream}\OperatorTok{,}
\NormalTok{\} }\OperatorTok{=} \PreprocessorTok{require}\NormalTok{(}\StringTok{\textquotesingle{}node:fs\textquotesingle{}}\NormalTok{)}\OperatorTok{;}
\KeywordTok{const}\NormalTok{ \{}
\NormalTok{  createHmac}\OperatorTok{,}
\NormalTok{\} }\OperatorTok{=} \PreprocessorTok{require}\NormalTok{(}\StringTok{\textquotesingle{}node:crypto\textquotesingle{}}\NormalTok{)}\OperatorTok{;}
\KeywordTok{const}\NormalTok{ \{ argv \} }\OperatorTok{=} \PreprocessorTok{require}\NormalTok{(}\StringTok{\textquotesingle{}node:process\textquotesingle{}}\NormalTok{)}\OperatorTok{;}

\KeywordTok{const}\NormalTok{ filename }\OperatorTok{=}\NormalTok{ argv[}\DecValTok{2}\NormalTok{]}\OperatorTok{;}

\KeywordTok{const}\NormalTok{ hmac }\OperatorTok{=} \FunctionTok{createHmac}\NormalTok{(}\StringTok{\textquotesingle{}sha256\textquotesingle{}}\OperatorTok{,} \StringTok{\textquotesingle{}a secret\textquotesingle{}}\NormalTok{)}\OperatorTok{;}

\KeywordTok{const}\NormalTok{ input }\OperatorTok{=} \FunctionTok{createReadStream}\NormalTok{(filename)}\OperatorTok{;}
\NormalTok{input}\OperatorTok{.}\FunctionTok{on}\NormalTok{(}\StringTok{\textquotesingle{}readable\textquotesingle{}}\OperatorTok{,}\NormalTok{ () }\KeywordTok{=\textgreater{}}\NormalTok{ \{}
  \CommentTok{// Only one element is going to be produced by the}
  \CommentTok{// hash stream.}
  \KeywordTok{const}\NormalTok{ data }\OperatorTok{=}\NormalTok{ input}\OperatorTok{.}\FunctionTok{read}\NormalTok{()}\OperatorTok{;}
  \ControlFlowTok{if}\NormalTok{ (data)}
\NormalTok{    hmac}\OperatorTok{.}\FunctionTok{update}\NormalTok{(data)}\OperatorTok{;}
  \ControlFlowTok{else}\NormalTok{ \{}
    \BuiltInTok{console}\OperatorTok{.}\FunctionTok{log}\NormalTok{(}\VerbatimStringTok{\textasciigrave{}}\SpecialCharTok{$\{}\NormalTok{hmac}\OperatorTok{.}\FunctionTok{digest}\NormalTok{(}\StringTok{\textquotesingle{}hex\textquotesingle{}}\NormalTok{)}\SpecialCharTok{\}}\VerbatimStringTok{ }\SpecialCharTok{$\{}\NormalTok{filename}\SpecialCharTok{\}}\VerbatimStringTok{\textasciigrave{}}\NormalTok{)}\OperatorTok{;}
\NormalTok{  \}}
\NormalTok{\})}\OperatorTok{;}
\end{Highlighting}
\end{Shaded}

\subsubsection{\texorpdfstring{\texttt{crypto.createPrivateKey(key)}}{crypto.createPrivateKey(key)}}\label{crypto.createprivatekeykey}

\begin{itemize}
\tightlist
\item
  \texttt{key}
  \{Object\textbar string\textbar ArrayBuffer\textbar Buffer\textbar TypedArray\textbar DataView\}

  \begin{itemize}
  \tightlist
  \item
    \texttt{key}:
    \{string\textbar ArrayBuffer\textbar Buffer\textbar TypedArray\textbar DataView\textbar Object\}
    The key material, either in PEM, DER, or JWK format.
  \item
    \texttt{format}: \{string\} Must be
    \texttt{\textquotesingle{}pem\textquotesingle{}},
    \texttt{\textquotesingle{}der\textquotesingle{}}, or
    '\texttt{\textquotesingle{}jwk\textquotesingle{}}. \textbf{Default:}
    \texttt{\textquotesingle{}pem\textquotesingle{}}.
  \item
    \texttt{type}: \{string\} Must be
    \texttt{\textquotesingle{}pkcs1\textquotesingle{}},
    \texttt{\textquotesingle{}pkcs8\textquotesingle{}} or
    \texttt{\textquotesingle{}sec1\textquotesingle{}}. This option is
    required only if the \texttt{format} is
    \texttt{\textquotesingle{}der\textquotesingle{}} and ignored
    otherwise.
  \item
    \texttt{passphrase}: \{string \textbar{} Buffer\} The passphrase to
    use for decryption.
  \item
    \texttt{encoding}: \{string\} The string encoding to use when
    \texttt{key} is a string.
  \end{itemize}
\item
  Returns: \{KeyObject\}
\end{itemize}

Creates and returns a new key object containing a private key. If
\texttt{key} is a string or \texttt{Buffer}, \texttt{format} is assumed
to be \texttt{\textquotesingle{}pem\textquotesingle{}}; otherwise,
\texttt{key} must be an object with the properties described above.

If the private key is encrypted, a \texttt{passphrase} must be
specified. The length of the passphrase is limited to 1024 bytes.

\subsubsection{\texorpdfstring{\texttt{crypto.createPublicKey(key)}}{crypto.createPublicKey(key)}}\label{crypto.createpublickeykey}

\begin{itemize}
\tightlist
\item
  \texttt{key}
  \{Object\textbar string\textbar ArrayBuffer\textbar Buffer\textbar TypedArray\textbar DataView\}

  \begin{itemize}
  \tightlist
  \item
    \texttt{key}:
    \{string\textbar ArrayBuffer\textbar Buffer\textbar TypedArray\textbar DataView\textbar Object\}
    The key material, either in PEM, DER, or JWK format.
  \item
    \texttt{format}: \{string\} Must be
    \texttt{\textquotesingle{}pem\textquotesingle{}},
    \texttt{\textquotesingle{}der\textquotesingle{}}, or
    \texttt{\textquotesingle{}jwk\textquotesingle{}}. \textbf{Default:}
    \texttt{\textquotesingle{}pem\textquotesingle{}}.
  \item
    \texttt{type}: \{string\} Must be
    \texttt{\textquotesingle{}pkcs1\textquotesingle{}} or
    \texttt{\textquotesingle{}spki\textquotesingle{}}. This option is
    required only if the \texttt{format} is
    \texttt{\textquotesingle{}der\textquotesingle{}} and ignored
    otherwise.
  \item
    \texttt{encoding} \{string\} The string encoding to use when
    \texttt{key} is a string.
  \end{itemize}
\item
  Returns: \{KeyObject\}
\end{itemize}

Creates and returns a new key object containing a public key. If
\texttt{key} is a string or \texttt{Buffer}, \texttt{format} is assumed
to be \texttt{\textquotesingle{}pem\textquotesingle{}}; if \texttt{key}
is a \texttt{KeyObject} with type
\texttt{\textquotesingle{}private\textquotesingle{}}, the public key is
derived from the given private key; otherwise, \texttt{key} must be an
object with the properties described above.

If the format is \texttt{\textquotesingle{}pem\textquotesingle{}}, the
\texttt{\textquotesingle{}key\textquotesingle{}} may also be an X.509
certificate.

Because public keys can be derived from private keys, a private key may
be passed instead of a public key. In that case, this function behaves
as if
\hyperref[cryptocreateprivatekeykey]{\texttt{crypto.createPrivateKey()}}
had been called, except that the type of the returned \texttt{KeyObject}
will be \texttt{\textquotesingle{}public\textquotesingle{}} and that the
private key cannot be extracted from the returned \texttt{KeyObject}.
Similarly, if a \texttt{KeyObject} with type
\texttt{\textquotesingle{}private\textquotesingle{}} is given, a new
\texttt{KeyObject} with type
\texttt{\textquotesingle{}public\textquotesingle{}} will be returned and
it will be impossible to extract the private key from the returned
object.

\subsubsection{\texorpdfstring{\texttt{crypto.createSecretKey(key{[},\ encoding{]})}}{crypto.createSecretKey(key{[}, encoding{]})}}\label{crypto.createsecretkeykey-encoding}

\begin{itemize}
\tightlist
\item
  \texttt{key}
  \{string\textbar ArrayBuffer\textbar Buffer\textbar TypedArray\textbar DataView\}
\item
  \texttt{encoding} \{string\} The string encoding when \texttt{key} is
  a string.
\item
  Returns: \{KeyObject\}
\end{itemize}

Creates and returns a new key object containing a secret key for
symmetric encryption or \texttt{Hmac}.

\subsubsection{\texorpdfstring{\texttt{crypto.createSign(algorithm{[},\ options{]})}}{crypto.createSign(algorithm{[}, options{]})}}\label{crypto.createsignalgorithm-options}

\begin{itemize}
\tightlist
\item
  \texttt{algorithm} \{string\}
\item
  \texttt{options} \{Object\}
  \href{stream.md\#new-streamwritableoptions}{\texttt{stream.Writable}
  options}
\item
  Returns: \{Sign\}
\end{itemize}

Creates and returns a \texttt{Sign} object that uses the given
\texttt{algorithm}. Use
\hyperref[cryptogethashes]{\texttt{crypto.getHashes()}} to obtain the
names of the available digest algorithms. Optional \texttt{options}
argument controls the \texttt{stream.Writable} behavior.

In some cases, a \texttt{Sign} instance can be created using the name of
a signature algorithm, such as
\texttt{\textquotesingle{}RSA-SHA256\textquotesingle{}}, instead of a
digest algorithm. This will use the corresponding digest algorithm. This
does not work for all signature algorithms, such as
\texttt{\textquotesingle{}ecdsa-with-SHA256\textquotesingle{}}, so it is
best to always use digest algorithm names.

\subsubsection{\texorpdfstring{\texttt{crypto.createVerify(algorithm{[},\ options{]})}}{crypto.createVerify(algorithm{[}, options{]})}}\label{crypto.createverifyalgorithm-options}

\begin{itemize}
\tightlist
\item
  \texttt{algorithm} \{string\}
\item
  \texttt{options} \{Object\}
  \href{stream.md\#new-streamwritableoptions}{\texttt{stream.Writable}
  options}
\item
  Returns: \{Verify\}
\end{itemize}

Creates and returns a \texttt{Verify} object that uses the given
algorithm. Use \hyperref[cryptogethashes]{\texttt{crypto.getHashes()}}
to obtain an array of names of the available signing algorithms.
Optional \texttt{options} argument controls the \texttt{stream.Writable}
behavior.

In some cases, a \texttt{Verify} instance can be created using the name
of a signature algorithm, such as
\texttt{\textquotesingle{}RSA-SHA256\textquotesingle{}}, instead of a
digest algorithm. This will use the corresponding digest algorithm. This
does not work for all signature algorithms, such as
\texttt{\textquotesingle{}ecdsa-with-SHA256\textquotesingle{}}, so it is
best to always use digest algorithm names.

\subsubsection{\texorpdfstring{\texttt{crypto.diffieHellman(options)}}{crypto.diffieHellman(options)}}\label{crypto.diffiehellmanoptions}

\begin{itemize}
\tightlist
\item
  \texttt{options}: \{Object\}

  \begin{itemize}
  \tightlist
  \item
    \texttt{privateKey}: \{KeyObject\}
  \item
    \texttt{publicKey}: \{KeyObject\}
  \end{itemize}
\item
  Returns: \{Buffer\}
\end{itemize}

Computes the Diffie-Hellman secret based on a \texttt{privateKey} and a
\texttt{publicKey}. Both keys must have the same
\texttt{asymmetricKeyType}, which must be one of
\texttt{\textquotesingle{}dh\textquotesingle{}} (for Diffie-Hellman),
\texttt{\textquotesingle{}ec\textquotesingle{}} (for ECDH),
\texttt{\textquotesingle{}x448\textquotesingle{}}, or
\texttt{\textquotesingle{}x25519\textquotesingle{}} (for ECDH-ES).

\subsubsection{\texorpdfstring{\texttt{crypto.generateKey(type,\ options,\ callback)}}{crypto.generateKey(type, options, callback)}}\label{crypto.generatekeytype-options-callback}

\begin{itemize}
\tightlist
\item
  \texttt{type}: \{string\} The intended use of the generated secret
  key. Currently accepted values are
  \texttt{\textquotesingle{}hmac\textquotesingle{}} and
  \texttt{\textquotesingle{}aes\textquotesingle{}}.
\item
  \texttt{options}: \{Object\}

  \begin{itemize}
  \tightlist
  \item
    \texttt{length}: \{number\} The bit length of the key to generate.
    This must be a value greater than 0.

    \begin{itemize}
    \tightlist
    \item
      If \texttt{type} is
      \texttt{\textquotesingle{}hmac\textquotesingle{}}, the minimum is
      8, and the maximum length is 231-1. If the value is not a multiple
      of 8, the generated key will be truncated to
      \texttt{Math.floor(length\ /\ 8)}.
    \item
      If \texttt{type} is
      \texttt{\textquotesingle{}aes\textquotesingle{}}, the length must
      be one of \texttt{128}, \texttt{192}, or \texttt{256}.
    \end{itemize}
  \end{itemize}
\item
  \texttt{callback}: \{Function\}

  \begin{itemize}
  \tightlist
  \item
    \texttt{err}: \{Error\}
  \item
    \texttt{key}: \{KeyObject\}
  \end{itemize}
\end{itemize}

Asynchronously generates a new random secret key of the given
\texttt{length}. The \texttt{type} will determine which validations will
be performed on the \texttt{length}.

\begin{Shaded}
\begin{Highlighting}[]
\KeywordTok{const}\NormalTok{ \{}
\NormalTok{  generateKey}\OperatorTok{,}
\NormalTok{\} }\OperatorTok{=} \ControlFlowTok{await} \ImportTok{import}\NormalTok{(}\StringTok{\textquotesingle{}node:crypto\textquotesingle{}}\NormalTok{)}\OperatorTok{;}

\FunctionTok{generateKey}\NormalTok{(}\StringTok{\textquotesingle{}hmac\textquotesingle{}}\OperatorTok{,}\NormalTok{ \{ }\DataTypeTok{length}\OperatorTok{:} \DecValTok{512}\NormalTok{ \}}\OperatorTok{,}\NormalTok{ (err}\OperatorTok{,}\NormalTok{ key) }\KeywordTok{=\textgreater{}}\NormalTok{ \{}
  \ControlFlowTok{if}\NormalTok{ (err) }\ControlFlowTok{throw}\NormalTok{ err}\OperatorTok{;}
  \BuiltInTok{console}\OperatorTok{.}\FunctionTok{log}\NormalTok{(key}\OperatorTok{.}\FunctionTok{export}\NormalTok{()}\OperatorTok{.}\FunctionTok{toString}\NormalTok{(}\StringTok{\textquotesingle{}hex\textquotesingle{}}\NormalTok{))}\OperatorTok{;}  \CommentTok{// 46e..........620}
\NormalTok{\})}\OperatorTok{;}
\end{Highlighting}
\end{Shaded}

\begin{Shaded}
\begin{Highlighting}[]
\KeywordTok{const}\NormalTok{ \{}
\NormalTok{  generateKey}\OperatorTok{,}
\NormalTok{\} }\OperatorTok{=} \PreprocessorTok{require}\NormalTok{(}\StringTok{\textquotesingle{}node:crypto\textquotesingle{}}\NormalTok{)}\OperatorTok{;}

\FunctionTok{generateKey}\NormalTok{(}\StringTok{\textquotesingle{}hmac\textquotesingle{}}\OperatorTok{,}\NormalTok{ \{ }\DataTypeTok{length}\OperatorTok{:} \DecValTok{512}\NormalTok{ \}}\OperatorTok{,}\NormalTok{ (err}\OperatorTok{,}\NormalTok{ key) }\KeywordTok{=\textgreater{}}\NormalTok{ \{}
  \ControlFlowTok{if}\NormalTok{ (err) }\ControlFlowTok{throw}\NormalTok{ err}\OperatorTok{;}
  \BuiltInTok{console}\OperatorTok{.}\FunctionTok{log}\NormalTok{(key}\OperatorTok{.}\FunctionTok{export}\NormalTok{()}\OperatorTok{.}\FunctionTok{toString}\NormalTok{(}\StringTok{\textquotesingle{}hex\textquotesingle{}}\NormalTok{))}\OperatorTok{;}  \CommentTok{// 46e..........620}
\NormalTok{\})}\OperatorTok{;}
\end{Highlighting}
\end{Shaded}

The size of a generated HMAC key should not exceed the block size of the
underlying hash function. See
\hyperref[cryptocreatehmacalgorithm-key-options]{\texttt{crypto.createHmac()}}
for more information.

\subsubsection{\texorpdfstring{\texttt{crypto.generateKeyPair(type,\ options,\ callback)}}{crypto.generateKeyPair(type, options, callback)}}\label{crypto.generatekeypairtype-options-callback}

\begin{itemize}
\tightlist
\item
  \texttt{type}: \{string\} Must be
  \texttt{\textquotesingle{}rsa\textquotesingle{}},
  \texttt{\textquotesingle{}rsa-pss\textquotesingle{}},
  \texttt{\textquotesingle{}dsa\textquotesingle{}},
  \texttt{\textquotesingle{}ec\textquotesingle{}},
  \texttt{\textquotesingle{}ed25519\textquotesingle{}},
  \texttt{\textquotesingle{}ed448\textquotesingle{}},
  \texttt{\textquotesingle{}x25519\textquotesingle{}},
  \texttt{\textquotesingle{}x448\textquotesingle{}}, or
  \texttt{\textquotesingle{}dh\textquotesingle{}}.
\item
  \texttt{options}: \{Object\}

  \begin{itemize}
  \tightlist
  \item
    \texttt{modulusLength}: \{number\} Key size in bits (RSA, DSA).
  \item
    \texttt{publicExponent}: \{number\} Public exponent (RSA).
    \textbf{Default:} \texttt{0x10001}.
  \item
    \texttt{hashAlgorithm}: \{string\} Name of the message digest
    (RSA-PSS).
  \item
    \texttt{mgf1HashAlgorithm}: \{string\} Name of the message digest
    used by MGF1 (RSA-PSS).
  \item
    \texttt{saltLength}: \{number\} Minimal salt length in bytes
    (RSA-PSS).
  \item
    \texttt{divisorLength}: \{number\} Size of \texttt{q} in bits (DSA).
  \item
    \texttt{namedCurve}: \{string\} Name of the curve to use (EC).
  \item
    \texttt{prime}: \{Buffer\} The prime parameter (DH).
  \item
    \texttt{primeLength}: \{number\} Prime length in bits (DH).
  \item
    \texttt{generator}: \{number\} Custom generator (DH).
    \textbf{Default:} \texttt{2}.
  \item
    \texttt{groupName}: \{string\} Diffie-Hellman group name (DH). See
    \hyperref[cryptogetdiffiehellmangroupname]{\texttt{crypto.getDiffieHellman()}}.
  \item
    \texttt{paramEncoding}: \{string\} Must be
    \texttt{\textquotesingle{}named\textquotesingle{}} or
    \texttt{\textquotesingle{}explicit\textquotesingle{}} (EC).
    \textbf{Default:}
    \texttt{\textquotesingle{}named\textquotesingle{}}.
  \item
    \texttt{publicKeyEncoding}: \{Object\} See
    \hyperref[keyobjectexportoptions]{\texttt{keyObject.export()}}.
  \item
    \texttt{privateKeyEncoding}: \{Object\} See
    \hyperref[keyobjectexportoptions]{\texttt{keyObject.export()}}.
  \end{itemize}
\item
  \texttt{callback}: \{Function\}

  \begin{itemize}
  \tightlist
  \item
    \texttt{err}: \{Error\}
  \item
    \texttt{publicKey}: \{string \textbar{} Buffer \textbar{}
    KeyObject\}
  \item
    \texttt{privateKey}: \{string \textbar{} Buffer \textbar{}
    KeyObject\}
  \end{itemize}
\end{itemize}

Generates a new asymmetric key pair of the given \texttt{type}. RSA,
RSA-PSS, DSA, EC, Ed25519, Ed448, X25519, X448, and DH are currently
supported.

If a \texttt{publicKeyEncoding} or \texttt{privateKeyEncoding} was
specified, this function behaves as if
\hyperref[keyobjectexportoptions]{\texttt{keyObject.export()}} had been
called on its result. Otherwise, the respective part of the key is
returned as a \hyperref[class-keyobject]{\texttt{KeyObject}}.

It is recommended to encode public keys as
\texttt{\textquotesingle{}spki\textquotesingle{}} and private keys as
\texttt{\textquotesingle{}pkcs8\textquotesingle{}} with encryption for
long-term storage:

\begin{Shaded}
\begin{Highlighting}[]
\KeywordTok{const}\NormalTok{ \{}
\NormalTok{  generateKeyPair}\OperatorTok{,}
\NormalTok{\} }\OperatorTok{=} \ControlFlowTok{await} \ImportTok{import}\NormalTok{(}\StringTok{\textquotesingle{}node:crypto\textquotesingle{}}\NormalTok{)}\OperatorTok{;}

\FunctionTok{generateKeyPair}\NormalTok{(}\StringTok{\textquotesingle{}rsa\textquotesingle{}}\OperatorTok{,}\NormalTok{ \{}
  \DataTypeTok{modulusLength}\OperatorTok{:} \DecValTok{4096}\OperatorTok{,}
  \DataTypeTok{publicKeyEncoding}\OperatorTok{:}\NormalTok{ \{}
    \DataTypeTok{type}\OperatorTok{:} \StringTok{\textquotesingle{}spki\textquotesingle{}}\OperatorTok{,}
    \DataTypeTok{format}\OperatorTok{:} \StringTok{\textquotesingle{}pem\textquotesingle{}}\OperatorTok{,}
\NormalTok{  \}}\OperatorTok{,}
  \DataTypeTok{privateKeyEncoding}\OperatorTok{:}\NormalTok{ \{}
    \DataTypeTok{type}\OperatorTok{:} \StringTok{\textquotesingle{}pkcs8\textquotesingle{}}\OperatorTok{,}
    \DataTypeTok{format}\OperatorTok{:} \StringTok{\textquotesingle{}pem\textquotesingle{}}\OperatorTok{,}
    \DataTypeTok{cipher}\OperatorTok{:} \StringTok{\textquotesingle{}aes{-}256{-}cbc\textquotesingle{}}\OperatorTok{,}
    \DataTypeTok{passphrase}\OperatorTok{:} \StringTok{\textquotesingle{}top secret\textquotesingle{}}\OperatorTok{,}
\NormalTok{  \}}\OperatorTok{,}
\NormalTok{\}}\OperatorTok{,}\NormalTok{ (err}\OperatorTok{,}\NormalTok{ publicKey}\OperatorTok{,}\NormalTok{ privateKey) }\KeywordTok{=\textgreater{}}\NormalTok{ \{}
  \CommentTok{// Handle errors and use the generated key pair.}
\NormalTok{\})}\OperatorTok{;}
\end{Highlighting}
\end{Shaded}

\begin{Shaded}
\begin{Highlighting}[]
\KeywordTok{const}\NormalTok{ \{}
\NormalTok{  generateKeyPair}\OperatorTok{,}
\NormalTok{\} }\OperatorTok{=} \PreprocessorTok{require}\NormalTok{(}\StringTok{\textquotesingle{}node:crypto\textquotesingle{}}\NormalTok{)}\OperatorTok{;}

\FunctionTok{generateKeyPair}\NormalTok{(}\StringTok{\textquotesingle{}rsa\textquotesingle{}}\OperatorTok{,}\NormalTok{ \{}
  \DataTypeTok{modulusLength}\OperatorTok{:} \DecValTok{4096}\OperatorTok{,}
  \DataTypeTok{publicKeyEncoding}\OperatorTok{:}\NormalTok{ \{}
    \DataTypeTok{type}\OperatorTok{:} \StringTok{\textquotesingle{}spki\textquotesingle{}}\OperatorTok{,}
    \DataTypeTok{format}\OperatorTok{:} \StringTok{\textquotesingle{}pem\textquotesingle{}}\OperatorTok{,}
\NormalTok{  \}}\OperatorTok{,}
  \DataTypeTok{privateKeyEncoding}\OperatorTok{:}\NormalTok{ \{}
    \DataTypeTok{type}\OperatorTok{:} \StringTok{\textquotesingle{}pkcs8\textquotesingle{}}\OperatorTok{,}
    \DataTypeTok{format}\OperatorTok{:} \StringTok{\textquotesingle{}pem\textquotesingle{}}\OperatorTok{,}
    \DataTypeTok{cipher}\OperatorTok{:} \StringTok{\textquotesingle{}aes{-}256{-}cbc\textquotesingle{}}\OperatorTok{,}
    \DataTypeTok{passphrase}\OperatorTok{:} \StringTok{\textquotesingle{}top secret\textquotesingle{}}\OperatorTok{,}
\NormalTok{  \}}\OperatorTok{,}
\NormalTok{\}}\OperatorTok{,}\NormalTok{ (err}\OperatorTok{,}\NormalTok{ publicKey}\OperatorTok{,}\NormalTok{ privateKey) }\KeywordTok{=\textgreater{}}\NormalTok{ \{}
  \CommentTok{// Handle errors and use the generated key pair.}
\NormalTok{\})}\OperatorTok{;}
\end{Highlighting}
\end{Shaded}

On completion, \texttt{callback} will be called with \texttt{err} set to
\texttt{undefined} and \texttt{publicKey} / \texttt{privateKey}
representing the generated key pair.

If this method is invoked as its
\href{util.md\#utilpromisifyoriginal}{\texttt{util.promisify()}}ed
version, it returns a \texttt{Promise} for an \texttt{Object} with
\texttt{publicKey} and \texttt{privateKey} properties.

\subsubsection{\texorpdfstring{\texttt{crypto.generateKeyPairSync(type,\ options)}}{crypto.generateKeyPairSync(type, options)}}\label{crypto.generatekeypairsynctype-options}

\begin{itemize}
\tightlist
\item
  \texttt{type}: \{string\} Must be
  \texttt{\textquotesingle{}rsa\textquotesingle{}},
  \texttt{\textquotesingle{}rsa-pss\textquotesingle{}},
  \texttt{\textquotesingle{}dsa\textquotesingle{}},
  \texttt{\textquotesingle{}ec\textquotesingle{}},
  \texttt{\textquotesingle{}ed25519\textquotesingle{}},
  \texttt{\textquotesingle{}ed448\textquotesingle{}},
  \texttt{\textquotesingle{}x25519\textquotesingle{}},
  \texttt{\textquotesingle{}x448\textquotesingle{}}, or
  \texttt{\textquotesingle{}dh\textquotesingle{}}.
\item
  \texttt{options}: \{Object\}

  \begin{itemize}
  \tightlist
  \item
    \texttt{modulusLength}: \{number\} Key size in bits (RSA, DSA).
  \item
    \texttt{publicExponent}: \{number\} Public exponent (RSA).
    \textbf{Default:} \texttt{0x10001}.
  \item
    \texttt{hashAlgorithm}: \{string\} Name of the message digest
    (RSA-PSS).
  \item
    \texttt{mgf1HashAlgorithm}: \{string\} Name of the message digest
    used by MGF1 (RSA-PSS).
  \item
    \texttt{saltLength}: \{number\} Minimal salt length in bytes
    (RSA-PSS).
  \item
    \texttt{divisorLength}: \{number\} Size of \texttt{q} in bits (DSA).
  \item
    \texttt{namedCurve}: \{string\} Name of the curve to use (EC).
  \item
    \texttt{prime}: \{Buffer\} The prime parameter (DH).
  \item
    \texttt{primeLength}: \{number\} Prime length in bits (DH).
  \item
    \texttt{generator}: \{number\} Custom generator (DH).
    \textbf{Default:} \texttt{2}.
  \item
    \texttt{groupName}: \{string\} Diffie-Hellman group name (DH). See
    \hyperref[cryptogetdiffiehellmangroupname]{\texttt{crypto.getDiffieHellman()}}.
  \item
    \texttt{paramEncoding}: \{string\} Must be
    \texttt{\textquotesingle{}named\textquotesingle{}} or
    \texttt{\textquotesingle{}explicit\textquotesingle{}} (EC).
    \textbf{Default:}
    \texttt{\textquotesingle{}named\textquotesingle{}}.
  \item
    \texttt{publicKeyEncoding}: \{Object\} See
    \hyperref[keyobjectexportoptions]{\texttt{keyObject.export()}}.
  \item
    \texttt{privateKeyEncoding}: \{Object\} See
    \hyperref[keyobjectexportoptions]{\texttt{keyObject.export()}}.
  \end{itemize}
\item
  Returns: \{Object\}

  \begin{itemize}
  \tightlist
  \item
    \texttt{publicKey}: \{string \textbar{} Buffer \textbar{}
    KeyObject\}
  \item
    \texttt{privateKey}: \{string \textbar{} Buffer \textbar{}
    KeyObject\}
  \end{itemize}
\end{itemize}

Generates a new asymmetric key pair of the given \texttt{type}. RSA,
RSA-PSS, DSA, EC, Ed25519, Ed448, X25519, X448, and DH are currently
supported.

If a \texttt{publicKeyEncoding} or \texttt{privateKeyEncoding} was
specified, this function behaves as if
\hyperref[keyobjectexportoptions]{\texttt{keyObject.export()}} had been
called on its result. Otherwise, the respective part of the key is
returned as a \hyperref[class-keyobject]{\texttt{KeyObject}}.

When encoding public keys, it is recommended to use
\texttt{\textquotesingle{}spki\textquotesingle{}}. When encoding private
keys, it is recommended to use
\texttt{\textquotesingle{}pkcs8\textquotesingle{}} with a strong
passphrase, and to keep the passphrase confidential.

\begin{Shaded}
\begin{Highlighting}[]
\KeywordTok{const}\NormalTok{ \{}
\NormalTok{  generateKeyPairSync}\OperatorTok{,}
\NormalTok{\} }\OperatorTok{=} \ControlFlowTok{await} \ImportTok{import}\NormalTok{(}\StringTok{\textquotesingle{}node:crypto\textquotesingle{}}\NormalTok{)}\OperatorTok{;}

\KeywordTok{const}\NormalTok{ \{}
\NormalTok{  publicKey}\OperatorTok{,}
\NormalTok{  privateKey}\OperatorTok{,}
\NormalTok{\} }\OperatorTok{=} \FunctionTok{generateKeyPairSync}\NormalTok{(}\StringTok{\textquotesingle{}rsa\textquotesingle{}}\OperatorTok{,}\NormalTok{ \{}
  \DataTypeTok{modulusLength}\OperatorTok{:} \DecValTok{4096}\OperatorTok{,}
  \DataTypeTok{publicKeyEncoding}\OperatorTok{:}\NormalTok{ \{}
    \DataTypeTok{type}\OperatorTok{:} \StringTok{\textquotesingle{}spki\textquotesingle{}}\OperatorTok{,}
    \DataTypeTok{format}\OperatorTok{:} \StringTok{\textquotesingle{}pem\textquotesingle{}}\OperatorTok{,}
\NormalTok{  \}}\OperatorTok{,}
  \DataTypeTok{privateKeyEncoding}\OperatorTok{:}\NormalTok{ \{}
    \DataTypeTok{type}\OperatorTok{:} \StringTok{\textquotesingle{}pkcs8\textquotesingle{}}\OperatorTok{,}
    \DataTypeTok{format}\OperatorTok{:} \StringTok{\textquotesingle{}pem\textquotesingle{}}\OperatorTok{,}
    \DataTypeTok{cipher}\OperatorTok{:} \StringTok{\textquotesingle{}aes{-}256{-}cbc\textquotesingle{}}\OperatorTok{,}
    \DataTypeTok{passphrase}\OperatorTok{:} \StringTok{\textquotesingle{}top secret\textquotesingle{}}\OperatorTok{,}
\NormalTok{  \}}\OperatorTok{,}
\NormalTok{\})}\OperatorTok{;}
\end{Highlighting}
\end{Shaded}

\begin{Shaded}
\begin{Highlighting}[]
\KeywordTok{const}\NormalTok{ \{}
\NormalTok{  generateKeyPairSync}\OperatorTok{,}
\NormalTok{\} }\OperatorTok{=} \PreprocessorTok{require}\NormalTok{(}\StringTok{\textquotesingle{}node:crypto\textquotesingle{}}\NormalTok{)}\OperatorTok{;}

\KeywordTok{const}\NormalTok{ \{}
\NormalTok{  publicKey}\OperatorTok{,}
\NormalTok{  privateKey}\OperatorTok{,}
\NormalTok{\} }\OperatorTok{=} \FunctionTok{generateKeyPairSync}\NormalTok{(}\StringTok{\textquotesingle{}rsa\textquotesingle{}}\OperatorTok{,}\NormalTok{ \{}
  \DataTypeTok{modulusLength}\OperatorTok{:} \DecValTok{4096}\OperatorTok{,}
  \DataTypeTok{publicKeyEncoding}\OperatorTok{:}\NormalTok{ \{}
    \DataTypeTok{type}\OperatorTok{:} \StringTok{\textquotesingle{}spki\textquotesingle{}}\OperatorTok{,}
    \DataTypeTok{format}\OperatorTok{:} \StringTok{\textquotesingle{}pem\textquotesingle{}}\OperatorTok{,}
\NormalTok{  \}}\OperatorTok{,}
  \DataTypeTok{privateKeyEncoding}\OperatorTok{:}\NormalTok{ \{}
    \DataTypeTok{type}\OperatorTok{:} \StringTok{\textquotesingle{}pkcs8\textquotesingle{}}\OperatorTok{,}
    \DataTypeTok{format}\OperatorTok{:} \StringTok{\textquotesingle{}pem\textquotesingle{}}\OperatorTok{,}
    \DataTypeTok{cipher}\OperatorTok{:} \StringTok{\textquotesingle{}aes{-}256{-}cbc\textquotesingle{}}\OperatorTok{,}
    \DataTypeTok{passphrase}\OperatorTok{:} \StringTok{\textquotesingle{}top secret\textquotesingle{}}\OperatorTok{,}
\NormalTok{  \}}\OperatorTok{,}
\NormalTok{\})}\OperatorTok{;}
\end{Highlighting}
\end{Shaded}

The return value \texttt{\{\ publicKey,\ privateKey\ \}} represents the
generated key pair. When PEM encoding was selected, the respective key
will be a string, otherwise it will be a buffer containing the data
encoded as DER.

\subsubsection{\texorpdfstring{\texttt{crypto.generateKeySync(type,\ options)}}{crypto.generateKeySync(type, options)}}\label{crypto.generatekeysynctype-options}

\begin{itemize}
\tightlist
\item
  \texttt{type}: \{string\} The intended use of the generated secret
  key. Currently accepted values are
  \texttt{\textquotesingle{}hmac\textquotesingle{}} and
  \texttt{\textquotesingle{}aes\textquotesingle{}}.
\item
  \texttt{options}: \{Object\}

  \begin{itemize}
  \tightlist
  \item
    \texttt{length}: \{number\} The bit length of the key to generate.

    \begin{itemize}
    \tightlist
    \item
      If \texttt{type} is
      \texttt{\textquotesingle{}hmac\textquotesingle{}}, the minimum is
      8, and the maximum length is 231-1. If the value is not a multiple
      of 8, the generated key will be truncated to
      \texttt{Math.floor(length\ /\ 8)}.
    \item
      If \texttt{type} is
      \texttt{\textquotesingle{}aes\textquotesingle{}}, the length must
      be one of \texttt{128}, \texttt{192}, or \texttt{256}.
    \end{itemize}
  \end{itemize}
\item
  Returns: \{KeyObject\}
\end{itemize}

Synchronously generates a new random secret key of the given
\texttt{length}. The \texttt{type} will determine which validations will
be performed on the \texttt{length}.

\begin{Shaded}
\begin{Highlighting}[]
\KeywordTok{const}\NormalTok{ \{}
\NormalTok{  generateKeySync}\OperatorTok{,}
\NormalTok{\} }\OperatorTok{=} \ControlFlowTok{await} \ImportTok{import}\NormalTok{(}\StringTok{\textquotesingle{}node:crypto\textquotesingle{}}\NormalTok{)}\OperatorTok{;}

\KeywordTok{const}\NormalTok{ key }\OperatorTok{=} \FunctionTok{generateKeySync}\NormalTok{(}\StringTok{\textquotesingle{}hmac\textquotesingle{}}\OperatorTok{,}\NormalTok{ \{ }\DataTypeTok{length}\OperatorTok{:} \DecValTok{512}\NormalTok{ \})}\OperatorTok{;}
\BuiltInTok{console}\OperatorTok{.}\FunctionTok{log}\NormalTok{(key}\OperatorTok{.}\FunctionTok{export}\NormalTok{()}\OperatorTok{.}\FunctionTok{toString}\NormalTok{(}\StringTok{\textquotesingle{}hex\textquotesingle{}}\NormalTok{))}\OperatorTok{;}  \CommentTok{// e89..........41e}
\end{Highlighting}
\end{Shaded}

\begin{Shaded}
\begin{Highlighting}[]
\KeywordTok{const}\NormalTok{ \{}
\NormalTok{  generateKeySync}\OperatorTok{,}
\NormalTok{\} }\OperatorTok{=} \PreprocessorTok{require}\NormalTok{(}\StringTok{\textquotesingle{}node:crypto\textquotesingle{}}\NormalTok{)}\OperatorTok{;}

\KeywordTok{const}\NormalTok{ key }\OperatorTok{=} \FunctionTok{generateKeySync}\NormalTok{(}\StringTok{\textquotesingle{}hmac\textquotesingle{}}\OperatorTok{,}\NormalTok{ \{ }\DataTypeTok{length}\OperatorTok{:} \DecValTok{512}\NormalTok{ \})}\OperatorTok{;}
\BuiltInTok{console}\OperatorTok{.}\FunctionTok{log}\NormalTok{(key}\OperatorTok{.}\FunctionTok{export}\NormalTok{()}\OperatorTok{.}\FunctionTok{toString}\NormalTok{(}\StringTok{\textquotesingle{}hex\textquotesingle{}}\NormalTok{))}\OperatorTok{;}  \CommentTok{// e89..........41e}
\end{Highlighting}
\end{Shaded}

The size of a generated HMAC key should not exceed the block size of the
underlying hash function. See
\hyperref[cryptocreatehmacalgorithm-key-options]{\texttt{crypto.createHmac()}}
for more information.

\subsubsection{\texorpdfstring{\texttt{crypto.generatePrime(size{[},\ options{[},\ callback{]}{]})}}{crypto.generatePrime(size{[}, options{[}, callback{]}{]})}}\label{crypto.generateprimesize-options-callback}

\begin{itemize}
\tightlist
\item
  \texttt{size} \{number\} The size (in bits) of the prime to generate.
\item
  \texttt{options} \{Object\}

  \begin{itemize}
  \tightlist
  \item
    \texttt{add}
    \{ArrayBuffer\textbar SharedArrayBuffer\textbar TypedArray\textbar Buffer\textbar DataView\textbar bigint\}
  \item
    \texttt{rem}
    \{ArrayBuffer\textbar SharedArrayBuffer\textbar TypedArray\textbar Buffer\textbar DataView\textbar bigint\}
  \item
    \texttt{safe} \{boolean\} \textbf{Default:} \texttt{false}.
  \item
    \texttt{bigint} \{boolean\} When \texttt{true}, the generated prime
    is returned as a \texttt{bigint}.
  \end{itemize}
\item
  \texttt{callback} \{Function\}

  \begin{itemize}
  \tightlist
  \item
    \texttt{err} \{Error\}
  \item
    \texttt{prime} \{ArrayBuffer\textbar bigint\}
  \end{itemize}
\end{itemize}

Generates a pseudorandom prime of \texttt{size} bits.

If \texttt{options.safe} is \texttt{true}, the prime will be a safe
prime -- that is, \texttt{(prime\ -\ 1)\ /\ 2} will also be a prime.

The \texttt{options.add} and \texttt{options.rem} parameters can be used
to enforce additional requirements, e.g., for Diffie-Hellman:

\begin{itemize}
\tightlist
\item
  If \texttt{options.add} and \texttt{options.rem} are both set, the
  prime will satisfy the condition that \texttt{prime\ \%\ add\ =\ rem}.
\item
  If only \texttt{options.add} is set and \texttt{options.safe} is not
  \texttt{true}, the prime will satisfy the condition that
  \texttt{prime\ \%\ add\ =\ 1}.
\item
  If only \texttt{options.add} is set and \texttt{options.safe} is set
  to \texttt{true}, the prime will instead satisfy the condition that
  \texttt{prime\ \%\ add\ =\ 3}. This is necessary because
  \texttt{prime\ \%\ add\ =\ 1} for
  \texttt{options.add\ \textgreater{}\ 2} would contradict the condition
  enforced by \texttt{options.safe}.
\item
  \texttt{options.rem} is ignored if \texttt{options.add} is not given.
\end{itemize}

Both \texttt{options.add} and \texttt{options.rem} must be encoded as
big-endian sequences if given as an \texttt{ArrayBuffer},
\texttt{SharedArrayBuffer}, \texttt{TypedArray}, \texttt{Buffer}, or
\texttt{DataView}.

By default, the prime is encoded as a big-endian sequence of octets in
an \{ArrayBuffer\}. If the \texttt{bigint} option is \texttt{true}, then
a \{bigint\} is provided.

\subsubsection{\texorpdfstring{\texttt{crypto.generatePrimeSync(size{[},\ options{]})}}{crypto.generatePrimeSync(size{[}, options{]})}}\label{crypto.generateprimesyncsize-options}

\begin{itemize}
\tightlist
\item
  \texttt{size} \{number\} The size (in bits) of the prime to generate.
\item
  \texttt{options} \{Object\}

  \begin{itemize}
  \tightlist
  \item
    \texttt{add}
    \{ArrayBuffer\textbar SharedArrayBuffer\textbar TypedArray\textbar Buffer\textbar DataView\textbar bigint\}
  \item
    \texttt{rem}
    \{ArrayBuffer\textbar SharedArrayBuffer\textbar TypedArray\textbar Buffer\textbar DataView\textbar bigint\}
  \item
    \texttt{safe} \{boolean\} \textbf{Default:} \texttt{false}.
  \item
    \texttt{bigint} \{boolean\} When \texttt{true}, the generated prime
    is returned as a \texttt{bigint}.
  \end{itemize}
\item
  Returns: \{ArrayBuffer\textbar bigint\}
\end{itemize}

Generates a pseudorandom prime of \texttt{size} bits.

If \texttt{options.safe} is \texttt{true}, the prime will be a safe
prime -- that is, \texttt{(prime\ -\ 1)\ /\ 2} will also be a prime.

The \texttt{options.add} and \texttt{options.rem} parameters can be used
to enforce additional requirements, e.g., for Diffie-Hellman:

\begin{itemize}
\tightlist
\item
  If \texttt{options.add} and \texttt{options.rem} are both set, the
  prime will satisfy the condition that \texttt{prime\ \%\ add\ =\ rem}.
\item
  If only \texttt{options.add} is set and \texttt{options.safe} is not
  \texttt{true}, the prime will satisfy the condition that
  \texttt{prime\ \%\ add\ =\ 1}.
\item
  If only \texttt{options.add} is set and \texttt{options.safe} is set
  to \texttt{true}, the prime will instead satisfy the condition that
  \texttt{prime\ \%\ add\ =\ 3}. This is necessary because
  \texttt{prime\ \%\ add\ =\ 1} for
  \texttt{options.add\ \textgreater{}\ 2} would contradict the condition
  enforced by \texttt{options.safe}.
\item
  \texttt{options.rem} is ignored if \texttt{options.add} is not given.
\end{itemize}

Both \texttt{options.add} and \texttt{options.rem} must be encoded as
big-endian sequences if given as an \texttt{ArrayBuffer},
\texttt{SharedArrayBuffer}, \texttt{TypedArray}, \texttt{Buffer}, or
\texttt{DataView}.

By default, the prime is encoded as a big-endian sequence of octets in
an \{ArrayBuffer\}. If the \texttt{bigint} option is \texttt{true}, then
a \{bigint\} is provided.

\subsubsection{\texorpdfstring{\texttt{crypto.getCipherInfo(nameOrNid{[},\ options{]})}}{crypto.getCipherInfo(nameOrNid{[}, options{]})}}\label{crypto.getcipherinfonameornid-options}

\begin{itemize}
\tightlist
\item
  \texttt{nameOrNid}: \{string\textbar number\} The name or nid of the
  cipher to query.
\item
  \texttt{options}: \{Object\}

  \begin{itemize}
  \tightlist
  \item
    \texttt{keyLength}: \{number\} A test key length.
  \item
    \texttt{ivLength}: \{number\} A test IV length.
  \end{itemize}
\item
  Returns: \{Object\}

  \begin{itemize}
  \tightlist
  \item
    \texttt{name} \{string\} The name of the cipher
  \item
    \texttt{nid} \{number\} The nid of the cipher
  \item
    \texttt{blockSize} \{number\} The block size of the cipher in bytes.
    This property is omitted when \texttt{mode} is
    \texttt{\textquotesingle{}stream\textquotesingle{}}.
  \item
    \texttt{ivLength} \{number\} The expected or default initialization
    vector length in bytes. This property is omitted if the cipher does
    not use an initialization vector.
  \item
    \texttt{keyLength} \{number\} The expected or default key length in
    bytes.
  \item
    \texttt{mode} \{string\} The cipher mode. One of
    \texttt{\textquotesingle{}cbc\textquotesingle{}},
    \texttt{\textquotesingle{}ccm\textquotesingle{}},
    \texttt{\textquotesingle{}cfb\textquotesingle{}},
    \texttt{\textquotesingle{}ctr\textquotesingle{}},
    \texttt{\textquotesingle{}ecb\textquotesingle{}},
    \texttt{\textquotesingle{}gcm\textquotesingle{}},
    \texttt{\textquotesingle{}ocb\textquotesingle{}},
    \texttt{\textquotesingle{}ofb\textquotesingle{}},
    \texttt{\textquotesingle{}stream\textquotesingle{}},
    \texttt{\textquotesingle{}wrap\textquotesingle{}},
    \texttt{\textquotesingle{}xts\textquotesingle{}}.
  \end{itemize}
\end{itemize}

Returns information about a given cipher.

Some ciphers accept variable length keys and initialization vectors. By
default, the \texttt{crypto.getCipherInfo()} method will return the
default values for these ciphers. To test if a given key length or iv
length is acceptable for given cipher, use the \texttt{keyLength} and
\texttt{ivLength} options. If the given values are unacceptable,
\texttt{undefined} will be returned.

\subsubsection{\texorpdfstring{\texttt{crypto.getCiphers()}}{crypto.getCiphers()}}\label{crypto.getciphers}

\begin{itemize}
\tightlist
\item
  Returns: \{string{[}{]}\} An array with the names of the supported
  cipher algorithms.
\end{itemize}

\begin{Shaded}
\begin{Highlighting}[]
\KeywordTok{const}\NormalTok{ \{}
\NormalTok{  getCiphers}\OperatorTok{,}
\NormalTok{\} }\OperatorTok{=} \ControlFlowTok{await} \ImportTok{import}\NormalTok{(}\StringTok{\textquotesingle{}node:crypto\textquotesingle{}}\NormalTok{)}\OperatorTok{;}

\BuiltInTok{console}\OperatorTok{.}\FunctionTok{log}\NormalTok{(}\FunctionTok{getCiphers}\NormalTok{())}\OperatorTok{;} \CommentTok{// [\textquotesingle{}aes{-}128{-}cbc\textquotesingle{}, \textquotesingle{}aes{-}128{-}ccm\textquotesingle{}, ...]}
\end{Highlighting}
\end{Shaded}

\begin{Shaded}
\begin{Highlighting}[]
\KeywordTok{const}\NormalTok{ \{}
\NormalTok{  getCiphers}\OperatorTok{,}
\NormalTok{\} }\OperatorTok{=} \PreprocessorTok{require}\NormalTok{(}\StringTok{\textquotesingle{}node:crypto\textquotesingle{}}\NormalTok{)}\OperatorTok{;}

\BuiltInTok{console}\OperatorTok{.}\FunctionTok{log}\NormalTok{(}\FunctionTok{getCiphers}\NormalTok{())}\OperatorTok{;} \CommentTok{// [\textquotesingle{}aes{-}128{-}cbc\textquotesingle{}, \textquotesingle{}aes{-}128{-}ccm\textquotesingle{}, ...]}
\end{Highlighting}
\end{Shaded}

\subsubsection{\texorpdfstring{\texttt{crypto.getCurves()}}{crypto.getCurves()}}\label{crypto.getcurves}

\begin{itemize}
\tightlist
\item
  Returns: \{string{[}{]}\} An array with the names of the supported
  elliptic curves.
\end{itemize}

\begin{Shaded}
\begin{Highlighting}[]
\KeywordTok{const}\NormalTok{ \{}
\NormalTok{  getCurves}\OperatorTok{,}
\NormalTok{\} }\OperatorTok{=} \ControlFlowTok{await} \ImportTok{import}\NormalTok{(}\StringTok{\textquotesingle{}node:crypto\textquotesingle{}}\NormalTok{)}\OperatorTok{;}

\BuiltInTok{console}\OperatorTok{.}\FunctionTok{log}\NormalTok{(}\FunctionTok{getCurves}\NormalTok{())}\OperatorTok{;} \CommentTok{// [\textquotesingle{}Oakley{-}EC2N{-}3\textquotesingle{}, \textquotesingle{}Oakley{-}EC2N{-}4\textquotesingle{}, ...]}
\end{Highlighting}
\end{Shaded}

\begin{Shaded}
\begin{Highlighting}[]
\KeywordTok{const}\NormalTok{ \{}
\NormalTok{  getCurves}\OperatorTok{,}
\NormalTok{\} }\OperatorTok{=} \PreprocessorTok{require}\NormalTok{(}\StringTok{\textquotesingle{}node:crypto\textquotesingle{}}\NormalTok{)}\OperatorTok{;}

\BuiltInTok{console}\OperatorTok{.}\FunctionTok{log}\NormalTok{(}\FunctionTok{getCurves}\NormalTok{())}\OperatorTok{;} \CommentTok{// [\textquotesingle{}Oakley{-}EC2N{-}3\textquotesingle{}, \textquotesingle{}Oakley{-}EC2N{-}4\textquotesingle{}, ...]}
\end{Highlighting}
\end{Shaded}

\subsubsection{\texorpdfstring{\texttt{crypto.getDiffieHellman(groupName)}}{crypto.getDiffieHellman(groupName)}}\label{crypto.getdiffiehellmangroupname}

\begin{itemize}
\tightlist
\item
  \texttt{groupName} \{string\}
\item
  Returns: \{DiffieHellmanGroup\}
\end{itemize}

Creates a predefined \texttt{DiffieHellmanGroup} key exchange object.
The supported groups are listed in the documentation for
\hyperref[class-diffiehellmangroup]{\texttt{DiffieHellmanGroup}}.

The returned object mimics the interface of objects created by
\hyperref[cryptocreatediffiehellmanprime-primeencoding-generator-generatorencoding]{\texttt{crypto.createDiffieHellman()}},
but will not allow changing the keys (with
\hyperref[diffiehellmansetpublickeypublickey-encoding]{\texttt{diffieHellman.setPublicKey()}},
for example). The advantage of using this method is that the parties do
not have to generate nor exchange a group modulus beforehand, saving
both processor and communication time.

Example (obtaining a shared secret):

\begin{Shaded}
\begin{Highlighting}[]
\KeywordTok{const}\NormalTok{ \{}
\NormalTok{  getDiffieHellman}\OperatorTok{,}
\NormalTok{\} }\OperatorTok{=} \ControlFlowTok{await} \ImportTok{import}\NormalTok{(}\StringTok{\textquotesingle{}node:crypto\textquotesingle{}}\NormalTok{)}\OperatorTok{;}
\KeywordTok{const}\NormalTok{ alice }\OperatorTok{=} \FunctionTok{getDiffieHellman}\NormalTok{(}\StringTok{\textquotesingle{}modp14\textquotesingle{}}\NormalTok{)}\OperatorTok{;}
\KeywordTok{const}\NormalTok{ bob }\OperatorTok{=} \FunctionTok{getDiffieHellman}\NormalTok{(}\StringTok{\textquotesingle{}modp14\textquotesingle{}}\NormalTok{)}\OperatorTok{;}

\NormalTok{alice}\OperatorTok{.}\FunctionTok{generateKeys}\NormalTok{()}\OperatorTok{;}
\NormalTok{bob}\OperatorTok{.}\FunctionTok{generateKeys}\NormalTok{()}\OperatorTok{;}

\KeywordTok{const}\NormalTok{ aliceSecret }\OperatorTok{=}\NormalTok{ alice}\OperatorTok{.}\FunctionTok{computeSecret}\NormalTok{(bob}\OperatorTok{.}\FunctionTok{getPublicKey}\NormalTok{()}\OperatorTok{,} \KeywordTok{null}\OperatorTok{,} \StringTok{\textquotesingle{}hex\textquotesingle{}}\NormalTok{)}\OperatorTok{;}
\KeywordTok{const}\NormalTok{ bobSecret }\OperatorTok{=}\NormalTok{ bob}\OperatorTok{.}\FunctionTok{computeSecret}\NormalTok{(alice}\OperatorTok{.}\FunctionTok{getPublicKey}\NormalTok{()}\OperatorTok{,} \KeywordTok{null}\OperatorTok{,} \StringTok{\textquotesingle{}hex\textquotesingle{}}\NormalTok{)}\OperatorTok{;}

\CommentTok{/* aliceSecret and bobSecret should be the same */}
\BuiltInTok{console}\OperatorTok{.}\FunctionTok{log}\NormalTok{(aliceSecret }\OperatorTok{===}\NormalTok{ bobSecret)}\OperatorTok{;}
\end{Highlighting}
\end{Shaded}

\begin{Shaded}
\begin{Highlighting}[]
\KeywordTok{const}\NormalTok{ \{}
\NormalTok{  getDiffieHellman}\OperatorTok{,}
\NormalTok{\} }\OperatorTok{=} \PreprocessorTok{require}\NormalTok{(}\StringTok{\textquotesingle{}node:crypto\textquotesingle{}}\NormalTok{)}\OperatorTok{;}

\KeywordTok{const}\NormalTok{ alice }\OperatorTok{=} \FunctionTok{getDiffieHellman}\NormalTok{(}\StringTok{\textquotesingle{}modp14\textquotesingle{}}\NormalTok{)}\OperatorTok{;}
\KeywordTok{const}\NormalTok{ bob }\OperatorTok{=} \FunctionTok{getDiffieHellman}\NormalTok{(}\StringTok{\textquotesingle{}modp14\textquotesingle{}}\NormalTok{)}\OperatorTok{;}

\NormalTok{alice}\OperatorTok{.}\FunctionTok{generateKeys}\NormalTok{()}\OperatorTok{;}
\NormalTok{bob}\OperatorTok{.}\FunctionTok{generateKeys}\NormalTok{()}\OperatorTok{;}

\KeywordTok{const}\NormalTok{ aliceSecret }\OperatorTok{=}\NormalTok{ alice}\OperatorTok{.}\FunctionTok{computeSecret}\NormalTok{(bob}\OperatorTok{.}\FunctionTok{getPublicKey}\NormalTok{()}\OperatorTok{,} \KeywordTok{null}\OperatorTok{,} \StringTok{\textquotesingle{}hex\textquotesingle{}}\NormalTok{)}\OperatorTok{;}
\KeywordTok{const}\NormalTok{ bobSecret }\OperatorTok{=}\NormalTok{ bob}\OperatorTok{.}\FunctionTok{computeSecret}\NormalTok{(alice}\OperatorTok{.}\FunctionTok{getPublicKey}\NormalTok{()}\OperatorTok{,} \KeywordTok{null}\OperatorTok{,} \StringTok{\textquotesingle{}hex\textquotesingle{}}\NormalTok{)}\OperatorTok{;}

\CommentTok{/* aliceSecret and bobSecret should be the same */}
\BuiltInTok{console}\OperatorTok{.}\FunctionTok{log}\NormalTok{(aliceSecret }\OperatorTok{===}\NormalTok{ bobSecret)}\OperatorTok{;}
\end{Highlighting}
\end{Shaded}

\subsubsection{\texorpdfstring{\texttt{crypto.getFips()}}{crypto.getFips()}}\label{crypto.getfips}

\begin{itemize}
\tightlist
\item
  Returns: \{number\} \texttt{1} if and only if a FIPS compliant crypto
  provider is currently in use, \texttt{0} otherwise. A future
  semver-major release may change the return type of this API to a
  \{boolean\}.
\end{itemize}

\subsubsection{\texorpdfstring{\texttt{crypto.getHashes()}}{crypto.getHashes()}}\label{crypto.gethashes}

\begin{itemize}
\tightlist
\item
  Returns: \{string{[}{]}\} An array of the names of the supported hash
  algorithms, such as
  \texttt{\textquotesingle{}RSA-SHA256\textquotesingle{}}. Hash
  algorithms are also called ``digest'' algorithms.
\end{itemize}

\begin{Shaded}
\begin{Highlighting}[]
\KeywordTok{const}\NormalTok{ \{}
\NormalTok{  getHashes}\OperatorTok{,}
\NormalTok{\} }\OperatorTok{=} \ControlFlowTok{await} \ImportTok{import}\NormalTok{(}\StringTok{\textquotesingle{}node:crypto\textquotesingle{}}\NormalTok{)}\OperatorTok{;}

\BuiltInTok{console}\OperatorTok{.}\FunctionTok{log}\NormalTok{(}\FunctionTok{getHashes}\NormalTok{())}\OperatorTok{;} \CommentTok{// [\textquotesingle{}DSA\textquotesingle{}, \textquotesingle{}DSA{-}SHA\textquotesingle{}, \textquotesingle{}DSA{-}SHA1\textquotesingle{}, ...]}
\end{Highlighting}
\end{Shaded}

\begin{Shaded}
\begin{Highlighting}[]
\KeywordTok{const}\NormalTok{ \{}
\NormalTok{  getHashes}\OperatorTok{,}
\NormalTok{\} }\OperatorTok{=} \PreprocessorTok{require}\NormalTok{(}\StringTok{\textquotesingle{}node:crypto\textquotesingle{}}\NormalTok{)}\OperatorTok{;}

\BuiltInTok{console}\OperatorTok{.}\FunctionTok{log}\NormalTok{(}\FunctionTok{getHashes}\NormalTok{())}\OperatorTok{;} \CommentTok{// [\textquotesingle{}DSA\textquotesingle{}, \textquotesingle{}DSA{-}SHA\textquotesingle{}, \textquotesingle{}DSA{-}SHA1\textquotesingle{}, ...]}
\end{Highlighting}
\end{Shaded}

\subsubsection{\texorpdfstring{\texttt{crypto.getRandomValues(typedArray)}}{crypto.getRandomValues(typedArray)}}\label{crypto.getrandomvaluestypedarray}

\begin{itemize}
\tightlist
\item
  \texttt{typedArray}
  \{Buffer\textbar TypedArray\textbar DataView\textbar ArrayBuffer\}
\item
  Returns:
  \{Buffer\textbar TypedArray\textbar DataView\textbar ArrayBuffer\}
  Returns \texttt{typedArray}.
\end{itemize}

A convenient alias for
\href{webcrypto.md\#cryptogetrandomvaluestypedarray}{\texttt{crypto.webcrypto.getRandomValues()}}.
This implementation is not compliant with the Web Crypto spec, to write
web-compatible code use
\href{webcrypto.md\#cryptogetrandomvaluestypedarray}{\texttt{crypto.webcrypto.getRandomValues()}}
instead.

\subsubsection{\texorpdfstring{\texttt{crypto.hkdf(digest,\ ikm,\ salt,\ info,\ keylen,\ callback)}}{crypto.hkdf(digest, ikm, salt, info, keylen, callback)}}\label{crypto.hkdfdigest-ikm-salt-info-keylen-callback}

\begin{itemize}
\tightlist
\item
  \texttt{digest} \{string\} The digest algorithm to use.
\item
  \texttt{ikm}
  \{string\textbar ArrayBuffer\textbar Buffer\textbar TypedArray\textbar DataView\textbar KeyObject\}
  The input keying material. Must be provided but can be zero-length.
\item
  \texttt{salt}
  \{string\textbar ArrayBuffer\textbar Buffer\textbar TypedArray\textbar DataView\}
  The salt value. Must be provided but can be zero-length.
\item
  \texttt{info}
  \{string\textbar ArrayBuffer\textbar Buffer\textbar TypedArray\textbar DataView\}
  Additional info value. Must be provided but can be zero-length, and
  cannot be more than 1024 bytes.
\item
  \texttt{keylen} \{number\} The length of the key to generate. Must be
  greater than 0. The maximum allowable value is \texttt{255} times the
  number of bytes produced by the selected digest function
  (e.g.~\texttt{sha512} generates 64-byte hashes, making the maximum
  HKDF output 16320 bytes).
\item
  \texttt{callback} \{Function\}

  \begin{itemize}
  \tightlist
  \item
    \texttt{err} \{Error\}
  \item
    \texttt{derivedKey} \{ArrayBuffer\}
  \end{itemize}
\end{itemize}

HKDF is a simple key derivation function defined in RFC 5869. The given
\texttt{ikm}, \texttt{salt} and \texttt{info} are used with the
\texttt{digest} to derive a key of \texttt{keylen} bytes.

The supplied \texttt{callback} function is called with two arguments:
\texttt{err} and \texttt{derivedKey}. If an errors occurs while deriving
the key, \texttt{err} will be set; otherwise \texttt{err} will be
\texttt{null}. The successfully generated \texttt{derivedKey} will be
passed to the callback as an \{ArrayBuffer\}. An error will be thrown if
any of the input arguments specify invalid values or types.

\begin{Shaded}
\begin{Highlighting}[]
\ImportTok{import}\NormalTok{ \{ }\BuiltInTok{Buffer}\NormalTok{ \} }\ImportTok{from} \StringTok{\textquotesingle{}node:buffer\textquotesingle{}}\OperatorTok{;}
\KeywordTok{const}\NormalTok{ \{}
\NormalTok{  hkdf}\OperatorTok{,}
\NormalTok{\} }\OperatorTok{=} \ControlFlowTok{await} \ImportTok{import}\NormalTok{(}\StringTok{\textquotesingle{}node:crypto\textquotesingle{}}\NormalTok{)}\OperatorTok{;}

\FunctionTok{hkdf}\NormalTok{(}\StringTok{\textquotesingle{}sha512\textquotesingle{}}\OperatorTok{,} \StringTok{\textquotesingle{}key\textquotesingle{}}\OperatorTok{,} \StringTok{\textquotesingle{}salt\textquotesingle{}}\OperatorTok{,} \StringTok{\textquotesingle{}info\textquotesingle{}}\OperatorTok{,} \DecValTok{64}\OperatorTok{,}\NormalTok{ (err}\OperatorTok{,}\NormalTok{ derivedKey) }\KeywordTok{=\textgreater{}}\NormalTok{ \{}
  \ControlFlowTok{if}\NormalTok{ (err) }\ControlFlowTok{throw}\NormalTok{ err}\OperatorTok{;}
  \BuiltInTok{console}\OperatorTok{.}\FunctionTok{log}\NormalTok{(}\BuiltInTok{Buffer}\OperatorTok{.}\FunctionTok{from}\NormalTok{(derivedKey)}\OperatorTok{.}\FunctionTok{toString}\NormalTok{(}\StringTok{\textquotesingle{}hex\textquotesingle{}}\NormalTok{))}\OperatorTok{;}  \CommentTok{// \textquotesingle{}24156e2...5391653\textquotesingle{}}
\NormalTok{\})}\OperatorTok{;}
\end{Highlighting}
\end{Shaded}

\begin{Shaded}
\begin{Highlighting}[]
\KeywordTok{const}\NormalTok{ \{}
\NormalTok{  hkdf}\OperatorTok{,}
\NormalTok{\} }\OperatorTok{=} \PreprocessorTok{require}\NormalTok{(}\StringTok{\textquotesingle{}node:crypto\textquotesingle{}}\NormalTok{)}\OperatorTok{;}
\KeywordTok{const}\NormalTok{ \{ }\BuiltInTok{Buffer}\NormalTok{ \} }\OperatorTok{=} \PreprocessorTok{require}\NormalTok{(}\StringTok{\textquotesingle{}node:buffer\textquotesingle{}}\NormalTok{)}\OperatorTok{;}

\FunctionTok{hkdf}\NormalTok{(}\StringTok{\textquotesingle{}sha512\textquotesingle{}}\OperatorTok{,} \StringTok{\textquotesingle{}key\textquotesingle{}}\OperatorTok{,} \StringTok{\textquotesingle{}salt\textquotesingle{}}\OperatorTok{,} \StringTok{\textquotesingle{}info\textquotesingle{}}\OperatorTok{,} \DecValTok{64}\OperatorTok{,}\NormalTok{ (err}\OperatorTok{,}\NormalTok{ derivedKey) }\KeywordTok{=\textgreater{}}\NormalTok{ \{}
  \ControlFlowTok{if}\NormalTok{ (err) }\ControlFlowTok{throw}\NormalTok{ err}\OperatorTok{;}
  \BuiltInTok{console}\OperatorTok{.}\FunctionTok{log}\NormalTok{(}\BuiltInTok{Buffer}\OperatorTok{.}\FunctionTok{from}\NormalTok{(derivedKey)}\OperatorTok{.}\FunctionTok{toString}\NormalTok{(}\StringTok{\textquotesingle{}hex\textquotesingle{}}\NormalTok{))}\OperatorTok{;}  \CommentTok{// \textquotesingle{}24156e2...5391653\textquotesingle{}}
\NormalTok{\})}\OperatorTok{;}
\end{Highlighting}
\end{Shaded}

\subsubsection{\texorpdfstring{\texttt{crypto.hkdfSync(digest,\ ikm,\ salt,\ info,\ keylen)}}{crypto.hkdfSync(digest, ikm, salt, info, keylen)}}\label{crypto.hkdfsyncdigest-ikm-salt-info-keylen}

\begin{itemize}
\tightlist
\item
  \texttt{digest} \{string\} The digest algorithm to use.
\item
  \texttt{ikm}
  \{string\textbar ArrayBuffer\textbar Buffer\textbar TypedArray\textbar DataView\textbar KeyObject\}
  The input keying material. Must be provided but can be zero-length.
\item
  \texttt{salt}
  \{string\textbar ArrayBuffer\textbar Buffer\textbar TypedArray\textbar DataView\}
  The salt value. Must be provided but can be zero-length.
\item
  \texttt{info}
  \{string\textbar ArrayBuffer\textbar Buffer\textbar TypedArray\textbar DataView\}
  Additional info value. Must be provided but can be zero-length, and
  cannot be more than 1024 bytes.
\item
  \texttt{keylen} \{number\} The length of the key to generate. Must be
  greater than 0. The maximum allowable value is \texttt{255} times the
  number of bytes produced by the selected digest function
  (e.g.~\texttt{sha512} generates 64-byte hashes, making the maximum
  HKDF output 16320 bytes).
\item
  Returns: \{ArrayBuffer\}
\end{itemize}

Provides a synchronous HKDF key derivation function as defined in RFC
5869. The given \texttt{ikm}, \texttt{salt} and \texttt{info} are used
with the \texttt{digest} to derive a key of \texttt{keylen} bytes.

The successfully generated \texttt{derivedKey} will be returned as an
\{ArrayBuffer\}.

An error will be thrown if any of the input arguments specify invalid
values or types, or if the derived key cannot be generated.

\begin{Shaded}
\begin{Highlighting}[]
\ImportTok{import}\NormalTok{ \{ }\BuiltInTok{Buffer}\NormalTok{ \} }\ImportTok{from} \StringTok{\textquotesingle{}node:buffer\textquotesingle{}}\OperatorTok{;}
\KeywordTok{const}\NormalTok{ \{}
\NormalTok{  hkdfSync}\OperatorTok{,}
\NormalTok{\} }\OperatorTok{=} \ControlFlowTok{await} \ImportTok{import}\NormalTok{(}\StringTok{\textquotesingle{}node:crypto\textquotesingle{}}\NormalTok{)}\OperatorTok{;}

\KeywordTok{const}\NormalTok{ derivedKey }\OperatorTok{=} \FunctionTok{hkdfSync}\NormalTok{(}\StringTok{\textquotesingle{}sha512\textquotesingle{}}\OperatorTok{,} \StringTok{\textquotesingle{}key\textquotesingle{}}\OperatorTok{,} \StringTok{\textquotesingle{}salt\textquotesingle{}}\OperatorTok{,} \StringTok{\textquotesingle{}info\textquotesingle{}}\OperatorTok{,} \DecValTok{64}\NormalTok{)}\OperatorTok{;}
\BuiltInTok{console}\OperatorTok{.}\FunctionTok{log}\NormalTok{(}\BuiltInTok{Buffer}\OperatorTok{.}\FunctionTok{from}\NormalTok{(derivedKey)}\OperatorTok{.}\FunctionTok{toString}\NormalTok{(}\StringTok{\textquotesingle{}hex\textquotesingle{}}\NormalTok{))}\OperatorTok{;}  \CommentTok{// \textquotesingle{}24156e2...5391653\textquotesingle{}}
\end{Highlighting}
\end{Shaded}

\begin{Shaded}
\begin{Highlighting}[]
\KeywordTok{const}\NormalTok{ \{}
\NormalTok{  hkdfSync}\OperatorTok{,}
\NormalTok{\} }\OperatorTok{=} \PreprocessorTok{require}\NormalTok{(}\StringTok{\textquotesingle{}node:crypto\textquotesingle{}}\NormalTok{)}\OperatorTok{;}
\KeywordTok{const}\NormalTok{ \{ }\BuiltInTok{Buffer}\NormalTok{ \} }\OperatorTok{=} \PreprocessorTok{require}\NormalTok{(}\StringTok{\textquotesingle{}node:buffer\textquotesingle{}}\NormalTok{)}\OperatorTok{;}

\KeywordTok{const}\NormalTok{ derivedKey }\OperatorTok{=} \FunctionTok{hkdfSync}\NormalTok{(}\StringTok{\textquotesingle{}sha512\textquotesingle{}}\OperatorTok{,} \StringTok{\textquotesingle{}key\textquotesingle{}}\OperatorTok{,} \StringTok{\textquotesingle{}salt\textquotesingle{}}\OperatorTok{,} \StringTok{\textquotesingle{}info\textquotesingle{}}\OperatorTok{,} \DecValTok{64}\NormalTok{)}\OperatorTok{;}
\BuiltInTok{console}\OperatorTok{.}\FunctionTok{log}\NormalTok{(}\BuiltInTok{Buffer}\OperatorTok{.}\FunctionTok{from}\NormalTok{(derivedKey)}\OperatorTok{.}\FunctionTok{toString}\NormalTok{(}\StringTok{\textquotesingle{}hex\textquotesingle{}}\NormalTok{))}\OperatorTok{;}  \CommentTok{// \textquotesingle{}24156e2...5391653\textquotesingle{}}
\end{Highlighting}
\end{Shaded}

\subsubsection{\texorpdfstring{\texttt{crypto.pbkdf2(password,\ salt,\ iterations,\ keylen,\ digest,\ callback)}}{crypto.pbkdf2(password, salt, iterations, keylen, digest, callback)}}\label{crypto.pbkdf2password-salt-iterations-keylen-digest-callback}

\begin{itemize}
\tightlist
\item
  \texttt{password}
  \{string\textbar ArrayBuffer\textbar Buffer\textbar TypedArray\textbar DataView\}
\item
  \texttt{salt}
  \{string\textbar ArrayBuffer\textbar Buffer\textbar TypedArray\textbar DataView\}
\item
  \texttt{iterations} \{number\}
\item
  \texttt{keylen} \{number\}
\item
  \texttt{digest} \{string\}
\item
  \texttt{callback} \{Function\}

  \begin{itemize}
  \tightlist
  \item
    \texttt{err} \{Error\}
  \item
    \texttt{derivedKey} \{Buffer\}
  \end{itemize}
\end{itemize}

Provides an asynchronous Password-Based Key Derivation Function 2
(PBKDF2) implementation. A selected HMAC digest algorithm specified by
\texttt{digest} is applied to derive a key of the requested byte length
(\texttt{keylen}) from the \texttt{password}, \texttt{salt} and
\texttt{iterations}.

The supplied \texttt{callback} function is called with two arguments:
\texttt{err} and \texttt{derivedKey}. If an error occurs while deriving
the key, \texttt{err} will be set; otherwise \texttt{err} will be
\texttt{null}. By default, the successfully generated
\texttt{derivedKey} will be passed to the callback as a
\href{buffer.md}{\texttt{Buffer}}. An error will be thrown if any of the
input arguments specify invalid values or types.

The \texttt{iterations} argument must be a number set as high as
possible. The higher the number of iterations, the more secure the
derived key will be, but will take a longer amount of time to complete.

The \texttt{salt} should be as unique as possible. It is recommended
that a salt is random and at least 16 bytes long. See
\href{https://nvlpubs.nist.gov/nistpubs/Legacy/SP/nistspecialpublication800-132.pdf}{NIST
SP 800-132} for details.

When passing strings for \texttt{password} or \texttt{salt}, please
consider
\hyperref[using-strings-as-inputs-to-cryptographic-apis]{caveats when
using strings as inputs to cryptographic APIs}.

\begin{Shaded}
\begin{Highlighting}[]
\KeywordTok{const}\NormalTok{ \{}
\NormalTok{  pbkdf2}\OperatorTok{,}
\NormalTok{\} }\OperatorTok{=} \ControlFlowTok{await} \ImportTok{import}\NormalTok{(}\StringTok{\textquotesingle{}node:crypto\textquotesingle{}}\NormalTok{)}\OperatorTok{;}

\FunctionTok{pbkdf2}\NormalTok{(}\StringTok{\textquotesingle{}secret\textquotesingle{}}\OperatorTok{,} \StringTok{\textquotesingle{}salt\textquotesingle{}}\OperatorTok{,} \DecValTok{100000}\OperatorTok{,} \DecValTok{64}\OperatorTok{,} \StringTok{\textquotesingle{}sha512\textquotesingle{}}\OperatorTok{,}\NormalTok{ (err}\OperatorTok{,}\NormalTok{ derivedKey) }\KeywordTok{=\textgreater{}}\NormalTok{ \{}
  \ControlFlowTok{if}\NormalTok{ (err) }\ControlFlowTok{throw}\NormalTok{ err}\OperatorTok{;}
  \BuiltInTok{console}\OperatorTok{.}\FunctionTok{log}\NormalTok{(derivedKey}\OperatorTok{.}\FunctionTok{toString}\NormalTok{(}\StringTok{\textquotesingle{}hex\textquotesingle{}}\NormalTok{))}\OperatorTok{;}  \CommentTok{// \textquotesingle{}3745e48...08d59ae\textquotesingle{}}
\NormalTok{\})}\OperatorTok{;}
\end{Highlighting}
\end{Shaded}

\begin{Shaded}
\begin{Highlighting}[]
\KeywordTok{const}\NormalTok{ \{}
\NormalTok{  pbkdf2}\OperatorTok{,}
\NormalTok{\} }\OperatorTok{=} \PreprocessorTok{require}\NormalTok{(}\StringTok{\textquotesingle{}node:crypto\textquotesingle{}}\NormalTok{)}\OperatorTok{;}

\FunctionTok{pbkdf2}\NormalTok{(}\StringTok{\textquotesingle{}secret\textquotesingle{}}\OperatorTok{,} \StringTok{\textquotesingle{}salt\textquotesingle{}}\OperatorTok{,} \DecValTok{100000}\OperatorTok{,} \DecValTok{64}\OperatorTok{,} \StringTok{\textquotesingle{}sha512\textquotesingle{}}\OperatorTok{,}\NormalTok{ (err}\OperatorTok{,}\NormalTok{ derivedKey) }\KeywordTok{=\textgreater{}}\NormalTok{ \{}
  \ControlFlowTok{if}\NormalTok{ (err) }\ControlFlowTok{throw}\NormalTok{ err}\OperatorTok{;}
  \BuiltInTok{console}\OperatorTok{.}\FunctionTok{log}\NormalTok{(derivedKey}\OperatorTok{.}\FunctionTok{toString}\NormalTok{(}\StringTok{\textquotesingle{}hex\textquotesingle{}}\NormalTok{))}\OperatorTok{;}  \CommentTok{// \textquotesingle{}3745e48...08d59ae\textquotesingle{}}
\NormalTok{\})}\OperatorTok{;}
\end{Highlighting}
\end{Shaded}

An array of supported digest functions can be retrieved using
\hyperref[cryptogethashes]{\texttt{crypto.getHashes()}}.

This API uses libuv's threadpool, which can have surprising and negative
performance implications for some applications; see the
\href{cli.md\#uv_threadpool_sizesize}{\texttt{UV\_THREADPOOL\_SIZE}}
documentation for more information.

\subsubsection{\texorpdfstring{\texttt{crypto.pbkdf2Sync(password,\ salt,\ iterations,\ keylen,\ digest)}}{crypto.pbkdf2Sync(password, salt, iterations, keylen, digest)}}\label{crypto.pbkdf2syncpassword-salt-iterations-keylen-digest}

\begin{itemize}
\tightlist
\item
  \texttt{password}
  \{string\textbar Buffer\textbar TypedArray\textbar DataView\}
\item
  \texttt{salt}
  \{string\textbar Buffer\textbar TypedArray\textbar DataView\}
\item
  \texttt{iterations} \{number\}
\item
  \texttt{keylen} \{number\}
\item
  \texttt{digest} \{string\}
\item
  Returns: \{Buffer\}
\end{itemize}

Provides a synchronous Password-Based Key Derivation Function 2 (PBKDF2)
implementation. A selected HMAC digest algorithm specified by
\texttt{digest} is applied to derive a key of the requested byte length
(\texttt{keylen}) from the \texttt{password}, \texttt{salt} and
\texttt{iterations}.

If an error occurs an \texttt{Error} will be thrown, otherwise the
derived key will be returned as a \href{buffer.md}{\texttt{Buffer}}.

The \texttt{iterations} argument must be a number set as high as
possible. The higher the number of iterations, the more secure the
derived key will be, but will take a longer amount of time to complete.

The \texttt{salt} should be as unique as possible. It is recommended
that a salt is random and at least 16 bytes long. See
\href{https://nvlpubs.nist.gov/nistpubs/Legacy/SP/nistspecialpublication800-132.pdf}{NIST
SP 800-132} for details.

When passing strings for \texttt{password} or \texttt{salt}, please
consider
\hyperref[using-strings-as-inputs-to-cryptographic-apis]{caveats when
using strings as inputs to cryptographic APIs}.

\begin{Shaded}
\begin{Highlighting}[]
\KeywordTok{const}\NormalTok{ \{}
\NormalTok{  pbkdf2Sync}\OperatorTok{,}
\NormalTok{\} }\OperatorTok{=} \ControlFlowTok{await} \ImportTok{import}\NormalTok{(}\StringTok{\textquotesingle{}node:crypto\textquotesingle{}}\NormalTok{)}\OperatorTok{;}

\KeywordTok{const}\NormalTok{ key }\OperatorTok{=} \FunctionTok{pbkdf2Sync}\NormalTok{(}\StringTok{\textquotesingle{}secret\textquotesingle{}}\OperatorTok{,} \StringTok{\textquotesingle{}salt\textquotesingle{}}\OperatorTok{,} \DecValTok{100000}\OperatorTok{,} \DecValTok{64}\OperatorTok{,} \StringTok{\textquotesingle{}sha512\textquotesingle{}}\NormalTok{)}\OperatorTok{;}
\BuiltInTok{console}\OperatorTok{.}\FunctionTok{log}\NormalTok{(key}\OperatorTok{.}\FunctionTok{toString}\NormalTok{(}\StringTok{\textquotesingle{}hex\textquotesingle{}}\NormalTok{))}\OperatorTok{;}  \CommentTok{// \textquotesingle{}3745e48...08d59ae\textquotesingle{}}
\end{Highlighting}
\end{Shaded}

\begin{Shaded}
\begin{Highlighting}[]
\KeywordTok{const}\NormalTok{ \{}
\NormalTok{  pbkdf2Sync}\OperatorTok{,}
\NormalTok{\} }\OperatorTok{=} \PreprocessorTok{require}\NormalTok{(}\StringTok{\textquotesingle{}node:crypto\textquotesingle{}}\NormalTok{)}\OperatorTok{;}

\KeywordTok{const}\NormalTok{ key }\OperatorTok{=} \FunctionTok{pbkdf2Sync}\NormalTok{(}\StringTok{\textquotesingle{}secret\textquotesingle{}}\OperatorTok{,} \StringTok{\textquotesingle{}salt\textquotesingle{}}\OperatorTok{,} \DecValTok{100000}\OperatorTok{,} \DecValTok{64}\OperatorTok{,} \StringTok{\textquotesingle{}sha512\textquotesingle{}}\NormalTok{)}\OperatorTok{;}
\BuiltInTok{console}\OperatorTok{.}\FunctionTok{log}\NormalTok{(key}\OperatorTok{.}\FunctionTok{toString}\NormalTok{(}\StringTok{\textquotesingle{}hex\textquotesingle{}}\NormalTok{))}\OperatorTok{;}  \CommentTok{// \textquotesingle{}3745e48...08d59ae\textquotesingle{}}
\end{Highlighting}
\end{Shaded}

An array of supported digest functions can be retrieved using
\hyperref[cryptogethashes]{\texttt{crypto.getHashes()}}.

\subsubsection{\texorpdfstring{\texttt{crypto.privateDecrypt(privateKey,\ buffer)}}{crypto.privateDecrypt(privateKey, buffer)}}\label{crypto.privatedecryptprivatekey-buffer}

\begin{itemize}
\tightlist
\item
  \texttt{privateKey}
  \{Object\textbar string\textbar ArrayBuffer\textbar Buffer\textbar TypedArray\textbar DataView\textbar KeyObject\textbar CryptoKey\}

  \begin{itemize}
  \tightlist
  \item
    \texttt{oaepHash} \{string\} The hash function to use for OAEP
    padding and MGF1. \textbf{Default:}
    \texttt{\textquotesingle{}sha1\textquotesingle{}}
  \item
    \texttt{oaepLabel}
    \{string\textbar ArrayBuffer\textbar Buffer\textbar TypedArray\textbar DataView\}
    The label to use for OAEP padding. If not specified, no label is
    used.
  \item
    \texttt{padding} \{crypto.constants\} An optional padding value
    defined in \texttt{crypto.constants}, which may be:
    \texttt{crypto.constants.RSA\_NO\_PADDING},
    \texttt{crypto.constants.RSA\_PKCS1\_PADDING}, or
    \texttt{crypto.constants.RSA\_PKCS1\_OAEP\_PADDING}.
  \end{itemize}
\item
  \texttt{buffer}
  \{string\textbar ArrayBuffer\textbar Buffer\textbar TypedArray\textbar DataView\}
\item
  Returns: \{Buffer\} A new \texttt{Buffer} with the decrypted content.
\end{itemize}

Decrypts \texttt{buffer} with \texttt{privateKey}. \texttt{buffer} was
previously encrypted using the corresponding public key, for example
using
\hyperref[cryptopublicencryptkey-buffer]{\texttt{crypto.publicEncrypt()}}.

If \texttt{privateKey} is not a
\hyperref[class-keyobject]{\texttt{KeyObject}}, this function behaves as
if \texttt{privateKey} had been passed to
\hyperref[cryptocreateprivatekeykey]{\texttt{crypto.createPrivateKey()}}.
If it is an object, the \texttt{padding} property can be passed.
Otherwise, this function uses \texttt{RSA\_PKCS1\_OAEP\_PADDING}.

\subsubsection{\texorpdfstring{\texttt{crypto.privateEncrypt(privateKey,\ buffer)}}{crypto.privateEncrypt(privateKey, buffer)}}\label{crypto.privateencryptprivatekey-buffer}

\begin{itemize}
\tightlist
\item
  \texttt{privateKey}
  \{Object\textbar string\textbar ArrayBuffer\textbar Buffer\textbar TypedArray\textbar DataView\textbar KeyObject\textbar CryptoKey\}

  \begin{itemize}
  \tightlist
  \item
    \texttt{key}
    \{string\textbar ArrayBuffer\textbar Buffer\textbar TypedArray\textbar DataView\textbar KeyObject\textbar CryptoKey\}
    A PEM encoded private key.
  \item
    \texttt{passphrase}
    \{string\textbar ArrayBuffer\textbar Buffer\textbar TypedArray\textbar DataView\}
    An optional passphrase for the private key.
  \item
    \texttt{padding} \{crypto.constants\} An optional padding value
    defined in \texttt{crypto.constants}, which may be:
    \texttt{crypto.constants.RSA\_NO\_PADDING} or
    \texttt{crypto.constants.RSA\_PKCS1\_PADDING}.
  \item
    \texttt{encoding} \{string\} The string encoding to use when
    \texttt{buffer}, \texttt{key}, or \texttt{passphrase} are strings.
  \end{itemize}
\item
  \texttt{buffer}
  \{string\textbar ArrayBuffer\textbar Buffer\textbar TypedArray\textbar DataView\}
\item
  Returns: \{Buffer\} A new \texttt{Buffer} with the encrypted content.
\end{itemize}

Encrypts \texttt{buffer} with \texttt{privateKey}. The returned data can
be decrypted using the corresponding public key, for example using
\hyperref[cryptopublicdecryptkey-buffer]{\texttt{crypto.publicDecrypt()}}.

If \texttt{privateKey} is not a
\hyperref[class-keyobject]{\texttt{KeyObject}}, this function behaves as
if \texttt{privateKey} had been passed to
\hyperref[cryptocreateprivatekeykey]{\texttt{crypto.createPrivateKey()}}.
If it is an object, the \texttt{padding} property can be passed.
Otherwise, this function uses \texttt{RSA\_PKCS1\_PADDING}.

\subsubsection{\texorpdfstring{\texttt{crypto.publicDecrypt(key,\ buffer)}}{crypto.publicDecrypt(key, buffer)}}\label{crypto.publicdecryptkey-buffer}

\begin{itemize}
\tightlist
\item
  \texttt{key}
  \{Object\textbar string\textbar ArrayBuffer\textbar Buffer\textbar TypedArray\textbar DataView\textbar KeyObject\textbar CryptoKey\}

  \begin{itemize}
  \tightlist
  \item
    \texttt{passphrase}
    \{string\textbar ArrayBuffer\textbar Buffer\textbar TypedArray\textbar DataView\}
    An optional passphrase for the private key.
  \item
    \texttt{padding} \{crypto.constants\} An optional padding value
    defined in \texttt{crypto.constants}, which may be:
    \texttt{crypto.constants.RSA\_NO\_PADDING} or
    \texttt{crypto.constants.RSA\_PKCS1\_PADDING}.
  \item
    \texttt{encoding} \{string\} The string encoding to use when
    \texttt{buffer}, \texttt{key}, or \texttt{passphrase} are strings.
  \end{itemize}
\item
  \texttt{buffer}
  \{string\textbar ArrayBuffer\textbar Buffer\textbar TypedArray\textbar DataView\}
\item
  Returns: \{Buffer\} A new \texttt{Buffer} with the decrypted content.
\end{itemize}

Decrypts \texttt{buffer} with \texttt{key}.\texttt{buffer} was
previously encrypted using the corresponding private key, for example
using
\hyperref[cryptoprivateencryptprivatekey-buffer]{\texttt{crypto.privateEncrypt()}}.

If \texttt{key} is not a \hyperref[class-keyobject]{\texttt{KeyObject}},
this function behaves as if \texttt{key} had been passed to
\hyperref[cryptocreatepublickeykey]{\texttt{crypto.createPublicKey()}}.
If it is an object, the \texttt{padding} property can be passed.
Otherwise, this function uses \texttt{RSA\_PKCS1\_PADDING}.

Because RSA public keys can be derived from private keys, a private key
may be passed instead of a public key.

\subsubsection{\texorpdfstring{\texttt{crypto.publicEncrypt(key,\ buffer)}}{crypto.publicEncrypt(key, buffer)}}\label{crypto.publicencryptkey-buffer}

\begin{itemize}
\tightlist
\item
  \texttt{key}
  \{Object\textbar string\textbar ArrayBuffer\textbar Buffer\textbar TypedArray\textbar DataView\textbar KeyObject\textbar CryptoKey\}

  \begin{itemize}
  \tightlist
  \item
    \texttt{key}
    \{string\textbar ArrayBuffer\textbar Buffer\textbar TypedArray\textbar DataView\textbar KeyObject\textbar CryptoKey\}
    A PEM encoded public or private key, \{KeyObject\}, or
    \{CryptoKey\}.
  \item
    \texttt{oaepHash} \{string\} The hash function to use for OAEP
    padding and MGF1. \textbf{Default:}
    \texttt{\textquotesingle{}sha1\textquotesingle{}}
  \item
    \texttt{oaepLabel}
    \{string\textbar ArrayBuffer\textbar Buffer\textbar TypedArray\textbar DataView\}
    The label to use for OAEP padding. If not specified, no label is
    used.
  \item
    \texttt{passphrase}
    \{string\textbar ArrayBuffer\textbar Buffer\textbar TypedArray\textbar DataView\}
    An optional passphrase for the private key.
  \item
    \texttt{padding} \{crypto.constants\} An optional padding value
    defined in \texttt{crypto.constants}, which may be:
    \texttt{crypto.constants.RSA\_NO\_PADDING},
    \texttt{crypto.constants.RSA\_PKCS1\_PADDING}, or
    \texttt{crypto.constants.RSA\_PKCS1\_OAEP\_PADDING}.
  \item
    \texttt{encoding} \{string\} The string encoding to use when
    \texttt{buffer}, \texttt{key}, \texttt{oaepLabel}, or
    \texttt{passphrase} are strings.
  \end{itemize}
\item
  \texttt{buffer}
  \{string\textbar ArrayBuffer\textbar Buffer\textbar TypedArray\textbar DataView\}
\item
  Returns: \{Buffer\} A new \texttt{Buffer} with the encrypted content.
\end{itemize}

Encrypts the content of \texttt{buffer} with \texttt{key} and returns a
new \href{buffer.md}{\texttt{Buffer}} with encrypted content. The
returned data can be decrypted using the corresponding private key, for
example using
\hyperref[cryptoprivatedecryptprivatekey-buffer]{\texttt{crypto.privateDecrypt()}}.

If \texttt{key} is not a \hyperref[class-keyobject]{\texttt{KeyObject}},
this function behaves as if \texttt{key} had been passed to
\hyperref[cryptocreatepublickeykey]{\texttt{crypto.createPublicKey()}}.
If it is an object, the \texttt{padding} property can be passed.
Otherwise, this function uses \texttt{RSA\_PKCS1\_OAEP\_PADDING}.

Because RSA public keys can be derived from private keys, a private key
may be passed instead of a public key.

\subsubsection{\texorpdfstring{\texttt{crypto.randomBytes(size{[},\ callback{]})}}{crypto.randomBytes(size{[}, callback{]})}}\label{crypto.randombytessize-callback}

\begin{itemize}
\tightlist
\item
  \texttt{size} \{number\} The number of bytes to generate. The
  \texttt{size} must not be larger than \texttt{2**31\ -\ 1}.
\item
  \texttt{callback} \{Function\}

  \begin{itemize}
  \tightlist
  \item
    \texttt{err} \{Error\}
  \item
    \texttt{buf} \{Buffer\}
  \end{itemize}
\item
  Returns: \{Buffer\} if the \texttt{callback} function is not provided.
\end{itemize}

Generates cryptographically strong pseudorandom data. The \texttt{size}
argument is a number indicating the number of bytes to generate.

If a \texttt{callback} function is provided, the bytes are generated
asynchronously and the \texttt{callback} function is invoked with two
arguments: \texttt{err} and \texttt{buf}. If an error occurs,
\texttt{err} will be an \texttt{Error} object; otherwise it is
\texttt{null}. The \texttt{buf} argument is a
\href{buffer.md}{\texttt{Buffer}} containing the generated bytes.

\begin{Shaded}
\begin{Highlighting}[]
\CommentTok{// Asynchronous}
\KeywordTok{const}\NormalTok{ \{}
\NormalTok{  randomBytes}\OperatorTok{,}
\NormalTok{\} }\OperatorTok{=} \ControlFlowTok{await} \ImportTok{import}\NormalTok{(}\StringTok{\textquotesingle{}node:crypto\textquotesingle{}}\NormalTok{)}\OperatorTok{;}

\FunctionTok{randomBytes}\NormalTok{(}\DecValTok{256}\OperatorTok{,}\NormalTok{ (err}\OperatorTok{,}\NormalTok{ buf) }\KeywordTok{=\textgreater{}}\NormalTok{ \{}
  \ControlFlowTok{if}\NormalTok{ (err) }\ControlFlowTok{throw}\NormalTok{ err}\OperatorTok{;}
  \BuiltInTok{console}\OperatorTok{.}\FunctionTok{log}\NormalTok{(}\VerbatimStringTok{\textasciigrave{}}\SpecialCharTok{$\{}\NormalTok{buf}\OperatorTok{.}\AttributeTok{length}\SpecialCharTok{\}}\VerbatimStringTok{ bytes of random data: }\SpecialCharTok{$\{}\NormalTok{buf}\OperatorTok{.}\FunctionTok{toString}\NormalTok{(}\StringTok{\textquotesingle{}hex\textquotesingle{}}\NormalTok{)}\SpecialCharTok{\}}\VerbatimStringTok{\textasciigrave{}}\NormalTok{)}\OperatorTok{;}
\NormalTok{\})}\OperatorTok{;}
\end{Highlighting}
\end{Shaded}

\begin{Shaded}
\begin{Highlighting}[]
\CommentTok{// Asynchronous}
\KeywordTok{const}\NormalTok{ \{}
\NormalTok{  randomBytes}\OperatorTok{,}
\NormalTok{\} }\OperatorTok{=} \PreprocessorTok{require}\NormalTok{(}\StringTok{\textquotesingle{}node:crypto\textquotesingle{}}\NormalTok{)}\OperatorTok{;}

\FunctionTok{randomBytes}\NormalTok{(}\DecValTok{256}\OperatorTok{,}\NormalTok{ (err}\OperatorTok{,}\NormalTok{ buf) }\KeywordTok{=\textgreater{}}\NormalTok{ \{}
  \ControlFlowTok{if}\NormalTok{ (err) }\ControlFlowTok{throw}\NormalTok{ err}\OperatorTok{;}
  \BuiltInTok{console}\OperatorTok{.}\FunctionTok{log}\NormalTok{(}\VerbatimStringTok{\textasciigrave{}}\SpecialCharTok{$\{}\NormalTok{buf}\OperatorTok{.}\AttributeTok{length}\SpecialCharTok{\}}\VerbatimStringTok{ bytes of random data: }\SpecialCharTok{$\{}\NormalTok{buf}\OperatorTok{.}\FunctionTok{toString}\NormalTok{(}\StringTok{\textquotesingle{}hex\textquotesingle{}}\NormalTok{)}\SpecialCharTok{\}}\VerbatimStringTok{\textasciigrave{}}\NormalTok{)}\OperatorTok{;}
\NormalTok{\})}\OperatorTok{;}
\end{Highlighting}
\end{Shaded}

If the \texttt{callback} function is not provided, the random bytes are
generated synchronously and returned as a
\href{buffer.md}{\texttt{Buffer}}. An error will be thrown if there is a
problem generating the bytes.

\begin{Shaded}
\begin{Highlighting}[]
\CommentTok{// Synchronous}
\KeywordTok{const}\NormalTok{ \{}
\NormalTok{  randomBytes}\OperatorTok{,}
\NormalTok{\} }\OperatorTok{=} \ControlFlowTok{await} \ImportTok{import}\NormalTok{(}\StringTok{\textquotesingle{}node:crypto\textquotesingle{}}\NormalTok{)}\OperatorTok{;}

\KeywordTok{const}\NormalTok{ buf }\OperatorTok{=} \FunctionTok{randomBytes}\NormalTok{(}\DecValTok{256}\NormalTok{)}\OperatorTok{;}
\BuiltInTok{console}\OperatorTok{.}\FunctionTok{log}\NormalTok{(}
  \VerbatimStringTok{\textasciigrave{}}\SpecialCharTok{$\{}\NormalTok{buf}\OperatorTok{.}\AttributeTok{length}\SpecialCharTok{\}}\VerbatimStringTok{ bytes of random data: }\SpecialCharTok{$\{}\NormalTok{buf}\OperatorTok{.}\FunctionTok{toString}\NormalTok{(}\StringTok{\textquotesingle{}hex\textquotesingle{}}\NormalTok{)}\SpecialCharTok{\}}\VerbatimStringTok{\textasciigrave{}}\NormalTok{)}\OperatorTok{;}
\end{Highlighting}
\end{Shaded}

\begin{Shaded}
\begin{Highlighting}[]
\CommentTok{// Synchronous}
\KeywordTok{const}\NormalTok{ \{}
\NormalTok{  randomBytes}\OperatorTok{,}
\NormalTok{\} }\OperatorTok{=} \PreprocessorTok{require}\NormalTok{(}\StringTok{\textquotesingle{}node:crypto\textquotesingle{}}\NormalTok{)}\OperatorTok{;}

\KeywordTok{const}\NormalTok{ buf }\OperatorTok{=} \FunctionTok{randomBytes}\NormalTok{(}\DecValTok{256}\NormalTok{)}\OperatorTok{;}
\BuiltInTok{console}\OperatorTok{.}\FunctionTok{log}\NormalTok{(}
  \VerbatimStringTok{\textasciigrave{}}\SpecialCharTok{$\{}\NormalTok{buf}\OperatorTok{.}\AttributeTok{length}\SpecialCharTok{\}}\VerbatimStringTok{ bytes of random data: }\SpecialCharTok{$\{}\NormalTok{buf}\OperatorTok{.}\FunctionTok{toString}\NormalTok{(}\StringTok{\textquotesingle{}hex\textquotesingle{}}\NormalTok{)}\SpecialCharTok{\}}\VerbatimStringTok{\textasciigrave{}}\NormalTok{)}\OperatorTok{;}
\end{Highlighting}
\end{Shaded}

The \texttt{crypto.randomBytes()} method will not complete until there
is sufficient entropy available. This should normally never take longer
than a few milliseconds. The only time when generating the random bytes
may conceivably block for a longer period of time is right after boot,
when the whole system is still low on entropy.

This API uses libuv's threadpool, which can have surprising and negative
performance implications for some applications; see the
\href{cli.md\#uv_threadpool_sizesize}{\texttt{UV\_THREADPOOL\_SIZE}}
documentation for more information.

The asynchronous version of \texttt{crypto.randomBytes()} is carried out
in a single threadpool request. To minimize threadpool task length
variation, partition large \texttt{randomBytes} requests when doing so
as part of fulfilling a client request.

\subsubsection{\texorpdfstring{\texttt{crypto.randomFillSync(buffer{[},\ offset{]}{[},\ size{]})}}{crypto.randomFillSync(buffer{[}, offset{]}{[}, size{]})}}\label{crypto.randomfillsyncbuffer-offset-size}

\begin{itemize}
\tightlist
\item
  \texttt{buffer}
  \{ArrayBuffer\textbar Buffer\textbar TypedArray\textbar DataView\}
  Must be supplied. The size of the provided \texttt{buffer} must not be
  larger than \texttt{2**31\ -\ 1}.
\item
  \texttt{offset} \{number\} \textbf{Default:} \texttt{0}
\item
  \texttt{size} \{number\} \textbf{Default:}
  \texttt{buffer.length\ -\ offset}. The \texttt{size} must not be
  larger than \texttt{2**31\ -\ 1}.
\item
  Returns:
  \{ArrayBuffer\textbar Buffer\textbar TypedArray\textbar DataView\} The
  object passed as \texttt{buffer} argument.
\end{itemize}

Synchronous version of
\hyperref[cryptorandomfillbuffer-offset-size-callback]{\texttt{crypto.randomFill()}}.

\begin{Shaded}
\begin{Highlighting}[]
\ImportTok{import}\NormalTok{ \{ }\BuiltInTok{Buffer}\NormalTok{ \} }\ImportTok{from} \StringTok{\textquotesingle{}node:buffer\textquotesingle{}}\OperatorTok{;}
\KeywordTok{const}\NormalTok{ \{ randomFillSync \} }\OperatorTok{=} \ControlFlowTok{await} \ImportTok{import}\NormalTok{(}\StringTok{\textquotesingle{}node:crypto\textquotesingle{}}\NormalTok{)}\OperatorTok{;}

\KeywordTok{const}\NormalTok{ buf }\OperatorTok{=} \BuiltInTok{Buffer}\OperatorTok{.}\FunctionTok{alloc}\NormalTok{(}\DecValTok{10}\NormalTok{)}\OperatorTok{;}
\BuiltInTok{console}\OperatorTok{.}\FunctionTok{log}\NormalTok{(}\FunctionTok{randomFillSync}\NormalTok{(buf)}\OperatorTok{.}\FunctionTok{toString}\NormalTok{(}\StringTok{\textquotesingle{}hex\textquotesingle{}}\NormalTok{))}\OperatorTok{;}

\FunctionTok{randomFillSync}\NormalTok{(buf}\OperatorTok{,} \DecValTok{5}\NormalTok{)}\OperatorTok{;}
\BuiltInTok{console}\OperatorTok{.}\FunctionTok{log}\NormalTok{(buf}\OperatorTok{.}\FunctionTok{toString}\NormalTok{(}\StringTok{\textquotesingle{}hex\textquotesingle{}}\NormalTok{))}\OperatorTok{;}

\CommentTok{// The above is equivalent to the following:}
\FunctionTok{randomFillSync}\NormalTok{(buf}\OperatorTok{,} \DecValTok{5}\OperatorTok{,} \DecValTok{5}\NormalTok{)}\OperatorTok{;}
\BuiltInTok{console}\OperatorTok{.}\FunctionTok{log}\NormalTok{(buf}\OperatorTok{.}\FunctionTok{toString}\NormalTok{(}\StringTok{\textquotesingle{}hex\textquotesingle{}}\NormalTok{))}\OperatorTok{;}
\end{Highlighting}
\end{Shaded}

\begin{Shaded}
\begin{Highlighting}[]
\KeywordTok{const}\NormalTok{ \{ randomFillSync \} }\OperatorTok{=} \PreprocessorTok{require}\NormalTok{(}\StringTok{\textquotesingle{}node:crypto\textquotesingle{}}\NormalTok{)}\OperatorTok{;}
\KeywordTok{const}\NormalTok{ \{ }\BuiltInTok{Buffer}\NormalTok{ \} }\OperatorTok{=} \PreprocessorTok{require}\NormalTok{(}\StringTok{\textquotesingle{}node:buffer\textquotesingle{}}\NormalTok{)}\OperatorTok{;}

\KeywordTok{const}\NormalTok{ buf }\OperatorTok{=} \BuiltInTok{Buffer}\OperatorTok{.}\FunctionTok{alloc}\NormalTok{(}\DecValTok{10}\NormalTok{)}\OperatorTok{;}
\BuiltInTok{console}\OperatorTok{.}\FunctionTok{log}\NormalTok{(}\FunctionTok{randomFillSync}\NormalTok{(buf)}\OperatorTok{.}\FunctionTok{toString}\NormalTok{(}\StringTok{\textquotesingle{}hex\textquotesingle{}}\NormalTok{))}\OperatorTok{;}

\FunctionTok{randomFillSync}\NormalTok{(buf}\OperatorTok{,} \DecValTok{5}\NormalTok{)}\OperatorTok{;}
\BuiltInTok{console}\OperatorTok{.}\FunctionTok{log}\NormalTok{(buf}\OperatorTok{.}\FunctionTok{toString}\NormalTok{(}\StringTok{\textquotesingle{}hex\textquotesingle{}}\NormalTok{))}\OperatorTok{;}

\CommentTok{// The above is equivalent to the following:}
\FunctionTok{randomFillSync}\NormalTok{(buf}\OperatorTok{,} \DecValTok{5}\OperatorTok{,} \DecValTok{5}\NormalTok{)}\OperatorTok{;}
\BuiltInTok{console}\OperatorTok{.}\FunctionTok{log}\NormalTok{(buf}\OperatorTok{.}\FunctionTok{toString}\NormalTok{(}\StringTok{\textquotesingle{}hex\textquotesingle{}}\NormalTok{))}\OperatorTok{;}
\end{Highlighting}
\end{Shaded}

Any \texttt{ArrayBuffer}, \texttt{TypedArray} or \texttt{DataView}
instance may be passed as \texttt{buffer}.

\begin{Shaded}
\begin{Highlighting}[]
\ImportTok{import}\NormalTok{ \{ }\BuiltInTok{Buffer}\NormalTok{ \} }\ImportTok{from} \StringTok{\textquotesingle{}node:buffer\textquotesingle{}}\OperatorTok{;}
\KeywordTok{const}\NormalTok{ \{ randomFillSync \} }\OperatorTok{=} \ControlFlowTok{await} \ImportTok{import}\NormalTok{(}\StringTok{\textquotesingle{}node:crypto\textquotesingle{}}\NormalTok{)}\OperatorTok{;}

\KeywordTok{const}\NormalTok{ a }\OperatorTok{=} \KeywordTok{new} \BuiltInTok{Uint32Array}\NormalTok{(}\DecValTok{10}\NormalTok{)}\OperatorTok{;}
\BuiltInTok{console}\OperatorTok{.}\FunctionTok{log}\NormalTok{(}\BuiltInTok{Buffer}\OperatorTok{.}\FunctionTok{from}\NormalTok{(}\FunctionTok{randomFillSync}\NormalTok{(a)}\OperatorTok{.}\AttributeTok{buffer}\OperatorTok{,}
\NormalTok{                        a}\OperatorTok{.}\AttributeTok{byteOffset}\OperatorTok{,}\NormalTok{ a}\OperatorTok{.}\AttributeTok{byteLength}\NormalTok{)}\OperatorTok{.}\FunctionTok{toString}\NormalTok{(}\StringTok{\textquotesingle{}hex\textquotesingle{}}\NormalTok{))}\OperatorTok{;}

\KeywordTok{const}\NormalTok{ b }\OperatorTok{=} \KeywordTok{new} \BuiltInTok{DataView}\NormalTok{(}\KeywordTok{new} \BuiltInTok{ArrayBuffer}\NormalTok{(}\DecValTok{10}\NormalTok{))}\OperatorTok{;}
\BuiltInTok{console}\OperatorTok{.}\FunctionTok{log}\NormalTok{(}\BuiltInTok{Buffer}\OperatorTok{.}\FunctionTok{from}\NormalTok{(}\FunctionTok{randomFillSync}\NormalTok{(b)}\OperatorTok{.}\AttributeTok{buffer}\OperatorTok{,}
\NormalTok{                        b}\OperatorTok{.}\AttributeTok{byteOffset}\OperatorTok{,}\NormalTok{ b}\OperatorTok{.}\AttributeTok{byteLength}\NormalTok{)}\OperatorTok{.}\FunctionTok{toString}\NormalTok{(}\StringTok{\textquotesingle{}hex\textquotesingle{}}\NormalTok{))}\OperatorTok{;}

\KeywordTok{const}\NormalTok{ c }\OperatorTok{=} \KeywordTok{new} \BuiltInTok{ArrayBuffer}\NormalTok{(}\DecValTok{10}\NormalTok{)}\OperatorTok{;}
\BuiltInTok{console}\OperatorTok{.}\FunctionTok{log}\NormalTok{(}\BuiltInTok{Buffer}\OperatorTok{.}\FunctionTok{from}\NormalTok{(}\FunctionTok{randomFillSync}\NormalTok{(c))}\OperatorTok{.}\FunctionTok{toString}\NormalTok{(}\StringTok{\textquotesingle{}hex\textquotesingle{}}\NormalTok{))}\OperatorTok{;}
\end{Highlighting}
\end{Shaded}

\begin{Shaded}
\begin{Highlighting}[]
\KeywordTok{const}\NormalTok{ \{ randomFillSync \} }\OperatorTok{=} \PreprocessorTok{require}\NormalTok{(}\StringTok{\textquotesingle{}node:crypto\textquotesingle{}}\NormalTok{)}\OperatorTok{;}
\KeywordTok{const}\NormalTok{ \{ }\BuiltInTok{Buffer}\NormalTok{ \} }\OperatorTok{=} \PreprocessorTok{require}\NormalTok{(}\StringTok{\textquotesingle{}node:buffer\textquotesingle{}}\NormalTok{)}\OperatorTok{;}

\KeywordTok{const}\NormalTok{ a }\OperatorTok{=} \KeywordTok{new} \BuiltInTok{Uint32Array}\NormalTok{(}\DecValTok{10}\NormalTok{)}\OperatorTok{;}
\BuiltInTok{console}\OperatorTok{.}\FunctionTok{log}\NormalTok{(}\BuiltInTok{Buffer}\OperatorTok{.}\FunctionTok{from}\NormalTok{(}\FunctionTok{randomFillSync}\NormalTok{(a)}\OperatorTok{.}\AttributeTok{buffer}\OperatorTok{,}
\NormalTok{                        a}\OperatorTok{.}\AttributeTok{byteOffset}\OperatorTok{,}\NormalTok{ a}\OperatorTok{.}\AttributeTok{byteLength}\NormalTok{)}\OperatorTok{.}\FunctionTok{toString}\NormalTok{(}\StringTok{\textquotesingle{}hex\textquotesingle{}}\NormalTok{))}\OperatorTok{;}

\KeywordTok{const}\NormalTok{ b }\OperatorTok{=} \KeywordTok{new} \BuiltInTok{DataView}\NormalTok{(}\KeywordTok{new} \BuiltInTok{ArrayBuffer}\NormalTok{(}\DecValTok{10}\NormalTok{))}\OperatorTok{;}
\BuiltInTok{console}\OperatorTok{.}\FunctionTok{log}\NormalTok{(}\BuiltInTok{Buffer}\OperatorTok{.}\FunctionTok{from}\NormalTok{(}\FunctionTok{randomFillSync}\NormalTok{(b)}\OperatorTok{.}\AttributeTok{buffer}\OperatorTok{,}
\NormalTok{                        b}\OperatorTok{.}\AttributeTok{byteOffset}\OperatorTok{,}\NormalTok{ b}\OperatorTok{.}\AttributeTok{byteLength}\NormalTok{)}\OperatorTok{.}\FunctionTok{toString}\NormalTok{(}\StringTok{\textquotesingle{}hex\textquotesingle{}}\NormalTok{))}\OperatorTok{;}

\KeywordTok{const}\NormalTok{ c }\OperatorTok{=} \KeywordTok{new} \BuiltInTok{ArrayBuffer}\NormalTok{(}\DecValTok{10}\NormalTok{)}\OperatorTok{;}
\BuiltInTok{console}\OperatorTok{.}\FunctionTok{log}\NormalTok{(}\BuiltInTok{Buffer}\OperatorTok{.}\FunctionTok{from}\NormalTok{(}\FunctionTok{randomFillSync}\NormalTok{(c))}\OperatorTok{.}\FunctionTok{toString}\NormalTok{(}\StringTok{\textquotesingle{}hex\textquotesingle{}}\NormalTok{))}\OperatorTok{;}
\end{Highlighting}
\end{Shaded}

\subsubsection{\texorpdfstring{\texttt{crypto.randomFill(buffer{[},\ offset{]}{[},\ size{]},\ callback)}}{crypto.randomFill(buffer{[}, offset{]}{[}, size{]}, callback)}}\label{crypto.randomfillbuffer-offset-size-callback}

\begin{itemize}
\tightlist
\item
  \texttt{buffer}
  \{ArrayBuffer\textbar Buffer\textbar TypedArray\textbar DataView\}
  Must be supplied. The size of the provided \texttt{buffer} must not be
  larger than \texttt{2**31\ -\ 1}.
\item
  \texttt{offset} \{number\} \textbf{Default:} \texttt{0}
\item
  \texttt{size} \{number\} \textbf{Default:}
  \texttt{buffer.length\ -\ offset}. The \texttt{size} must not be
  larger than \texttt{2**31\ -\ 1}.
\item
  \texttt{callback} \{Function\} \texttt{function(err,\ buf)\ \{\}}.
\end{itemize}

This function is similar to
\hyperref[cryptorandombytessize-callback]{\texttt{crypto.randomBytes()}}
but requires the first argument to be a
\href{buffer.md}{\texttt{Buffer}} that will be filled. It also requires
that a callback is passed in.

If the \texttt{callback} function is not provided, an error will be
thrown.

\begin{Shaded}
\begin{Highlighting}[]
\ImportTok{import}\NormalTok{ \{ }\BuiltInTok{Buffer}\NormalTok{ \} }\ImportTok{from} \StringTok{\textquotesingle{}node:buffer\textquotesingle{}}\OperatorTok{;}
\KeywordTok{const}\NormalTok{ \{ randomFill \} }\OperatorTok{=} \ControlFlowTok{await} \ImportTok{import}\NormalTok{(}\StringTok{\textquotesingle{}node:crypto\textquotesingle{}}\NormalTok{)}\OperatorTok{;}

\KeywordTok{const}\NormalTok{ buf }\OperatorTok{=} \BuiltInTok{Buffer}\OperatorTok{.}\FunctionTok{alloc}\NormalTok{(}\DecValTok{10}\NormalTok{)}\OperatorTok{;}
\FunctionTok{randomFill}\NormalTok{(buf}\OperatorTok{,}\NormalTok{ (err}\OperatorTok{,}\NormalTok{ buf) }\KeywordTok{=\textgreater{}}\NormalTok{ \{}
  \ControlFlowTok{if}\NormalTok{ (err) }\ControlFlowTok{throw}\NormalTok{ err}\OperatorTok{;}
  \BuiltInTok{console}\OperatorTok{.}\FunctionTok{log}\NormalTok{(buf}\OperatorTok{.}\FunctionTok{toString}\NormalTok{(}\StringTok{\textquotesingle{}hex\textquotesingle{}}\NormalTok{))}\OperatorTok{;}
\NormalTok{\})}\OperatorTok{;}

\FunctionTok{randomFill}\NormalTok{(buf}\OperatorTok{,} \DecValTok{5}\OperatorTok{,}\NormalTok{ (err}\OperatorTok{,}\NormalTok{ buf) }\KeywordTok{=\textgreater{}}\NormalTok{ \{}
  \ControlFlowTok{if}\NormalTok{ (err) }\ControlFlowTok{throw}\NormalTok{ err}\OperatorTok{;}
  \BuiltInTok{console}\OperatorTok{.}\FunctionTok{log}\NormalTok{(buf}\OperatorTok{.}\FunctionTok{toString}\NormalTok{(}\StringTok{\textquotesingle{}hex\textquotesingle{}}\NormalTok{))}\OperatorTok{;}
\NormalTok{\})}\OperatorTok{;}

\CommentTok{// The above is equivalent to the following:}
\FunctionTok{randomFill}\NormalTok{(buf}\OperatorTok{,} \DecValTok{5}\OperatorTok{,} \DecValTok{5}\OperatorTok{,}\NormalTok{ (err}\OperatorTok{,}\NormalTok{ buf) }\KeywordTok{=\textgreater{}}\NormalTok{ \{}
  \ControlFlowTok{if}\NormalTok{ (err) }\ControlFlowTok{throw}\NormalTok{ err}\OperatorTok{;}
  \BuiltInTok{console}\OperatorTok{.}\FunctionTok{log}\NormalTok{(buf}\OperatorTok{.}\FunctionTok{toString}\NormalTok{(}\StringTok{\textquotesingle{}hex\textquotesingle{}}\NormalTok{))}\OperatorTok{;}
\NormalTok{\})}\OperatorTok{;}
\end{Highlighting}
\end{Shaded}

\begin{Shaded}
\begin{Highlighting}[]
\KeywordTok{const}\NormalTok{ \{ randomFill \} }\OperatorTok{=} \PreprocessorTok{require}\NormalTok{(}\StringTok{\textquotesingle{}node:crypto\textquotesingle{}}\NormalTok{)}\OperatorTok{;}
\KeywordTok{const}\NormalTok{ \{ }\BuiltInTok{Buffer}\NormalTok{ \} }\OperatorTok{=} \PreprocessorTok{require}\NormalTok{(}\StringTok{\textquotesingle{}node:buffer\textquotesingle{}}\NormalTok{)}\OperatorTok{;}

\KeywordTok{const}\NormalTok{ buf }\OperatorTok{=} \BuiltInTok{Buffer}\OperatorTok{.}\FunctionTok{alloc}\NormalTok{(}\DecValTok{10}\NormalTok{)}\OperatorTok{;}
\FunctionTok{randomFill}\NormalTok{(buf}\OperatorTok{,}\NormalTok{ (err}\OperatorTok{,}\NormalTok{ buf) }\KeywordTok{=\textgreater{}}\NormalTok{ \{}
  \ControlFlowTok{if}\NormalTok{ (err) }\ControlFlowTok{throw}\NormalTok{ err}\OperatorTok{;}
  \BuiltInTok{console}\OperatorTok{.}\FunctionTok{log}\NormalTok{(buf}\OperatorTok{.}\FunctionTok{toString}\NormalTok{(}\StringTok{\textquotesingle{}hex\textquotesingle{}}\NormalTok{))}\OperatorTok{;}
\NormalTok{\})}\OperatorTok{;}

\FunctionTok{randomFill}\NormalTok{(buf}\OperatorTok{,} \DecValTok{5}\OperatorTok{,}\NormalTok{ (err}\OperatorTok{,}\NormalTok{ buf) }\KeywordTok{=\textgreater{}}\NormalTok{ \{}
  \ControlFlowTok{if}\NormalTok{ (err) }\ControlFlowTok{throw}\NormalTok{ err}\OperatorTok{;}
  \BuiltInTok{console}\OperatorTok{.}\FunctionTok{log}\NormalTok{(buf}\OperatorTok{.}\FunctionTok{toString}\NormalTok{(}\StringTok{\textquotesingle{}hex\textquotesingle{}}\NormalTok{))}\OperatorTok{;}
\NormalTok{\})}\OperatorTok{;}

\CommentTok{// The above is equivalent to the following:}
\FunctionTok{randomFill}\NormalTok{(buf}\OperatorTok{,} \DecValTok{5}\OperatorTok{,} \DecValTok{5}\OperatorTok{,}\NormalTok{ (err}\OperatorTok{,}\NormalTok{ buf) }\KeywordTok{=\textgreater{}}\NormalTok{ \{}
  \ControlFlowTok{if}\NormalTok{ (err) }\ControlFlowTok{throw}\NormalTok{ err}\OperatorTok{;}
  \BuiltInTok{console}\OperatorTok{.}\FunctionTok{log}\NormalTok{(buf}\OperatorTok{.}\FunctionTok{toString}\NormalTok{(}\StringTok{\textquotesingle{}hex\textquotesingle{}}\NormalTok{))}\OperatorTok{;}
\NormalTok{\})}\OperatorTok{;}
\end{Highlighting}
\end{Shaded}

Any \texttt{ArrayBuffer}, \texttt{TypedArray}, or \texttt{DataView}
instance may be passed as \texttt{buffer}.

While this includes instances of \texttt{Float32Array} and
\texttt{Float64Array}, this function should not be used to generate
random floating-point numbers. The result may contain
\texttt{+Infinity}, \texttt{-Infinity}, and \texttt{NaN}, and even if
the array contains finite numbers only, they are not drawn from a
uniform random distribution and have no meaningful lower or upper
bounds.

\begin{Shaded}
\begin{Highlighting}[]
\ImportTok{import}\NormalTok{ \{ }\BuiltInTok{Buffer}\NormalTok{ \} }\ImportTok{from} \StringTok{\textquotesingle{}node:buffer\textquotesingle{}}\OperatorTok{;}
\KeywordTok{const}\NormalTok{ \{ randomFill \} }\OperatorTok{=} \ControlFlowTok{await} \ImportTok{import}\NormalTok{(}\StringTok{\textquotesingle{}node:crypto\textquotesingle{}}\NormalTok{)}\OperatorTok{;}

\KeywordTok{const}\NormalTok{ a }\OperatorTok{=} \KeywordTok{new} \BuiltInTok{Uint32Array}\NormalTok{(}\DecValTok{10}\NormalTok{)}\OperatorTok{;}
\FunctionTok{randomFill}\NormalTok{(a}\OperatorTok{,}\NormalTok{ (err}\OperatorTok{,}\NormalTok{ buf) }\KeywordTok{=\textgreater{}}\NormalTok{ \{}
  \ControlFlowTok{if}\NormalTok{ (err) }\ControlFlowTok{throw}\NormalTok{ err}\OperatorTok{;}
  \BuiltInTok{console}\OperatorTok{.}\FunctionTok{log}\NormalTok{(}\BuiltInTok{Buffer}\OperatorTok{.}\FunctionTok{from}\NormalTok{(buf}\OperatorTok{.}\AttributeTok{buffer}\OperatorTok{,}\NormalTok{ buf}\OperatorTok{.}\AttributeTok{byteOffset}\OperatorTok{,}\NormalTok{ buf}\OperatorTok{.}\AttributeTok{byteLength}\NormalTok{)}
    \OperatorTok{.}\FunctionTok{toString}\NormalTok{(}\StringTok{\textquotesingle{}hex\textquotesingle{}}\NormalTok{))}\OperatorTok{;}
\NormalTok{\})}\OperatorTok{;}

\KeywordTok{const}\NormalTok{ b }\OperatorTok{=} \KeywordTok{new} \BuiltInTok{DataView}\NormalTok{(}\KeywordTok{new} \BuiltInTok{ArrayBuffer}\NormalTok{(}\DecValTok{10}\NormalTok{))}\OperatorTok{;}
\FunctionTok{randomFill}\NormalTok{(b}\OperatorTok{,}\NormalTok{ (err}\OperatorTok{,}\NormalTok{ buf) }\KeywordTok{=\textgreater{}}\NormalTok{ \{}
  \ControlFlowTok{if}\NormalTok{ (err) }\ControlFlowTok{throw}\NormalTok{ err}\OperatorTok{;}
  \BuiltInTok{console}\OperatorTok{.}\FunctionTok{log}\NormalTok{(}\BuiltInTok{Buffer}\OperatorTok{.}\FunctionTok{from}\NormalTok{(buf}\OperatorTok{.}\AttributeTok{buffer}\OperatorTok{,}\NormalTok{ buf}\OperatorTok{.}\AttributeTok{byteOffset}\OperatorTok{,}\NormalTok{ buf}\OperatorTok{.}\AttributeTok{byteLength}\NormalTok{)}
    \OperatorTok{.}\FunctionTok{toString}\NormalTok{(}\StringTok{\textquotesingle{}hex\textquotesingle{}}\NormalTok{))}\OperatorTok{;}
\NormalTok{\})}\OperatorTok{;}

\KeywordTok{const}\NormalTok{ c }\OperatorTok{=} \KeywordTok{new} \BuiltInTok{ArrayBuffer}\NormalTok{(}\DecValTok{10}\NormalTok{)}\OperatorTok{;}
\FunctionTok{randomFill}\NormalTok{(c}\OperatorTok{,}\NormalTok{ (err}\OperatorTok{,}\NormalTok{ buf) }\KeywordTok{=\textgreater{}}\NormalTok{ \{}
  \ControlFlowTok{if}\NormalTok{ (err) }\ControlFlowTok{throw}\NormalTok{ err}\OperatorTok{;}
  \BuiltInTok{console}\OperatorTok{.}\FunctionTok{log}\NormalTok{(}\BuiltInTok{Buffer}\OperatorTok{.}\FunctionTok{from}\NormalTok{(buf)}\OperatorTok{.}\FunctionTok{toString}\NormalTok{(}\StringTok{\textquotesingle{}hex\textquotesingle{}}\NormalTok{))}\OperatorTok{;}
\NormalTok{\})}\OperatorTok{;}
\end{Highlighting}
\end{Shaded}

\begin{Shaded}
\begin{Highlighting}[]
\KeywordTok{const}\NormalTok{ \{ randomFill \} }\OperatorTok{=} \PreprocessorTok{require}\NormalTok{(}\StringTok{\textquotesingle{}node:crypto\textquotesingle{}}\NormalTok{)}\OperatorTok{;}
\KeywordTok{const}\NormalTok{ \{ }\BuiltInTok{Buffer}\NormalTok{ \} }\OperatorTok{=} \PreprocessorTok{require}\NormalTok{(}\StringTok{\textquotesingle{}node:buffer\textquotesingle{}}\NormalTok{)}\OperatorTok{;}

\KeywordTok{const}\NormalTok{ a }\OperatorTok{=} \KeywordTok{new} \BuiltInTok{Uint32Array}\NormalTok{(}\DecValTok{10}\NormalTok{)}\OperatorTok{;}
\FunctionTok{randomFill}\NormalTok{(a}\OperatorTok{,}\NormalTok{ (err}\OperatorTok{,}\NormalTok{ buf) }\KeywordTok{=\textgreater{}}\NormalTok{ \{}
  \ControlFlowTok{if}\NormalTok{ (err) }\ControlFlowTok{throw}\NormalTok{ err}\OperatorTok{;}
  \BuiltInTok{console}\OperatorTok{.}\FunctionTok{log}\NormalTok{(}\BuiltInTok{Buffer}\OperatorTok{.}\FunctionTok{from}\NormalTok{(buf}\OperatorTok{.}\AttributeTok{buffer}\OperatorTok{,}\NormalTok{ buf}\OperatorTok{.}\AttributeTok{byteOffset}\OperatorTok{,}\NormalTok{ buf}\OperatorTok{.}\AttributeTok{byteLength}\NormalTok{)}
    \OperatorTok{.}\FunctionTok{toString}\NormalTok{(}\StringTok{\textquotesingle{}hex\textquotesingle{}}\NormalTok{))}\OperatorTok{;}
\NormalTok{\})}\OperatorTok{;}

\KeywordTok{const}\NormalTok{ b }\OperatorTok{=} \KeywordTok{new} \BuiltInTok{DataView}\NormalTok{(}\KeywordTok{new} \BuiltInTok{ArrayBuffer}\NormalTok{(}\DecValTok{10}\NormalTok{))}\OperatorTok{;}
\FunctionTok{randomFill}\NormalTok{(b}\OperatorTok{,}\NormalTok{ (err}\OperatorTok{,}\NormalTok{ buf) }\KeywordTok{=\textgreater{}}\NormalTok{ \{}
  \ControlFlowTok{if}\NormalTok{ (err) }\ControlFlowTok{throw}\NormalTok{ err}\OperatorTok{;}
  \BuiltInTok{console}\OperatorTok{.}\FunctionTok{log}\NormalTok{(}\BuiltInTok{Buffer}\OperatorTok{.}\FunctionTok{from}\NormalTok{(buf}\OperatorTok{.}\AttributeTok{buffer}\OperatorTok{,}\NormalTok{ buf}\OperatorTok{.}\AttributeTok{byteOffset}\OperatorTok{,}\NormalTok{ buf}\OperatorTok{.}\AttributeTok{byteLength}\NormalTok{)}
    \OperatorTok{.}\FunctionTok{toString}\NormalTok{(}\StringTok{\textquotesingle{}hex\textquotesingle{}}\NormalTok{))}\OperatorTok{;}
\NormalTok{\})}\OperatorTok{;}

\KeywordTok{const}\NormalTok{ c }\OperatorTok{=} \KeywordTok{new} \BuiltInTok{ArrayBuffer}\NormalTok{(}\DecValTok{10}\NormalTok{)}\OperatorTok{;}
\FunctionTok{randomFill}\NormalTok{(c}\OperatorTok{,}\NormalTok{ (err}\OperatorTok{,}\NormalTok{ buf) }\KeywordTok{=\textgreater{}}\NormalTok{ \{}
  \ControlFlowTok{if}\NormalTok{ (err) }\ControlFlowTok{throw}\NormalTok{ err}\OperatorTok{;}
  \BuiltInTok{console}\OperatorTok{.}\FunctionTok{log}\NormalTok{(}\BuiltInTok{Buffer}\OperatorTok{.}\FunctionTok{from}\NormalTok{(buf)}\OperatorTok{.}\FunctionTok{toString}\NormalTok{(}\StringTok{\textquotesingle{}hex\textquotesingle{}}\NormalTok{))}\OperatorTok{;}
\NormalTok{\})}\OperatorTok{;}
\end{Highlighting}
\end{Shaded}

This API uses libuv's threadpool, which can have surprising and negative
performance implications for some applications; see the
\href{cli.md\#uv_threadpool_sizesize}{\texttt{UV\_THREADPOOL\_SIZE}}
documentation for more information.

The asynchronous version of \texttt{crypto.randomFill()} is carried out
in a single threadpool request. To minimize threadpool task length
variation, partition large \texttt{randomFill} requests when doing so as
part of fulfilling a client request.

\subsubsection{\texorpdfstring{\texttt{crypto.randomInt({[}min,\ {]}max{[},\ callback{]})}}{crypto.randomInt({[}min, {]}max{[}, callback{]})}}\label{crypto.randomintmin-max-callback}

\begin{itemize}
\tightlist
\item
  \texttt{min} \{integer\} Start of random range (inclusive).
  \textbf{Default:} \texttt{0}.
\item
  \texttt{max} \{integer\} End of random range (exclusive).
\item
  \texttt{callback} \{Function\} \texttt{function(err,\ n)\ \{\}}.
\end{itemize}

Return a random integer \texttt{n} such that
\texttt{min\ \textless{}=\ n\ \textless{}\ max}. This implementation
avoids
\href{https://en.wikipedia.org/wiki/Fisher\%E2\%80\%93Yates_shuffle\#Modulo_bias}{modulo
bias}.

The range (\texttt{max\ -\ min}) must be less than 248. \texttt{min} and
\texttt{max} must be
\href{https://developer.mozilla.org/en-US/docs/Web/JavaScript/Reference/Global_Objects/Number/isSafeInteger}{safe
integers}.

If the \texttt{callback} function is not provided, the random integer is
generated synchronously.

\begin{Shaded}
\begin{Highlighting}[]
\CommentTok{// Asynchronous}
\KeywordTok{const}\NormalTok{ \{}
\NormalTok{  randomInt}\OperatorTok{,}
\NormalTok{\} }\OperatorTok{=} \ControlFlowTok{await} \ImportTok{import}\NormalTok{(}\StringTok{\textquotesingle{}node:crypto\textquotesingle{}}\NormalTok{)}\OperatorTok{;}

\FunctionTok{randomInt}\NormalTok{(}\DecValTok{3}\OperatorTok{,}\NormalTok{ (err}\OperatorTok{,}\NormalTok{ n) }\KeywordTok{=\textgreater{}}\NormalTok{ \{}
  \ControlFlowTok{if}\NormalTok{ (err) }\ControlFlowTok{throw}\NormalTok{ err}\OperatorTok{;}
  \BuiltInTok{console}\OperatorTok{.}\FunctionTok{log}\NormalTok{(}\VerbatimStringTok{\textasciigrave{}Random number chosen from (0, 1, 2): }\SpecialCharTok{$\{}\NormalTok{n}\SpecialCharTok{\}}\VerbatimStringTok{\textasciigrave{}}\NormalTok{)}\OperatorTok{;}
\NormalTok{\})}\OperatorTok{;}
\end{Highlighting}
\end{Shaded}

\begin{Shaded}
\begin{Highlighting}[]
\CommentTok{// Asynchronous}
\KeywordTok{const}\NormalTok{ \{}
\NormalTok{  randomInt}\OperatorTok{,}
\NormalTok{\} }\OperatorTok{=} \PreprocessorTok{require}\NormalTok{(}\StringTok{\textquotesingle{}node:crypto\textquotesingle{}}\NormalTok{)}\OperatorTok{;}

\FunctionTok{randomInt}\NormalTok{(}\DecValTok{3}\OperatorTok{,}\NormalTok{ (err}\OperatorTok{,}\NormalTok{ n) }\KeywordTok{=\textgreater{}}\NormalTok{ \{}
  \ControlFlowTok{if}\NormalTok{ (err) }\ControlFlowTok{throw}\NormalTok{ err}\OperatorTok{;}
  \BuiltInTok{console}\OperatorTok{.}\FunctionTok{log}\NormalTok{(}\VerbatimStringTok{\textasciigrave{}Random number chosen from (0, 1, 2): }\SpecialCharTok{$\{}\NormalTok{n}\SpecialCharTok{\}}\VerbatimStringTok{\textasciigrave{}}\NormalTok{)}\OperatorTok{;}
\NormalTok{\})}\OperatorTok{;}
\end{Highlighting}
\end{Shaded}

\begin{Shaded}
\begin{Highlighting}[]
\CommentTok{// Synchronous}
\KeywordTok{const}\NormalTok{ \{}
\NormalTok{  randomInt}\OperatorTok{,}
\NormalTok{\} }\OperatorTok{=} \ControlFlowTok{await} \ImportTok{import}\NormalTok{(}\StringTok{\textquotesingle{}node:crypto\textquotesingle{}}\NormalTok{)}\OperatorTok{;}

\KeywordTok{const}\NormalTok{ n }\OperatorTok{=} \FunctionTok{randomInt}\NormalTok{(}\DecValTok{3}\NormalTok{)}\OperatorTok{;}
\BuiltInTok{console}\OperatorTok{.}\FunctionTok{log}\NormalTok{(}\VerbatimStringTok{\textasciigrave{}Random number chosen from (0, 1, 2): }\SpecialCharTok{$\{}\NormalTok{n}\SpecialCharTok{\}}\VerbatimStringTok{\textasciigrave{}}\NormalTok{)}\OperatorTok{;}
\end{Highlighting}
\end{Shaded}

\begin{Shaded}
\begin{Highlighting}[]
\CommentTok{// Synchronous}
\KeywordTok{const}\NormalTok{ \{}
\NormalTok{  randomInt}\OperatorTok{,}
\NormalTok{\} }\OperatorTok{=} \PreprocessorTok{require}\NormalTok{(}\StringTok{\textquotesingle{}node:crypto\textquotesingle{}}\NormalTok{)}\OperatorTok{;}

\KeywordTok{const}\NormalTok{ n }\OperatorTok{=} \FunctionTok{randomInt}\NormalTok{(}\DecValTok{3}\NormalTok{)}\OperatorTok{;}
\BuiltInTok{console}\OperatorTok{.}\FunctionTok{log}\NormalTok{(}\VerbatimStringTok{\textasciigrave{}Random number chosen from (0, 1, 2): }\SpecialCharTok{$\{}\NormalTok{n}\SpecialCharTok{\}}\VerbatimStringTok{\textasciigrave{}}\NormalTok{)}\OperatorTok{;}
\end{Highlighting}
\end{Shaded}

\begin{Shaded}
\begin{Highlighting}[]
\CommentTok{// With \textasciigrave{}min\textasciigrave{} argument}
\KeywordTok{const}\NormalTok{ \{}
\NormalTok{  randomInt}\OperatorTok{,}
\NormalTok{\} }\OperatorTok{=} \ControlFlowTok{await} \ImportTok{import}\NormalTok{(}\StringTok{\textquotesingle{}node:crypto\textquotesingle{}}\NormalTok{)}\OperatorTok{;}

\KeywordTok{const}\NormalTok{ n }\OperatorTok{=} \FunctionTok{randomInt}\NormalTok{(}\DecValTok{1}\OperatorTok{,} \DecValTok{7}\NormalTok{)}\OperatorTok{;}
\BuiltInTok{console}\OperatorTok{.}\FunctionTok{log}\NormalTok{(}\VerbatimStringTok{\textasciigrave{}The dice rolled: }\SpecialCharTok{$\{}\NormalTok{n}\SpecialCharTok{\}}\VerbatimStringTok{\textasciigrave{}}\NormalTok{)}\OperatorTok{;}
\end{Highlighting}
\end{Shaded}

\begin{Shaded}
\begin{Highlighting}[]
\CommentTok{// With \textasciigrave{}min\textasciigrave{} argument}
\KeywordTok{const}\NormalTok{ \{}
\NormalTok{  randomInt}\OperatorTok{,}
\NormalTok{\} }\OperatorTok{=} \PreprocessorTok{require}\NormalTok{(}\StringTok{\textquotesingle{}node:crypto\textquotesingle{}}\NormalTok{)}\OperatorTok{;}

\KeywordTok{const}\NormalTok{ n }\OperatorTok{=} \FunctionTok{randomInt}\NormalTok{(}\DecValTok{1}\OperatorTok{,} \DecValTok{7}\NormalTok{)}\OperatorTok{;}
\BuiltInTok{console}\OperatorTok{.}\FunctionTok{log}\NormalTok{(}\VerbatimStringTok{\textasciigrave{}The dice rolled: }\SpecialCharTok{$\{}\NormalTok{n}\SpecialCharTok{\}}\VerbatimStringTok{\textasciigrave{}}\NormalTok{)}\OperatorTok{;}
\end{Highlighting}
\end{Shaded}

\subsubsection{\texorpdfstring{\texttt{crypto.randomUUID({[}options{]})}}{crypto.randomUUID({[}options{]})}}\label{crypto.randomuuidoptions}

\begin{itemize}
\tightlist
\item
  \texttt{options} \{Object\}

  \begin{itemize}
  \tightlist
  \item
    \texttt{disableEntropyCache} \{boolean\} By default, to improve
    performance, Node.js generates and caches enough random data to
    generate up to 128 random UUIDs. To generate a UUID without using
    the cache, set \texttt{disableEntropyCache} to \texttt{true}.
    \textbf{Default:} \texttt{false}.
  \end{itemize}
\item
  Returns: \{string\}
\end{itemize}

Generates a random \href{https://www.rfc-editor.org/rfc/rfc4122.txt}{RFC
4122} version 4 UUID. The UUID is generated using a cryptographic
pseudorandom number generator.

\subsubsection{\texorpdfstring{\texttt{crypto.scrypt(password,\ salt,\ keylen{[},\ options{]},\ callback)}}{crypto.scrypt(password, salt, keylen{[}, options{]}, callback)}}\label{crypto.scryptpassword-salt-keylen-options-callback}

\begin{itemize}
\tightlist
\item
  \texttt{password}
  \{string\textbar ArrayBuffer\textbar Buffer\textbar TypedArray\textbar DataView\}
\item
  \texttt{salt}
  \{string\textbar ArrayBuffer\textbar Buffer\textbar TypedArray\textbar DataView\}
\item
  \texttt{keylen} \{number\}
\item
  \texttt{options} \{Object\}

  \begin{itemize}
  \tightlist
  \item
    \texttt{cost} \{number\} CPU/memory cost parameter. Must be a power
    of two greater than one. \textbf{Default:} \texttt{16384}.
  \item
    \texttt{blockSize} \{number\} Block size parameter.
    \textbf{Default:} \texttt{8}.
  \item
    \texttt{parallelization} \{number\} Parallelization parameter.
    \textbf{Default:} \texttt{1}.
  \item
    \texttt{N} \{number\} Alias for \texttt{cost}. Only one of both may
    be specified.
  \item
    \texttt{r} \{number\} Alias for \texttt{blockSize}. Only one of both
    may be specified.
  \item
    \texttt{p} \{number\} Alias for \texttt{parallelization}. Only one
    of both may be specified.
  \item
    \texttt{maxmem} \{number\} Memory upper bound. It is an error when
    (approximately) \texttt{128\ *\ N\ *\ r\ \textgreater{}\ maxmem}.
    \textbf{Default:} \texttt{32\ *\ 1024\ *\ 1024}.
  \end{itemize}
\item
  \texttt{callback} \{Function\}

  \begin{itemize}
  \tightlist
  \item
    \texttt{err} \{Error\}
  \item
    \texttt{derivedKey} \{Buffer\}
  \end{itemize}
\end{itemize}

Provides an asynchronous
\href{https://en.wikipedia.org/wiki/Scrypt}{scrypt} implementation.
Scrypt is a password-based key derivation function that is designed to
be expensive computationally and memory-wise in order to make
brute-force attacks unrewarding.

The \texttt{salt} should be as unique as possible. It is recommended
that a salt is random and at least 16 bytes long. See
\href{https://nvlpubs.nist.gov/nistpubs/Legacy/SP/nistspecialpublication800-132.pdf}{NIST
SP 800-132} for details.

When passing strings for \texttt{password} or \texttt{salt}, please
consider
\hyperref[using-strings-as-inputs-to-cryptographic-apis]{caveats when
using strings as inputs to cryptographic APIs}.

The \texttt{callback} function is called with two arguments:
\texttt{err} and \texttt{derivedKey}. \texttt{err} is an exception
object when key derivation fails, otherwise \texttt{err} is
\texttt{null}. \texttt{derivedKey} is passed to the callback as a
\href{buffer.md}{\texttt{Buffer}}.

An exception is thrown when any of the input arguments specify invalid
values or types.

\begin{Shaded}
\begin{Highlighting}[]
\KeywordTok{const}\NormalTok{ \{}
\NormalTok{  scrypt}\OperatorTok{,}
\NormalTok{\} }\OperatorTok{=} \ControlFlowTok{await} \ImportTok{import}\NormalTok{(}\StringTok{\textquotesingle{}node:crypto\textquotesingle{}}\NormalTok{)}\OperatorTok{;}

\CommentTok{// Using the factory defaults.}
\FunctionTok{scrypt}\NormalTok{(}\StringTok{\textquotesingle{}password\textquotesingle{}}\OperatorTok{,} \StringTok{\textquotesingle{}salt\textquotesingle{}}\OperatorTok{,} \DecValTok{64}\OperatorTok{,}\NormalTok{ (err}\OperatorTok{,}\NormalTok{ derivedKey) }\KeywordTok{=\textgreater{}}\NormalTok{ \{}
  \ControlFlowTok{if}\NormalTok{ (err) }\ControlFlowTok{throw}\NormalTok{ err}\OperatorTok{;}
  \BuiltInTok{console}\OperatorTok{.}\FunctionTok{log}\NormalTok{(derivedKey}\OperatorTok{.}\FunctionTok{toString}\NormalTok{(}\StringTok{\textquotesingle{}hex\textquotesingle{}}\NormalTok{))}\OperatorTok{;}  \CommentTok{// \textquotesingle{}3745e48...08d59ae\textquotesingle{}}
\NormalTok{\})}\OperatorTok{;}
\CommentTok{// Using a custom N parameter. Must be a power of two.}
\FunctionTok{scrypt}\NormalTok{(}\StringTok{\textquotesingle{}password\textquotesingle{}}\OperatorTok{,} \StringTok{\textquotesingle{}salt\textquotesingle{}}\OperatorTok{,} \DecValTok{64}\OperatorTok{,}\NormalTok{ \{ }\DataTypeTok{N}\OperatorTok{:} \DecValTok{1024}\NormalTok{ \}}\OperatorTok{,}\NormalTok{ (err}\OperatorTok{,}\NormalTok{ derivedKey) }\KeywordTok{=\textgreater{}}\NormalTok{ \{}
  \ControlFlowTok{if}\NormalTok{ (err) }\ControlFlowTok{throw}\NormalTok{ err}\OperatorTok{;}
  \BuiltInTok{console}\OperatorTok{.}\FunctionTok{log}\NormalTok{(derivedKey}\OperatorTok{.}\FunctionTok{toString}\NormalTok{(}\StringTok{\textquotesingle{}hex\textquotesingle{}}\NormalTok{))}\OperatorTok{;}  \CommentTok{// \textquotesingle{}3745e48...aa39b34\textquotesingle{}}
\NormalTok{\})}\OperatorTok{;}
\end{Highlighting}
\end{Shaded}

\begin{Shaded}
\begin{Highlighting}[]
\KeywordTok{const}\NormalTok{ \{}
\NormalTok{  scrypt}\OperatorTok{,}
\NormalTok{\} }\OperatorTok{=} \PreprocessorTok{require}\NormalTok{(}\StringTok{\textquotesingle{}node:crypto\textquotesingle{}}\NormalTok{)}\OperatorTok{;}

\CommentTok{// Using the factory defaults.}
\FunctionTok{scrypt}\NormalTok{(}\StringTok{\textquotesingle{}password\textquotesingle{}}\OperatorTok{,} \StringTok{\textquotesingle{}salt\textquotesingle{}}\OperatorTok{,} \DecValTok{64}\OperatorTok{,}\NormalTok{ (err}\OperatorTok{,}\NormalTok{ derivedKey) }\KeywordTok{=\textgreater{}}\NormalTok{ \{}
  \ControlFlowTok{if}\NormalTok{ (err) }\ControlFlowTok{throw}\NormalTok{ err}\OperatorTok{;}
  \BuiltInTok{console}\OperatorTok{.}\FunctionTok{log}\NormalTok{(derivedKey}\OperatorTok{.}\FunctionTok{toString}\NormalTok{(}\StringTok{\textquotesingle{}hex\textquotesingle{}}\NormalTok{))}\OperatorTok{;}  \CommentTok{// \textquotesingle{}3745e48...08d59ae\textquotesingle{}}
\NormalTok{\})}\OperatorTok{;}
\CommentTok{// Using a custom N parameter. Must be a power of two.}
\FunctionTok{scrypt}\NormalTok{(}\StringTok{\textquotesingle{}password\textquotesingle{}}\OperatorTok{,} \StringTok{\textquotesingle{}salt\textquotesingle{}}\OperatorTok{,} \DecValTok{64}\OperatorTok{,}\NormalTok{ \{ }\DataTypeTok{N}\OperatorTok{:} \DecValTok{1024}\NormalTok{ \}}\OperatorTok{,}\NormalTok{ (err}\OperatorTok{,}\NormalTok{ derivedKey) }\KeywordTok{=\textgreater{}}\NormalTok{ \{}
  \ControlFlowTok{if}\NormalTok{ (err) }\ControlFlowTok{throw}\NormalTok{ err}\OperatorTok{;}
  \BuiltInTok{console}\OperatorTok{.}\FunctionTok{log}\NormalTok{(derivedKey}\OperatorTok{.}\FunctionTok{toString}\NormalTok{(}\StringTok{\textquotesingle{}hex\textquotesingle{}}\NormalTok{))}\OperatorTok{;}  \CommentTok{// \textquotesingle{}3745e48...aa39b34\textquotesingle{}}
\NormalTok{\})}\OperatorTok{;}
\end{Highlighting}
\end{Shaded}

\subsubsection{\texorpdfstring{\texttt{crypto.scryptSync(password,\ salt,\ keylen{[},\ options{]})}}{crypto.scryptSync(password, salt, keylen{[}, options{]})}}\label{crypto.scryptsyncpassword-salt-keylen-options}

\begin{itemize}
\tightlist
\item
  \texttt{password}
  \{string\textbar Buffer\textbar TypedArray\textbar DataView\}
\item
  \texttt{salt}
  \{string\textbar Buffer\textbar TypedArray\textbar DataView\}
\item
  \texttt{keylen} \{number\}
\item
  \texttt{options} \{Object\}

  \begin{itemize}
  \tightlist
  \item
    \texttt{cost} \{number\} CPU/memory cost parameter. Must be a power
    of two greater than one. \textbf{Default:} \texttt{16384}.
  \item
    \texttt{blockSize} \{number\} Block size parameter.
    \textbf{Default:} \texttt{8}.
  \item
    \texttt{parallelization} \{number\} Parallelization parameter.
    \textbf{Default:} \texttt{1}.
  \item
    \texttt{N} \{number\} Alias for \texttt{cost}. Only one of both may
    be specified.
  \item
    \texttt{r} \{number\} Alias for \texttt{blockSize}. Only one of both
    may be specified.
  \item
    \texttt{p} \{number\} Alias for \texttt{parallelization}. Only one
    of both may be specified.
  \item
    \texttt{maxmem} \{number\} Memory upper bound. It is an error when
    (approximately) \texttt{128\ *\ N\ *\ r\ \textgreater{}\ maxmem}.
    \textbf{Default:} \texttt{32\ *\ 1024\ *\ 1024}.
  \end{itemize}
\item
  Returns: \{Buffer\}
\end{itemize}

Provides a synchronous
\href{https://en.wikipedia.org/wiki/Scrypt}{scrypt} implementation.
Scrypt is a password-based key derivation function that is designed to
be expensive computationally and memory-wise in order to make
brute-force attacks unrewarding.

The \texttt{salt} should be as unique as possible. It is recommended
that a salt is random and at least 16 bytes long. See
\href{https://nvlpubs.nist.gov/nistpubs/Legacy/SP/nistspecialpublication800-132.pdf}{NIST
SP 800-132} for details.

When passing strings for \texttt{password} or \texttt{salt}, please
consider
\hyperref[using-strings-as-inputs-to-cryptographic-apis]{caveats when
using strings as inputs to cryptographic APIs}.

An exception is thrown when key derivation fails, otherwise the derived
key is returned as a \href{buffer.md}{\texttt{Buffer}}.

An exception is thrown when any of the input arguments specify invalid
values or types.

\begin{Shaded}
\begin{Highlighting}[]
\KeywordTok{const}\NormalTok{ \{}
\NormalTok{  scryptSync}\OperatorTok{,}
\NormalTok{\} }\OperatorTok{=} \ControlFlowTok{await} \ImportTok{import}\NormalTok{(}\StringTok{\textquotesingle{}node:crypto\textquotesingle{}}\NormalTok{)}\OperatorTok{;}
\CommentTok{// Using the factory defaults.}

\KeywordTok{const}\NormalTok{ key1 }\OperatorTok{=} \FunctionTok{scryptSync}\NormalTok{(}\StringTok{\textquotesingle{}password\textquotesingle{}}\OperatorTok{,} \StringTok{\textquotesingle{}salt\textquotesingle{}}\OperatorTok{,} \DecValTok{64}\NormalTok{)}\OperatorTok{;}
\BuiltInTok{console}\OperatorTok{.}\FunctionTok{log}\NormalTok{(key1}\OperatorTok{.}\FunctionTok{toString}\NormalTok{(}\StringTok{\textquotesingle{}hex\textquotesingle{}}\NormalTok{))}\OperatorTok{;}  \CommentTok{// \textquotesingle{}3745e48...08d59ae\textquotesingle{}}
\CommentTok{// Using a custom N parameter. Must be a power of two.}
\KeywordTok{const}\NormalTok{ key2 }\OperatorTok{=} \FunctionTok{scryptSync}\NormalTok{(}\StringTok{\textquotesingle{}password\textquotesingle{}}\OperatorTok{,} \StringTok{\textquotesingle{}salt\textquotesingle{}}\OperatorTok{,} \DecValTok{64}\OperatorTok{,}\NormalTok{ \{ }\DataTypeTok{N}\OperatorTok{:} \DecValTok{1024}\NormalTok{ \})}\OperatorTok{;}
\BuiltInTok{console}\OperatorTok{.}\FunctionTok{log}\NormalTok{(key2}\OperatorTok{.}\FunctionTok{toString}\NormalTok{(}\StringTok{\textquotesingle{}hex\textquotesingle{}}\NormalTok{))}\OperatorTok{;}  \CommentTok{// \textquotesingle{}3745e48...aa39b34\textquotesingle{}}
\end{Highlighting}
\end{Shaded}

\begin{Shaded}
\begin{Highlighting}[]
\KeywordTok{const}\NormalTok{ \{}
\NormalTok{  scryptSync}\OperatorTok{,}
\NormalTok{\} }\OperatorTok{=} \PreprocessorTok{require}\NormalTok{(}\StringTok{\textquotesingle{}node:crypto\textquotesingle{}}\NormalTok{)}\OperatorTok{;}
\CommentTok{// Using the factory defaults.}

\KeywordTok{const}\NormalTok{ key1 }\OperatorTok{=} \FunctionTok{scryptSync}\NormalTok{(}\StringTok{\textquotesingle{}password\textquotesingle{}}\OperatorTok{,} \StringTok{\textquotesingle{}salt\textquotesingle{}}\OperatorTok{,} \DecValTok{64}\NormalTok{)}\OperatorTok{;}
\BuiltInTok{console}\OperatorTok{.}\FunctionTok{log}\NormalTok{(key1}\OperatorTok{.}\FunctionTok{toString}\NormalTok{(}\StringTok{\textquotesingle{}hex\textquotesingle{}}\NormalTok{))}\OperatorTok{;}  \CommentTok{// \textquotesingle{}3745e48...08d59ae\textquotesingle{}}
\CommentTok{// Using a custom N parameter. Must be a power of two.}
\KeywordTok{const}\NormalTok{ key2 }\OperatorTok{=} \FunctionTok{scryptSync}\NormalTok{(}\StringTok{\textquotesingle{}password\textquotesingle{}}\OperatorTok{,} \StringTok{\textquotesingle{}salt\textquotesingle{}}\OperatorTok{,} \DecValTok{64}\OperatorTok{,}\NormalTok{ \{ }\DataTypeTok{N}\OperatorTok{:} \DecValTok{1024}\NormalTok{ \})}\OperatorTok{;}
\BuiltInTok{console}\OperatorTok{.}\FunctionTok{log}\NormalTok{(key2}\OperatorTok{.}\FunctionTok{toString}\NormalTok{(}\StringTok{\textquotesingle{}hex\textquotesingle{}}\NormalTok{))}\OperatorTok{;}  \CommentTok{// \textquotesingle{}3745e48...aa39b34\textquotesingle{}}
\end{Highlighting}
\end{Shaded}

\subsubsection{\texorpdfstring{\texttt{crypto.secureHeapUsed()}}{crypto.secureHeapUsed()}}\label{crypto.secureheapused}

\begin{itemize}
\tightlist
\item
  Returns: \{Object\}

  \begin{itemize}
  \tightlist
  \item
    \texttt{total} \{number\} The total allocated secure heap size as
    specified using the \texttt{-\/-secure-heap=n} command-line flag.
  \item
    \texttt{min} \{number\} The minimum allocation from the secure heap
    as specified using the \texttt{-\/-secure-heap-min} command-line
    flag.
  \item
    \texttt{used} \{number\} The total number of bytes currently
    allocated from the secure heap.
  \item
    \texttt{utilization} \{number\} The calculated ratio of
    \texttt{used} to \texttt{total} allocated bytes.
  \end{itemize}
\end{itemize}

\subsubsection{\texorpdfstring{\texttt{crypto.setEngine(engine{[},\ flags{]})}}{crypto.setEngine(engine{[}, flags{]})}}\label{crypto.setengineengine-flags}

\begin{itemize}
\tightlist
\item
  \texttt{engine} \{string\}
\item
  \texttt{flags} \{crypto.constants\} \textbf{Default:}
  \texttt{crypto.constants.ENGINE\_METHOD\_ALL}
\end{itemize}

Load and set the \texttt{engine} for some or all OpenSSL functions
(selected by flags).

\texttt{engine} could be either an id or a path to the engine's shared
library.

The optional \texttt{flags} argument uses \texttt{ENGINE\_METHOD\_ALL}
by default. The \texttt{flags} is a bit field taking one of or a mix of
the following flags (defined in \texttt{crypto.constants}):

\begin{itemize}
\tightlist
\item
  \texttt{crypto.constants.ENGINE\_METHOD\_RSA}
\item
  \texttt{crypto.constants.ENGINE\_METHOD\_DSA}
\item
  \texttt{crypto.constants.ENGINE\_METHOD\_DH}
\item
  \texttt{crypto.constants.ENGINE\_METHOD\_RAND}
\item
  \texttt{crypto.constants.ENGINE\_METHOD\_EC}
\item
  \texttt{crypto.constants.ENGINE\_METHOD\_CIPHERS}
\item
  \texttt{crypto.constants.ENGINE\_METHOD\_DIGESTS}
\item
  \texttt{crypto.constants.ENGINE\_METHOD\_PKEY\_METHS}
\item
  \texttt{crypto.constants.ENGINE\_METHOD\_PKEY\_ASN1\_METHS}
\item
  \texttt{crypto.constants.ENGINE\_METHOD\_ALL}
\item
  \texttt{crypto.constants.ENGINE\_METHOD\_NONE}
\end{itemize}

\subsubsection{\texorpdfstring{\texttt{crypto.setFips(bool)}}{crypto.setFips(bool)}}\label{crypto.setfipsbool}

\begin{itemize}
\tightlist
\item
  \texttt{bool} \{boolean\} \texttt{true} to enable FIPS mode.
\end{itemize}

Enables the FIPS compliant crypto provider in a FIPS-enabled Node.js
build. Throws an error if FIPS mode is not available.

\subsubsection{\texorpdfstring{\texttt{crypto.sign(algorithm,\ data,\ key{[},\ callback{]})}}{crypto.sign(algorithm, data, key{[}, callback{]})}}\label{crypto.signalgorithm-data-key-callback}

\begin{itemize}
\tightlist
\item
  \texttt{algorithm} \{string \textbar{} null \textbar{} undefined\}
\item
  \texttt{data}
  \{ArrayBuffer\textbar Buffer\textbar TypedArray\textbar DataView\}
\item
  \texttt{key}
  \{Object\textbar string\textbar ArrayBuffer\textbar Buffer\textbar TypedArray\textbar DataView\textbar KeyObject\textbar CryptoKey\}
\item
  \texttt{callback} \{Function\}

  \begin{itemize}
  \tightlist
  \item
    \texttt{err} \{Error\}
  \item
    \texttt{signature} \{Buffer\}
  \end{itemize}
\item
  Returns: \{Buffer\} if the \texttt{callback} function is not provided.
\end{itemize}

Calculates and returns the signature for \texttt{data} using the given
private key and algorithm. If \texttt{algorithm} is \texttt{null} or
\texttt{undefined}, then the algorithm is dependent upon the key type
(especially Ed25519 and Ed448).

If \texttt{key} is not a \hyperref[class-keyobject]{\texttt{KeyObject}},
this function behaves as if \texttt{key} had been passed to
\hyperref[cryptocreateprivatekeykey]{\texttt{crypto.createPrivateKey()}}.
If it is an object, the following additional properties can be passed:

\begin{itemize}
\item
  \texttt{dsaEncoding} \{string\} For DSA and ECDSA, this option
  specifies the format of the generated signature. It can be one of the
  following:

  \begin{itemize}
  \tightlist
  \item
    \texttt{\textquotesingle{}der\textquotesingle{}} (default):
    DER-encoded ASN.1 signature structure encoding \texttt{(r,\ s)}.
  \item
    \texttt{\textquotesingle{}ieee-p1363\textquotesingle{}}: Signature
    format \texttt{r\ \textbar{}\textbar{}\ s} as proposed in
    IEEE-P1363.
  \end{itemize}
\item
  \texttt{padding} \{integer\} Optional padding value for RSA, one of
  the following:

  \begin{itemize}
  \tightlist
  \item
    \texttt{crypto.constants.RSA\_PKCS1\_PADDING} (default)
  \item
    \texttt{crypto.constants.RSA\_PKCS1\_PSS\_PADDING}
  \end{itemize}

  \texttt{RSA\_PKCS1\_PSS\_PADDING} will use MGF1 with the same hash
  function used to sign the message as specified in section 3.1 of
  \href{https://www.rfc-editor.org/rfc/rfc4055.txt}{RFC 4055}.
\item
  \texttt{saltLength} \{integer\} Salt length for when padding is
  \texttt{RSA\_PKCS1\_PSS\_PADDING}. The special value
  \texttt{crypto.constants.RSA\_PSS\_SALTLEN\_DIGEST} sets the salt
  length to the digest size,
  \texttt{crypto.constants.RSA\_PSS\_SALTLEN\_MAX\_SIGN} (default) sets
  it to the maximum permissible value.
\end{itemize}

If the \texttt{callback} function is provided this function uses libuv's
threadpool.

\subsubsection{\texorpdfstring{\texttt{crypto.subtle}}{crypto.subtle}}\label{crypto.subtle}

\begin{itemize}
\tightlist
\item
  Type: \{SubtleCrypto\}
\end{itemize}

A convenient alias for
\href{webcrypto.md\#class-subtlecrypto}{\texttt{crypto.webcrypto.subtle}}.

\subsubsection{\texorpdfstring{\texttt{crypto.timingSafeEqual(a,\ b)}}{crypto.timingSafeEqual(a, b)}}\label{crypto.timingsafeequala-b}

\begin{itemize}
\tightlist
\item
  \texttt{a}
  \{ArrayBuffer\textbar Buffer\textbar TypedArray\textbar DataView\}
\item
  \texttt{b}
  \{ArrayBuffer\textbar Buffer\textbar TypedArray\textbar DataView\}
\item
  Returns: \{boolean\}
\end{itemize}

This function compares the underlying bytes that represent the given
\texttt{ArrayBuffer}, \texttt{TypedArray}, or \texttt{DataView}
instances using a constant-time algorithm.

This function does not leak timing information that would allow an
attacker to guess one of the values. This is suitable for comparing HMAC
digests or secret values like authentication cookies or
\href{https://www.w3.org/TR/capability-urls/}{capability urls}.

\texttt{a} and \texttt{b} must both be \texttt{Buffer}s,
\texttt{TypedArray}s, or \texttt{DataView}s, and they must have the same
byte length. An error is thrown if \texttt{a} and \texttt{b} have
different byte lengths.

If at least one of \texttt{a} and \texttt{b} is a \texttt{TypedArray}
with more than one byte per entry, such as \texttt{Uint16Array}, the
result will be computed using the platform byte order.

When both of the inputs are \texttt{Float32Array}s or
\texttt{Float64Array}s, this function might return unexpected results
due to IEEE 754 encoding of floating-point numbers. In particular,
neither \texttt{x\ ===\ y} nor \texttt{Object.is(x,\ y)} implies that
the byte representations of two floating-point numbers \texttt{x} and
\texttt{y} are equal.

Use of \texttt{crypto.timingSafeEqual} does not guarantee that the
\emph{surrounding} code is timing-safe. Care should be taken to ensure
that the surrounding code does not introduce timing vulnerabilities.

\subsubsection{\texorpdfstring{\texttt{crypto.verify(algorithm,\ data,\ key,\ signature{[},\ callback{]})}}{crypto.verify(algorithm, data, key, signature{[}, callback{]})}}\label{crypto.verifyalgorithm-data-key-signature-callback}

\begin{itemize}
\tightlist
\item
  \texttt{algorithm} \{string\textbar null\textbar undefined\}
\item
  \texttt{data} \{ArrayBuffer\textbar{}
  Buffer\textbar TypedArray\textbar DataView\}
\item
  \texttt{key}
  \{Object\textbar string\textbar ArrayBuffer\textbar Buffer\textbar TypedArray\textbar DataView\textbar KeyObject\textbar CryptoKey\}
\item
  \texttt{signature}
  \{ArrayBuffer\textbar Buffer\textbar TypedArray\textbar DataView\}
\item
  \texttt{callback} \{Function\}

  \begin{itemize}
  \tightlist
  \item
    \texttt{err} \{Error\}
  \item
    \texttt{result} \{boolean\}
  \end{itemize}
\item
  Returns: \{boolean\} \texttt{true} or \texttt{false} depending on the
  validity of the signature for the data and public key if the
  \texttt{callback} function is not provided.
\end{itemize}

Verifies the given signature for \texttt{data} using the given key and
algorithm. If \texttt{algorithm} is \texttt{null} or \texttt{undefined},
then the algorithm is dependent upon the key type (especially Ed25519
and Ed448).

If \texttt{key} is not a \hyperref[class-keyobject]{\texttt{KeyObject}},
this function behaves as if \texttt{key} had been passed to
\hyperref[cryptocreatepublickeykey]{\texttt{crypto.createPublicKey()}}.
If it is an object, the following additional properties can be passed:

\begin{itemize}
\item
  \texttt{dsaEncoding} \{string\} For DSA and ECDSA, this option
  specifies the format of the signature. It can be one of the following:

  \begin{itemize}
  \tightlist
  \item
    \texttt{\textquotesingle{}der\textquotesingle{}} (default):
    DER-encoded ASN.1 signature structure encoding \texttt{(r,\ s)}.
  \item
    \texttt{\textquotesingle{}ieee-p1363\textquotesingle{}}: Signature
    format \texttt{r\ \textbar{}\textbar{}\ s} as proposed in
    IEEE-P1363.
  \end{itemize}
\item
  \texttt{padding} \{integer\} Optional padding value for RSA, one of
  the following:

  \begin{itemize}
  \tightlist
  \item
    \texttt{crypto.constants.RSA\_PKCS1\_PADDING} (default)
  \item
    \texttt{crypto.constants.RSA\_PKCS1\_PSS\_PADDING}
  \end{itemize}

  \texttt{RSA\_PKCS1\_PSS\_PADDING} will use MGF1 with the same hash
  function used to sign the message as specified in section 3.1 of
  \href{https://www.rfc-editor.org/rfc/rfc4055.txt}{RFC 4055}.
\item
  \texttt{saltLength} \{integer\} Salt length for when padding is
  \texttt{RSA\_PKCS1\_PSS\_PADDING}. The special value
  \texttt{crypto.constants.RSA\_PSS\_SALTLEN\_DIGEST} sets the salt
  length to the digest size,
  \texttt{crypto.constants.RSA\_PSS\_SALTLEN\_MAX\_SIGN} (default) sets
  it to the maximum permissible value.
\end{itemize}

The \texttt{signature} argument is the previously calculated signature
for the \texttt{data}.

Because public keys can be derived from private keys, a private key or a
public key may be passed for \texttt{key}.

If the \texttt{callback} function is provided this function uses libuv's
threadpool.

\subsubsection{\texorpdfstring{\texttt{crypto.webcrypto}}{crypto.webcrypto}}\label{crypto.webcrypto}

Type: \{Crypto\} An implementation of the Web Crypto API standard.

See the \href{webcrypto.md}{Web Crypto API documentation} for details.

\subsection{Notes}\label{notes}

\subsubsection{Using strings as inputs to cryptographic
APIs}\label{using-strings-as-inputs-to-cryptographic-apis}

For historical reasons, many cryptographic APIs provided by Node.js
accept strings as inputs where the underlying cryptographic algorithm
works on byte sequences. These instances include plaintexts,
ciphertexts, symmetric keys, initialization vectors, passphrases, salts,
authentication tags, and additional authenticated data.

When passing strings to cryptographic APIs, consider the following
factors.

\begin{itemize}
\item
  Not all byte sequences are valid UTF-8 strings. Therefore, when a byte
  sequence of length \texttt{n} is derived from a string, its entropy is
  generally lower than the entropy of a random or pseudorandom
  \texttt{n} byte sequence. For example, no UTF-8 string will result in
  the byte sequence \texttt{c0\ af}. Secret keys should almost
  exclusively be random or pseudorandom byte sequences.
\item
  Similarly, when converting random or pseudorandom byte sequences to
  UTF-8 strings, subsequences that do not represent valid code points
  may be replaced by the Unicode replacement character
  (\texttt{U+FFFD}). The byte representation of the resulting Unicode
  string may, therefore, not be equal to the byte sequence that the
  string was created from.

\begin{Shaded}
\begin{Highlighting}[]
\KeywordTok{const}\NormalTok{ original }\OperatorTok{=}\NormalTok{ [}\BaseNTok{0xc0}\OperatorTok{,} \BaseNTok{0xaf}\NormalTok{]}\OperatorTok{;}
\KeywordTok{const}\NormalTok{ bytesAsString }\OperatorTok{=} \BuiltInTok{Buffer}\OperatorTok{.}\FunctionTok{from}\NormalTok{(original)}\OperatorTok{.}\FunctionTok{toString}\NormalTok{(}\StringTok{\textquotesingle{}utf8\textquotesingle{}}\NormalTok{)}\OperatorTok{;}
\KeywordTok{const}\NormalTok{ stringAsBytes }\OperatorTok{=} \BuiltInTok{Buffer}\OperatorTok{.}\FunctionTok{from}\NormalTok{(bytesAsString}\OperatorTok{,} \StringTok{\textquotesingle{}utf8\textquotesingle{}}\NormalTok{)}\OperatorTok{;}
\BuiltInTok{console}\OperatorTok{.}\FunctionTok{log}\NormalTok{(stringAsBytes)}\OperatorTok{;}
\CommentTok{// Prints \textquotesingle{}\textless{}Buffer ef bf bd ef bf bd\textgreater{}\textquotesingle{}.}
\end{Highlighting}
\end{Shaded}

  The outputs of ciphers, hash functions, signature algorithms, and key
  derivation functions are pseudorandom byte sequences and should not be
  used as Unicode strings.
\item
  When strings are obtained from user input, some Unicode characters can
  be represented in multiple equivalent ways that result in different
  byte sequences. For example, when passing a user passphrase to a key
  derivation function, such as PBKDF2 or scrypt, the result of the key
  derivation function depends on whether the string uses composed or
  decomposed characters. Node.js does not normalize character
  representations. Developers should consider using
  \href{https://developer.mozilla.org/en-US/docs/Web/JavaScript/Reference/Global_Objects/String/normalize}{\texttt{String.prototype.normalize()}}
  on user inputs before passing them to cryptographic APIs.
\end{itemize}

\subsubsection{Legacy streams API (prior to Node.js
0.10)}\label{legacy-streams-api-prior-to-node.js-0.10}

The Crypto module was added to Node.js before there was the concept of a
unified Stream API, and before there were
\href{buffer.md}{\texttt{Buffer}} objects for handling binary data. As
such, many \texttt{crypto} classes have methods not typically found on
other Node.js classes that implement the \href{stream.md}{streams} API
(e.g.~\texttt{update()}, \texttt{final()}, or \texttt{digest()}). Also,
many methods accepted and returned
\texttt{\textquotesingle{}latin1\textquotesingle{}} encoded strings by
default rather than \texttt{Buffer}s. This default was changed after
Node.js v0.8 to use \href{buffer.md}{\texttt{Buffer}} objects by default
instead.

\subsubsection{Support for weak or compromised
algorithms}\label{support-for-weak-or-compromised-algorithms}

The \texttt{node:crypto} module still supports some algorithms which are
already compromised and are not recommended for use. The API also allows
the use of ciphers and hashes with a small key size that are too weak
for safe use.

Users should take full responsibility for selecting the crypto algorithm
and key size according to their security requirements.

Based on the recommendations of
\href{https://nvlpubs.nist.gov/nistpubs/SpecialPublications/NIST.SP.800-131Ar2.pdf}{NIST
SP 800-131A}:

\begin{itemize}
\tightlist
\item
  MD5 and SHA-1 are no longer acceptable where collision resistance is
  required such as digital signatures.
\item
  The key used with RSA, DSA, and DH algorithms is recommended to have
  at least 2048 bits and that of the curve of ECDSA and ECDH at least
  224 bits, to be safe to use for several years.
\item
  The DH groups of \texttt{modp1}, \texttt{modp2} and \texttt{modp5}
  have a key size smaller than 2048 bits and are not recommended.
\end{itemize}

See the reference for other recommendations and details.

Some algorithms that have known weaknesses and are of little relevance
in practice are only available through the
\href{cli.md\#--openssl-legacy-provider}{legacy provider}, which is not
enabled by default.

\subsubsection{CCM mode}\label{ccm-mode}

CCM is one of the supported
\href{https://en.wikipedia.org/wiki/Authenticated_encryption}{AEAD
algorithms}. Applications which use this mode must adhere to certain
restrictions when using the cipher API:

\begin{itemize}
\tightlist
\item
  The authentication tag length must be specified during cipher creation
  by setting the \texttt{authTagLength} option and must be one of 4, 6,
  8, 10, 12, 14 or 16 bytes.
\item
  The length of the initialization vector (nonce) \texttt{N} must be
  between 7 and 13 bytes (\texttt{7\ ≤\ N\ ≤\ 13}).
\item
  The length of the plaintext is limited to
  \texttt{2\ **\ (8\ *\ (15\ -\ N))} bytes.
\item
  When decrypting, the authentication tag must be set via
  \texttt{setAuthTag()} before calling \texttt{update()}. Otherwise,
  decryption will fail and \texttt{final()} will throw an error in
  compliance with section 2.6 of
  \href{https://www.rfc-editor.org/rfc/rfc3610.txt}{RFC 3610}.
\item
  Using stream methods such as \texttt{write(data)}, \texttt{end(data)}
  or \texttt{pipe()} in CCM mode might fail as CCM cannot handle more
  than one chunk of data per instance.
\item
  When passing additional authenticated data (AAD), the length of the
  actual message in bytes must be passed to \texttt{setAAD()} via the
  \texttt{plaintextLength} option. Many crypto libraries include the
  authentication tag in the ciphertext, which means that they produce
  ciphertexts of the length \texttt{plaintextLength\ +\ authTagLength}.
  Node.js does not include the authentication tag, so the ciphertext
  length is always \texttt{plaintextLength}. This is not necessary if no
  AAD is used.
\item
  As CCM processes the whole message at once, \texttt{update()} must be
  called exactly once.
\item
  Even though calling \texttt{update()} is sufficient to encrypt/decrypt
  the message, applications \emph{must} call \texttt{final()} to compute
  or verify the authentication tag.
\end{itemize}

\begin{Shaded}
\begin{Highlighting}[]
\ImportTok{import}\NormalTok{ \{ }\BuiltInTok{Buffer}\NormalTok{ \} }\ImportTok{from} \StringTok{\textquotesingle{}node:buffer\textquotesingle{}}\OperatorTok{;}
\KeywordTok{const}\NormalTok{ \{}
\NormalTok{  createCipheriv}\OperatorTok{,}
\NormalTok{  createDecipheriv}\OperatorTok{,}
\NormalTok{  randomBytes}\OperatorTok{,}
\NormalTok{\} }\OperatorTok{=} \ControlFlowTok{await} \ImportTok{import}\NormalTok{(}\StringTok{\textquotesingle{}node:crypto\textquotesingle{}}\NormalTok{)}\OperatorTok{;}

\KeywordTok{const}\NormalTok{ key }\OperatorTok{=} \StringTok{\textquotesingle{}keykeykeykeykeykeykeykey\textquotesingle{}}\OperatorTok{;}
\KeywordTok{const}\NormalTok{ nonce }\OperatorTok{=} \FunctionTok{randomBytes}\NormalTok{(}\DecValTok{12}\NormalTok{)}\OperatorTok{;}

\KeywordTok{const}\NormalTok{ aad }\OperatorTok{=} \BuiltInTok{Buffer}\OperatorTok{.}\FunctionTok{from}\NormalTok{(}\StringTok{\textquotesingle{}0123456789\textquotesingle{}}\OperatorTok{,} \StringTok{\textquotesingle{}hex\textquotesingle{}}\NormalTok{)}\OperatorTok{;}

\KeywordTok{const}\NormalTok{ cipher }\OperatorTok{=} \FunctionTok{createCipheriv}\NormalTok{(}\StringTok{\textquotesingle{}aes{-}192{-}ccm\textquotesingle{}}\OperatorTok{,}\NormalTok{ key}\OperatorTok{,}\NormalTok{ nonce}\OperatorTok{,}\NormalTok{ \{}
  \DataTypeTok{authTagLength}\OperatorTok{:} \DecValTok{16}\OperatorTok{,}
\NormalTok{\})}\OperatorTok{;}
\KeywordTok{const}\NormalTok{ plaintext }\OperatorTok{=} \StringTok{\textquotesingle{}Hello world\textquotesingle{}}\OperatorTok{;}
\NormalTok{cipher}\OperatorTok{.}\FunctionTok{setAAD}\NormalTok{(aad}\OperatorTok{,}\NormalTok{ \{}
  \DataTypeTok{plaintextLength}\OperatorTok{:} \BuiltInTok{Buffer}\OperatorTok{.}\FunctionTok{byteLength}\NormalTok{(plaintext)}\OperatorTok{,}
\NormalTok{\})}\OperatorTok{;}
\KeywordTok{const}\NormalTok{ ciphertext }\OperatorTok{=}\NormalTok{ cipher}\OperatorTok{.}\FunctionTok{update}\NormalTok{(plaintext}\OperatorTok{,} \StringTok{\textquotesingle{}utf8\textquotesingle{}}\NormalTok{)}\OperatorTok{;}
\NormalTok{cipher}\OperatorTok{.}\FunctionTok{final}\NormalTok{()}\OperatorTok{;}
\KeywordTok{const}\NormalTok{ tag }\OperatorTok{=}\NormalTok{ cipher}\OperatorTok{.}\FunctionTok{getAuthTag}\NormalTok{()}\OperatorTok{;}

\CommentTok{// Now transmit \{ ciphertext, nonce, tag \}.}

\KeywordTok{const}\NormalTok{ decipher }\OperatorTok{=} \FunctionTok{createDecipheriv}\NormalTok{(}\StringTok{\textquotesingle{}aes{-}192{-}ccm\textquotesingle{}}\OperatorTok{,}\NormalTok{ key}\OperatorTok{,}\NormalTok{ nonce}\OperatorTok{,}\NormalTok{ \{}
  \DataTypeTok{authTagLength}\OperatorTok{:} \DecValTok{16}\OperatorTok{,}
\NormalTok{\})}\OperatorTok{;}
\NormalTok{decipher}\OperatorTok{.}\FunctionTok{setAuthTag}\NormalTok{(tag)}\OperatorTok{;}
\NormalTok{decipher}\OperatorTok{.}\FunctionTok{setAAD}\NormalTok{(aad}\OperatorTok{,}\NormalTok{ \{}
  \DataTypeTok{plaintextLength}\OperatorTok{:}\NormalTok{ ciphertext}\OperatorTok{.}\AttributeTok{length}\OperatorTok{,}
\NormalTok{\})}\OperatorTok{;}
\KeywordTok{const}\NormalTok{ receivedPlaintext }\OperatorTok{=}\NormalTok{ decipher}\OperatorTok{.}\FunctionTok{update}\NormalTok{(ciphertext}\OperatorTok{,} \KeywordTok{null}\OperatorTok{,} \StringTok{\textquotesingle{}utf8\textquotesingle{}}\NormalTok{)}\OperatorTok{;}

\ControlFlowTok{try}\NormalTok{ \{}
\NormalTok{  decipher}\OperatorTok{.}\FunctionTok{final}\NormalTok{()}\OperatorTok{;}
\NormalTok{\} }\ControlFlowTok{catch}\NormalTok{ (err) \{}
  \ControlFlowTok{throw} \KeywordTok{new} \BuiltInTok{Error}\NormalTok{(}\StringTok{\textquotesingle{}Authentication failed!\textquotesingle{}}\OperatorTok{,}\NormalTok{ \{ }\DataTypeTok{cause}\OperatorTok{:}\NormalTok{ err \})}\OperatorTok{;}
\NormalTok{\}}

\BuiltInTok{console}\OperatorTok{.}\FunctionTok{log}\NormalTok{(receivedPlaintext)}\OperatorTok{;}
\end{Highlighting}
\end{Shaded}

\begin{Shaded}
\begin{Highlighting}[]
\KeywordTok{const}\NormalTok{ \{ }\BuiltInTok{Buffer}\NormalTok{ \} }\OperatorTok{=} \PreprocessorTok{require}\NormalTok{(}\StringTok{\textquotesingle{}node:buffer\textquotesingle{}}\NormalTok{)}\OperatorTok{;}
\KeywordTok{const}\NormalTok{ \{}
\NormalTok{  createCipheriv}\OperatorTok{,}
\NormalTok{  createDecipheriv}\OperatorTok{,}
\NormalTok{  randomBytes}\OperatorTok{,}
\NormalTok{\} }\OperatorTok{=} \PreprocessorTok{require}\NormalTok{(}\StringTok{\textquotesingle{}node:crypto\textquotesingle{}}\NormalTok{)}\OperatorTok{;}

\KeywordTok{const}\NormalTok{ key }\OperatorTok{=} \StringTok{\textquotesingle{}keykeykeykeykeykeykeykey\textquotesingle{}}\OperatorTok{;}
\KeywordTok{const}\NormalTok{ nonce }\OperatorTok{=} \FunctionTok{randomBytes}\NormalTok{(}\DecValTok{12}\NormalTok{)}\OperatorTok{;}

\KeywordTok{const}\NormalTok{ aad }\OperatorTok{=} \BuiltInTok{Buffer}\OperatorTok{.}\FunctionTok{from}\NormalTok{(}\StringTok{\textquotesingle{}0123456789\textquotesingle{}}\OperatorTok{,} \StringTok{\textquotesingle{}hex\textquotesingle{}}\NormalTok{)}\OperatorTok{;}

\KeywordTok{const}\NormalTok{ cipher }\OperatorTok{=} \FunctionTok{createCipheriv}\NormalTok{(}\StringTok{\textquotesingle{}aes{-}192{-}ccm\textquotesingle{}}\OperatorTok{,}\NormalTok{ key}\OperatorTok{,}\NormalTok{ nonce}\OperatorTok{,}\NormalTok{ \{}
  \DataTypeTok{authTagLength}\OperatorTok{:} \DecValTok{16}\OperatorTok{,}
\NormalTok{\})}\OperatorTok{;}
\KeywordTok{const}\NormalTok{ plaintext }\OperatorTok{=} \StringTok{\textquotesingle{}Hello world\textquotesingle{}}\OperatorTok{;}
\NormalTok{cipher}\OperatorTok{.}\FunctionTok{setAAD}\NormalTok{(aad}\OperatorTok{,}\NormalTok{ \{}
  \DataTypeTok{plaintextLength}\OperatorTok{:} \BuiltInTok{Buffer}\OperatorTok{.}\FunctionTok{byteLength}\NormalTok{(plaintext)}\OperatorTok{,}
\NormalTok{\})}\OperatorTok{;}
\KeywordTok{const}\NormalTok{ ciphertext }\OperatorTok{=}\NormalTok{ cipher}\OperatorTok{.}\FunctionTok{update}\NormalTok{(plaintext}\OperatorTok{,} \StringTok{\textquotesingle{}utf8\textquotesingle{}}\NormalTok{)}\OperatorTok{;}
\NormalTok{cipher}\OperatorTok{.}\FunctionTok{final}\NormalTok{()}\OperatorTok{;}
\KeywordTok{const}\NormalTok{ tag }\OperatorTok{=}\NormalTok{ cipher}\OperatorTok{.}\FunctionTok{getAuthTag}\NormalTok{()}\OperatorTok{;}

\CommentTok{// Now transmit \{ ciphertext, nonce, tag \}.}

\KeywordTok{const}\NormalTok{ decipher }\OperatorTok{=} \FunctionTok{createDecipheriv}\NormalTok{(}\StringTok{\textquotesingle{}aes{-}192{-}ccm\textquotesingle{}}\OperatorTok{,}\NormalTok{ key}\OperatorTok{,}\NormalTok{ nonce}\OperatorTok{,}\NormalTok{ \{}
  \DataTypeTok{authTagLength}\OperatorTok{:} \DecValTok{16}\OperatorTok{,}
\NormalTok{\})}\OperatorTok{;}
\NormalTok{decipher}\OperatorTok{.}\FunctionTok{setAuthTag}\NormalTok{(tag)}\OperatorTok{;}
\NormalTok{decipher}\OperatorTok{.}\FunctionTok{setAAD}\NormalTok{(aad}\OperatorTok{,}\NormalTok{ \{}
  \DataTypeTok{plaintextLength}\OperatorTok{:}\NormalTok{ ciphertext}\OperatorTok{.}\AttributeTok{length}\OperatorTok{,}
\NormalTok{\})}\OperatorTok{;}
\KeywordTok{const}\NormalTok{ receivedPlaintext }\OperatorTok{=}\NormalTok{ decipher}\OperatorTok{.}\FunctionTok{update}\NormalTok{(ciphertext}\OperatorTok{,} \KeywordTok{null}\OperatorTok{,} \StringTok{\textquotesingle{}utf8\textquotesingle{}}\NormalTok{)}\OperatorTok{;}

\ControlFlowTok{try}\NormalTok{ \{}
\NormalTok{  decipher}\OperatorTok{.}\FunctionTok{final}\NormalTok{()}\OperatorTok{;}
\NormalTok{\} }\ControlFlowTok{catch}\NormalTok{ (err) \{}
  \ControlFlowTok{throw} \KeywordTok{new} \BuiltInTok{Error}\NormalTok{(}\StringTok{\textquotesingle{}Authentication failed!\textquotesingle{}}\OperatorTok{,}\NormalTok{ \{ }\DataTypeTok{cause}\OperatorTok{:}\NormalTok{ err \})}\OperatorTok{;}
\NormalTok{\}}

\BuiltInTok{console}\OperatorTok{.}\FunctionTok{log}\NormalTok{(receivedPlaintext)}\OperatorTok{;}
\end{Highlighting}
\end{Shaded}

\subsubsection{FIPS mode}\label{fips-mode}

When using OpenSSL 3, Node.js supports FIPS 140-2 when used with an
appropriate OpenSSL 3 provider, such as the
\href{https://www.openssl.org/docs/man3.0/man7/crypto.html\#FIPS-provider}{FIPS
provider from OpenSSL 3} which can be installed by following the
instructions in
\href{https://github.com/openssl/openssl/blob/openssl-3.0/README-FIPS.md}{OpenSSL's
FIPS README file}.

For FIPS support in Node.js you will need:

\begin{itemize}
\tightlist
\item
  A correctly installed OpenSSL 3 FIPS provider.
\item
  An OpenSSL 3
  \href{https://www.openssl.org/docs/man3.0/man5/fips_config.html}{FIPS
  module configuration file}.
\item
  An OpenSSL 3 configuration file that references the FIPS module
  configuration file.
\end{itemize}

Node.js will need to be configured with an OpenSSL configuration file
that points to the FIPS provider. An example configuration file looks
like this:

\begin{Shaded}
\begin{Highlighting}[]
\NormalTok{nodejs\_conf = nodejs\_init}

\NormalTok{.include /\textless{}absolute path\textgreater{}/fipsmodule.cnf}

\NormalTok{[nodejs\_init]}
\NormalTok{providers = provider\_sect}

\NormalTok{[provider\_sect]}
\NormalTok{default = default\_sect}
\NormalTok{\# The fips section name should match the section name inside the}
\NormalTok{\# included fipsmodule.cnf.}
\NormalTok{fips = fips\_sect}

\NormalTok{[default\_sect]}
\NormalTok{activate = 1}
\end{Highlighting}
\end{Shaded}

where \texttt{fipsmodule.cnf} is the FIPS module configuration file
generated from the FIPS provider installation step:

\begin{Shaded}
\begin{Highlighting}[]
\ExtensionTok{openssl}\NormalTok{ fipsinstall}
\end{Highlighting}
\end{Shaded}

Set the \texttt{OPENSSL\_CONF} environment variable to point to your
configuration file and \texttt{OPENSSL\_MODULES} to the location of the
FIPS provider dynamic library. e.g.

\begin{Shaded}
\begin{Highlighting}[]
\BuiltInTok{export} \VariableTok{OPENSSL\_CONF}\OperatorTok{=}\NormalTok{/}\OperatorTok{\textless{}}\NormalTok{path to configuration file}\OperatorTok{\textgreater{}}\NormalTok{/nodejs.cnf}
\BuiltInTok{export} \VariableTok{OPENSSL\_MODULES}\OperatorTok{=}\NormalTok{/}\OperatorTok{\textless{}}\NormalTok{path to openssl lib}\OperatorTok{\textgreater{}}\NormalTok{/ossl{-}modules}
\end{Highlighting}
\end{Shaded}

FIPS mode can then be enabled in Node.js either by:

\begin{itemize}
\tightlist
\item
  Starting Node.js with \texttt{-\/-enable-fips} or
  \texttt{-\/-force-fips} command line flags.
\item
  Programmatically calling \texttt{crypto.setFips(true)}.
\end{itemize}

Optionally FIPS mode can be enabled in Node.js via the OpenSSL
configuration file. e.g.

\begin{Shaded}
\begin{Highlighting}[]
\NormalTok{nodejs\_conf = nodejs\_init}

\NormalTok{.include /\textless{}absolute path\textgreater{}/fipsmodule.cnf}

\NormalTok{[nodejs\_init]}
\NormalTok{providers = provider\_sect}
\NormalTok{alg\_section = algorithm\_sect}

\NormalTok{[provider\_sect]}
\NormalTok{default = default\_sect}
\NormalTok{\# The fips section name should match the section name inside the}
\NormalTok{\# included fipsmodule.cnf.}
\NormalTok{fips = fips\_sect}

\NormalTok{[default\_sect]}
\NormalTok{activate = 1}

\NormalTok{[algorithm\_sect]}
\NormalTok{default\_properties = fips=yes}
\end{Highlighting}
\end{Shaded}

\subsection{Crypto constants}\label{crypto-constants}

The following constants exported by \texttt{crypto.constants} apply to
various uses of the \texttt{node:crypto}, \texttt{node:tls}, and
\texttt{node:https} modules and are generally specific to OpenSSL.

\subsubsection{OpenSSL options}\label{openssl-options}

See the
\href{https://wiki.openssl.org/index.php/List_of_SSL_OP_Flags\#Table_of_Options}{list
of SSL OP Flags} for details.

Constant

Description

SSL\_OP\_ALL

Applies multiple bug workarounds within OpenSSL. See
https://www.openssl.org/docs/man3.0/man3/SSL\_CTX\_set\_options.html for
detail.

SSL\_OP\_ALLOW\_NO\_DHE\_KEX

Instructs OpenSSL to allow a non-{[}EC{]}DHE-based key exchange mode for
TLS v1.3

SSL\_OP\_ALLOW\_UNSAFE\_LEGACY\_RENEGOTIATION

Allows legacy insecure renegotiation between OpenSSL and unpatched
clients or servers. See
https://www.openssl.org/docs/man3.0/man3/SSL\_CTX\_set\_options.html.

SSL\_OP\_CIPHER\_SERVER\_PREFERENCE

Attempts to use the server's preferences instead of the client's when
selecting a cipher. Behavior depends on protocol version. See
https://www.openssl.org/docs/man3.0/man3/SSL\_CTX\_set\_options.html.

SSL\_OP\_CISCO\_ANYCONNECT

Instructs OpenSSL to use Cisco's ``speshul'' version of DTLS\_BAD\_VER.

SSL\_OP\_COOKIE\_EXCHANGE

Instructs OpenSSL to turn on cookie exchange.

SSL\_OP\_CRYPTOPRO\_TLSEXT\_BUG

Instructs OpenSSL to add server-hello extension from an early version of
the cryptopro draft.

SSL\_OP\_DONT\_INSERT\_EMPTY\_FRAGMENTS

Instructs OpenSSL to disable a SSL 3.0/TLS 1.0 vulnerability workaround
added in OpenSSL 0.9.6d.

SSL\_OP\_LEGACY\_SERVER\_CONNECT

Allows initial connection to servers that do not support RI.

SSL\_OP\_NO\_COMPRESSION

Instructs OpenSSL to disable support for SSL/TLS compression.

SSL\_OP\_NO\_ENCRYPT\_THEN\_MAC

Instructs OpenSSL to disable encrypt-then-MAC.

SSL\_OP\_NO\_QUERY\_MTU

SSL\_OP\_NO\_RENEGOTIATION

Instructs OpenSSL to disable renegotiation.

SSL\_OP\_NO\_SESSION\_RESUMPTION\_ON\_RENEGOTIATION

Instructs OpenSSL to always start a new session when performing
renegotiation.

SSL\_OP\_NO\_SSLv2

Instructs OpenSSL to turn off SSL v2

SSL\_OP\_NO\_SSLv3

Instructs OpenSSL to turn off SSL v3

SSL\_OP\_NO\_TICKET

Instructs OpenSSL to disable use of RFC4507bis tickets.

SSL\_OP\_NO\_TLSv1

Instructs OpenSSL to turn off TLS v1

SSL\_OP\_NO\_TLSv1\_1

Instructs OpenSSL to turn off TLS v1.1

SSL\_OP\_NO\_TLSv1\_2

Instructs OpenSSL to turn off TLS v1.2

SSL\_OP\_NO\_TLSv1\_3

Instructs OpenSSL to turn off TLS v1.3

SSL\_OP\_PRIORITIZE\_CHACHA

Instructs OpenSSL server to prioritize ChaCha20-Poly1305 when the client
does. This option has no effect if SSL\_OP\_CIPHER\_SERVER\_PREFERENCE
is not enabled.

SSL\_OP\_TLS\_ROLLBACK\_BUG

Instructs OpenSSL to disable version rollback attack detection.

\subsubsection{OpenSSL engine constants}\label{openssl-engine-constants}

Constant

Description

ENGINE\_METHOD\_RSA

Limit engine usage to RSA

ENGINE\_METHOD\_DSA

Limit engine usage to DSA

ENGINE\_METHOD\_DH

Limit engine usage to DH

ENGINE\_METHOD\_RAND

Limit engine usage to RAND

ENGINE\_METHOD\_EC

Limit engine usage to EC

ENGINE\_METHOD\_CIPHERS

Limit engine usage to CIPHERS

ENGINE\_METHOD\_DIGESTS

Limit engine usage to DIGESTS

ENGINE\_METHOD\_PKEY\_METHS

Limit engine usage to PKEY\_METHDS

ENGINE\_METHOD\_PKEY\_ASN1\_METHS

Limit engine usage to PKEY\_ASN1\_METHS

ENGINE\_METHOD\_ALL

ENGINE\_METHOD\_NONE

\subsubsection{Other OpenSSL constants}\label{other-openssl-constants}

Constant

Description

DH\_CHECK\_P\_NOT\_SAFE\_PRIME

DH\_CHECK\_P\_NOT\_PRIME

DH\_UNABLE\_TO\_CHECK\_GENERATOR

DH\_NOT\_SUITABLE\_GENERATOR

RSA\_PKCS1\_PADDING

RSA\_SSLV23\_PADDING

RSA\_NO\_PADDING

RSA\_PKCS1\_OAEP\_PADDING

RSA\_X931\_PADDING

RSA\_PKCS1\_PSS\_PADDING

RSA\_PSS\_SALTLEN\_DIGEST

Sets the salt length for RSA\_PKCS1\_PSS\_PADDING to the digest size
when signing or verifying.

RSA\_PSS\_SALTLEN\_MAX\_SIGN

Sets the salt length for RSA\_PKCS1\_PSS\_PADDING to the maximum
permissible value when signing data.

RSA\_PSS\_SALTLEN\_AUTO

Causes the salt length for RSA\_PKCS1\_PSS\_PADDING to be determined
automatically when verifying a signature.

POINT\_CONVERSION\_COMPRESSED

POINT\_CONVERSION\_UNCOMPRESSED

POINT\_CONVERSION\_HYBRID

\subsubsection{Node.js crypto constants}\label{node.js-crypto-constants}

Constant

Description

defaultCoreCipherList

Specifies the built-in default cipher list used by Node.js.

defaultCipherList

Specifies the active default cipher list used by the current Node.js
process.
