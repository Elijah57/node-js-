\section{String decoder}\label{string-decoder}

\begin{quote}
Stability: 2 - Stable
\end{quote}

The \texttt{node:string\_decoder} module provides an API for decoding
\texttt{Buffer} objects into strings in a manner that preserves encoded
multi-byte UTF-8 and UTF-16 characters. It can be accessed using:

\begin{Shaded}
\begin{Highlighting}[]
\KeywordTok{const}\NormalTok{ \{ StringDecoder \} }\OperatorTok{=} \PreprocessorTok{require}\NormalTok{(}\StringTok{\textquotesingle{}node:string\_decoder\textquotesingle{}}\NormalTok{)}\OperatorTok{;}
\end{Highlighting}
\end{Shaded}

The following example shows the basic use of the \texttt{StringDecoder}
class.

\begin{Shaded}
\begin{Highlighting}[]
\KeywordTok{const}\NormalTok{ \{ StringDecoder \} }\OperatorTok{=} \PreprocessorTok{require}\NormalTok{(}\StringTok{\textquotesingle{}node:string\_decoder\textquotesingle{}}\NormalTok{)}\OperatorTok{;}
\KeywordTok{const}\NormalTok{ decoder }\OperatorTok{=} \KeywordTok{new} \FunctionTok{StringDecoder}\NormalTok{(}\StringTok{\textquotesingle{}utf8\textquotesingle{}}\NormalTok{)}\OperatorTok{;}

\KeywordTok{const}\NormalTok{ cent }\OperatorTok{=} \BuiltInTok{Buffer}\OperatorTok{.}\FunctionTok{from}\NormalTok{([}\BaseNTok{0xC2}\OperatorTok{,} \BaseNTok{0xA2}\NormalTok{])}\OperatorTok{;}
\BuiltInTok{console}\OperatorTok{.}\FunctionTok{log}\NormalTok{(decoder}\OperatorTok{.}\FunctionTok{write}\NormalTok{(cent))}\OperatorTok{;} \CommentTok{// Prints: ¢}

\KeywordTok{const}\NormalTok{ euro }\OperatorTok{=} \BuiltInTok{Buffer}\OperatorTok{.}\FunctionTok{from}\NormalTok{([}\BaseNTok{0xE2}\OperatorTok{,} \BaseNTok{0x82}\OperatorTok{,} \BaseNTok{0xAC}\NormalTok{])}\OperatorTok{;}
\BuiltInTok{console}\OperatorTok{.}\FunctionTok{log}\NormalTok{(decoder}\OperatorTok{.}\FunctionTok{write}\NormalTok{(euro))}\OperatorTok{;} \CommentTok{// Prints: €}
\end{Highlighting}
\end{Shaded}

When a \texttt{Buffer} instance is written to the \texttt{StringDecoder}
instance, an internal buffer is used to ensure that the decoded string
does not contain any incomplete multibyte characters. These are held in
the buffer until the next call to \texttt{stringDecoder.write()} or
until \texttt{stringDecoder.end()} is called.

In the following example, the three UTF-8 encoded bytes of the European
Euro symbol (\texttt{€}) are written over three separate operations:

\begin{Shaded}
\begin{Highlighting}[]
\KeywordTok{const}\NormalTok{ \{ StringDecoder \} }\OperatorTok{=} \PreprocessorTok{require}\NormalTok{(}\StringTok{\textquotesingle{}node:string\_decoder\textquotesingle{}}\NormalTok{)}\OperatorTok{;}
\KeywordTok{const}\NormalTok{ decoder }\OperatorTok{=} \KeywordTok{new} \FunctionTok{StringDecoder}\NormalTok{(}\StringTok{\textquotesingle{}utf8\textquotesingle{}}\NormalTok{)}\OperatorTok{;}

\NormalTok{decoder}\OperatorTok{.}\FunctionTok{write}\NormalTok{(}\BuiltInTok{Buffer}\OperatorTok{.}\FunctionTok{from}\NormalTok{([}\BaseNTok{0xE2}\NormalTok{]))}\OperatorTok{;}
\NormalTok{decoder}\OperatorTok{.}\FunctionTok{write}\NormalTok{(}\BuiltInTok{Buffer}\OperatorTok{.}\FunctionTok{from}\NormalTok{([}\BaseNTok{0x82}\NormalTok{]))}\OperatorTok{;}
\BuiltInTok{console}\OperatorTok{.}\FunctionTok{log}\NormalTok{(decoder}\OperatorTok{.}\FunctionTok{end}\NormalTok{(}\BuiltInTok{Buffer}\OperatorTok{.}\FunctionTok{from}\NormalTok{([}\BaseNTok{0xAC}\NormalTok{])))}\OperatorTok{;} \CommentTok{// Prints: €}
\end{Highlighting}
\end{Shaded}

\subsection{\texorpdfstring{Class:
\texttt{StringDecoder}}{Class: StringDecoder}}\label{class-stringdecoder}

\subsubsection{\texorpdfstring{\texttt{new\ StringDecoder({[}encoding{]})}}{new StringDecoder({[}encoding{]})}}\label{new-stringdecoderencoding}

\begin{itemize}
\tightlist
\item
  \texttt{encoding} \{string\} The character
  \href{buffer.md\#buffers-and-character-encodings}{encoding} the
  \texttt{StringDecoder} will use. \textbf{Default:}
  \texttt{\textquotesingle{}utf8\textquotesingle{}}.
\end{itemize}

Creates a new \texttt{StringDecoder} instance.

\subsubsection{\texorpdfstring{\texttt{stringDecoder.end({[}buffer{]})}}{stringDecoder.end({[}buffer{]})}}\label{stringdecoder.endbuffer}

\begin{itemize}
\tightlist
\item
  \texttt{buffer}
  \{string\textbar Buffer\textbar TypedArray\textbar DataView\} The
  bytes to decode.
\item
  Returns: \{string\}
\end{itemize}

Returns any remaining input stored in the internal buffer as a string.
Bytes representing incomplete UTF-8 and UTF-16 characters will be
replaced with substitution characters appropriate for the character
encoding.

If the \texttt{buffer} argument is provided, one final call to
\texttt{stringDecoder.write()} is performed before returning the
remaining input. After \texttt{end()} is called, the
\texttt{stringDecoder} object can be reused for new input.

\subsubsection{\texorpdfstring{\texttt{stringDecoder.write(buffer)}}{stringDecoder.write(buffer)}}\label{stringdecoder.writebuffer}

\begin{itemize}
\tightlist
\item
  \texttt{buffer}
  \{string\textbar Buffer\textbar TypedArray\textbar DataView\} The
  bytes to decode.
\item
  Returns: \{string\}
\end{itemize}

Returns a decoded string, ensuring that any incomplete multibyte
characters at the end of the \texttt{Buffer}, or \texttt{TypedArray}, or
\texttt{DataView} are omitted from the returned string and stored in an
internal buffer for the next call to \texttt{stringDecoder.write()} or
\texttt{stringDecoder.end()}.
