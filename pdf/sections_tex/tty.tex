\section{TTY}\label{tty}

\begin{quote}
Stability: 2 - Stable
\end{quote}

The \texttt{node:tty} module provides the \texttt{tty.ReadStream} and
\texttt{tty.WriteStream} classes. In most cases, it will not be
necessary or possible to use this module directly. However, it can be
accessed using:

\begin{Shaded}
\begin{Highlighting}[]
\KeywordTok{const}\NormalTok{ tty }\OperatorTok{=} \PreprocessorTok{require}\NormalTok{(}\StringTok{\textquotesingle{}node:tty\textquotesingle{}}\NormalTok{)}\OperatorTok{;}
\end{Highlighting}
\end{Shaded}

When Node.js detects that it is being run with a text terminal (``TTY'')
attached, \href{process.md\#processstdin}{\texttt{process.stdin}} will,
by default, be initialized as an instance of \texttt{tty.ReadStream} and
both \href{process.md\#processstdout}{\texttt{process.stdout}} and
\href{process.md\#processstderr}{\texttt{process.stderr}} will, by
default, be instances of \texttt{tty.WriteStream}. The preferred method
of determining whether Node.js is being run within a TTY context is to
check that the value of the \texttt{process.stdout.isTTY} property is
\texttt{true}:

\begin{Shaded}
\begin{Highlighting}[]
\NormalTok{$ node {-}p {-}e "Boolean(process.stdout.isTTY)"}
\NormalTok{true}
\NormalTok{$ node {-}p {-}e "Boolean(process.stdout.isTTY)" | cat}
\NormalTok{false}
\end{Highlighting}
\end{Shaded}

In most cases, there should be little to no reason for an application to
manually create instances of the \texttt{tty.ReadStream} and
\texttt{tty.WriteStream} classes.

\subsection{\texorpdfstring{Class:
\texttt{tty.ReadStream}}{Class: tty.ReadStream}}\label{class-tty.readstream}

\begin{itemize}
\tightlist
\item
  Extends: \{net.Socket\}
\end{itemize}

Represents the readable side of a TTY. In normal circumstances
\href{process.md\#processstdin}{\texttt{process.stdin}} will be the only
\texttt{tty.ReadStream} instance in a Node.js process and there should
be no reason to create additional instances.

\subsubsection{\texorpdfstring{\texttt{readStream.isRaw}}{readStream.isRaw}}\label{readstream.israw}

A \texttt{boolean} that is \texttt{true} if the TTY is currently
configured to operate as a raw device.

This flag is always \texttt{false} when a process starts, even if the
terminal is operating in raw mode. Its value will change with subsequent
calls to \texttt{setRawMode}.

\subsubsection{\texorpdfstring{\texttt{readStream.isTTY}}{readStream.isTTY}}\label{readstream.istty}

A \texttt{boolean} that is always \texttt{true} for
\texttt{tty.ReadStream} instances.

\subsubsection{\texorpdfstring{\texttt{readStream.setRawMode(mode)}}{readStream.setRawMode(mode)}}\label{readstream.setrawmodemode}

\begin{itemize}
\tightlist
\item
  \texttt{mode} \{boolean\} If \texttt{true}, configures the
  \texttt{tty.ReadStream} to operate as a raw device. If \texttt{false},
  configures the \texttt{tty.ReadStream} to operate in its default mode.
  The \texttt{readStream.isRaw} property will be set to the resulting
  mode.
\item
  Returns: \{this\} The read stream instance.
\end{itemize}

Allows configuration of \texttt{tty.ReadStream} so that it operates as a
raw device.

When in raw mode, input is always available character-by-character, not
including modifiers. Additionally, all special processing of characters
by the terminal is disabled, including echoing input characters. Ctrl+C
will no longer cause a \texttt{SIGINT} when in this mode.

\subsection{\texorpdfstring{Class:
\texttt{tty.WriteStream}}{Class: tty.WriteStream}}\label{class-tty.writestream}

\begin{itemize}
\tightlist
\item
  Extends: \{net.Socket\}
\end{itemize}

Represents the writable side of a TTY. In normal circumstances,
\href{process.md\#processstdout}{\texttt{process.stdout}} and
\href{process.md\#processstderr}{\texttt{process.stderr}} will be the
only \texttt{tty.WriteStream} instances created for a Node.js process
and there should be no reason to create additional instances.

\subsubsection{\texorpdfstring{Event:
\texttt{\textquotesingle{}resize\textquotesingle{}}}{Event: \textquotesingle resize\textquotesingle{}}}\label{event-resize}

The \texttt{\textquotesingle{}resize\textquotesingle{}} event is emitted
whenever either of the \texttt{writeStream.columns} or
\texttt{writeStream.rows} properties have changed. No arguments are
passed to the listener callback when called.

\begin{Shaded}
\begin{Highlighting}[]
\BuiltInTok{process}\OperatorTok{.}\AttributeTok{stdout}\OperatorTok{.}\FunctionTok{on}\NormalTok{(}\StringTok{\textquotesingle{}resize\textquotesingle{}}\OperatorTok{,}\NormalTok{ () }\KeywordTok{=\textgreater{}}\NormalTok{ \{}
  \BuiltInTok{console}\OperatorTok{.}\FunctionTok{log}\NormalTok{(}\StringTok{\textquotesingle{}screen size has changed!\textquotesingle{}}\NormalTok{)}\OperatorTok{;}
  \BuiltInTok{console}\OperatorTok{.}\FunctionTok{log}\NormalTok{(}\VerbatimStringTok{\textasciigrave{}}\SpecialCharTok{$\{}\BuiltInTok{process}\OperatorTok{.}\AttributeTok{stdout}\OperatorTok{.}\AttributeTok{columns}\SpecialCharTok{\}}\VerbatimStringTok{x}\SpecialCharTok{$\{}\BuiltInTok{process}\OperatorTok{.}\AttributeTok{stdout}\OperatorTok{.}\AttributeTok{rows}\SpecialCharTok{\}}\VerbatimStringTok{\textasciigrave{}}\NormalTok{)}\OperatorTok{;}
\NormalTok{\})}\OperatorTok{;}
\end{Highlighting}
\end{Shaded}

\subsubsection{\texorpdfstring{\texttt{writeStream.clearLine(dir{[},\ callback{]})}}{writeStream.clearLine(dir{[}, callback{]})}}\label{writestream.clearlinedir-callback}

\begin{itemize}
\tightlist
\item
  \texttt{dir} \{number\}

  \begin{itemize}
  \tightlist
  \item
    \texttt{-1}: to the left from cursor
  \item
    \texttt{1}: to the right from cursor
  \item
    \texttt{0}: the entire line
  \end{itemize}
\item
  \texttt{callback} \{Function\} Invoked once the operation completes.
\item
  Returns: \{boolean\} \texttt{false} if the stream wishes for the
  calling code to wait for the
  \texttt{\textquotesingle{}drain\textquotesingle{}} event to be emitted
  before continuing to write additional data; otherwise \texttt{true}.
\end{itemize}

\texttt{writeStream.clearLine()} clears the current line of this
\texttt{WriteStream} in a direction identified by \texttt{dir}.

\subsubsection{\texorpdfstring{\texttt{writeStream.clearScreenDown({[}callback{]})}}{writeStream.clearScreenDown({[}callback{]})}}\label{writestream.clearscreendowncallback}

\begin{itemize}
\tightlist
\item
  \texttt{callback} \{Function\} Invoked once the operation completes.
\item
  Returns: \{boolean\} \texttt{false} if the stream wishes for the
  calling code to wait for the
  \texttt{\textquotesingle{}drain\textquotesingle{}} event to be emitted
  before continuing to write additional data; otherwise \texttt{true}.
\end{itemize}

\texttt{writeStream.clearScreenDown()} clears this \texttt{WriteStream}
from the current cursor down.

\subsubsection{\texorpdfstring{\texttt{writeStream.columns}}{writeStream.columns}}\label{writestream.columns}

A \texttt{number} specifying the number of columns the TTY currently
has. This property is updated whenever the
\texttt{\textquotesingle{}resize\textquotesingle{}} event is emitted.

\subsubsection{\texorpdfstring{\texttt{writeStream.cursorTo(x{[},\ y{]}{[},\ callback{]})}}{writeStream.cursorTo(x{[}, y{]}{[}, callback{]})}}\label{writestream.cursortox-y-callback}

\begin{itemize}
\tightlist
\item
  \texttt{x} \{number\}
\item
  \texttt{y} \{number\}
\item
  \texttt{callback} \{Function\} Invoked once the operation completes.
\item
  Returns: \{boolean\} \texttt{false} if the stream wishes for the
  calling code to wait for the
  \texttt{\textquotesingle{}drain\textquotesingle{}} event to be emitted
  before continuing to write additional data; otherwise \texttt{true}.
\end{itemize}

\texttt{writeStream.cursorTo()} moves this \texttt{WriteStream}'s cursor
to the specified position.

\subsubsection{\texorpdfstring{\texttt{writeStream.getColorDepth({[}env{]})}}{writeStream.getColorDepth({[}env{]})}}\label{writestream.getcolordepthenv}

\begin{itemize}
\tightlist
\item
  \texttt{env} \{Object\} An object containing the environment variables
  to check. This enables simulating the usage of a specific terminal.
  \textbf{Default:} \texttt{process.env}.
\item
  Returns: \{number\}
\end{itemize}

Returns:

\begin{itemize}
\tightlist
\item
  \texttt{1} for 2,
\item
  \texttt{4} for 16,
\item
  \texttt{8} for 256,
\item
  \texttt{24} for 16,777,216 colors supported.
\end{itemize}

Use this to determine what colors the terminal supports. Due to the
nature of colors in terminals it is possible to either have false
positives or false negatives. It depends on process information and the
environment variables that may lie about what terminal is used. It is
possible to pass in an \texttt{env} object to simulate the usage of a
specific terminal. This can be useful to check how specific environment
settings behave.

To enforce a specific color support, use one of the below environment
settings.

\begin{itemize}
\tightlist
\item
  2 colors: \texttt{FORCE\_COLOR\ =\ 0} (Disables colors)
\item
  16 colors: \texttt{FORCE\_COLOR\ =\ 1}
\item
  256 colors: \texttt{FORCE\_COLOR\ =\ 2}
\item
  16,777,216 colors: \texttt{FORCE\_COLOR\ =\ 3}
\end{itemize}

Disabling color support is also possible by using the \texttt{NO\_COLOR}
and \texttt{NODE\_DISABLE\_COLORS} environment variables.

\subsubsection{\texorpdfstring{\texttt{writeStream.getWindowSize()}}{writeStream.getWindowSize()}}\label{writestream.getwindowsize}

\begin{itemize}
\tightlist
\item
  Returns: \{number{[}{]}\}
\end{itemize}

\texttt{writeStream.getWindowSize()} returns the size of the TTY
corresponding to this \texttt{WriteStream}. The array is of the type
\texttt{{[}numColumns,\ numRows{]}} where \texttt{numColumns} and
\texttt{numRows} represent the number of columns and rows in the
corresponding TTY.

\subsubsection{\texorpdfstring{\texttt{writeStream.hasColors({[}count{]}{[},\ env{]})}}{writeStream.hasColors({[}count{]}{[}, env{]})}}\label{writestream.hascolorscount-env}

\begin{itemize}
\tightlist
\item
  \texttt{count} \{integer\} The number of colors that are requested
  (minimum 2). \textbf{Default:} 16.
\item
  \texttt{env} \{Object\} An object containing the environment variables
  to check. This enables simulating the usage of a specific terminal.
  \textbf{Default:} \texttt{process.env}.
\item
  Returns: \{boolean\}
\end{itemize}

Returns \texttt{true} if the \texttt{writeStream} supports at least as
many colors as provided in \texttt{count}. Minimum support is 2 (black
and white).

This has the same false positives and negatives as described in
\hyperref[writestreamgetcolordepthenv]{\texttt{writeStream.getColorDepth()}}.

\begin{Shaded}
\begin{Highlighting}[]
\BuiltInTok{process}\OperatorTok{.}\AttributeTok{stdout}\OperatorTok{.}\FunctionTok{hasColors}\NormalTok{()}\OperatorTok{;}
\CommentTok{// Returns true or false depending on if \textasciigrave{}stdout\textasciigrave{} supports at least 16 colors.}
\BuiltInTok{process}\OperatorTok{.}\AttributeTok{stdout}\OperatorTok{.}\FunctionTok{hasColors}\NormalTok{(}\DecValTok{256}\NormalTok{)}\OperatorTok{;}
\CommentTok{// Returns true or false depending on if \textasciigrave{}stdout\textasciigrave{} supports at least 256 colors.}
\BuiltInTok{process}\OperatorTok{.}\AttributeTok{stdout}\OperatorTok{.}\FunctionTok{hasColors}\NormalTok{(\{ }\DataTypeTok{TMUX}\OperatorTok{:} \StringTok{\textquotesingle{}1\textquotesingle{}}\NormalTok{ \})}\OperatorTok{;}
\CommentTok{// Returns true.}
\BuiltInTok{process}\OperatorTok{.}\AttributeTok{stdout}\OperatorTok{.}\FunctionTok{hasColors}\NormalTok{(}\DecValTok{2} \OperatorTok{**} \DecValTok{24}\OperatorTok{,}\NormalTok{ \{ }\DataTypeTok{TMUX}\OperatorTok{:} \StringTok{\textquotesingle{}1\textquotesingle{}}\NormalTok{ \})}\OperatorTok{;}
\CommentTok{// Returns false (the environment setting pretends to support 2 ** 8 colors).}
\end{Highlighting}
\end{Shaded}

\subsubsection{\texorpdfstring{\texttt{writeStream.isTTY}}{writeStream.isTTY}}\label{writestream.istty}

A \texttt{boolean} that is always \texttt{true}.

\subsubsection{\texorpdfstring{\texttt{writeStream.moveCursor(dx,\ dy{[},\ callback{]})}}{writeStream.moveCursor(dx, dy{[}, callback{]})}}\label{writestream.movecursordx-dy-callback}

\begin{itemize}
\tightlist
\item
  \texttt{dx} \{number\}
\item
  \texttt{dy} \{number\}
\item
  \texttt{callback} \{Function\} Invoked once the operation completes.
\item
  Returns: \{boolean\} \texttt{false} if the stream wishes for the
  calling code to wait for the
  \texttt{\textquotesingle{}drain\textquotesingle{}} event to be emitted
  before continuing to write additional data; otherwise \texttt{true}.
\end{itemize}

\texttt{writeStream.moveCursor()} moves this \texttt{WriteStream}'s
cursor \emph{relative} to its current position.

\subsubsection{\texorpdfstring{\texttt{writeStream.rows}}{writeStream.rows}}\label{writestream.rows}

A \texttt{number} specifying the number of rows the TTY currently has.
This property is updated whenever the
\texttt{\textquotesingle{}resize\textquotesingle{}} event is emitted.

\subsection{\texorpdfstring{\texttt{tty.isatty(fd)}}{tty.isatty(fd)}}\label{tty.isattyfd}

\begin{itemize}
\tightlist
\item
  \texttt{fd} \{number\} A numeric file descriptor
\item
  Returns: \{boolean\}
\end{itemize}

The \texttt{tty.isatty()} method returns \texttt{true} if the given
\texttt{fd} is associated with a TTY and \texttt{false} if it is not,
including whenever \texttt{fd} is not a non-negative integer.
